\documentclass[main.tex]{subfiles}
\begin{document}
\section{Introduction}
A \emph{lattice valuation}
is a map~$\varphi$ from a lattice~$L$
to an ordered Abelian group~$E$
which is order preserving and \emph{modular}, i.e.,
\begin{equation*}
\varphi(x)+\varphi(y) \ =\ 
\varphi(x\wedge y)+ \varphi(x \vee y)
\qquad(x,y\in L).
\end{equation*}
Both the Lebesgue measure (considered
a function from the lattice of subsets of finite Lebesgue measure 
to~$\R$)
and the Lebesgue integral (considered
a function from the lattice of Lebesgue integrable 
functions to~$\R$) are lattice valuations.
\todo{Add historical remarks? 
E.g.:  Birkhoff already studied lattice valuations.
Hausdorff tried to axiomatise
measure theory with valuations on lattices of subsets.}

In this sense, a lattice valuation
is a generalisation of measure and integral.
In this thesis,
we formulate and prove a few elementary
theorems concerning measure and integral.


We give more examples 
of lattice valuations
and list some properties
in Section~\ref{S:vals}.

One can introduce the Lebesgue measure
by first assign


\end{document}
