\documentclass[main.tex]{subfiles}
\begin{document}
\begin{flushright}
\begin{minipage}{.7\columnwidth}
\begin{flushright}
The theory of integration,
because of its 
central r\^ole 
in mathematical analysis and geometry,
continues to afford opportunities 
for serious investigation.\\
--- \textsc{M.H. Stone}, 1948, \cite{Stone48}
\end{flushright}
\end{minipage}
\end{flushright}

\section{Introduction}
\noindent
A \emph{(lattice) valuation}
is a map~$\varphi$ from a lattice~$L$
to an ordered Abelian group~$E$
which is order preserving and \emph{modular}, i.e.,
\begin{equation*}
\varphi(x)+\varphi(y) \ =\ 
\varphi(x\wedge y)+ \varphi(x \vee y)
\qquad(x,y\in L).
\end{equation*}
Both the Lebesgue measure 
$\Lmu\colon \LA \ra \R$
(where $\LA$ is the lattice of subsets of~$\R$ with finite Lebesgue measure)
and the Lebesgue integral 
$\Lphi\colon \LF \ra \R$ 
(where $\LF$ is the lattice of Lebesgue integrable functions from~$\R$
to~$\E$) 
are valuations.

So in a sense, a  valuation
is a generalisation of measure and integral.



In this thesis,
we support this claim by generalising some
of the measure and integration theory to 
the setting of  valuations.
We give an overview in this section.

Most notably,
we generalise the `completion procedures'
by which the Lebesgue measure and
Lebesgue integral can be introduced.


\subsection{Completion}
Let us describe these `completion procedures'.

To introduce the Lebesgue measure $\Lmu$
(that is,
to identify the Lebesgue measurable sets~$\LA$
and assign to each set~$A\in\LA$ a Lebesgue measure)
one can start by
assigning a measure to a finite disjoint union
of intervals 
$I_1 \cup\dotsb\cup I_N$:
the sum of the lengths of the intervals,
$|I_1|+\dots+|I_N|$.
One can prove that the set of these disjoint
unions of intervals is a lattice, $\SA$,
and that measure we have just assigned
to each element
yields  a valuation $\Smu\colon \SA\ra\R$.

The Lebesgue measure~$\Lmu$
(which we wish to obtain from~$\Smu$)
is then an extension of~$\Smu$ which is complete
in the several ways. For one, we have the following.
\begin{equation*}
\left[\quad
\begin{minipage}{.7\columnwidth}
Let $A_1\subseteq,A_2\subseteq\dotsb$
be elements of~$\LA$
such that 
\begin{equation*}
\textstyle{\bv_n \Lmu(A_n)}\text{\quad exists.}
\end{equation*}
Then $\bigcup_n A_n \in \LA$,
and we have 
\begin{equation*}
\textstyle{\Lmu (\,\bigcup_n A_n\,) \ =\  \bv_n \Lmu(A_n)}.
\end{equation*}
\end{minipage}
\right.
\end{equation*}

\vspace{.3em}
\noindent For the Lebesgue integral
we can set up a similar situation.
If one wants to introduce the Lebesgue integral~$\Lphi$,
one can start by assigning to each step function
the obvious integral.
The step functions form a lattice~$\SF$,
and the assignment of an integral to the step functions
gives a valuation $\Lphi\colon \SF\ra\R$.

The Lebesgue measure $\Lphi$
(which we wish to obtain from~$\Sphi$)
is then an extension of~$\Sphi$
which is complete in the following sense.
\begin{equation*}
\left[\quad
\begin{minipage}{.7\columnwidth}
Let $f_1 \leq f_2 \leq \dotsb$
be from~$\LF$
such that 
\begin{equation*}
\textstyle{\bv_n \Lphi(f_n)}\text{\quad exists.}
\end{equation*}
Then $\bv_n f_n \in \LF$,
and we have 
\begin{equation*}
\textstyle{\Lphi (\,\bv_n A_n\,) \ =\  \bv_n \Lphi(f_n)}.
\end{equation*}
\end{minipage}
\right.
\end{equation*}


\vspace{.3em}
\noindent
So we see that both the Lebesgue measure
and the Lebesgue integral
are extensions of relatively simple valuations,
and that they are complete in a certain sense.
The similarity does not end here.
We formulate a notion of completeness for
valuations~$\varphi\colon L\ra E$ relative
to a lattice~$V$ which is complete in an appropriate sense
and such that~$L$ is a sublattice of~$V$.
So we consider systems of the following form.
\begin{equation*}
\vs{V}{L}\varphi{E}.
\end{equation*}
We call these systems \emph{valuation systems}
and study them in Section~\ref{S:valuation-systems}.
The notion of completeness for valuation systems
 generalizes the aforementioned
notions of completeness.
In particular, the following valuation systems are complete.
\begin{equation*}
\vsLA,\qquad\qquad\vsLF.
\end{equation*}

For these complete valuation systems,
we can formulate and prove variants
of some of the convergence theorems from
integration theory,
such as the Lemma of Fatou (Lemma~\ref{L:fatou})
and Lebesgue's Dominated Convergence Theorem
 (Theorem~\ref{T:Lebesgue}).

\subsection{Hierarchy}
The main question
of this thesis 
is: 
\begin{quote}
Given a valuation system $\vs{V}{L}\varphi{E}$,
 under which conditions can we extend $\varphi$
to a valuation $\psi$ such that
$\vs{V}{C}\psi{E}$ is complete?
\end{quote}
The question 
naturally leads us to consider an
hierarchy 
(see Section~\ref{S:completion})
of partial completions
akin to the
Borel Hierarchy in the case of~$\Smu$
and the Baire Hierarchy in the case of~$\Sphi$.
The corresponding picture may be familiar to the reader.
\begin{equation*}
\xymatrix @=10pt {
& \Sigma\varphi \ar @{-} [rr] \ar @{-} [rrdd]
&& \Sigma_2 \varphi  \ar @{-} [r]\ar @{-} [rd]
&& \dotsb
& \Sigma_{\omega} \varphi \ar @{-} [rr] \ar @{-} [rrdd]
                          \ar @{=} [dd]
&& \Sigma_{\omega+1}\varphi \ar @{-} [rr] \ar @{-} [rrdd]
&& \Sigma_{\omega+2} \varphi  \ar @{-} [r]\ar @{-} [rd]
&& \dotsb
\\  
\varphi \ar @{-} [ru] \ar @{-} [rd] 
&&&&&\dotsb
&&
&&&&&\dotsb\\
& \Pi\varphi \ar @{-} [rr] \ar @{-} [rruu]
&& \Pi_{2} \varphi \ar @{-} [r] \ar @{-} [ru]
&& \dotsb
& \Pi_{\omega}\varphi \ar @{-} [rr] \ar @{-} [rruu]
&& \Pi_{\omega+1}\varphi \ar @{-} [rr] \ar @{-} [rruu]
&& \Pi_{\omega+2} \varphi \ar @{-} [r] \ar @{-} [ru]
&& \dotsb
}
\end{equation*}


We will see that it is not easy to determine whether
a given valuation system 
$\vs{V}{L}\varphi{E}$
can be `completed'.
Fortunately, 
the situation is more tractable
for some $E$. 
We will not go into the details yet,
but can say
that we call such~$E$ \emph{benign}
and that $\R$ is one example.
We study these~$E$ in Section~\ref{S:benign}.

Let us for the moment assume that $\varphi$ can be completed.
Then we are able to prove that there is
a smallest complete extension $\vs{V}{\overline L}{\overline\varphi}{E}$.
We will see that $\overline{\Smu}$ is essentially 
the Lebesgue measure $\Lmu$:
$\overline{\Smu}$ is the Lebesgue measure restricted to the Borel sets.
Similarly,
 $\overline{\Sphi}$ turns out to be what one might
call the Baire integral.\todo{add reference}

Moreover,
we will prove that this completion $\overline L$
is closed under certain operations (see Section~\ref{S:closedness}).
For instance,
we will prove that if $f,g\in \overline{\SF}$
and $f$ and $g$ do not attain the values $+\infty$ and $-\infty$,
then $f+g\in\overline{\SF}$ and 
$\overline{\Sphi}(f+g)=\overline{\Sphi}(f) + \overline{\Sphi}(g)$.

\subsection{Fitting Uniformity}
To prove that $\Smu$ and $\Sphi$
can be `completed'
(without presupposing $\Lmu$ and $\Lphi$ exist)
we draw on the topological structure of~$\R$.
More precisely,
we will consider ordered Abelian groups~$E$
which are endowed with a certain uniform structure
(such as~$\R$) and prove
that these are benign (see Section~\ref{S:unif}).

Interestingly,
with the additional uniform structure on~$E$,
we can prove that certain operations
that are (initially only) defined on~$L$
can be extended to the completion~$\ol{L}$
(see Theorem~\ref{T:fubext}).

As an example,
we will deduce a variant of Fubini's Theorem.
\todo{Add ref.}

\subsection{Rings and Riesz Spaces}
There is a long list of axiomatisations of measure and
axiomatisations of integral
which we could compare to the theory set forth in this text.
We will not do this.

Nevertheless,
as an example,
we will apply our theory
to a positive additive map $\mu$ on a ring $\mathcal{A}$ of
subsets of~$X$
(which is an axiomatisation of measure)
and 
to a positive linear 
map $\varphi$ on a Riesz space~$F$ of functions on~$X$
(which is a axiomatisation of integral).
That is,
we consider the following valuation systems.
\begin{equation*}
\vs{\wp X}{\mathcal{A}}{\mu}{E},
\qquad\qquad
\vs{\EX}{F}{\varphi}{E}.
\end{equation*}
We will return to these two valuation systems throughtout the paper
and hope that this will clarify the theory
and the similarity between measure and integral.


\end{document}
