\documentclass[main.tex]{subfiles}
\begin{document}
\thispagestyle{empty}
\begin{flushright}
\begin{minipage}{.7\columnwidth}
\begin{flushright}
The theory of integration,
because of its 
central r\^ole 
in mathematical analysis and geometry,
continues to afford opportunities 
for serious investigation.\\
--- \textsc{M.H. Stone}, 1948, \cite{Stone48}
\end{flushright}
\end{minipage}
\end{flushright}
\clearpage
\section{Introduction}
\noindent
There are many ways (some more popular than
others) to introduce the Lebesgue measure
and the Lebesgue integral.
For the purposes of this introduction,
we define the Lebesgue measure and integral
in such a way that the similarity between them is obvious.
This similarity is the basis of this thesis.
We leave it to the reader to 
compare the  definitions below
to those that are familiar to him/her.
\begin{dfn}
\label{D:lebesgue-measure}
The \keyword{Lebesgue measure}
$\Lmu \colon \LA \ra \R$
is the smallest\footnote{%
``Smallest'' with respect to the following order.
We say that $\mu_1$ \emph{is extended by}~$\mu_2$
where $\mu_i\colon \mathcal{A}_i \ra \R$
and $\mathcal{A}_i \subseteq \wp(\R)$
provided that
$\mathcal{A}_1 \subseteq \mathcal{A}_2$,
and $\mu_1(A) = \mu_2(A)$
for all~$A\in\mathcal{A}_1$.}
$\mu\colon \mathcal{A} \ra \R$
where $\mathcal{A}$ is a subset of~$\wp(\R)$
that has the following properties.
\begin{enumerate}
\item
\label{prop:measure-1}
Let $a,b\in\R$ with $a\leq b$. Then $[a,b]\in\mathcal{A}$
and $(a,b)\in\mathcal{A}$, 
and
\begin{equation*}
\mu(\,[a,b]\,)
\ = \ \mu(\,(a,b)\,)
\ =\ b-a.
\end{equation*}

\item
\label{prop:measure-2}
\textit{(Monotonicity)}\ 
Let $A,B\in \mathcal{A}$.
Then $\mu(A)\leq \mu(B)$
when  $A\subseteq B$.

\item
\label{prop:measure-3}
\textit{(Modularity)}\ 
Let $A,B\in \mathcal{A}$.
Then $A\cap B\in\mathcal{A}$ and $A\cup B \in \mathcal{A}$, and
\begin{equation*}
\mu(\,A\cap B\,)\,+\,\mu(\,A\cup B\,)\ =\ \mu(A)\,+\,\mu(B).
\end{equation*}

\item
\label{prop:measure-4}
\textit{($\Pi$-Completeness)}\ 
Let $A_1 \supseteq A_2 \supseteq \dotsb$
from~$\mathcal{A}$ be given.\\
Assume that the set $\{\,\mu(A_1),\, \mu(A_2),\, \dotsc\,\}$
has an infimum, $\bw_n \mu(A_n)$.\\
Then we have $\bigcap_n A_n \in \mathcal{A}$.
Moreover,
\begin{equation*}
\mu(\, \textstyle{ \bigcap_n A_n }\,) \ =\ \bw_n \,\mu(A_n).
\end{equation*}

\item
\label{prop:measure-5}
\textit{($\Sigma$-Completeness)}\ 
Let $A_1 \subseteq A_2 \subseteq \dotsb$
from~$\mathcal{A}$ 
be such that $\bv_n \mu(A_n)$ exists.\\
Then we have $\bigcup_n A_n \in \mathcal{A}$.
Moreover,
\begin{equation*}
\mu(\, \textstyle{ \bigcup_n A_n }\,) \ =\ \bv_n \,\mu(A_n).
\end{equation*}

\item
\label{prop:measure-6}
\textit{(Convexity)}
Let $A\subseteq Z \subseteq B$ be subsets of~$\R$.\\
Assume that $A,B\in\mathcal{A}$ and $\mu(A)=\mu(B)$.\\
Then we have $Z\in \mathcal{A}$ and $\mu(A) = \mu(Z)= \mu(B)$.
\end{enumerate}
\end{dfn}



\begin{dfn}
\label{D:lebesgue-integral}
The \keyword{Lebesgue integral}
$\Lphi \colon \LF \ra \R$
is the smallest  $\varphi\colon F \ra \R$
where $F$ is a subset of~$\E^\R$
that has the following properties.
\begin{enumerate}
\item
\label{prop:integral-1}
Let $a,b,\lambda\in\R$ with $a\leq b$. 
Then $\lambda\cdot\mathbf{1}_{[a,b]}\in F$
and $\lambda\cdot\mathbf{1}_{(a,b)}\in F$,
and
\begin{equation*}
\varphi(\,\lambda\cdot \mathbf{1}_{[a,b]}\,)
\ =\ \varphi(\,\lambda\cdot \mathbf{1}_{(a,b)}\,)
\ =\ \lambda\cdot(b-a).
\end{equation*}

\item
\label{prop:integral-2}
\textit{(Monotonicity)}\ 
Let $f,g\in F$.
Then $\varphi(f)\leq \varphi(g)$
when  $f\leq g$.

\item
\label{prop:integral-3}
\textit{(Modularity)}\ 
Let $f,g\in F$.
Then $f\wedge g\in F$ and $f \vee g \in F$, and
\begin{equation*}
\varphi(\,f\wedge g\,)\,+\,\varphi(\,f\vee g\,)\ =\ \varphi(f)\,+\,\varphi(g).
\end{equation*}

\item
\label{prop:integral-4}
\textit{($\Pi$-Completeness)}\ 
Let $f_1 \geq f_2 \geq \dotsb$
from~$F$ be such that $\bw_n \varphi(f_n)$ exists.\\
Then we have $\bw_n f_n \in F$.
Moreover,
\begin{equation*}
\varphi(\, \textstyle{ \bigwedge_n f_n }\,) \ =\ \bw_n \,\varphi(f_n).
\end{equation*}
Here $\bw_n f_n$
is the infimum of $\{\,f_1,\,f_2,\,\dotsc\,\}$ in $\E^\R$;
more concretely,
it is the \emph{pointwise infimum}, i.e., $(\bw_n f_n)(x) = \bw_n f_n(x)$
for all~$x\in \R$.

\item
\label{prop:integral-5}
\textit{($\Sigma$-Completeness)}\ 
Let $f_1 \leq f_2 \leq \dotsb$
from~$F$ 
be such that $\bv_n \varphi(f_n)$ exists.\\
Then we have $\bigvee_n f_n \in F$.
Moreover,
\begin{equation*}
\varphi(\, \textstyle{ \bigvee_n f_n }\,) \ =\ \bv_n \,\varphi(f_n).
\end{equation*}

\item
\label{prop:integral-6}
\textit{(Convexity)}
Let $f\leq z \leq g$ be $\E$-valued functions on~$\R$.\\
Assume that $f,g\in F$ and $\varphi(f)=\varphi(g)$.\\
Then we have $z\in F$ and $\varphi(f) = \varphi(z)= \varphi(g)$.
\end{enumerate}
\end{dfn}



\noindent
In this thesis
we present an abstract theory based on the properties
(Monotonicity), (Modularity), ($\Pi$-Completeness),
($\Sigma$-Completeness) and (Convexity)
and we try to fit some of the results
of measure and integration theory
in this framework.

\subsection{Valuations}
We begin by considering (Monotonicity) and (Modularity).

Maps with these two properties 
are called \emph{(lattice) valuations}.
More precisely,
let~$L$ be a lattice,
and let~$E$ be an
ordered Abelian group (e.g.~$\R$, see Appendix~\ref{S:ag}).
A map $\varphi\colon L\ra E$ is a \emph{valuation}
if it is order preserving and \emph{modular}, i.e.,
\begin{equation*}
\varphi(x)\,+\,\varphi(y) \ =\ 
\varphi(x\wedge y)\,+\, \varphi(x \vee y)
\qquad(x,y\in L).
\end{equation*}

Of course,
the Lebesgue measure~$\Lmu$
and the Lebesgue integral~$\Lphi$
are valuations,
and there are many more examples.
We study valuation in Section~\ref{S:vals}.



\subsection{Valuation Systems}
Let us now look at ($\Pi$-Completeness).
For the Lebesgue measure it
involves intersections, ``$\bigcap_n A_n$'',
i.e., infima in~$\wp(\R)$.
Similarly,
for the Lebesgue integral
it
involves pointwise infima, ``$\bw_n f_n$'',
i.e., infima in~$\smash{\E^\R}$.
In order to
generalise 
the notion of 
($\Pi$-Completeness) 
to any valuation $\varphi\colon L\ra E$
we involve a `surrounding' lattice,~$V$.
That is, we will define what
it means for an object of the following shape
to be \emph{$\Pi$-complete}
(see Definition~\ref{D:system-complete}).
\begin{equation*}
\vs{V}{L}\varphi{E}
\end{equation*}
We call these objects \emph{valuation systems},
and we study them in Section~\ref{S:system}.

The Lebesgue measure and the Lebesgue integral give us valuation systems:
\begin{equation*}
\vsLA\qquad\text{and}\qquad\vsLF.
\end{equation*}
Of course
these valuation systems are $\Pi$-complete
by ($\Pi$-Completeness).\\
They are also \emph{$\Sigma$-complete},
which is a generalisation of  ($\Sigma$-Completeness).

Finally,
(Convexity)
can easily generalised to valuation systems as well.
We will define what it means
for a valuation system to be \emph{convex}
in  Definition~\ref{D:convex}.
We study these convex valuation systems in Subsection~\ref{SS:convex}.

\vspace{1em}
\noindent
Now that we have introduced
the main objects of study,
valuations and valuation systems,
let us spend some words on the theorems that we will prove.


\subsection{Completion and Convexification}
Recall that we defined
the Lebesgue measure~$\Lmu$
as the smallest map $\mu\colon \mathcal{A}\ra \R$
that has properties~\ref{prop:measure-1}--\ref{prop:measure-6}.
It is important to note that
it is not obvious at all that such a map exists.
While it is relatively easy to see
that if there is a map $\mu\colon \mathcal{A} \ra \R$
that has properties~\ref{prop:measure-1}--\ref{prop:measure-6},
then there is also a smallest one,
it takes quite some effort
to prove that there is any map $\mu\colon \mathcal{A}\ra \R$
with properties~\ref{prop:measure-1}--\ref{prop:measure-6} to begin with.

One could call this statement the \emph{Extension Theorem
for the Lebesgue measure}.
Similarly,
to define~$\Lphi$,
we need an \emph{Extension Theorem
for the Lebesgue integral}.

We will generalise (a part) of these two theorems
to the setting of valuations.
To see how we could do this,
note that to prove
the Extension Theorem for the Lebesgue measure,
one could take the following three steps.
\begin{enumerate}
\item
\label{extension-step-1}
\emph{Find the smallest map $\Smu\colon \SA\ra \R$
that has properties~\ref{prop:measure-1}--\ref{prop:measure-3}.}

This is not so difficult.
Let $\mathcal{S}$ be the family of subsets of~$\R$ of
the form~$[a,b]$ or~$(a,b)$ where $a,b\in \R$ with $a\leq b$.
Let~$\SA$ be the set of all unions
of finite disjoint subsets of~$\mathcal{S}$,
and let $\Smu\colon \SA\ra \R$
be given by
\begin{equation*}
\Smu(\,I_1 \cup\dotsb\cup I_N\,) \ = \ |I_1| \,+\,\dotsb\,+\,|I_N|,
\end{equation*}
where $I_1,\dotsc,I_N\,\in\, \mathcal{S}$
with $I_n\cap I_m = \varnothing$ when $n\neq m$.

Of course, it requires some calculations
to see that such a map~$\Smu$ exists,
and that~$\Smu$ will have the 
properties~\ref{prop:measure-1}--\ref{prop:measure-3}
(see Example~\ref{E:smeas-val}).

\item
\label{extension-step-2}
\emph{Extend~$\Smu$
to the smallest map $\overline{\Smu}\colon\overline{\SA}\ra\R$
that has properties~\ref{prop:measure-1}--\ref{prop:measure-5}.}

This is the most interesting and the most difficult step.
To give an idea of how one could try obtain such~$\overline{\Smu}$,
consider the following `algorithm'.
\begin{equation*}
\left[\quad
\begin{minipage}{.75\columnwidth}
Let $\mu\colon\mathcal{A} \ra \R$
be a variable. To begin, set $\mu \eqdf \Smu$.\\
$\mathbf{(*)}$ For all~$A_1,A_2,\dotsc$ from~$\mathcal{A}$ 
do the following.
\begin{enumerate}
\item[$\bullet$] If $A_1 \supseteq A_2 \supseteq \dotsb$
and  $\bw_n \mu(A_n)$ exists
and  $\bigcap_n A_n \notin \mathcal{A}$,\\
then add $\bigcap_n A_n$ to~$\mathcal{A}$,
and set $\mu(\bigcap_n A_n) \eqdf \bw_n \mu(A_n)$.
\item[$\bullet$] If $A_1 \subseteq A_2 \subseteq \dotsb$
and  $\bv_n \mu(A_n)$ exists
and  $\bigcup_n A_n \notin \mathcal{A}$,\\
then add $\bigcup_n A_n$ to~$\mathcal{A}$,
and set $\mu(\bigcup_n A_n) \eqdf \bv_n \mu(A_n)$.
\end{enumerate}
If~$\mu$ was changed, repeat~$\mathbf{(*)}$.
\end{minipage}
\right.
\end{equation*}
There are many problems with this `algoritm'.
Perhaps the most serious problem
is, loosely speaking, that the same set~$A$ may be obtained
in several ways
and it is not clear that
$\mu(A)$ would be given the same value each time.

Note that the `algorithm' resembles the  definition
of the Borel sets.
In fact, $\overline{\Smu}$
will be the family of all Borel subsets of~$\R$
with finite measure.


\item
\label{extension-step-3}
\emph{Extend~$\overline{\Smu}$
to the smallest map~$\Lmu\colon \LA \ra \R$
that has properties~\ref{prop:measure-1}--\ref{prop:measure-6}.}

This is straightforward.
Simply
define $\LA$ to be the family of all subsets of~$\R$
that are `sandwiched' between elements of~$\overline{\SA}$,
that is,
all $Z\in\wp(\R)$
for which
there are $A,B\in \overline{\SA}$
such that $A\subseteq Z\subseteq B$ and 
$\overline{\Smu}(A) = \overline{\Smu}(B)$.

Now, define $\Lmu\colon \LA \ra \R$
by $\Lmu (Z) = \overline{\Smu} (A)$ for $Z$ and~$A$ as above.
\end{enumerate}

\noindent
We have sketched how 
to get
the Lebesgue measure~$\Lmu\colon \LA \ra \R$
in three steps, 
\begin{equation*}
\xymatrix{
\ar @{-->}[rr]^{\text{\ref{extension-step-1}}}
&&
\Smu
\ar @{-->}[rr]^{\text{\ref{extension-step-2}}}
&&
\Bmu
\ar @{-->}[rr]^{\text{\ref{extension-step-3}}}
&&
\Lmu.
}
\end{equation*}
We will generalise step~\ref{extension-step-2}
and step~\ref{extension-step-3}
to the setting of valuations.
More precisely:
\begin{enumerate}
\item
Let $\vs{V}{L}\varphi{E}$
be a valuation system.
We will give
a necessary and sufficient condition,
namely~$\vs{V}{L}\varphi{E}$
is \emph{extendible}
(see Definition~\ref{D:extendible}),
for when 
there is a smallest
valuation~$\overline\varphi\colon\overline{L}\ra E$
which extends~$\varphi$
where~$\overline{L}$ is a sublattice of~$V$
such that the valuation system
$\vs{V}{\overline{L}}{\overline\varphi}{E}$
is both $\Pi$-complete and $\Sigma$-complete
(see Lemma~\ref{L:complete} and Proposition~\ref{P:comp-minimal}).\\
We will call~$\overline\varphi$
the \emph{completion}
of~$\varphi$ (relative to~$V$).
\item
Let $\vs{V}{L}\varphi{E}$
be a valuation sytem.
We will prove the following.\\
There is smallest valuation~$\varphi^\bullet\colon L^\bullet\ra E$
extending~$\varphi$
with~$L^\bullet$ a sublattice of~$V$
such that 
$\vs{V}{L^\bullet}{\varphi^\bullet}{E}$ is convex
(see Propisition~\ref{P:convexification}).\\
Moreover,
$\vs{V}{L^\bullet}{\varphi^\bullet}{E}$
is $\Pi$-complete and $\Sigma$-complete
provided that
$\vs{V}{L}\varphi{E}$ 
is $\Pi$-complete and $\Sigma$-complete
(see Proposition~\ref{P:convexification_versus_completion}).\\
We will call $\varphi^\bullet$
the \emph{convexification} of~$\varphi$
(relative to~$V$).
\end{enumerate}
By the discussion above
we see that 
the Lebesgue measure~$\Lmu$
is the convexification of the completion of~$\Smu$
relative to~$\wp(\R)$:
\begin{equation*}
\xymatrix{
\Smu
\ar @{-->}[rrr]^{\text{completion}}
&&&
\Bmu
\ar @{-->}[rrr]^{\text{convexification}}
&&&
{\Lmu}
}
\end{equation*}
Similarly,
the Lebesgue integral~$\Lphi$
is the convexification 
of the completion of~$\Sphi$
relative to~$\smash{\E^\R}$,
where~$\Sphi\colon \SF\ra \R$ is the obvious
valuation on the set of \emph{step functions}~$\SF$
(see Example~\ref{E:sint-val}).
So we get the following diagram.
\begin{equation*}
\xymatrix{
\Sphi
\ar @{-->}[rrr]^{\text{completion}}
&&&
\ol\Sphi
\ar @{-->}[rrr]^{\text{convexification}}
&&&
{\Lphi}
}
\end{equation*}
Let us note that $\ol\Sphi\colon \ol{\SF}\ra \R$
will be the restriction of the Lebesgue integral
to the set~$\ol{\SF}$
of Lebesgue integrable \emph{Baire functions}.
We will not prove this.

We believe that
the completion is the most important step,
and that the convexification
is mere decoration.
In line with this believe,
we spend most words
on the completion,
and we leave it to the reader
to think about the convexification.


\subsection{Closedness under Operations}
We have found an abstract method
to get the Lebesgue measure~$\Lmu$ and
the Lebesgue integral~$\Lphi$.
However, such a method is nothing but a curiocity
if we cannot use it to derive some
basic properties of~$\Lmu$ and~$\Lphi$.
One such property might be:
\begin{equation*}
\left[\quad
\begin{minipage}{.7\columnwidth}
If $f,g\in \R^\R$
are Lebesgue integrable,\\
then $f+g$ is Lebesgue integrable,\\
and $\Lphi(f+g) = \Lphi(f)+\Lphi(g)$.
\end{minipage}
\right.
\end{equation*}
So, roughly speaking,
$\Lphi$ is \emph{closed under the operation~``$+$''}.
Instead of this,
we will prove that~$\overline{\Sphi}$ is closed under the operation~``$+$''.
We leave it to the reader to use this to prove that the convexification
 of~$\overline{\Sphi}$, i.e.~$\Lphi$, is closed under~``$+$'' as well.

More generally,
in Section~\ref{S:closedness}
we will prove some  statements of the following shape.
If $\vs{V}{L}\varphi{E}$
is a valuation system,
and~$\varphi$ is closed under some operation in some sense,
then the completion~$\ol{\varphi}$ is closed under the same operation as well.


\subsection{Convergence Theorems}
An important part of the theory of integration
is that of the convergence theorems.
So we have studied whether
these make sense in the setting of valuations.
We will show in Subsection~\ref{SS:complete-val_convergence}
that it is possible to formulate
and prove
the Lemma of Fatou
and Lebesgue's Dominated Convergence Theorem
for complete valuation systems.
Interestingly,
the surrounding lattice~$V$ will play no role.
This leads to the study 
of \emph{complete valuations}
(as opposed to complete valuation systems),
see Section~\ref{S:complete-val} for more details.


\subsection{Fubini's Theorem}
Another important part of the theory of integration
is Fubini's Theorem.
Unfortunately,
it seems that that it not possible
to make sense of Fubini's Theorem in
the general setting of valuations.

Nevertheless,
in Section~\ref{S:fubini} 
we will split the proof of Fubini's Theorem for the Lebesgue integral
into two parts.
The first part concerns step functions and is specific
to the Lebesgue integral,
while the second part is a consequence of a
general extension theorem for valuations
(see Theorem~\ref{T:fubext}).


\subsection{Extendibility}
We have remarked that a valuation system
$\vs{V}{L}\varphi{E}$
has a completion if and only if~$\varphi$ is \emph{extendible}.
As the reader will see in Subsection~\ref{SS:hierarchy-abstract}
the definition of ``$\varphi$ is extendible''
is rather involved.

Fortunately, 
the situation is simpler
for some choices of~$E$.
We say that~$E$ is \emph{benign}
if 
for every valuation system $\vs{V}{L}\varphi{E}$
we have  that $\varphi$ is extendible iff
\begin{equation*}
\left[\quad
\begin{minipage}{.7\columnwidth}
Let $a_1 \geq a_2 \geq \dotsb$ in~$L$
with $\bw_n\varphi(a_n)$ exists be given.\\
Let $b_1 \leq b_2 \leq \dotsb$ in~$L$ 
with $\bv_n \varphi(b_n)$ eixsts be given. \\
Then we have the following implication.
\begin{equation*}
\bw_n a_n  \,\leq\,\bv_n b_n
\quad\implies\quad
\varphi(\bw_n a_n)
\,\leq\,
\varphi(\bv_n b_n),
\end{equation*}
Here, $\bw_n a_n$
is the infimum of $a_1 \geq a_2 \geq \dotsb$ in~$V$,\\
and 
$\bv_n b_n$ is the supremum of~$b_1 \leq b_2 \leq \dotsb$ in~$V$.
\end{minipage}
\right.
\end{equation*}
We will prove that~$\R$ is benign.
More generally,
we will prove in Section~\ref{S:unif}
any ordered Abelian group~$E$
that has a suitable unformity
(see Definition~\ref{D:uniformity})
is benign.

\subsection{Historical Remarks}
All







\end{document}
