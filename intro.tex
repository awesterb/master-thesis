\documentclass[main.tex]{subfiles}
\begin{document}
\thispagestyle{empty}
\begin{flushright}
\begin{minipage}{.7\columnwidth}
\begin{flushright}
The theory of integration,
because of its 
central r\^ole 
in mathematical analysis and geometry,
continues to afford opportunities 
for serious investigation.\\
--- \textsc{M.H. Stone}, 1948, \cite{Stone48}
\end{flushright}
\end{minipage}
\end{flushright}
\clearpage
\section{Introduction}
\noindent
There are many ways (some more popular than
others) to introduce the Lebesgue measure
and the Lebesgue integral.
For the purposes of this text,
we define the Lebesgue measure and integral
in such a way that the similarity between them is obvious.
This similarity is the basis of this thesis.
We leave it to the reader to 
compare the  definitions below
to those that are familiar to him/her.
\begin{dfn}
\label{D:lebesgue-measure}
The \keyword{Lebesgue measure}
$\Lmu \colon \LA \ra \R$
is the smallest\footnote{%
``Smallest'' with respect to the following order.
We say that $\mu_1$ \emph{is extended by}~$\mu_2$
where $\mu_i\colon \mathcal{A}_i \ra \R$
and $\mathcal{A}_i \subseteq \wp(\R)$
provided that
$\mathcal{A}_1 \subseteq \mathcal{A}_2$,
and $\mu_1(A) = \mu_2(A)$
for all~$A\in\mathcal{A}_1$.}
$\mu\colon \mathcal{A} \ra \R$
where $\mathcal{A}$ is a subset of~$\wp(\R)$
that has the following properties.
\begin{enumerate}
\item
\label{prop:measure-1}
\textit{(Base)}\ 
Let $a,b\in\R$ with $a\leq b$. Then $[a,b]\in\mathcal{A}$, and
\begin{equation*}
\mu(\,[a,b]\,)\ =\ b-a.
\end{equation*}

\item
\label{prop:measure-2}
\textit{(Monotonicity)}\ 
Let $A,B\in \mathcal{A}$.
Then $\mu(A)\leq \mu(B)$
when  $A\subseteq B$.

\item
\label{prop:measure-3}
\textit{(Modularity)}\ 
Let $A,B\in \mathcal{A}$.
Then $A\cap B\in\mathcal{A}$ and $A\cup B \in \mathcal{A}$, and
\begin{equation*}
\mu(\,A\cap B\,)\,+\,\mu(\,A\cup B\,)\ =\ \mu(A)\,+\,\mu(B).
\end{equation*}

\item
\label{prop:measure-4}
\textit{($\Pi$-Completeness)}\ 
Let $A_1 \supseteq A_2 \supseteq \dotsb$
from~$\mathcal{A}$ be given.\\
Assume that the set $\{\,\mu(A_1),\, \mu(A_2),\, \dotsc\,\}$
has an infimum, $\bw_n \mu(A_n)$.\\
Then we have $\bigcap_n A_n \in \mathcal{A}$.
Moreover,
\begin{equation*}
\mu(\, \textstyle{ \bigcap_n A_n }\,) \ =\ \bw_n \,\mu(A_n).
\end{equation*}

\item
\label{prop:measure-5}
\textit{($\Sigma$-Completeness)}\ 
Let $A_1 \subseteq A_2 \subseteq \dotsb$
from~$\mathcal{A}$ 
be such that $\bv_n \mu(A_n)$ exists.\\
Then we have $\bigcup_n A_n \in \mathcal{A}$.
Moreover,
\begin{equation*}
\mu(\, \textstyle{ \bigcup_n A_n }\,) \ =\ \bv_n \,\mu(A_n).
\end{equation*}

\item
\label{prop:measure-6}
\textit{(Convexity)}
Let $A\subseteq Z \subseteq B$ be subsets of~$\R$.\\
Assume that $A,B\in\mathcal{A}$ and $\mu(A)=\mu(B)$.\\
Then we have $Z\in \mathcal{A}$ and $\mu(A) = \mu(Z)= \mu(B)$.
\end{enumerate}
\end{dfn}



\begin{dfn}
\label{D:lebesgue-integral}
The \keyword{Lebesgue integral}
$\Lphi \colon \LF \ra \R$
is the smallest  $\varphi\colon F \ra \R$
where $F$ is a subset of~$\E^\R$
that has the following properties.
\begin{enumerate}
\item
\label{prop:integral-1}
\textit{(Base)}\ 
Let $a,b,\lambda\in\R$ with $a\leq b$. 
Then $\lambda\cdot\mathbf{1}_{[a,b]}\in F$,
and
\begin{equation*}
\varphi(\,\lambda\cdot \mathbf{1}_{[a,b]}\,)\ =\ \lambda\cdot(b-a).
\end{equation*}

\item
\label{prop:integral-2}
\textit{(Monotonicity)}\ 
Let $f,g\in F$.
Then $\varphi(f)\leq \varphi(g)$
when  $f\leq g$.

\item
\label{prop:integral-3}
\textit{(Modularity)}\ 
Let $f,g\in F$.
Then $f\wedge g\in F$ and $f \vee g \in F$, and
\begin{equation*}
\varphi(\,f\wedge g\,)\,+\,\varphi(\,f\vee g\,)\ =\ \varphi(f)\,+\,\varphi(g).
\end{equation*}

\item
\label{prop:integral-4}
\textit{($\Pi$-Completeness)}\ 
Let $f_1 \geq f_2 \geq \dotsb$
from~$F$ be such that $\bw_n \varphi(f_n)$ exists.\\
Then we have $\bw_n f_n \in F$.
Moreover,
\begin{equation*}
\varphi(\, \textstyle{ \bigwedge_n f_n }\,) \ =\ \bw_n \,\varphi(f_n).
\end{equation*}
Here $\bw_n f_n$
is the infimum of $\{\,f_1,\,f_2,\,\dotsc\,\}$ in $\E^\R$;
more concretely,
it is the \emph{pointwise infimum}, i.e., $(\bw_n f_n)(x) = \bw_n f_n(x)$
for all~$x\in \R$.

\item
\label{prop:integral-5}
\textit{($\Sigma$-Completeness)}\ 
Let $f_1 \leq f_2 \leq \dotsb$
from~$F$ 
be such that $\bv_n \varphi(f_n)$ exists.\\
Then we have $\bigvee_n f_n \in F$.
Moreover,
\begin{equation*}
\varphi(\, \textstyle{ \bigvee_n f_n }\,) \ =\ \bv_n \,\varphi(f_n).
\end{equation*}

\item
\label{prop:integral-6}
\textit{(Convexity)}
Let $f\leq z \leq g$ be $\E$-valued functions on~$\R$.\\
Assume that $f,g\in F$ and $\varphi(f)=\varphi(g)$.\\
Then we have $z\in F$ and $\varphi(f) = \varphi(z)= \varphi(g)$.
\end{enumerate}
\end{dfn}



\noindent
In this thesis
we present an abstract theory based on the properties
(Monotonicity), (Modularity), ($\Pi$-Completeness),
($\Sigma$-Completeness) and (Convexity)
and we try to fit some of the classical results
of measure and integration theory
in this framework.

\subsection{Valuations}
We begin by considering (Monotonicity) and (Modularity).

Maps with these two properties 
are called \emph{(lattice) valuations}.
More precisely,
let~$L$ be a lattice,
and let~$E$ be an
ordered Abelian group (e.g.~$\R$, see Appendix~\ref{S:ag}).
A map $\varphi\colon L\ra E$ is a \emph{valuation}
if it is order preserving and \emph{modular}, i.e.,
\begin{equation*}
\varphi(x)\,+\,\varphi(y) \ =\ 
\varphi(x\wedge y)\,+\, \varphi(x \vee y)
\qquad(x,y\in L).
\end{equation*}

Of course,
the Lebesgue measure~$\Lmu$
and the Lebesgue integral~$\Lphi$
are valuations,
and there are many more examples.
We study valuation in Section~\ref{S:vals}.



\subsection{Valuation Systems}
Let us now look at ($\Pi$-Completeness).
For the Lebesgue measure it
involves intersections, ``$\bigcap_n A_n$'',
i.e., infima in~$\wp(\R)$.
Similarly,
for the Lebesgue integral
it
involves pointwise infima, ``$\bw_n f_n$'',
i.e., infima in~$\smash{\E^\R}$.
In order to
generalise 
the notion of 
($\Pi$-Completeness) 
to any valuation $\varphi\colon L\ra E$
we involve a `surrounding' lattice,~$V$.
That is, we will define what
it means for an object of the following shape
to be \emph{$\Pi$-complete}
(see Definition~\ref{D:system-complete}).
\begin{equation*}
\vs{V}{L}\varphi{E}
\end{equation*}
We call these objects \emph{valuation systems},
and we study them in Section~\ref{S:system}.

The Lebesgue measure and the Lebesgue integral give us valuation systems:
\begin{equation*}
\vsLA\qquad\text{and}\qquad\vsLF.
\end{equation*}
Of course
these valuation systems are $\Pi$-complete
by ($\Pi$-Completeness).\\
They are also \emph{$\Sigma$-complete},
which is a generalisation of  ($\Sigma$-Completeness).

Finally,
(Convexity)
can easily generalised to valuation systems as well.
We will define what it means
for a valuation system to be \emph{convex}
in  Definition~\ref{D:convex}.
We study these convex valuation systems in Subsection~\ref{SS:convex}.

\vspace{1em}
\noindent
Now that we have introduced
the main objects of study,
valuations and valuation systems,
let us spend some words on the theorems that we will prove.


\subsection{The Completion}
Recall that we have defined
the Lebesgue measure~$\Lmu$
as the smallest map $\mu\colon \mathcal{A}\ra \R$
that has properties~\ref{prop:measure-1}--\ref{prop:measure-6}.
It is important to note that
it is not obvious at all that such a map exists.
While it is relatively easy to see
that if there is a map $\mu\colon \mathcal{A} \ra \R$
that has properties~\ref{prop:measure-1}--\ref{prop:measure-6},
then there is also a smallest one,
it takes quite some effort
to prove that there is any map $\mu\colon \mathcal{A}\ra \R$
with properties~\ref{prop:measure-1}--\ref{prop:measure-6} to begin with.
One could proceed in the following steps.
\begin{enumerate}
\item
\emph{Find the smallest map $\Smu\colon \SA\ra \R$
that has properties~\ref{prop:measure-1}--\ref{prop:measure-3}.}

This is not so difficult.
Let $\mathcal{S}$ be the family of subsets of~$\R$ of
the form~$[a,b]$ where $a,b\in \R$ with $a\leq b$.
Let~$\SA$ be the set of all finite unions
of disjoint elements from~$\mathcal{S}$,
and let $\Smu\colon \SA\ra \R$
be the map given by
\begin{equation*}
\Smu(\,I_1 \cup\dotsb\cup I_N\,) \ = \ |I_1| \,+\,\dotsb\,+\,|I_N|,
\end{equation*}
where $I_1,\dotsc,I_N\,\in\, \mathcal{S}$
with $I_n\cap I_m = \varnothing$ when $n\neq m$.
Of course, it takes some calculations
to prove that such a map~$\Smu$ indeed exists.

Note that $\Smu\colon \SA\ra \R$ is a valuation.

\item
\emph{Extend~$\Smu$
to the smallest map $\overline{\Smu}\colon\overline{\SA}\ra\R$
that has properties~\ref{prop:measure-1}--\ref{prop:measure-5}.}

This is the most interesting and the most difficult step.
Informally,
we can use the following `algorithm' to obtain~$\overline{\Smu}$.
Let $\mu\colon\mathcal{A} \ra \R$
be a variable. To begin, set $\mu \eqdf \Smu$.
For all~$A_1,A_2,\dotsc$ from~$\mathcal{A}$ do the folllowing.
\begin{enumerate}
\item If $A_1 \supseteq A_2 \supseteq \dotsb$,
and if $\bw_n \mu(A_n)$ exists,
and if $\bigcap_n A_n \notin \mathcal{A}$,\\
then add $\bigcap_n A_n$ to~$\mathcal{A}$,
and set $\mu(\bigcap_n A_n) \eqdf \bw_n \mu(A_n)$.
\item If $A_1 \subseteq A_2 \subseteq \dotsb$,
and if $\bv_n \mu(A_n)$ exists,
and if $\bigcup_n A_n \notin \mathcal{A}$,\\
then add $\bigcup_n A_n$ to~$\mathcal{A}$,
and set $\mu(\bigcup_n A_n) \eqdf \bv_n \mu(A_n)$.
\end{enumerate}
There are many problems with this `algoritm'.
Perhaps the most serious problem
is that the same set~$A$ may be obtained
in several ways
(e.g., as an intersection,~$\bigcap_n A_n$,
and as a union,~$\bigcup_n B_n$),
and it is not clear whether 
$\mu(A)$ would be assigned the same value each time,
that is, 
whether 
\begin{equation*}
\bw_n \mu(A_n)\  =\  \bv_n \mu(B_n).
\end{equation*}


Note that the `algorithm' resembles the  definition
of the Borel sets.
In fact, $\overline{\Smu}$
will be the family of all Borel subsets of~$\R$
with finite measure.


\item
\emph{Extend~$\overline{\Smu}$
to the smallest map~$\Lmu\colon \LA \ra \R$
that has properties~\ref{prop:measure-1}--\ref{prop:measure-6}.}

This is straightforward.
Simply
define $\LA$ to be the family of all subsets of~$\R$
that are `sandwiched' between elements of~$\overline{\SA}$,
that is,
all $Z\in\wp(\R)$
for which
there are $A,B\in \overline{\SA}$
such that $A\subseteq Z\subseteq B$ and 
$\overline{\Smu}(A) = \overline{\Smu}(B)$.

Now, define $\Lmu\colon \LA \ra \R$
by $\Lmu (Z) = \overline{\Smu} (A)$ for $Z$ and~$A$ as above.

For more details,
see Proposition~\ref{P:convex-extension}.
\end{enumerate}

\noindent
For the Lebesgue integral~$\Lphi\colon \LF\ra \R$ the situation is similar.
Again it is not obvious that
there is a smallest map $\varphi\colon F\ra \R$
(with $F\subseteq\smash{\E^\R}$)
which has 
properties~\ref{prop:integral-1}--\ref{prop:integral-6},
and as before,
there are three steps to obtain~$\Lphi$.

The second step gives rise to the following question.
\begin{quote}
Given a valuation system $\vs{V}{L}\varphi{E}$,
under which conditions
can~$\varphi$
be extended to a valuation~$\psi\colon C\ra E$
such that 
\begin{equation*}
\vs{V}{C}{\psi}{E}
\end{equation*}
is both $\Pi$-complete and $\Sigma$-complete?
\end{quote}









\subsection{Completion}
Let us describe these `completion procedures'.

To introduce the Lebesgue measure $\Lmu$
(that is,
to identify the Lebesgue measurable sets~$\LA$
and assign to each set~$A\in\LA$ a Lebesgue measure)
one can start by
assigning a measure to a finite disjoint union
of intervals 
$I_1 \cup\dotsb\cup I_N$:
the sum of the lengths of the intervals,
$|I_1|+\dots+|I_N|$.
One can prove that the set of these disjoint
unions of intervals is a lattice, $\SA$,
and that measure we have just assigned
to each element
yields  a valuation $\Smu\colon \SA\ra\R$.

The Lebesgue measure~$\Lmu$
(which we wish to obtain from~$\Smu$)
is then an extension of~$\Smu$ which is complete
in the several ways. For one, we have the following.
\begin{equation*}
\left[\quad
\begin{minipage}{.7\columnwidth}
Let $A_1\subseteq,A_2\subseteq\dotsb$
be elements of~$\LA$
such that 
\begin{equation*}
\textstyle{\bv_n \Lmu(A_n)}\text{\quad exists.}
\end{equation*}
Then $\bigcup_n A_n \in \LA$,
and we have 
\begin{equation*}
\textstyle{\Lmu (\,\bigcup_n A_n\,) \ =\  \bv_n \Lmu(A_n)}.
\end{equation*}
\end{minipage}
\right.
\end{equation*}

\vspace{.3em}
\noindent For the Lebesgue integral
we can set up a similar situation.
If one wants to introduce the Lebesgue integral~$\Lphi$,
one can start by assigning to each step function
the obvious integral.
The step functions form a lattice~$\SF$,
and the assignment of an integral to the step functions
gives a valuation $\Lphi\colon \SF\ra\R$.

The Lebesgue measure $\Lphi$
(which we wish to obtain from~$\Sphi$)
is then an extension of~$\Sphi$
which is complete in the following sense.
\begin{equation*}
\left[\quad
\begin{minipage}{.7\columnwidth}
Let $f_1 \leq f_2 \leq \dotsb$
be from~$\LF$
such that 
\begin{equation*}
\textstyle{\bv_n \Lphi(f_n)}\text{\quad exists.}
\end{equation*}
Then $\bv_n f_n \in \LF$,
and we have 
\begin{equation*}
\textstyle{\Lphi (\,\bv_n A_n\,) \ =\  \bv_n \Lphi(f_n)}.
\end{equation*}
\end{minipage}
\right.
\end{equation*}


\vspace{.3em}
\noindent
So we see that both the Lebesgue measure
and the Lebesgue integral
are extensions of relatively simple valuations,
and that they are complete in a certain sense.
The similarity does not end here.
We formulate a notion of completeness for
valuations~$\varphi\colon L\ra E$ relative
to a lattice~$V$ which is complete in an appropriate sense
and such that~$L$ is a sublattice of~$V$.
So we consider systems of the following form.
\begin{equation*}
\vs{V}{L}\varphi{E}.
\end{equation*}
We call these systems \emph{valuation systems}
and study them in Section~\ref{S:valuation-systems}.
The notion of completeness for valuation systems
 generalizes the aforementioned
notions of completeness.
In particular, the following valuation systems are complete.
\begin{equation*}
\vsLA,\qquad\qquad\vsLF.
\end{equation*}

For these complete valuation systems,
we can formulate and prove variants
of some of the convergence theorems from
integration theory,
such as the Lemma of Fatou (Lemma~\ref{L:fatou})
and Lebesgue's Dominated Convergence Theorem
 (Theorem~\ref{T:Lebesgue}).

\subsection{Hierarchy}
The main question
of this thesis 
is: 
\begin{quote}
Given a valuation system $\vs{V}{L}\varphi{E}$,
 under which conditions can we extend $\varphi$
to a valuation $\psi$ such that
$\vs{V}{C}\psi{E}$ is complete?
\end{quote}
The question 
naturally leads us to consider an
hierarchy 
(see Section~\ref{S:completion})
of partial completions
akin to the
Borel Hierarchy in the case of~$\Smu$
and the Baire Hierarchy in the case of~$\Sphi$.
The corresponding picture may be familiar to the reader.
\begin{equation*}
\xymatrix @=10pt {
& \Sigma\varphi \ar @{-} [rr] \ar @{-} [rrdd]
&& \Sigma_2 \varphi  \ar @{-} [r]\ar @{-} [rd]
&& \dotsb
& \Sigma_{\omega} \varphi \ar @{-} [rr] \ar @{-} [rrdd]
                          \ar @{=} [dd]
&& \Sigma_{\omega+1}\varphi \ar @{-} [rr] \ar @{-} [rrdd]
&& \Sigma_{\omega+2} \varphi  \ar @{-} [r]\ar @{-} [rd]
&& \dotsb
\\  
\varphi \ar @{-} [ru] \ar @{-} [rd] 
&&&&&\dotsb
&&
&&&&&\dotsb\\
& \Pi\varphi \ar @{-} [rr] \ar @{-} [rruu]
&& \Pi_{2} \varphi \ar @{-} [r] \ar @{-} [ru]
&& \dotsb
& \Pi_{\omega}\varphi \ar @{-} [rr] \ar @{-} [rruu]
&& \Pi_{\omega+1}\varphi \ar @{-} [rr] \ar @{-} [rruu]
&& \Pi_{\omega+2} \varphi \ar @{-} [r] \ar @{-} [ru]
&& \dotsb
}
\end{equation*}


We will see that it is not easy to determine whether
a given valuation system 
$\vs{V}{L}\varphi{E}$
can be `completed'.
Fortunately, 
the situation is more tractable
for some $E$. 
We will not go into the details yet,
but can say
that we call such~$E$ \emph{benign}
and that $\R$ is one example.
We study these~$E$ in Section~\ref{S:benign}.

Let us for the moment assume that $\varphi$ can be completed.
Then we are able to prove that there is
a smallest complete extension $\vs{V}{\overline L}{\overline\varphi}{E}$.
We will see that $\overline{\Smu}$ is essentially 
the Lebesgue measure $\Lmu$:
$\overline{\Smu}$ is the Lebesgue measure restricted to the Borel sets.
Similarly,
 $\overline{\Sphi}$ turns out to be what one might
call the Baire integral.\todo{add reference}

Moreover,
we will prove that this completion $\overline L$
is closed under certain operations (see Section~\ref{S:closedness}).
For instance,
we will prove that if $f,g\in \overline{\SF}$
and $f$ and $g$ do not attain the values $+\infty$ and $-\infty$,
then $f+g\in\overline{\SF}$ and 
$\overline{\Sphi}(f+g)=\overline{\Sphi}(f) + \overline{\Sphi}(g)$.

\subsection{Fitting Uniformity}
To prove that $\Smu$ and $\Sphi$
can be `completed'
(without presupposing $\Lmu$ and $\Lphi$ exist)
we draw on the topological structure of~$\R$.
More precisely,
we will consider ordered Abelian groups~$E$
which are endowed with a certain uniform structure
(such as~$\R$) and prove
that these are benign (see Section~\ref{S:unif}).

Interestingly,
with the additional uniform structure on~$E$,
we can prove that certain operations
that are (initially only) defined on~$L$
can be extended to the completion~$\ol{L}$
(see Theorem~\ref{T:fubext}).

As an example,
we will deduce a variant of Fubini's Theorem.
\todo{Add ref.}

\subsection{Rings and Riesz Spaces}
There is a long list of axiomatisations of measure and
axiomatisations of integral
which we could compare to the theory set forth in this text.
We will not do this.

Nevertheless,
as an example,
we will apply our theory
to a positive additive map $\mu$ on a ring $\mathcal{A}$ of
subsets of~$X$
(which is an axiomatisation of measure)
and 
to a positive linear 
map $\varphi$ on a Riesz space~$F$ of functions on~$X$
(which is a axiomatisation of integral).
That is,
we consider the following valuation systems.
\begin{equation*}
\vs{\wp X}{\mathcal{A}}{\mu}{E},
\qquad\qquad
\vs{\EX}{F}{\varphi}{E}.
\end{equation*}
We will return to these two valuation systems throughtout the paper
and hope that this will clarify the theory
and the similarity between measure and integral.


\end{document}
