\documentclass[main.tex]{subfiles}
\begin{document}
\section{Uniformity on $E$}
To prove that~$\R$
is benign (see Definition~\ref{D:benign}),
we study
ordered Abelian groups~$E$ which are endowed with
a certain uniformity (such as~$\R$)
in Subsection~\ref{SS:fitting}.
We prove that all such~$E$ are benign.
We do this in the following way.

Let $\vs{V}{L}\varphi{E}$ be a valuation system.
Recall that in order to prove that~$E$ is benign
 we must show that 
if $\varphi$ is continuous,
then $\varphi$ is extendible (see Definition~\ref{D:benign}).
We will first prove that
if $\varphi$ is continuous,
then both $\Pi \varphi$ and $\Sigma \varphi$ are continuous.

So, by induction,
we see that $\varphi$ is
both both $\Pi_n$-extendible and $\Sigma_n$-extendible,
and both $\Pi_n\varphi$  and $\Sigma_n \varphi$
are continuous,
for every~$n\in \N$.
Hence $\varphi$ is $\Pi_\omega$-extendible.
However,
it is not clear that~$\Pi_\omega\varphi$
is continuous.

Secondly,
we prove that if 
$\varphi$ is $\Pi_\lambda$-extendible
for some ordinal number~$\lambda$,
then $\Pi_\lambda\varphi$ is continuous.
So by induction
we see that $\varphi$ is both $\Pi_\alpha$-extendible
and $\Sigma_\alpha$-extendible,
and both $\Pi_\alpha\varphi$ and $\Sigma_\alpha\varphi$
are continuous,
for every ordinal number~$\alpha$.
Hence $\varphi$ is extendible.

To prove the second statement
we use the fact that
elements of $\Pi_\alpha L$ (or $\Sigma_\alpha L$)
can be approximated from below
by elements of $\Pi L$, in some sense.
We will express this by
$\Sigma L$ is \emph{lower $\Pi_\alpha \varphi$-dense}
in $\Pi_\alpha L$.
We will formally introduce this notion,
and study it, in Subsection~\ref{SS:dense}.

\subsection{Fitting uniformity}
\label{SS:fitting}
%
%                  DEFINITION OF A GOOD UNIFORMITY
%
\begin{dfn}
\label{D:uniformity}
Let~$E$ be an ordered Abelian group.
A \keyword{fitting uniformity} on~$E$
is a \emph{countable} set~$\Phi$ of binary relations on~$E$
with the following properties.
\newcounter{epropc}
\begin{enumerate}
\item 
\label{E-refl}
We have $s \se s$ for all $\ve \in \Phi$ and $s \in E$.
\item
\label{E-min}
There is a map $\wedge\colon \Phi\times\Phi \ra \Phi$
such that
\begin{equation*}
s \ \varepsilon\wedge\delta\  t
\quad\implies\quad
s \,\varepsilon\,t\ \text{ and }\ s\,\delta\,t
\qquad (\varepsilon,\delta\in\Phi,\ s,t\in E).
\end{equation*}

\item
\label{E-half}
There is a map $-/2\colon \Phi \ra \Phi$
such that
\begin{equation*}
r\ \varepsilon/2 \ s \ \varepsilon/2 \ t
\quad\implies\quad
r \se t\qquad(\varepsilon\in\Phi,\ r,s,t\in E).
\end{equation*}

\item \label{E-ord}
Given $\varepsilon\in\Phi$ and $r,s,t\in E$ 
with $r\leq s\leq t$,
we have
\begin{equation*}
r\,\varepsilon\,t
\quad\implies\quad
r\,\varepsilon\,s
\ \text{ and }\ 
s\,\varepsilon\,t.
\end{equation*}

\item \label{E-haus}
Let $s,t\in E$ with $s\leq t$.
Then $s=t$ provided that $s\,\varepsilon\,t$ for all~$\varepsilon\in\Phi$.

\item \label{E-inf-conv}
If a sequence $s_1 \geq s_2 \geq \dotsb$ from~$E$
has an infimum~$s\in E$,
then 
\begin{equation*}
\forall\varepsilon\in \Phi
\ \ \exists N\in \N
\ \ s \, \varepsilon\, s_N.
\end{equation*}

\item  \label{E-bound-inf}
Let $s_1\geq s_2 \geq \dotsb$ be a
sequence in~$E$,
and assume that
for every $\ve\in \Phi$
there is an~$N_\ve \in \N$ such that 
$s_{n} \ \varepsilon \ s_{N_\ve}\quad (n\geq N_\ve)$.

Then $s_1 \geq s_2 \geq \dotsb$ has an infimum~$\bw_n s_n$.

\item \label{E-add}
Let $r,s,t\in E$ and $\varepsilon\in\Phi$ be given.
Then $s\,\varepsilon\,t$ implies $r+s\ \varepsilon\ r+t$.
\setcounter{epropc}{\value{enumi}}
\end{enumerate}
\end{dfn}
\begin{ex}
We define a fitting uniformity on~$\R$.
For each natural number~$n$,
let $\varepsilon_n$ be the binary relation
on~$\R$ given by 
\begin{equation*}
s \ \varepsilon_n\  t
\quad\iff\quad 
s\leq t\quad\text{and}\quad t-s \leq 2^{-n}.
\end{equation*}
Then $\Phi\eqdf \{ \varepsilon_n \colon n\in \N \}$
is a fitting uniformity on~$\R$.

(Take $\varepsilon_n \wedge \varepsilon_m \eqdf \varepsilon_{n\vee m}$
and $\varepsilon_n / 2 \eqdf \varepsilon_{n+1}$
for all $n,m\in \N$.)
\end{ex}
\todo{Add more examples.}

%
%                  LEMMA ON FURTHER PROPERTIES
%
To the list of properties 
that fitting uniformity must have (see Definition~\ref{D:uniformity}),
we add 
some easy observations
in Lemma~\ref{L:E-prop}.
When we speak of ``property~($q$)'',
where~$q$ is some roman numeral~$q$,
we refer to this list.
\begin{lem}
\label{L:E-prop}
Let $E$ be an ordered Abelian group
with a fitting uniformity~$\Phi$.
\begin{enumerate}
\setcounter{enumi}{\value{epropc}}
\item \label{E-minus}
Let $s,t\in E$ and $\ve \in \Phi$ be given.
Then $s \se t$ implies\, $-t \ \ve\,-\!\!s$.

\item \label{E-sup-conv}
If a sequence $s_1 \leq s_2 \leq \dotsb$ from~$E$
has an supremum~$s\in E$,
then 
\begin{equation*}
\forall\varepsilon\in \Phi
\ \ \exists N\in \N
\ \ s_N \, \varepsilon\, s.
\end{equation*}

\item  \label{E-bound-sup}
Let $s_1\leq s_2 \leq \dotsb$ be a
sequence in~$E$,
and assume that
for every $\ve\in \Phi$
there is an~$N_\ve \in \N$ such that 
$s_{N_\ve} \ \varepsilon \ s_n\quad (n\geq N_\ve)$.

Then $s_1 \leq s_2 \leq \dotsb$ has a supremum~$\bv_n s_n$.
\end{enumerate}
\end{lem}
\begin{proof}
\noindent\ref{E-minus}\ 
Let $s,t\in E$ and $\varepsilon\in \Phi$ be given,
and assume $s\se t$.
We prove $-t\ \varepsilon\,-\!\!s$.
By property~\ref{E-refl},
we have $-(t+s) \ \varepsilon\ -(t+s)$.
Hence property~\ref{E-add} yields
\begin{equation*}
-t\,=\,s-(t+s)\quad\ve\quad t-(t+s) \,=\,-s.
\end{equation*}

\noindent\ref{E-sup-conv}
Let $s_1 \leq s_2 \leq \dotsb$
be a sequence in~$E$
which has a supremum~$s$ in~$E$.
Let $\varepsilon \in \Phi$ be given.
We need to find an~$N\in\N$
such that $s_N \ \varepsilon\ s$.

Let us consider the sequence
$-s_1 \geq -s_2 \geq\dotsb$.
By Lemma~\ref{L:oag-minus-preserves})
the sequence $-s_1 \geq -s_2 \geq \dotsb$
has an infimum, $-s$.
By property~\ref{E-inf-conv}
we have 
$-s\ \varepsilon\ {-s_N}$
for some~$N$.
Then by property~\ref{E-minus},
we get $s_N\ \varepsilon\ s$,
and we are done.

\vspace{.3em}
\noindent\ref{E-bound-sup}
Similar:
apply property~\ref{E-bound-inf}
to the sequence $-s_1 \geq -s_2 \geq \dotsb$.
\end{proof}

%
%                  NOTATION
%
\begin{nt}
\label{N:unif}
Let $E$ be an ordered Abelian group
with a fitting uniformity~$\Phi$.
\begin{enumerate}
\item
\label{N:unif-leq}
Given binary relations $\varepsilon$
and $\delta$ on~$E$
(for instance, $\varepsilon,\delta\in \Phi$),
we write
\begin{equation*}
\varepsilon \ \leq\ \delta
\quad\iff\quad 
\forall s,t\in E\ 
[\ s\ \varepsilon\ t
\implies
s\ \delta\ t\ ].
\end{equation*}

\item
\label{N:unif-plus}
Given binary relations $\varepsilon$ and $\delta$ on~$E$,
let $\varepsilon + \delta$
be the relation on~$E$ given by
\begin{equation*}
s\ \ \varepsilon + \delta\ \ t
\quad\iff\quad
\exists q\in E\ 
[\ s\ \varepsilon\ q\ \delta\ t].
\end{equation*}
\end{enumerate}
\end{nt}
\begin{rem}
The operation ``$+$''
defined in Notation~\ref{N:unif}\ref{N:unif-plus}
is associative,
but not in general commutative
(contrary to the suggestive symbol).

The chosen notation does have advantages:
property~\ref{E-half} can be written as
\begin{equation*}
\varepsilon/2 \,+\,\varepsilon/2 \ \leq\ \varepsilon
\qquad(\varepsilon \in \Phi).
\end{equation*}
\end{rem}

%
%                  DIRECTED SET INFIMUM LEMMA
%
The following lemma will be usefull.
\begin{lem}
\label{lem:conv-inf}
Let~$E$ be an ordered Abelian group with 
fitting uniformity~$\Phi$.

Let $S\subseteq E$
be non-empty and \emph{downwards directed},
i.e., for all~$s_1,s_2\in S$,
there is an~$s\in S$ such that $s\leq s_1$
and $s\leq s_2$.

Let $t\in E$ be a lower bound of~$S$
which is close to~$S$ in the sense that
\begin{equation}
\label{eq:L:conv-inf}
\forall\varepsilon\in\Phi\ \exists s\in S\quad t\,\varepsilon\, s\text{.}
\end{equation}

Then~$t$ is the infimum of~$S$.
\end{lem}
\begin{proof}
To show that $t$ is the infimum of~$S$,
we need to prove that $\ell \leq t$
for every lower bound~$\ell$ of~$S$.
To do this, we take a detour.

Let~$\varepsilon_1,\varepsilon_2,\dotsb$
be an enumeration of~$\Phi$.
Using Equation~\eqref{eq:L:conv-inf},
and the fact that~$S$ is non-empty and directed,
choose~$s_1 \geq s_2 \geq \dotsb$ in~$S$
such that 
\begin{equation}
\label{eq:L:conv-inf-1}
t\ \varepsilon_n \ s_n\qquad(n\in \N).
\end{equation}

We will prove that
$s_1 \geq s_2 \geq \dotsb$
has an infimum~$s$ and that $s=t$.

This is sufficient to prove that~$t$ is the infimum of~$S$.
Indeed,
if $\ell$ is a lower bound of~$S$,
then $\ell$ is a lower bound of~$s_1 \geq s_2 \geq\dotsb$,
and so $\ell \leq \bw_n s_n =t$.

We use property~\ref{E-inf-conv}
to show that $s_1 \geq s_2 \geq\dotsb$
has an infimum.
Given $\varepsilon\in\Phi$,
we need to find an~$N$ such that $s_n \ \ve\ s_N$
for all~$n\geq N$. Pick $k$ such that $\varepsilon=\varepsilon_k$
and take $N=k$. Let $n\geq N$ be given.
Note that $t \leq s_n \leq s_N =s_k$
and $t\ \varepsilon_k\ s_k$
by Equation~\eqref{eq:L:conv-inf-1}.
So we have $s_n\ \varepsilon_k\ s_N$ by property~\ref{E-ord}.

Hence property~\ref{E-inf-conv} implies that $s_1 \geq s_2 \geq \dotsb$
has an infimum, $s$.
It remains to be shown that $s=t$.
For this we use property~\ref{E-haus}.

Note that $t\leq s$ because $t\leq s_n$ for all~$n$.
Let $\varepsilon\in\Phi$ be given.
We need to prove that $t \se s$.
Choose $k$ such that $\ve = \ve_k$.
Then $t \leq s \leq s_k$
and $t \ \ve_k\ s_k$
by Equation~\eqref{eq:L:conv-inf-1}.
So $t\ \ve_k \  s$ by property~\ref{E-ord}.
Hence $s=t$ by property~\ref{E-haus}.
\end{proof}
%%%%%%%%%%%%%%%%%%%%%%%%%%%%%%%%%%%%%%%%%%%%%%%%%%%%%%%%%%%%%%%%%%%%%%%%%%%%%%%
%
%                  LOWER DENSENESS SUBSECTION
%
\subsection{Denseness}
\label{SS:dense}
Throughout this subsection,
$E$ will be an ordered Abelian group
endowed with a fitting uniformity~$\Phi$
(see Definition~\ref{D:uniformity}).
%
%                  LOWER DENSENESS
%
\begin{dfn}
\label{D:lower-dense}
Let $\vs{V}{L}\varphi{E}$ be a valuation system.
Let $S\subseteq T$ be subsets of~$L$.
We say $S$ is \keyword{lower $\varphi$-dense} in~$T$
provided that the following condition holds.
\begin{equation*}
\begin{minipage}{0.7\textwidth}
For every $a\in T$ and $\varepsilon\in\Phi$
there is an $\ell\in S$
such that
\begin{equation*}
\ell \leq a\quad\text{and}\quad\varphi (\ell)\,\varepsilon\, \varphi (a).
\end{equation*}
\end{minipage}
\end{equation*}
The notion of 
\keyword{upper $\varphi$-denseness} is defined similarly.
\end{dfn}
\begin{ex}
Let $\vs{V}{L}\varphi{E}$
be a $\Sigma$-extendible valuation system
(see Def.~\ref{D:Pi-extendible}).
\emph{Then $L$ is lower $\Sigma\varphi$-dense in~$\Sigma L$.}

Indeed,
given $a\in \Sigma L$
and $\ve \in \Phi$,
we need to find an~$\ell\in L$ 
such that $\varphi(\ell) \ \ve\ \Sigma\varphi(a)$.
Write $a=\bv_n a_n$ for
some $\varphi$-convergent sequence $a_1 \leq a_2 \leq \dotsb$.
Then we have
\begin{equation*}
\Sigma\varphi(a)\ =\ \bv_n \varphi(a_n).
\end{equation*}
By property~\ref{E-sup-conv},
there is an~$N$ such that $\varphi(a_N) \ \ve\ \Sigma\varphi(a)$.
So take $\ell =a_N$.
\end{ex}
%
%                  ELEMENTARY PROPERTIES OF LOWER DENSENESS
%
\begin{lem}
\label{L:ldense-prop}
Let $\vs{V}{L}\varphi{E}$ be a valuation system.
\begin{enumerate}
\item
\label{L:ldense-prop-1}
Let $R\subseteq S\subseteq T$ be subsets of~$L$.
Suppose $R$ is lower $\varphi$-dense in~$S$,
and suppose that $S$ is lower $\varphi$-dense in~$T$.
Then $R$ is lower $\varphi$-dense in~$T$.

\item
\label{L:ldense-prop-2}
Let $R$ be a subset of~$L$,
and let $\mathcal S$ be a family of subset of~$L$.\\
If $R$ is lower $\varphi$-dense in each~$S\in \mathcal{S}$,
then $R$ is lower $\varphi$-dense in~$\bigcup \mathcal{S}$.
\end{enumerate}
\end{lem}
\begin{proof}
\noindent\ref{L:ldense-prop-1}\ 
Let $t\in T$ and $\ve \in\Phi$ be given.
To prove $R$ is lower $\varphi$-dense in~$T$,
we need to find an $r\in R$
with $r\leq t$ and $\varphi(r) \ \ve\ \varphi(t)$.
This is easy.
Choose an $s\in S$
such that $s \leq t$ and $\varphi(s)\ \ve/2\ \varphi(t)$
(see Definition~\ref{D:uniformity}\ref{E-half}
for the meaning of ``$\ve/2$'').
Choose an $r\in R$
such that $r\leq s$ and $\varphi(r)\ \ve/2\ \varphi(s)$.
Then $r\leq s$ and $\varphi(r) \ \ve\ \varphi(t)$.
\vspace{.3em}

\noindent\ref{L:ldense-prop-2}\ 
We leave this to the reader.
\end{proof}
%
%
%                  MAIN LEMMA
%
%
The proof that~$E$
 is benign
hinges on the following lemma.
\begin{lem}
\label{lem:main}
Let $\vs{V}{L}\varphi{E}$ be a valuation space.\\
Let $K$ be a lower $\varphi$-dense sublattice of~$L$.\\
Then, for every $\varphi$-convergent sequence
 $a_1 \geq a_2 \geq \dotsb$ from~$L$,
and $\varepsilon\in \Phi$,\\
there is a $\varphi$-convergent sequence
$\tilde a_1 \geq \tilde a_2 \geq \dotsb$ from~$K$
with 
\begin{equation}
\label{eq:main}
\tilde a_n \ \leq\  a_n
\qquad\text{and}\qquad
\bw_n \varphi (\tilde a_n) \ \ \varepsilon\ \ \bw_n\varphi (a_n).
\end{equation}
\end{lem}
\begin{proof}
Let $a_1 \geq a_2 \geq \dotsb$ 
be a $\varphi$-convergent sequence in~$L$,
and let~$\varepsilon\in\Phi$ be given.
We need to find a $\varphi$-convergent sequence 
$\tilde a_1 \geq \tilde a_2 \geq \dotsb$ in~$K$
which satisfies Condition~\eqref{eq:main}.
To this end,
we find a sequence 
$\tilde a_1 \geq \tilde a_2 \geq \dotsb$ in~$K$
such that
\begin{equation}
\label{eq:lem:main-cond}
\varphi(\tilde a_n) \ \eta\ \varphi (a_n)
\qquad\text{and}\qquad
\forall i\in\N\ \ \exists N\in\N 
\ \ \forall n\geq N
\  [\ \  \varphi \tilde a_n \ \varepsilon_i\  \varphi \tilde a_N \ \ ],
\end{equation}
where $\varepsilon_1,\,\varepsilon_2,\,\dotsc$
is
an enumeration of~$\Phi$,
and $\eta\in\Phi$ with  $2\eta \leq \varepsilon$
(see Notation~\ref{N:unif}).

Such a sequence $\tilde a_1 \geq \tilde a_2 \geq \dotsb$
is $\varphi$-convergent 
(by property~\ref{E-bound-inf}).
We prove that $\tilde a_1 \geq \tilde a_2 \geq\dotsb$
satisfies Condition~\eqref{eq:main}. 
Indeed:
We know $\bw_n \varphi(\tilde a_n)$ exists.
Hence,
there is an~$N\in\N$
with 
 $\bw_n\varphi (\tilde a_n)\ \eta\ \varphi (\tilde a_N)$
by property~\ref{E-inf-conv}.
Then
\begin{equation*}
\bw_n \varphi (\tilde a_n) \quad \eta \quad \varphi(\tilde a_N)
\quad\eta\quad \varphi(a_N).
\end{equation*}
So we have 
$\bw_n \varphi (\tilde a_n) \ \ve\  \varphi(a_N)$.
But $\bw_n \varphi(\tilde a_n) \,\leq\,
\bw_n \varphi(a_n) \,\leq\,\varphi(a_N)$.
Thus 
$\bw_n \varphi(\tilde a_n) \ \ve\ 
\bw_n \varphi(a_n)$ by property~\ref{E-ord}.
Hence $\tilde a_1  \geq \tilde a_2 \geq \dotsb$
satisfies Condition~\eqref{eq:main}.

Finding a sequence $\tilde a_1 \geq \tilde a_2 \geq \dotsb$ 
which satisfy Condition~\eqref{eq:lem:main-cond}
is a subtle affair.
Pick $\eta_1, \eta_2,\dotsc$ and $\zeta_1,\zeta_2,\dotsc$
from~$\Phi$
(using properties~\ref{E-half}
and~\ref{E-min})
such that
\begin{equation*}
2 \eta_i \leq \varepsilon_i\text{,} \qquad
\eta_i \leq \eta\text{,} \qquad
2\zeta_i \leq \eta_i\text{,} \qquad
2\zeta_{i+1} \leq \zeta_{i}\text{.}
\end{equation*}
Then we have
\begin{equation}
\label{eq:lem:main-zeta--eta}
\zeta_i + \dotsb + \zeta_j \leq \eta_i \qquad (i,j\in\N,\  i\leq j)\text{.}
\end{equation}
Pick $\ell_1,\ell_2,\dotsc$ from $K$ 
such that $\ell_n \leq a_n$ and $\varphi(\ell_n)\ \ \zeta_n\  \ \varphi(a_n)$
and define 
\begin{equation*}
\tilde{a}_{ij} \ =\ \ell_i \wedge \dotsb\wedge \ell_j\text{,}
\qquad\qquad
\tilde{a}_n \ =\ \tilde{a}_{1n} =\ell_1 \wedge\dotsb\wedge \ell_n\text{,}
\end{equation*}
where $i,j,n\in N$ with $i\leq j$.
Then $\tilde a_{ij} \in K$ and  $\tilde a_n \leq \ell_n \leq a_n$.
We will prove that
the sequence $\tilde a_1 \geq \tilde a_2 \geq \dotsb$
satisfies Condition~\eqref{eq:lem:main-cond}.

Note that for all~$i,j\in\N$ with $i\leq j$, we have,
by Lemma~\ref{L:wv-unif},
\begin{equation*}
\ld\varphi(\tilde a_{ij},a_j)\ =\ 
\ld\varphi(\ell_i \wedge\dotsb\wedge \ell_j,\  a_i \wedge \dotsb \wedge a_j)
\ \leq\ \ld\varphi(\ell_i,a_i)+\dotsb+\ld\varphi(\ell_j,a_j)\text{.}
\end{equation*}
Since $\varphi(\ell_k)\ \zeta_k\ \varphi(a_k)$
for all~$k$, 
the inequality above 
yields,
using property~\ref{E-add},
\begin{equation*}
\varphi(\tilde a_{ij}) \quad \zeta_i + \dotsb + \zeta_j\quad \varphi(a_j).
\end{equation*}
So because
$\zeta_i+\dotsb+\zeta_j \leq \eta_i$ 
(see Inequality~\eqref{eq:lem:main-zeta--eta}),
we have
\begin{equation}
\label{eq:lem:main-3}
\varphi(\tilde a_{ij} )\quad\eta_i\quad\varphi (a_j).
\end{equation}
In particular,
we get $\varphi (\tilde a_n)\equiv \varphi(\tilde a_{1n}) \ \eta_1 \ \varphi (a_n)$.
Hence $\varphi(\tilde a_n) \ \eta\ \varphi(a_n)$
as $\eta_1 \leq \eta$.

Let~$i\in \N$ be given.
To prove that $\tilde a_1 \geq \tilde a_2 \geq \dotsb$
satisfies Condition~\eqref{eq:lem:main-cond},
it remains to be shown that 
there is an~$N\in\N$ such that
\begin{equation}
\label{eq:lem:main-2}
\varphi(\tilde a_n)\ \ \varepsilon_i \ \ \varphi(\tilde a_N)
\qquad (n \geq N)
\end{equation}
Using property~\ref{E-inf-conv},
determine $N\geq i$ such that $\bw_n \varphi (a_n) \ \eta_i\ \varphi(a_N)$.
We will show that Statement~\eqref{eq:lem:main-2} holds.
Let $n\geq N$ be given.
Note that 
 by Lemma~\ref{lem:curry-wv-unif},
\begin{equation*}
\ld\varphi (\tilde a_n, \tilde a_N)
\,=\,\ld\varphi(\,\tilde a_{i-1} \wedge \tilde a_{in},\,
\tilde a_{i-1} \wedge \tilde a_{iN}\,)
\ \leq\ \ld\varphi(\tilde a_{in},\tilde a_{iN}).
\end{equation*}
So to prove Statement~\eqref{eq:lem:main-2},
it suffices to show 
that $\varphi(\tilde a_{in} ) \ \ve_i\ \varphi(\tilde a_{iN})$.

Recall that $\bw_m\varphi(a_m)\ \ \eta_i\ \ \varphi(a_N)$
by choice of~$N$. Then in particular,
we get
$\varphi(a_n) \ \eta_i\ \varphi(a_N)$
by property~\ref{E-ord}.
Further, $\varphi(\tilde a_{in}) \ \eta_i\ \varphi(a_n)$
by Inequality~\eqref{eq:lem:main-3}. So
\begin{equation*}
\varphi (\tilde a_{in})\ \eta_i\ 
\varphi (a_{n})\ \eta_i\ \varphi (a_N).
\end{equation*}
Hence $\varphi (\tilde a_n) \ \ve_i\ \varphi (a_N)$,
because $2\eta_i \leq \ve_i$.
Note that $\varphi(\tilde a_{in}) \leq 
\varphi(\tilde a_{iN}) \leq \varphi(a_N)$.

So by property~\ref{E-ord},
we get
$\varphi(\tilde a_{in}) \ \ve_i\  \varphi(\tilde a_{iN})$.
\end{proof}
\begin{lem}
\label{lem:cont-ext-double}
Let $K$ be a sublattice of~$L$.
Then $\varphi$ is continuous provided that:
\begin{enumerate}
\item
The restriction $\varphi|K$ of $\varphi$ to~$K$ is continuous.
\item
$K$ is lower and upper dense in~$L$.
\end{enumerate}
\end{lem}
\begin{proof}
Let $\varphi$-convergent
$a_1 \geq a_2 \geq \dotsb$ and $b_1 \leq b_2 \leq \dotsb$ from~$L$
with $\bw a_n \leq \bv b_n$ be given;
we will show that $\bw \varphi a_n \leq \bv \varphi b_n$.
To this end we will prove that 
$\bw \varphi a_n = \bv L$ and
$\bv \varphi b_n = \bw U$
where:
\begin{alignat*}{3}
L\  &:=\ \bigl\{ \, \bw\varphi \tilde a_n \colon \ 
                 \tilde a_1 \geq \tilde a_2 \geq \dotsb \  &&
                 \varphi\text{-convergent in }
                 K\text{ with }&\tilde a_n\,&\leq\, a_n \bigr\} \\
U\  &:=\ \bigl\{ \, \bv\varphi \tilde b_n \colon \ 
                 \tilde b_1 \leq \tilde b_2 \leq \dotsb \ &&
                 \varphi\text{-convergent in }
                 K\text{ with }&\tilde b_n\,&\geq\,b_n \bigr\} \\
\end{alignat*}
This is sufficient,
because it is easily seen that $\bv L \leq \bw U$.
Indeed,
if---so to say---$\bw \varphi \tilde a_n \in L$
and $\bv \varphi \tilde b_n \in U$
then $\bw\tilde a_n \leq \bw a_n \leq \bv b_n \leq \bv \tilde b_n$
and $\bv \varphi \tilde b_n \leq \bw \varphi \tilde a_n$
by continuity of~$\varphi|K$.

Since $\bw \varphi a_n$ is easily seen to be an upper bound of~$L$,\quad
 $\bw \varphi a_n = \bv L$\quad
follows directly from lemmas~\ref{lem:conv-inf} and~\ref{lem:main}.
Similarly, $\bv\varphi b_n = \bw U$, 
and we are done. \qed
\end{proof}

\begin{lem}
\label{lem:cont-ext-single}
Let $K$ be a sublattice of~$L$.
Then~$\varphi$ is continuous provided that
\begin{enumerate}
\item
\label{cont-ext-single-1}
The restriction~$\varphi|K$
of~$\varphi$ to~$K$ is continuous.

\item
\label{cont-ext-single-2}
If $a\in L$,
then
$\bw\tilde a_n = a$ and $\bw \varphi \tilde a_n = \varphi a$
for some $\varphi$-convergent $\tilde a_1 \geq \tilde a_2 \geq \dotsb$
from~$K$.

\item
\label{cont-ext-single-3}
$\bw a_n \in L$ and $\varphi \bw a_n = \bw \varphi a_n$
for every $\varphi$-convergent sequence $a_1 \geq a_2 \geq \dotsb$
in~$L$.
\end{enumerate}
\end{lem}
\begin{proof}
Let $\varphi$-convergent sequences
$a_1 \geq a_2 \geq \dotsb$
and $b_1 \leq b_2 \leq \dotsb$
with $\bw a_n \leq \bv b_n$ be given.
By assumptions~\ref{cont-ext-single-2} and~\ref{cont-ext-single-3}
there is a $\varphi$-convergent sequence 
$\tilde a_1 \geq \tilde a_2 \geq \dotsb$ in~$K$
such that $\bw \tilde a_n = \bw a_n$
and $\bw\varphi\tilde a_n = \bw\varphi a_n$.
Define~$U$ and~$L$ as in lemma~\ref{lem:cont-ext-double}.
Again it suffices to show that~$\bv L = \bw \varphi a_n$
and~$\bw U = \bv \varphi b_n$.

By assumption~\ref{cont-ext-single-2}
and property~\ref{E-inf-conv} of~$E$,
$K$ is lower dense in~$L$.
Hence $\bw U = \bv \varphi b_n$ follows as before.
Finally, $\bv L = \bw \varphi a_n$  since $\bw \varphi \tilde a_n \in L$.
\end{proof}

\subsection{$E$ is benign}
\label{SS:fitting-benign}
\todo{write}


\end{document}
