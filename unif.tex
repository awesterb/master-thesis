\documentclass[main.tex]{subfiles}
\begin{document}
\section{Uniformity on $E$}
\label{S:unif}
\noindent
To prove that~$\R$
is benign (see Definition~\ref{D:benign}),
we study
ordered Abelian groups~$E$ which are endowed with
a certain uniformity (such as~$\R$)
in Subsection~\ref{SS:fitting}.
We prove that all such~$E$ are benign
(see Theorem~\ref{T:fitting-benign}),
in the following way.

Let $\vs{V}{L}\varphi{E}$ be a valuation system.
Recall that in order to prove that~$E$ is benign
 we must show that 
if $\varphi$ is continuous,
then $\varphi$ is extendible (see Definition~\ref{D:benign}).
We will first prove that
if $\varphi$ is continuous,
then both $\Pi \varphi$ and $\Sigma \varphi$ are continuous
(see Lemma~\ref{lem:cont-ext-single}).
Then, by induction,
we see that $\varphi$ is
 both $\Pi_n$-extendible and $\Sigma_n$-extendible,
and both $\Pi_n\varphi$  and $\Sigma_n \varphi$
are continuous,
for every~$n\in \N$.
Hence $\varphi$ is $\Pi_\omega$-extendible.
However,
it is not clear a priori that~$\Pi_\omega\varphi$
is continuous.

Secondly,
we prove that if 
$\varphi$ is $\Pi_\lambda$-extendible
for some ordinal number~$\lambda$,
then $\Pi_\lambda\varphi$ is continuous.
So by induction
we see that $\varphi$ is both $\Pi_\alpha$-extendible
and $\Sigma_\alpha$-extendible,
and both $\Pi_\alpha\varphi$ and $\Sigma_\alpha\varphi$
are continuous,
for every ordinal number~$\alpha$ 
(see Lemma~\ref{L:fitting-ext}).
Hence $\varphi$ is extendible
(see Corollary~\ref{C:aleph1}).

To prove the second statement
we use the fact that
elements of $\Pi_\alpha L$ (or $\Sigma_\alpha L$)
can be approximated from below
by elements of $\Pi L$, in some sense
(see Lemma~\ref{L:fitting-dense}).
We will express this by
$\Sigma L$ is \emph{lower $\Pi_\alpha \varphi$-dense}
in $\Pi_\alpha L$.
We will formally introduce this notion,
and study it, in Subsection~\ref{SS:dense}.

\subsection{Fitting Uniformity}
\label{SS:fitting}
%
%                  DEFINITION OF A GOOD UNIFORMITY
%
\begin{dfn}
\label{D:uniformity}
\label{D:fitting-uniformity}
Let~$E$ be an ordered Abelian group.
A \keyword{fitting uniformity} on~$E$
is a \emph{countable} set~$\Phi$ of binary relations on~$E$
with the following properties.
\newcounter{epropc}
\begin{enumerate}
\item 
\label{E-refl}
We have $s \se s$ for all $\ve \in \Phi$ and $s \in E$.
\item
\label{E-min}
There is a map $\wedge\colon \Phi\times\Phi \ra \Phi$
such that
\begin{equation*}
s \ \ \varepsilon\wedge\delta\ \   t
\quad\implies\quad
s \,\varepsilon\,t\ \text{ and }\ s\,\delta\,t
\qquad (\varepsilon,\delta\in\Phi,\ s,t\in E).
\end{equation*}

\item
\label{E-half}
There is a map $\dt-\colon \Phi \ra \Phi$
such that
\begin{equation*}
r\ \ \dt\varepsilon \ \ s \ \  \dt\varepsilon \ \ t
\quad\implies\quad
r \se t\qquad(\varepsilon\in\Phi,\ r,s,t\in E).
\end{equation*}

\item \label{E-ord}
Given $\varepsilon\in\Phi$ and $r,s,t\in E$ 
with $r\leq s\leq t$,
we have
\begin{equation*}
r\,\varepsilon\,t
\quad\implies\quad
r\,\varepsilon\,s
\ \text{ and }\ 
s\,\varepsilon\,t.
\end{equation*}

\item \label{E-haus}
Let $s,t\in E$ with $s\leq t$.
Then $s=t$ provided that $s\,\varepsilon\,t$ for all~$\varepsilon\in\Phi$.

\item \label{E-inf-conv}
If a sequence $s_1 \geq s_2 \geq \dotsb$ from~$E$
has an infimum~$s\in E$,
then 
\begin{equation*}
\forall\varepsilon\in \Phi
\ \ \exists N\in \N
\ \ s \, \varepsilon\, s_N.
\end{equation*}

\item  \label{E-bound-inf}
Let $s_1\geq s_2 \geq \dotsb$ be a
sequence in~$E$,
and assume that
for every $\ve\in \Phi$
there is an~$N \in \N$ such that 
$s_{n} \ \varepsilon \ s_{N_\ve}\quad (n\geq N)$.

Then $s_1 \geq s_2 \geq \dotsb$ has an infimum~$\bw_n s_n$.

\item \label{E-add}
Let $r,s,t\in E$ and $\varepsilon\in\Phi$ be given.
Then $s\,\varepsilon\,t$ implies $r+s\ \varepsilon\ r+t$.
\setcounter{epropc}{\value{enumi}}
\end{enumerate}
\end{dfn}
\begin{ex}
We define a fitting uniformity on~$\R$.
For each natural number~$n$,
let $\varepsilon_n$ be the binary relation
on~$\R$ given by 
\begin{equation*}
s \ \varepsilon_n\  t
\quad\iff\quad 
s\leq t\quad\text{and}\quad t-s \leq 2^{-n}.
\end{equation*}
Then $\Phi\eqdf \{ \varepsilon_n \colon n\in \N \}$
is a fitting uniformity on~$\R$.

(Take $\varepsilon_n \wedge \varepsilon_m \eqdf \varepsilon_{n\vee m}$
and $\dt{\varepsilon_n} \eqdf \varepsilon_{n+1}$
for all $n,m\in \N$.)
\end{ex}

\begin{rem}
The fitting uniformities
defined here
are related to
the \emph{uniform spaces}
(or more precisely,
\emph{quasi uniform spaces})
studied in topology, see~\cite{Willard70}.

However
we do not involve uniform spaces,
because  the usual way of reasoning about them
does not seem to fit well with
property~\ref{E-ord}.
Moreover,
we do not wish to assume that the reader is familiar with uniform spaces.
\end{rem}
%
%                  LEMMA ON FURTHER PROPERTIES
%
To the list of properties 
that a fitting uniformity must have (see Definition~\ref{D:uniformity}),
we add 
some easy observations
in Lemma~\ref{L:E-prop}.
When we speak of ``property~($q$)'',
where~$q$ is some Roman numeral,
we refer to this list.
\begin{lem}
\label{L:E-prop}
Let $E$ be an ordered Abelian group
with a fitting uniformity~$\Phi$.
\begin{enumerate}
\setcounter{enumi}{\value{epropc}}
\item \label{E-minus}
Let $s,t\in E$ and $\ve \in \Phi$ be given.
Then $s \se t$ implies\, $-t \ \ve\,-\!\!s$.

\item \label{E-sup-conv}
If a sequence $s_1 \leq s_2 \leq \dotsb$ from~$E$
has an supremum~$s\in E$,
then 
\begin{equation*}
\forall\varepsilon\in \Phi
\ \ \exists N\in \N
\ \ s_N \, \varepsilon\, s.
\end{equation*}

\item  \label{E-bound-sup}
Let $s_1\leq s_2 \leq \dotsb$ be a
sequence in~$E$,
and assume that
for every $\ve\in \Phi$
there is an~$N_\ve \in \N$ such that 
$s_{N_\ve} \ \varepsilon \ s_n\quad (n\geq N_\ve)$.

Then $s_1 \leq s_2 \leq \dotsb$ has a supremum~$\bv_n s_n$.
\end{enumerate}
\end{lem}
\begin{proof}
\noindent\ref{E-minus}\ 
Let $s,t\in E$ and $\varepsilon\in \Phi$ be given,
and assume $s\se t$.
We prove $-t\ \varepsilon\,-\!\!s$.
Indeed, by property~\ref{E-add}, we have
\begin{equation*}
-t\,=\,-(t+s)\,+\,s\quad\ve\quad -(t+s)\,+\,t \,=\,-s.
\end{equation*}

\noindent\ref{E-sup-conv}
Let $s_1 \leq s_2 \leq \dotsb$
be a sequence in~$E$
which has a supremum~$s$ in~$E$.
Let $\varepsilon \in \Phi$ be given.
We need to find an~$N\in\N$
such that $s_N \ \varepsilon\ s$.

Let us consider the sequence
$-s_1 \geq -s_2 \geq\dotsb$.
By Lemma~\ref{L:oag-minus-preserves}
the sequence $-s_1 \geq -s_2 \geq \dotsb$
has an infimum, $-s$.
By property~\ref{E-inf-conv}
we have 
$-s\ \varepsilon\ {-s_N}$
for some~$N$.
Then by property~\ref{E-minus},
we get $s_N\ \varepsilon\ s$,
and we are done.

\vspace{.3em}
\noindent\ref{E-bound-sup}
Similar:
apply property~\ref{E-bound-inf}
to the sequence $-s_1 \geq -s_2 \geq \dotsb$.
\end{proof}

%
%                  NOTATION
%
\begin{nt}
\label{N:unif}
Let $E$ be an ordered Abelian group
with a fitting uniformity~$\Phi$.
\begin{enumerate}
\item
\label{N:unif-leq}
Given binary relations $\varepsilon$
and $\delta$ on~$E$
(for instance, $\varepsilon,\delta\in \Phi$),
we write
\begin{equation*}
\varepsilon \ \leq\ \delta
\quad\iff\quad 
\forall s,t\in E\ 
[\ s\ \varepsilon\ t
\implies
s\ \delta\ t\ ].
\end{equation*}

\item
\label{N:unif-plus}
Given binary relations $\varepsilon$ and $\delta$ on~$E$,
let $\varepsilon + \delta$
be the relation on~$E$ given by
\begin{equation*}
s\ \ \varepsilon + \delta\ \ t
\quad\iff\quad
\exists q\in E\ 
[\ s\ \varepsilon\ q\ \delta\ t\ ].
\end{equation*}
\end{enumerate}
\end{nt}
\begin{rem}
The operation ``$+$''
defined in Notation~\ref{N:unif}\ref{N:unif-plus}
is associative,
but not in general commutative
(contrary to the expectation the symbol ``$+$'' evokes).

The chosen notation does have advantages:
property~\ref{E-half} can be written as
\begin{equation*}
\dt\varepsilon \,+\,\dt\varepsilon \ \leq\ \varepsilon
\qquad(\varepsilon \in \Phi).
\end{equation*}
\end{rem}

%
%                  WITH A FITTING UNIFORMITY,
%                  E IS R-COMPLETE
%
\begin{lem}
\label{L:E-R-complete}
Let~$E$ be an ordered Abelian group with fitting uniformity~$\Phi$.\\
Then $E$ is $R$-complete (see Definition~\ref{D:R-complete}).
\end{lem}
\begin{proof}
Let $x_1 \leq x_2 \leq \dotsb$
and $y_1 \leq y_2 \leq \dotsb$ from~$E$ be given such that
\begin{equation}
\label{eq:unif-implies-R-comp}
x_{N+1} - x_N \ \leq\ y_{N+1} - y_N
\qquad(N\in\N).
\end{equation}
Assume $\bv_n y_n$ exists.
To that $E$ is $R$-complete,
we must show that $\bv_n x_n$ exists.

Let $\ve \in \Phi$ be given.
By property~\ref{E-bound-sup},
we know that 
to prove  $\bv_n x_n$ exists,
it suffices to find $N\in \N$ such that $x_N \se x_n$
for all~$n\geq N$.

By property~\ref{E-sup-conv},
we know there is an~$N$ such that $y_N \se \bv_m y_m$.
Let $n\geq N$ be given.
We will prove that $x_N \se x_n$.
We already know  $y_N \se y_n$
by property~\ref{E-ord}
because $y_N \leq y_n \leq \bv_m y_m$
and $y_N \se \bv_n y_m$.
So 
 $0\,\se\  (y_n - y_N)$
by property~\ref{E-add}.

From Inequality~\eqref{eq:unif-implies-R-comp}
one can easily derive that
\begin{equation*}
0 \ \leq\ x_n - x_N \ \leq\ y_n - y_N.
\end{equation*}
Since $0 \,\se\,(y_n-y_N)$
we get $0 \,\se\, (x_n -x_N)$ by property~\ref{E-ord}.

Hence $x_N \se x_n$ by property~\ref{E-add}.
So we are done.
\end{proof}
%
%                  DIRECTED SET INFIMUM LEMMA
%
The following lemma will be useful.
\begin{lem}
\label{lem:conv-inf}
Let~$E$ be an ordered Abelian group with 
fitting uniformity~$\Phi$.

Let $S\subseteq E$
be non-empty and \emph{downwards directed},
i.e., for all~$s_1,s_2\in S$,
there is an~$s\in S$ such that $s\leq s_1$
and $s\leq s_2$.

Let $t\in E$ be a lower bound of~$S$
which is close to~$S$ in the sense that
\begin{equation}
\label{eq:L:conv-inf}
\forall\varepsilon\in\Phi\ \exists s\in S\quad t\,\varepsilon\, s\text{.}
\end{equation}

Then~$t$ is the infimum of~$S$.
\end{lem}
\begin{proof}
To show that $t$ is the infimum of~$S$,
we need to prove that $\ell \leq t$
for every lower bound~$\ell$ of~$S$.
To do this, we take a detour.

Let~$\varepsilon_1,\varepsilon_2,\dotsb$
be an enumeration of~$\Phi$.
Using Equation~\eqref{eq:L:conv-inf},
and the fact that~$S$ is non-empty and directed,
choose~$s_1 \geq s_2 \geq \dotsb$ in~$S$
such that 
\begin{equation}
\label{eq:L:conv-inf-1}
t\ \varepsilon_n \ s_n\qquad(n\in \N).
\end{equation}

We will prove that
$s_1 \geq s_2 \geq \dotsb$
has an infimum~$s$ and that $s=t$.

This is sufficient to prove that~$t$ is the infimum of~$S$.
Indeed,
if $\ell$ is a lower bound of~$S$,
then $\ell$ is a lower bound of~$s_1 \geq s_2 \geq\dotsb$,
and so $\ell \leq \bw_n s_n =t$.

We use property~\ref{E-inf-conv}
to show that $s_1 \geq s_2 \geq\dotsb$
has an infimum.
Given $\varepsilon\in\Phi$,
we need to find an~$N$ such that $s_n \ \ve\ s_N$
for all~$n\geq N$. Pick $k$ such that $\varepsilon=\varepsilon_k$
and take $N=k$. Let $n\geq N$ be given.
Note that $t \leq s_n \leq s_N =s_k$
and $t\ \varepsilon_k\ s_k$
by Equation~\eqref{eq:L:conv-inf-1}.
So we have $s_n\ \varepsilon_k\ s_N$ by property~\ref{E-ord}.

Hence property~\ref{E-inf-conv} implies that $s_1 \geq s_2 \geq \dotsb$
has an infimum, $s$.
It remains to be shown that $s=t$.
For this we use property~\ref{E-haus}.

Note that $t\leq s$ because $t\leq s_n$ for all~$n$.
Let $\varepsilon\in\Phi$ be given.
We need to prove that $t \se s$.
Choose $k$ such that $\ve = \ve_k$.
Then $t \leq s \leq s_k$
and $t \ \ve_k\ s_k$
by Equation~\eqref{eq:L:conv-inf-1}.
So $t\ \ve_k \  s$ by property~\ref{E-ord}.
Hence $s=t$ by property~\ref{E-haus}.
\end{proof}
%%%%%%%%%%%%%%%%%%%%%%%%%%%%%%%%%%%%%%%%%%%%%%%%%%%%%%%%%%%%%%%%%%%%%%%%%%%%%%%
%
%                  LOWER DENSENESS SUBSECTION
%
\subsection{Denseness}
\label{SS:dense}
Throughout this subsection,
$E$ will be an ordered Abelian group
endowed with a fitting uniformity~$\Phi$
(see Definition~\ref{D:uniformity}).
%
%                  LOWER DENSENESS
%
\begin{dfn}
\label{D:lower-dense}
Let $\vs{V}{L}\varphi{E}$ be a valuation system.
Let $S\subseteq T$ be subsets of~$L$.
We say $S$ is \keyword{lower $\varphi$-dense} in~$T$
provided that the following condition holds.
\begin{equation*}
\begin{minipage}{0.7\textwidth}
For every $a\in T$ and $\varepsilon\in\Phi$
there is an $\ell\in S$
such that
\begin{equation*}
\ell \leq a\quad\text{and}\quad\varphi (\ell)\,\varepsilon\, \varphi (a).
\end{equation*}
\end{minipage}
\end{equation*}
The notion of 
\keyword{upper $\varphi$-denseness} is defined similarly.
\end{dfn}
\begin{ex}
\label{E:sigma-dense}
Let $\vs{V}{L}\varphi{E}$
be a $\Sigma$-extendible valuation system
(see Def.~\ref{D:Pi-extendible}).
\emph{Then $L$ is lower $\Sigma\varphi$-dense in~$\Sigma L$.}

Indeed,
given $a\in \Sigma L$
and $\ve \in \Phi$,
we need to find an~$\ell\in L$ 
such that $\varphi(\ell) \ \ve\ \Sigma\varphi(a)$.
Write $a=\bv_n a_n$ for
some $\varphi$-convergent sequence $a_1 \leq a_2 \leq \dotsb$.
Then we have
\begin{equation*}
\Sigma\varphi(a)\ =\ \bv_n \varphi(a_n).
\end{equation*}
By property~\ref{E-sup-conv},
there is an~$N$ such that $\varphi(a_N) \ \ve\ \Sigma\varphi(a)$.
So take $\ell =a_N$.
\end{ex}
%
%                  ELEMENTARY PROPERTIES OF LOWER DENSENESS
%
\begin{lem}
\label{L:ldense-prop}
Let $\vs{V}{L}\varphi{E}$ be a valuation system.
\begin{enumerate}
\item
\label{L:ldense-prop-1}
Let $R\subseteq S\subseteq T$ be subsets of~$L$.
Suppose $R$ is lower $\varphi$-dense in~$S$,
and suppose that $S$ is lower $\varphi$-dense in~$T$.
Then $R$ is lower $\varphi$-dense in~$T$.

\item
\label{L:ldense-prop-2}
Let $R$ be a subset of~$L$,
and let $\mathcal S$ be a family of subsets of~$L$.\\
If $R$ is lower $\varphi$-dense in each~$S\in \mathcal{S}$,
then $R$ is lower $\varphi$-dense in~$\bigcup \mathcal{S}$.
\end{enumerate}
\end{lem}
\begin{proof}
\noindent\ref{L:ldense-prop-1}\ 
Let $t\in T$ and $\ve \in\Phi$ be given.
To prove $R$ is lower $\varphi$-dense in~$T$,
we need to find an $r\in R$
with $r\leq t$ and $\varphi(r) \ \ve\ \varphi(t)$.
This is easy.
Choose an $s\in S$
such that $s \leq t$ and $\varphi(s)\ \dt\ve\ \varphi(t)$
(see Definition~\ref{D:uniformity}\ref{E-half}
for the meaning of ``$\dt\ve$'').
Choose an $r\in R$
such that $r\leq s$ and $\varphi(r)\ \dt\ve\ \varphi(s)$.
Then $r\leq s$ and $\varphi(r) \ \ve\ \varphi(t)$.
\vspace{.3em}

\noindent\ref{L:ldense-prop-2}\ 
We leave this to the reader.
\end{proof}
%
%
%                  MAIN LEMMA
%
%
The proof that~$E$
 is benign
hinges on the following lemma.
\begin{lem}
\label{lem:main}
Let $\vs{V}{L}\varphi{E}$ be a valuation system.\\
Let $K$ be a lower $\varphi$-dense sublattice of~$L$.\\
Then for every $\varphi$-convergent sequence
 $a_1 \geq a_2 \geq \dotsb$ from~$L$
and $\varepsilon\in \Phi$\\
there is a $\varphi$-convergent sequence
$\tilde a_1 \geq \tilde a_2 \geq \dotsb$ from~$K$
with 
\begin{equation}
\label{eq:main}
\tilde a_n \ \leq\  a_n
\qquad\text{and}\qquad
\bw_n \varphi (\tilde a_n) \ \ \varepsilon\ \ \bw_n\varphi (a_n).
\end{equation}
\end{lem}
\begin{proof}
Let $a_1 \geq a_2 \geq \dotsb$ 
be a $\varphi$-convergent sequence in~$L$,
and let~$\varepsilon\in\Phi$ be given.
We need to find a $\varphi$-convergent sequence 
$\tilde a_1 \geq \tilde a_2 \geq \dotsb$ in~$K$
which satisfies Condition~\eqref{eq:main}.
To this end,
we seek a sequence 
$\tilde a_1 \geq \tilde a_2 \geq \dotsb$ in~$K$
such that
\begin{equation}
\label{eq:lem:main-cond}
\varphi(\tilde a_n) \ \eta\ \varphi (a_n)
\qquad\text{and}\qquad
\forall i\in\N\ \ \exists N\in\N 
\ \ \forall n\geq N
\  [\ \  \varphi \tilde a_n \ \varepsilon_i\  \varphi \tilde a_N \ \ ],
\end{equation}
where $\varepsilon_1,\,\varepsilon_2,\,\dotsc$
is
an enumeration of~$\Phi$,
and $\eta\in\Phi$ with  $2\eta \leq \varepsilon$
(see Notation~\ref{N:unif}).

Such a sequence $\tilde a_1 \geq \tilde a_2 \geq \dotsb$
is $\varphi$-convergent 
(by property~\ref{E-bound-inf}).
We prove that $\tilde a_1 \geq \tilde a_2 \geq\dotsb$
satisfies Condition~\eqref{eq:main}. 
Indeed:
We know $\bw_n \varphi(\tilde a_n)$ exists.
Hence,
there is an~$N\in\N$
with 
 $\bw_n\varphi (\tilde a_n)\ \eta\ \varphi (\tilde a_N)$
by property~\ref{E-inf-conv}.
Then
\begin{equation*}
\bw_n \varphi (\tilde a_n) \quad \eta \quad \varphi(\tilde a_N)
\quad\eta\quad \varphi(a_N).
\end{equation*}
So we have 
$\bw_n \varphi (\tilde a_n) \ \ve\  \varphi(a_N)$.
But $\bw_n \varphi(\tilde a_n) \,\leq\,
\bw_n \varphi(a_n) \,\leq\,\varphi(a_N)$.
Thus 
$\bw_n \varphi(\tilde a_n) \ \ve\ 
\bw_n \varphi(a_n)$ by property~\ref{E-ord}.
Hence $\tilde a_1  \geq \tilde a_2 \geq \dotsb$
satisfies Condition~\eqref{eq:main}.

Finding a sequence $\tilde a_1 \geq \tilde a_2 \geq \dotsb$ 
which satisfies Condition~\eqref{eq:lem:main-cond}
is a subtle affair.
Pick $\eta_1, \eta_2,\dotsc$ and $\zeta_1,\zeta_2,\dotsc$
from~$\Phi$
(using properties~\ref{E-half}
and~\ref{E-min})
such that
\begin{equation*}
2 \eta_i \leq \varepsilon_i\text{,} \qquad
\eta_i \leq \eta\text{,} \qquad
2\zeta_i \leq \eta_i\text{,} \qquad
2\zeta_{i+1} \leq \zeta_{i}\text{.}
\end{equation*}
Then we have
\begin{equation}
\label{eq:lem:main-zeta--eta}
\zeta_i + \dotsb + \zeta_j \leq \eta_i \qquad (i,j\in\N,\  i\leq j)\text{.}
\end{equation}
Pick $\ell_1,\ell_2,\dotsc$ from $K$ 
such that $\ell_n \leq a_n$ and $\varphi(\ell_n)\ \ \zeta_n\  \ \varphi(a_n)$
and define 
\begin{equation*}
\tilde{a}_{ij} \ =\ \ell_i \wedge \dotsb\wedge \ell_j\text{,}
\qquad\qquad
\tilde{a}_n \ =\ \tilde{a}_{1n} =\ell_1 \wedge\dotsb\wedge \ell_n\text{,}
\end{equation*}
where $i,j,n\in N$ with $i\leq j$.
Then $\tilde a_{ij} \in K$ and  $\tilde a_n \leq \ell_n \leq a_n$.
We will prove that
the sequence $\tilde a_1 \geq \tilde a_2 \geq \dotsb$
satisfies Condition~\eqref{eq:lem:main-cond}.

Note that for all~$i,j\in\N$ with $i\leq j$, we have,
by Lemma~\ref{L:wv-unif},
\begin{equation*}
\ld\varphi(\tilde a_{ij},a_j)\ =\ 
\ld\varphi(\ell_i \wedge\dotsb\wedge \ell_j,\  a_i \wedge \dotsb \wedge a_j)
\ \leq\ \ld\varphi(\ell_i,a_i)+\dotsb+\ld\varphi(\ell_j,a_j)\text{.}
\end{equation*}
Since $\varphi(\ell_k)\ \zeta_k\ \varphi(a_k)$
for all~$k$, 
the inequality above 
yields,
using property~\ref{E-add},
\begin{equation*}
\varphi(\tilde a_{ij}) \quad \zeta_i + \dotsb + \zeta_j\quad \varphi(a_j).
\end{equation*}
So because
$\zeta_i+\dotsb+\zeta_j \leq \eta_i$ 
(see Inequality~\eqref{eq:lem:main-zeta--eta}),
we have
\begin{equation}
\label{eq:lem:main-3}
\varphi(\tilde a_{ij} )\quad\eta_i\quad\varphi (a_j).
\end{equation}
In particular,
we get $\varphi (\tilde a_n)\equiv \varphi(\tilde a_{1n}) \ \eta_1 \ \varphi (a_n)$.
Hence $\varphi(\tilde a_n) \ \eta\ \varphi(a_n)$
as $\eta_1 \leq \eta$.

Let~$i\in \N$ be given.
To prove that $\tilde a_1 \geq \tilde a_2 \geq \dotsb$
satisfies Condition~\eqref{eq:lem:main-cond},
it remains to be shown that 
there is an~$N\in\N$ such that
\begin{equation}
\label{eq:lem:main-2}
\varphi(\tilde a_n)\ \ \varepsilon_i \ \ \varphi(\tilde a_N)
\qquad (n \geq N)
\end{equation}
Using property~\ref{E-inf-conv},
determine $N\geq i$ such that $\bw_n \varphi (a_n) \ \eta_i\ \varphi(a_N)$.
We will show that Statement~\eqref{eq:lem:main-2} holds.
Let $n\geq N$ be given.
Note that 
 by Lemma~\ref{L:curry-wc-unif},
\begin{equation*}
\ld\varphi (\tilde a_n, \tilde a_N)
\,=\,\ld\varphi(\,\tilde a_{i-1} \wedge \tilde a_{in},\,
\tilde a_{i-1} \wedge \tilde a_{iN}\,)
\ \leq\ \ld\varphi(\tilde a_{in},\tilde a_{iN}).
\end{equation*}
So to prove Statement~\eqref{eq:lem:main-2},
it suffices to show 
that $\varphi(\tilde a_{in} ) \ \ve_i\ \varphi(\tilde a_{iN})$.

Recall that $\bw_m\varphi(a_m)\ \ \eta_i\ \ \varphi(a_N)$
by choice of~$N$. Then in particular,
we get
$\varphi(a_n) \ \eta_i\ \varphi(a_N)$
by property~\ref{E-ord}.
Further, $\varphi(\tilde a_{in}) \ \eta_i\ \varphi(a_n)$
by Inequality~\eqref{eq:lem:main-3}. So
\begin{equation*}
\varphi (\tilde a_{in})\ \eta_i\ 
\varphi (a_{n})\ \eta_i\ \varphi (a_N).
\end{equation*}
Hence $\varphi (\tilde a_n) \ \ve_i\ \varphi (a_N)$,
because $2\eta_i \leq \ve_i$.
Note that $\varphi(\tilde a_{in}) \leq 
\varphi(\tilde a_{iN}) \leq \varphi(a_N)$.

So by property~\ref{E-ord},
we get
$\varphi(\tilde a_{in}) \ \ve_i\  \varphi(\tilde a_{iN})$.
\end{proof}
\begin{cor}
\label{C:main-nice}
Let $\vs{V}{L}\varphi{E}$ be a $\Pi$-extendible valuation system.\\
Let $K$ be a sublattice of~$L$. Then
\begin{equation*}
\text{ $K$ is lower dense in~$L$}
\quad\implies\quad
\text{$\Pi K$ is lower dense in~$\Pi L$.}
\end{equation*}
\end{cor}
\begin{proof}
Follows immediately from Lemma~\ref{lem:main}.
\end{proof}
%
%                  LEMMA ON DENSENESS IN EXTENSION
%
\begin{lem}
\label{L:fitting-dense}
Let $\vs{V}{L}\varphi{E}$ be a valuation system
which is both $\Sigma$-extendible
and $\Pi$-extendible.
Then for every ordinal number~$\alpha$:
\begin{enumerate}
\item 
\label{L:fitting-dense-1}
If $\varphi$ is $\Pi_\alpha$-extendible,
then 
\begin{equation*}
\text{$\Pi L$ is upper dense in $\Pi_\alpha L$,}
\quad\text{and}\qquad
\text{$\Sigma L$ is lower dense in $\Pi_\alpha L$.}
\end{equation*}

\item
\label{L:fitting-dense-2}
If $\varphi$ is $\Sigma_\alpha$-extendible,
then 
\begin{equation*}
\text{$\Pi L$ is upper dense in $\Sigma_\alpha L$},
\quad\text{and}\qquad
\text{$\Sigma L$ is lower dense in $\Sigma_\alpha L$.}
\end{equation*}
\end{enumerate}
\end{lem}
\begin{proof}
We use induction on~$\alpha$.

\vspace{.3em}
For $\alpha=0$,
Statements~\ref{L:fitting-dense-1}
and~\ref{L:fitting-dense-2} are trivial.

\vspace{.3em}
Let~$\alpha$ be an ordinal number
such that 
Statement~\ref{L:fitting-dense-1}
holds for~$\alpha$
in order to prove that
Statement~\ref{L:fitting-dense-2} holds for~$\alpha+1$.
Suppose~$\varphi$ is $\Sigma_{\alpha+1}$-extendible.
We need to prove that $\Pi L$ is upper dense in~$\Sigma_{\alpha+1} L$
and that $\Sigma L$ is lower dense in~$\Sigma_{\alpha+1} L$.

Note that 
$\varphi$ is $\Pi_\alpha$-extendible,
because  $\varphi$ is $\Sigma_{\alpha+1}$-extendible.

By Statement~\ref{L:fitting-dense-1} for~$\alpha$,
we know that $\Pi L$ is lower dense in $\Pi_\alpha L$.
Further, $\Pi_\alpha L$
is lower dense in $\Sigma(\Pi_\alpha L) = \Sigma_{\alpha+1} L$
by Example~\ref{E:sigma-dense}.
So we see that $\Pi L$ is lower dense in~$\Sigma_{\alpha+1} L$ 
by Lemma~\ref{L:ldense-prop}\ref{L:ldense-prop-1}.

By Statement~\ref{L:fitting-dense-1} for~$\alpha$,
we know that $\Sigma L$ is upper dense in $\Pi_\alpha L$.
So by the dual of Corollary~\ref{C:main-nice},
we have
$\Sigma L = \Sigma(\Sigma L)$ is upper dense in
$\Sigma(\Pi_\alpha L) = \Sigma_{\alpha+1} L$.

Hence, Statement~\ref{L:fitting-dense-2} holds for~$\alpha+1$
(if Statement~\ref{L:fitting-dense-1} holds for~$\alpha$).

Similarly,
if Statement~\ref{L:fitting-dense-2} holds for~$\alpha$,
then Statement~\ref{L:fitting-dense-1} holds for~$\alpha+1$.

\vspace{.3em}
Let
$\lambda$ be a limit ordinal
such that Statement~\ref{L:fitting-dense-1} holds
for all~$\alpha<\lambda$.
We prove that Statement~\ref{L:fitting-dense-1} holds
for~$\lambda$.
Suppose that $\varphi$ is $\Pi_\lambda$-extendible.
We need to prove that $\Pi L$ is upper dense in~$\Pi_\lambda L$
and $\Sigma L$ is lower dense in $\Pi_\lambda L$.

We know that $\varphi$ is $\Pi_\alpha$-extendible
for all~$\alpha < \lambda$.

As Statement~\ref{L:fitting-dense-1} holds for all~$\alpha <\lambda$,
we see that $\Pi L$ is upper dense in all~$\Pi_\alpha L$.
So by Lemma~\ref{L:ldense-prop}\ref{L:ldense-prop-2},
$\Pi L$ is upper dense in
 $\Pi_\lambda L = \bigcup_{\alpha <\lambda} \Pi_\alpha L$.

Similarly,
$\Sigma L$ is lower dense in
$\Sigma_\lambda L= \bigcup_{\alpha < \lambda} \Sigma_\alpha L$.
\end{proof}
%
%                  MAIN COROLLARY
%
\begin{cor}
\label{C:main}
Let $\vs{V}{L}\varphi{E}$ be a valuation system.\\
Let $K$ be a lower~$\varphi$-dense sublattice of~$L$
and assume that $\psi \eqdf \varphi | K$ is $\Pi$-extendible.\\
Let $a_1 \geq a_2 \geq \dotsb$ be a $\varphi$-convergent sequence in~$L$.
Then
\begin{equation}
\label{eq:C:main}
\bw_n \varphi(a_n) \ = \ 
\bv\ \bigl\{\ \Pi \psi(\ell) \colon \ 
 \ell \in S \ \bigr\},
\end{equation}
where 
$S \ \eqdf\ 
\bigl\{\ \bw_n \tilde a_n \colon\ 
\text{$\psi$-convergent $\tilde a_1 \geq \tilde a_2 \geq \dotsb$
with $\tilde a_n \leq a_n$ for all~$n$}\ \bigr\}$.
\end{cor}
\begin{proof}
To prove Statement~\eqref{eq:C:main},
we apply the dual of Lemma~\ref{lem:conv-inf}.
We need to verify 
that  $\Pi\psi(S)\eqdf\{\,\Pi\psi(\ell) \colon \ell \in S\,\}$
is upwards directed,
that~$\bw_n\varphi(a_n)$
is a lower bound of   $\Pi\psi(S)$,
and that
\begin{equation}
\label{eq:C:main-1}
\forall \varepsilon \in \Phi\ \ 
\exists \ell\in S\quad  \Pi\psi(\ell) \ \ \ve \ \ \bw_n \varphi(a_n).
\end{equation}
To begin,
note that Statement~\eqref{eq:C:main-1} follows immediately
from Lemma~\ref{lem:main}.

Let $\psi$-convergent 
$\tilde a_1 \geq \tilde a_2 \geq \dotsb$
with $\tilde a_n \leq a_n$ for all~$n$ be given.
Then we have  $\psi(\tilde a_n) = \varphi(\tilde a_n)\leq \varphi(a_n)$
for all~$n$, so $\Pi\psi(\bw_n\tilde a_n) = \bw_n \psi(\tilde a_n) 
\leq \bw_n \varphi(a_n)$.
Hence~$\bw_n\varphi(a_n)$
is a lower bound of $\Pi\psi(S)$.
 
To prove that~$\Pi\psi(S)$
is upwards directed,
it suffices to show that~$S$
is upwards directed 
(as $\Pi\psi$ is order preserving).
Let $\psi$-convergent sequences 
$\tilde a_1 \geq \tilde a_2 \geq \dotsb$
and
$\tilde a_1' \geq \tilde a_2' \geq \dotsb$
with $\tilde a_n \leq a_n$ and $\tilde a_n' \leq a_n$
be given.
Then 
\begin{equation*}
\tilde a_1 \vee \tilde a_1' \,\leq\, 
\tilde a_1 \vee \tilde a_2' \,\leq\,\dotsb
\end{equation*}
is again a $\psi$-convergent sequence by Proposition~\ref{P:R-main}.
Further $\tilde a_n \vee \tilde a_n' \leq a_n$
for all~$n$.
Hence $\bw_n \tilde a_n \vee \tilde a_n' \in S$.
But also $\bw_n\tilde a_n \leq \bw_n \tilde a_n \vee\tilde a_n'$
and $\bw_n\tilde a_n' \leq \bw_n \tilde a_n \vee \tilde a_n'$.
So we see that $S$ is upwards directed.
\end{proof}
%
%                  SINGLE SIDED CONTINUITY LEMMA
%
\begin{lem}
\label{lem:cont-ext-single}
Let $\vs{V}{L}\varphi{E}$ be a valuation system.\\
Assume $\varphi$ is continuous.
Then $\Pi \varphi$ is continuous.
\end{lem}
\begin{proof}
Note that~$L$ is an
upper $\Pi\varphi$-dense sublattice of~$\Pi L$
(see Example~\ref{E:sigma-dense}).
We apply Lemma~\ref{L:cont-ext}
to prove that $\varphi$ is continuous.
We must verify that Conditions~\ref{L:cont-ext-1}
and~\ref{L:cont-ext-2} of Lemma~\ref{L:cont-ext} hold.

\ref{L:cont-ext-1}\ 
Let $a_1 \geq a_2 \geq \dotsb$ be a $\Pi\varphi$-convergent
sequence in~$\Pi L$.
We need to find $S\subseteq \Pi L$
such that $\bw_n \varphi(a_n) = \bv S$
and $\ell \leq \bw_n a_n$ for all~$\ell \in S$.
By Lemma~\ref{L:Pi-complete},
we know that
$\Pi\varphi$ is $\Pi$-complete.
Hence $\bw_n a_n \in \Pi L$.
So simply take $S=\{ \bw_n a_n \}$.

\ref{L:cont-ext-2}\ 
Follows immediately from Corollary~\ref{C:main}.
\end{proof}
%
%                  DOUBLE SIDED CONTINUITY LEMMA
%
\begin{lem}
\label{lem:cont-ext-double}
Let $\vs{V}{L}\varphi{E}$ be a valuation system.

Let $K$ be a sublattice of~$L$.
Then $\varphi$ is continuous provided that:
\begin{enumerate}
\item
The restriction $\varphi|K$ of $\varphi$ to~$K$ is continuous.
\item
$K$ is lower and upper $\varphi$-dense in~$L$.
\end{enumerate}
\end{lem}
\begin{proof}
This follows from Lemma~\ref{L:cont-ext}.
Indeed, condition~\ref{L:cont-ext-1} holds
by Corollary~\ref{C:main},
and condition~\ref{L:cont-ext-2} holds
by the dual of Corollary~\ref{C:main}.
\end{proof}
%
%                  FITTING UNIFORMITY EXTENSION LEMMA
%
\begin{lem}
\label{L:fitting-ext}
Let $\vs{V}{L}\varphi{E}$ be a continuous valuation system,
and $\alpha$ an ordinal.
Then $\varphi$ is both $\Pi_\alpha$-extendible
and $\Sigma_\alpha$-extendible,
and $\Pi_\alpha\varphi$ and $\Sigma_\alpha\varphi$
are continuous.
\end{lem}
\begin{proof}
With induction on~$\alpha$.

\vspace{.3em}
For $\alpha=0$,
the statement is trivial.

\vspace{.3em}
Let~$\alpha$ be an ordinal number
and assume that $\varphi$
is $\Pi_\alpha$-extendible
and $\Pi_\alpha \varphi$ is continuous.
We prove that $\varphi$
is $\Sigma_{\alpha+1}$-extendible
and $\Sigma_{\alpha+1}\varphi$ is continuous.
Indeed,
since $\Pi_\alpha\varphi$ is continuous,
$\Pi_\alpha\varphi$ is $\Sigma$-extendible
and so $\varphi$ is $\Sigma_{\alpha+1}$-extendible.
Finally,
$\Sigma(\Pi_\alpha \varphi)=\Sigma_{\alpha+1}\varphi$
is continuous
by the dual of Lemma~\ref{lem:cont-ext-single}.

Similarly,
if $\varphi$ is $\Sigma_\alpha$-extendible
and $\Sigma_\alpha \varphi$ is continuous,
then $\varphi$
is $\Pi_{\alpha+1}$-extendible
and $\Pi_{\alpha+1}\varphi$ is continuous.

\vspace{.3em}
Let $\lambda$ be a limit ordinal
such that for each~$\alpha<\lambda$,
we have that $\varphi$ is $\Pi_\alpha$-extendible
and $\Pi_\alpha\varphi$ is continuous.
Note that $\varphi$ is $\Pi_\lambda$-extendible.
We prove that $\Pi_\lambda \varphi$ is continuous.
For this,
we use Lemma~\ref{lem:cont-ext-double}.
Consider $\psi \eqdf \Pi_2 \varphi$.
By assumption,
$\psi$ is continuous.
We know
that $\Pi_\lambda \varphi$ extends $\psi$,
and that $\psi$ extends both $\Pi\varphi$ and $\Sigma\varphi$.
Since $\Pi L$ is lower dense in $\Pi_\alpha L$,
and $\Sigma L$ is upper dense in $\Pi_\alpha L$
(by Lemma~\ref{L:fitting-dense}),
we get that $K\eqdf \Pi_2 L$ is both upper and lower dense in~$\Pi_\alpha L$.
So by Lemma~\ref{lem:cont-ext-double},
we see that~$\Pi_\lambda\varphi$ is continuous.
(Of course, 
the argument is also valid for other choices for~$\psi$,
such as $\Sigma_3\varphi$ and $\Pi_{42} \varphi$.)
\end{proof}
%
%                  THEOREM
%
\begin{thm}
\label{T:fitting-benign}
Let $E$ be an ordered Abelian group. \\
If $E$ has a fitting uniformity,
then~$E$ is benign.
\end{thm}
\begin{proof}
Let $\vs{V}{L}\varphi{E}$ be a continuous valuation system.
To prove that~$E$ is benign,
we must show that~$\varphi$ is extendible
(see Definition~\ref{D:benign}).
It suffices to prove that~$\varphi$ is $\Pi_{\aleph_1}$-extendible
by Corollary~\ref{C:aleph1}.
Now apply Lemma~\ref{L:fitting-ext}.
\end{proof}
\begin{cor}
\label{C:R-benign}
The ordered Abelian group~$\R$ is benign.
\end{cor}
\end{document}

