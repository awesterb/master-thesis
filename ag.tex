\section{Ordered Abelian Groups}
\label{S:ag}
\noindent
In this thesis
we do not only consider $\R$-valued measures and integrals,
but also~$E$-valued ones, 
where $E$ is an \emph{ordered Abelian group}.
Since we do not expect reader to be familiar with this
particular generalisation of~$\R$,
we have collected the relevant definitions
and basic results in this appendix.
\begin{dfn}
\label{D:oag}
An \keyword{ordered Abelian group} $E$
is a set that is endowed with an Abelian group operation, $+$,
and a partial order, $\leq$, 
such that 
\begin{equation*}
x \ \leq\ y \quad\implies\quad w+x \ \leq\ w+y
\qquad\quad(w,x,y\in E).
\end{equation*}
\end{dfn}
%
%                  EXAMPLES OF AOGs
%
\begin{exs}
\label{E:oag}
\begin{enumerate}
\item
\label{E:aog-1}
The integers, $\Z$, the rationals, $\Q$, and the reals, $\R$,
endowed with  the usual addition and order
are ordered Abelian groups.

\item
\label{E:oag_div}
Let
$\Q^{\circ}$
be the set of rational numbers $q$ with $q>0$.
Order $\Q^\circ$ by
\begin{equation*}
q \ \preccurlyeq \ r
\quad\iff\quad 
\exists n\in \N \ [\ q \cdot n = r \ ].
\end{equation*}
Then $\Q^\circ$ with 
the usual multiplication is an  ordered Abelian group.

\item
\label{E:aog-2}
Let $E_1$ and $E_2$ be ordered Abelian groups.
Then $E_1\times E_2$ with pointwise order
and pointwise group operation is an ordered Abelian group.

\item
\label{E:oag_lex}
Consider~$\R^2$ with the pointwise addition.
By point~\ref{E:aog-2},
$\R^2$ with the usual order
is an  ordered Abelian group.
Further,
$\R^2$ with the
 \emph{lexicograpgic order},
\begin{equation*}
(x_1,x_2)\ \leq\ (y_1,y_2)
\quad\iff\quad
\left[ \ \ 
\begin{alignedat}{3}
&x_1 < y_1 \quad\text{or}\\
&x_1 = y_1  \quad\text{and}\quad  x_2 \leq y_2 
\end{alignedat}
\right.,
\end{equation*}
is also an ordered Abelian group,
called the \keyword{lexicographic plane},
 $\Lex$.
\end{enumerate}
\end{exs}

\noindent
Let us prove some simple statements
concerning ordered Abelian groups.
%
%                  + IS AN ORDER ISOMORPHISM
%
\begin{lem}
\label{L:oag-plus-iso}
Let $E$ be an ordered Abelian group.
Then, for $x,y,w\in E$,
\begin{equation*}
x \,\leq\, y \quad\iff\quad w+x \,\leq\, w+y.
\end{equation*}
\end{lem}
\begin{proof}
``$\Longrightarrow$''\ 
By the definition of ordered Abelian group.

\noindent ``$\Longleftarrow$''\ 
If $w+x\leq w+y$, then $x \,=\,-w\,+\, (w+x) \ \leq\ 
-w\,+\,(w+y) \,=\, y$.
\end{proof}
%
%                  PRESERVATION /\ BY +
%
\begin{lem}
\label{L:oag-plus-preserves}
Let $E$ be an ordered Abelian group.
Let $A\subseteq E$ and $x\in E$ be given.
\begin{enumerate}
\item \label{L:oag-plus-preserves-meet}
If $A$ has an infimum,
then so has $x+A \eqdf \{ x+a\colon a\in A\}$,
and 
\begin{equation*}
\bw \,\,x+A \ =\  x+\bw A.
\end{equation*}
\item \label{L:oag-plus-preserves-join}
If $A$ has a supremum,
then so has $x+A$,
and 
\begin{equation*}
\bv \,\,x+A \ =\  x+\bv A.
\end{equation*}
\end{enumerate}
\end{lem}
\begin{proof}
It suffices to prove that the map $E\ra E$
given by $u\mapsto x + u$ is an order isomorphism.
This follows easily using Lemma~\ref{L:oag-plus-iso}.
\end{proof}
%
%                  - IS AN ORDER ISOMORPHISM
%
\begin{lem}
\label{L:oag-minus-iso}
Let $E$ be an ordered Abelian group.
Then, for $x,y\in E$,
\begin{equation*}
x \,\leq\, y \quad\iff\quad -x \,\geq\, -y.
\end{equation*}
\end{lem}
\begin{proof}
``$\Longrightarrow$''\ 
If $x\leq y$, then
$-y \,=\, (-x-y)\,+\,x
\,\leq\, (-x-y)\,+\, y
\,=\, -x$.

\noindent ``$\Longleftarrow$''\ 
If $-y \leq -x$,
then $x\,=\,-(-x) \,\leq\, -(-y) \,=\, y$
by ``$\Longrightarrow$''.
\end{proof}
%
%                  PRESERVATION OF /\ BY -
%
\begin{lem}
\label{L:oag-minus-preserves}
Let $E$ be an ordered Abelian group,
and $A\subseteq E$.
\begin{enumerate}
\item
If $A$ has an infimum,
then $-A\eqdf \{-a\colon a\in A\}$
has a supremum,
and 
\begin{equation*}
-\bw A \,=\, \bv -\hspace{-3pt}A.
\end{equation*}
\item
If $A$ has an supremum,
then $-A$
has an infimum, and
\begin{equation*}
-\bv A \,=\, \bw -\hspace{-3pt}A.
\end{equation*}
\end{enumerate}
\end{lem}
\begin{proof}
The map $E\rightarrow E$ given by
$u\mapsto -u$ is an order reversing isomorphism.
\end{proof}

\noindent
We use the following
lemma
regularly.
%
%                  ADDITION OF SEQUENCES
%
\begin{lem}
\label{L:addition-of-suprema}
Let~$E$ be an ordered Abelian group.\\
Let $x_1 \leq x_2 \leq \dotsb$
be from~$E$ such that $\bv_n x_n$ exists.\\
Let $y_1 \leq y_2 \leq \dotsb$
be from~$E$ such that $\bv_n y_n$ exists.
Then
\begin{equation}
\label{eq:addition-of-suprema}
(\bv_n x_n) \,+\, (\bv_n y_n)
\ =\ 
\bv_k\, x_k + y_k.
\end{equation}
\end{lem}
\begin{proof}
By Lemma~\ref{L:oag-plus-preserves}
we know that 
\begin{equation*}
(\bv_n x_n) \,+\, (\bv_m y_m) \ =\  \bv_{n,m}\ x_n + y_m.
\end{equation*}
So to prove Equation~\eqref{eq:addition-of-suprema} holds,
it suffices to show that 
\begin{equation*}
\bv_{n,m}\ x_n + y_m \ =\ \bv_k \ x_k + y_k.
\end{equation*}
That is,
writing $z\eqdf \bv_{n,m}\ x_n + y_m$,
we must show that~$z$ is the supremum of 
\begin{equation*}
S \ \eqdf\ \{\ x_1+y_1,\ x_2 + y_2,\  \dotsc \ \}.
\end{equation*}
That is,
we must show that~$z$ is the smallest upper bound of~$S$.

Given $s\in S$, we have $s\equiv x_k + y_k$ for some  $k\in \N$,
and 
\begin{equation*}
x_k + y_k \ \leq\  \bv_{n,m}\ x_n + y_m\,\equiv\, z.
\end{equation*}
So we see that~$z$ is an upper bound of~$S$.

Let $u\in E$ be an upper bound of~$S$.
To prove that~$z$ is the smallest upper bound
of~$S$, we must show that $z\leq u$.
It suffices to prove that,
for all $n,m\in \N$,
\begin{equation}
\label{L:addition-of-suprema_final-bit}
x_n + y_m \ \leq \ u.
\end{equation}
Let $n,m\in\N$ be given,
and define  $k\eqdf \max\{n,m\}$.
Then we see that
\begin{equation*}
x_n + y_m \ \leq\  x_k + y_k \ \leq\  u.
\end{equation*}
Hence Statement~\eqref{L:addition-of-suprema_final-bit}
holds,
and we are done.
\end{proof}
\noindent
Of course,
we have a similar statement concerning infima.
\begin{lem}
\label{L:addition-of-infima}
Let~$E$ be an ordered Abelian group.\\
Let $x_1 \geq x_2 \geq \dotsb$
be from~$E$ such that $\bw_n x_n$ exists.\\
Let $y_1 \geq y_2 \geq \dotsb$
be from~$E$ such that $\bw_n y_n$ exists.
Then
\begin{equation*}
(\bw_n x_n) \,+\, (\bw_n y_n)
\ =\ 
\bw_k\, x_k + y_k.
\end{equation*}
\end{lem}
\begin{proof}
Similar to the proof of Lemma~\ref{L:addition-of-suprema}.
\end{proof}
%
%                  E^+ and E^-
%
\noindent
We will occasionally use the following notation.
\begin{dfn}
Let $E$ be an ordered Abelian group.
We write
\begin{align*}
E^+ \ &\eqdf\ \{\  a\in E\colon\  a\geq 0\ \},&
E^- \ &\eqdf\ \{\  a\in E\colon\  a\leq 0\ \}.
\end{align*}
\end{dfn}
%
\noindent
Let us now turn to a special class of ordered Abelian groups.
%
%                  Lattice ordered Abelian groups
%
\begin{dfn}
\label{D:loag}
A \keyword{lattice ordered Abelian group}
is an ordered Abelian group~$E$,
such that the order~$\leq$ 
makes~$E$ a lattice,
i.e., each pair $x,y\in E$
has an infimum, $x\wedge y$,
and a supremum, $x\vee y$.
\end{dfn}
\begin{exs}
\label{E:loag}
\begin{enumerate}
\item
The sets
$\Z$, $\Q$, and $\R$ are lattices
under the usual order.
The supremum of two elements is their maximum,
the infimum is the minimum.

\item
More generally,
any partially ordered set~$E$
that is \emph{totally ordered},
i.e.,
\begin{equation*}
\text{either } \quad x\leq y\quad\text{ or } 
\quad y\leq x\quad \text{ for all }x,y\in E,
\end{equation*}
is a lattice.
The supremum of $x,y\in E$ is simply the maximum of~$x$ and $y$,
the infimum $x$ and $y$ is the minimum of~$x$ and~$y$.

\item
The space $\Lex$
(see Example~\ref{E:oag}\ref{E:oag_lex})
is totally ordered and hence a lattice.

\item
\label{E:loag_div}
The set~$\Q^\circ$ ordered by~$\preccurlyeq$
(see Examples~\ref{E:oag}\ref{E:oag_div})
is a lattice.

Let $m,n\in \Q^\circ$ be given.
If $m,n\in \Z$, then the supremum of $m$ and~$n$
is the least common multiple of~$m$ and~$n$,
and the infimum of~$m$ and~$n$ is the greatest common divisor
of~$m$ and~$n$.

\end{enumerate}
\end{exs}

\noindent
The following result is quite suprising.
%
%                  MODULARITY EQUATION FOR LATTICE ORDERED ABELIAN GROUPS
%
\begin{lem}
\label{L:1-valuation}
Let $E$ be a lattice ordered Abelian group.
Then we have 
\begin{equation*}
a\wedge b  \,+\, a\vee b \,=\, a\,+\,b \qquad(a,b\in E).
\end{equation*}
\end{lem}
\begin{proof}
$a\vee b - a - b
= (a - a - b) \vee (b - a - b)
= (-b)\vee(-a) = -(a\wedge b)$.
\end{proof}
\begin{exs}
\begin{enumerate}
\item
Let $x,y\in \R$ be given.
Then Lemma~\ref{L:1-valuation}
gives us
\begin{equation*}
x+y\ =\ \min\{x,y\} \,+\, \max\{x,y\}.
\end{equation*}
Of course, this is trivial.

\item
Let $m,n\in \Z$ with $m,n\geq 0$ be given.
Then Lemma~\ref{L:1-valuation}
gives us
\begin{equation*}
m\cdot n \ =\ \gcd\{m,n\} \,\cdot\,\lcm\{m,n\}.
\end{equation*}
The above equality is more difficult to derive directly.
\end{enumerate}
\end{exs}
%
%                  SIGMA-DEDEKIND COMPLETE ORDERED ABELIAN GROUPS
%
\noindent
We now turn to `complete' ordered Abelian groups.
\begin{dfn}
\label{D:sdc}
Let~$E$ be an ordered Abelian group.\\
We say~$E$ is \keyword{$\sigma$-Dedekind complete}
if the following statement holds.
\begin{equation*}
\left[\quad
\begin{minipage}{.7\columnwidth}
Let $x_1,x_2,\dotsc$ be a sequence in~$E$.\\
Assume $x_1,x_2,\dotsc$ has an upper bound.\\
Then $\bv_n x_n$ exists.
\end{minipage}
\right.
\end{equation*}
\end{dfn}
%
%                  EXAMPLES CONCERNING SIGMA-DEDEKIND COMPLETE OAGs
%
\begin{exs}
\label{E:sdc}
\begin{enumerate}
\item 
The ordered Abelian group $\R$ is $\sigma$-Dedekind complete.
\item 
The ordered Abelian $\Q$ is \emph{not} $\sigma$-Dedekind complete.
\item
\label{E:sdc_lex}
The lexicographic plane~$\Lex$
(see Examples~\ref{E:oag}\ref{E:oag_lex})
is \emph{not} $\sigma$-Dedekind complete.\\
Indeed,
consider the following elements of~$\Lex$.
\begin{equation*}
(0,0)\ \leq\ (0,1)\ \leq\ (0,2)\ \leq\ \dotsb
\ \leq \ (1,0)
\end{equation*}
If $\Lex$ were $\sigma$-Dedekind complete,
then $S\eqdf \{ \, (0,n)\colon\, n\in\N\,\}$
would have a supremum;
we will prove that~$S$ does not have a supremum.

Suppose (towards a contradiction) that~$S$ has a supremum, $(x,y)$.\\
Then we have $(0,n)\leq (x,y)$ for all~$n\in \N$.
In other words, for all $n\in\N$,
\begin{equation*}
0<x \qquad\text{or}\qquad 
(\ 0=x\quad\text{and}\quad n\leq y\ ).
\end{equation*}
Hence $0<x$,
because there is no $y\in \R$ such that $n\leq y$
for all~$n\in\N$.\\
But then $(x,y-1)$ is an upper bound of~$S$ as well.\\
Since $(x,y)$ is the smallest upper bound of~$S$,
we have $(x,y)\leq(x,y-1)$. 
So $y\leq y-1$,
which is absurd.
Hence $S$ has no supremum.
\end{enumerate}
\end{exs}
%
%                  REMARK ON BOUNDED
%
\begin{rem}
The requirement
in Definition~\ref{D:sdc}
 that $x_1,x_2,\dotsc$
has an upper bound 
is essential
to make the notion
of $\sigma$-Dedekind completeness non-trivial.

Indeed,
if~$E$ is an ordered Abelian group
in which \emph{every} sequence $x_1,x_2,\dotsc$ has a supremum~$\bv_n x_n$,
then we have~$E=\{0\}$~!

Let $a\in E^+$ be given. We prove that $a=0$.
Note that the sequence
\begin{equation*}
1\cdot a  \ \leq\ 2\cdot a \ \leq\  3\cdot a \ \leq\ \dotsb
\end{equation*}
has a supremum, $\bv_n \ n\cdot a$.
Note that by Lemma~\ref{L:oag-plus-preserves}, we have
\begin{equation*}
(\bv_n\ n\cdot a)\,-\,a\ =\ 
\bv_n \ (n-1) \cdot a \ = \ 
\bv_n\ n\cdot a.
\end{equation*}
So we see that $b-a = b$,
where $b\eqdf \bv_n \ n\cdot a$.
Hence $a=0$.

Let $a\in E$ be given.
We must prove that~$a=0$.
Note that
by Lemma~\ref{L:1-valuation},
\begin{equation}
\label{eq:R:sdc}
a\ = \ 0\wedge a \,+\, 0\vee a.
\end{equation}
We have $0\wedge a =0$,
since $0\wedge a \in E^+$.
We also have $0\vee a=0$,
because $0\vee a\in E^-$,
so $-(0\vee a)\in E^+$, so $-(0\vee a)=0$,
and thus $0\vee a=0$.

So we see that $a=0$ by Equation~\eqref{eq:R:sdc}.
Hence $E=\{0\}$.
\end{rem}
%
%                  REMARK ON EQUIVALENT DEFINITION
%
\begin{rem}
\label{R:sdc}
Let~$E$ be an ordered Abelian group.\\
Using the order reversing isomorphism $x\mapsto -x$,
the reader can easily verify that
$E$ is $\sigma$-Dedekind complete
if and only if the following statement holds.
\begin{equation*}
\left[\quad
\begin{minipage}{.7\columnwidth}
Let $x_1,x_2,\dotsc$ be a sequence in~$E$.\\
Assume $x_1,x_2,\dotsc$ has a lower bound.\\
Then $\bw_n x_n$ exists.
\end{minipage}
\right.
\end{equation*}
\end{rem}
