\documentclass[main.tex]{subfiles}
\begin{document}
\section{Ordered Abelian Groups}
\label{S:ag}
In this thesis
we do not only consider $\R$-valued measures and integrals,
but also~$E$-valued ones, 
where $E$ is an ordered Abelian group.
Since we do not expect reader to be familiar with this
particular generalisation of~$\R$,
we have collected the relevant definitions
and basic results in this appendix.
\begin{dfn}
\label{D:oag}
An \keyword{ordered Abelian group} $E$
is a set that is endowed with an Abelian group operation, $+$,
and a partial order, $\leq$, 
such that 
\begin{equation*}
x \ \leq\ y \quad\implies w+x \ \leq\ w+y
\qquad\quad(w,x,y\in E).
\end{equation*}
\end{dfn}
%
%                  EXAMPLES OF AOGs
%
\begin{exs}
\label{E:oag}
\begin{enumerate}
\item
\label{E:aog-1}
The integers, $\Z$, the rationals, $\Q$, and the reals, $\R$,
endowed with  the usual addition and order
are ordered Abelian groups.

\item
\label{E:oag_div}
Let
$\Q^{\circ}$
be the set of rational numbers $q$ with $q>0$.
Order $\Q^\circ$ by
\begin{equation*}
q \ \preccurlyeq \ r
\quad\iff\quad 
\exists n\in \N \ [\ q \cdot n = r \ ].
\end{equation*}
Then $\Q^\circ$ with 
the usual multiplication is an  ordered Abelian group.

\item
\label{E:aog-2}
Let $E_1$ and $E_2$ be Abelian ordered groups.
Then $E_1\times E_2$ with pointwise order
and pointwise group operation is an Abelian ordered group.

\item
\label{E:oag_lex}
Consider~$\R^2$ with the pointwise addition.
By point~\ref{E:aog-2},
$\R^2$ with the usual order
is an  ordered Abelian group.
Further,
$\R^2$ with the
 \emph{lexicograpgic order},
\begin{equation*}
(x_1,x_2)\ \leq\ (y_1,y_2)
\quad\iff\quad
\left[ \ \ 
\begin{alignedat}{3}
&x_1 < x_2 \quad\text{or}\\
&x_1 = x_2  \quad\text{and}\quad  y_1 \leq y_2 
\end{alignedat}
\right.,
\end{equation*}
is also an ordered Abelian group,
called the \keyword{lexicographic plane},
 $\Lex$.
\end{enumerate}
\end{exs}
%
%                  PRESERVATION /\ BY +
%
\begin{lem}
\label{L:oag-plus-preserves}
Let $E$ be an ordered Abelian group.
Let $A\subseteq E$ and $x\in E$ be given.
\begin{enumerate}
\item \label{L:oag-plus-preserves-meet}
If $A$ has an infimum,
then so has $x+A \eqdf \{ x+a\colon a\in A\}$,
and 
\begin{equation*}
\bw \,\,x+A \ =\  x+\bw A.
\end{equation*}
\item \label{L:oag-plus-preserves-join}
If $A$ has a supremum,
then so has $x+A$,
and 
\begin{equation*}
\bv \,\,x+A \ =\  x+\bv A.
\end{equation*}
\end{enumerate}
\end{lem}
\begin{proof}
It suffices to prove that the map $E\ra E$
given by $u\mapsto x + u$ is an order isomorphism.
Of course,
it is: $u\leq v\iff x+u\leq x+v$ for all $u,v\in E$.
\end{proof}
%
%                  PRESERVATION OF /\ BY -
%
\begin{lem}
\label{L:oag-minus-preserves}
Let $E$ be an ordered Abelian group,
and $A\subseteq E$.
\begin{enumerate}
\item
If $A$ has an infimum,
then $-A\eqdf \{-a\colon a\in A\}$
has a supremum,
and 
\begin{equation*}
-\bw A \,=\, \bv -\hspace{-3pt}A.
\end{equation*}
\item
If $A$ has an supremum,
then $-A$
has an infimum, and
\begin{equation*}
-\bv A \,=\, \bw -\hspace{-3pt}A.
\end{equation*}
\end{enumerate}
\end{lem}
\begin{proof}
The map $E\rightarrow E$ given by
$u\mapsto u$ is an order reversing isomorphism.
\end{proof}
%
%                  Lattice ordered Abelian groups
%
\begin{dfn}
\label{D:loag}
A \keyword{lattice ordered Abelian group}
is an ordered Abelian group~$E$,
such that the order~$\leq$ 
makes~$E$ a lattice,
i.e., each pair $x,y\in E$
has an infimum, $x\wedge y$,
and a supremum, $x\vee y$.
\end{dfn}
\begin{exs}
\label{E:loag}
\begin{enumerate}
\item
The sets
$\Z$, $\Q$, and $\R$ are lattices
under the usual order.
The supremum of two elements is their maximum,
the infimum is the minimum.

\item
More generally,
any partially ordered set~$E$
that is \emph{totally ordered},
i.e.,
\begin{equation*}
\text{either } \quad x\leq y\quad\text{ or } 
\quad y\leq x\quad \text{ for all }x,y\in E,
\end{equation*}
is a lattice.
The supremum of $x,y\in E$ is simply the maximum of~$x$ and $y$,
the infimum $x$ and $y$ is the minimum of~$x$ and~$y$.

\item
The space $\Lex$
(see Example~\ref{E:oag}\ref{E:oag_lex})
is totally ordered and hence a lattice.

\item
\label{E:loag_div}
The set~$\Q^\circ$ ordered by~$\preccurlyeq$
(see Examples~\ref{E:oag}\ref{E:oag_div})
is a lattice.

Let $m,n\in \Q^\circ$ be given.
If $m,n\in \Z$, then the supremum of $m$ and~$n$
is the least common multiple of~$m$ and~$n$,
and the infimum of~$m$ and~$n$ is the greatest common divisor
of~$m$ and~$n$.

\end{enumerate}
\end{exs}

The following result is quite suprising.
%
%                  MODULARITY EQUATION FOR LATTICE ORDERED ABELIAN GROUPS
%
\begin{lem}
\label{L:1-valuation}
Let $E$ be a lattice ordered Abelian group.
Then we have 
\begin{equation*}
a\wedge b  + a\vee b \,=\, a+b \qquad(a,b\in E).
\end{equation*}
\end{lem}
\begin{proof}
$a\vee b - a - b
= (a - a - b) \vee (b - a - b)
= (-b)\vee(-a) = -(a\wedge b)$.
\end{proof}
\begin{exs}
\begin{enumerate}
\item
Let $x,y\in \R$ be given.
Then Lemma~\ref{L:1-valuation}
gives us
\begin{equation*}
x+y\ =\ \min\{x,y\} \,+\, \max\{x,y\}.
\end{equation*}
Of course, this is trivial.

\item
Let $m,n\in \Z$ with $m,n\geq 0$ be given.
Then Lemma~\ref{L:1-valuation}
gives us
\begin{equation*}
m\cdot n \ =\ \gcd(m,n) \,\cdot\,\mathrm{lcm}(m,n),
\end{equation*}
The above equality is more difficult to derive directly.
\end{enumerate}
\end{exs}



\end{document}
