\documentclass[main.tex]{subfiles}
\begin{document}
\section{Closedness of the Completion under Operations}
\label{S:closedness}
\noindent
We have seen how we can
obtain the Lebesgue measure and the Lebesgue integral
as the (convexification of) the completion
of relatively simple valuations systems,
\begin{equation*}
\vsSA\qquad\text{and}\qquad\vsSF.
\end{equation*}
It is now time to derive some simple facts
about the completion.
In this section we will prove
statements of the following form.
\begin{enumerate}
\item
If $A,B\in\overline\SA$,
then $A\backslash B \in\ol\SA$
(see Example~\ref{E:ring-ext-2}).

\item 
If $f,g\in \ol\SF \cap \R^\R$,
then $f+g \in \ol\SF$
(see Example~\ref{E:riesz-space-ext}).
\end{enumerate}

%
%                  SIGMA-PRESERVING
%
\begin{dfn}
\label{D:sigma-preserving}
Let $P$ and $Q$ be posets.
Let $f\colon S\ra Q$ be a map, where $S\subseteq P$.
We say $f$ is \keyword{$\sigma$-preserving
with respect to~$P$}
provided that
\begin{enumerate}
\item
if 
$\bw_n{a_n}$ exists
(in~$P$)
for
$a_1 \geq a_2 \geq \dotsb$
from~$S$,
and if $\bw_n a_n\in S$,
then 
\begin{equation*}
f(\bw_n a_n) = \bw_n f(a_n);
\end{equation*}

\item
if 
$\bv_n{b_n}$ exists
for
$b_1 \leq b_2 \leq \dotsb$
from~$S$,
and if~$\bv_n b_n\in S$,
then 
\begin{equation*}
f(\bv_n b_n) = \bv_n f(b_n).
\end{equation*}
\end{enumerate}
Let $P$ and~$Q$ be posets. Let $f\colon P \ra Q$ be a map.
We say $f$ is \keyword{$\sigma$-preserving}
provided that~$f$ is $\sigma$-preserving with respect to~$P$.
\end{dfn}
\begin{rem}
If in the setting of Definition~\ref{D:sigma-preserving}
$f$ is $\sigma$-preserving
(with respect some~$S$),
then~$f$ is order preserving as well.
\end{rem}
%
%                  FIRST EXTENSION THEOREM
%
\begin{thm}
\label{T:ext1}
Let $\vs{V}{L}{\varphi}{E}$ and
 $\vs{W}{K}{\psi}{F}$ be extendible valuation systems.
Let
$A\colon V\ra W$
and 
$f\colon E\ra F$ be $\sigma$-preserving maps,
such that 
\begin{equation*}
A(L)\subseteq K
\qquad\text{and}\qquad
f \circ \varphi = \psi \circ A|L.
\end{equation*}
\begin{equation*}
\xymatrix{
V\ar[d]_{A} & 
  L\ar[r]^{\varphi} \ar[d]_{A|L} \ar @{_{(}->} [l] & 
  E\ar[d]^{f} \\
W &
  K\ar[r]^{\psi} \ar @{_{(}->} [l] & 
  F
}\end{equation*}
Then $A(\overline{L})\subseteq \overline{K}$
and $f \circ \overline\varphi = \overline\psi\circ A|\overline L$.
\end{thm}
\begin{proof}
We prove with induction that for every ordinal number~$\alpha$
we have 
\begin{equation}
\label{eq:L:ext1-1}
\begin{alignedat}{3}
A(\Pi_\alpha L)\,&\subseteq\, \overline{K}&
 \qquad f \circ \Pi_\alpha \varphi 
     \,&=\, \overline \psi \circ A | \Pi_\alpha L \\
A(\Sigma_\alpha L)\,&\subseteq\, \overline{K}&
 \qquad f \circ \Sigma_\alpha \varphi 
     \,&=\, \overline \psi \circ A | \Sigma_\alpha L. 
\end{alignedat}
\end{equation}
This is sufficient, because
$\overline{L}=\Pi_{\aleph_1} L$ and 
 $\overline \varphi = \Pi_{\aleph_1} \varphi$
(see Corollary~\ref{C:aleph1}).

\begin{enumerate}
\item We prove~\eqref{eq:L:ext1-1}
holds for $\alpha=0$.
Since $\Pi_0\varphi = \Sigma_0\varphi = \varphi$,
we need to prove that
$A(L)\subseteq \overline{K}$
and $f\circ \varphi = A| L$.
But this is valid by assumption.

\item
Let~$\alpha$ be an ordinal number
and assume~\eqref{eq:L:ext1-1}
holds for~$\alpha$;
we prove~\eqref{eq:L:ext1-1} holds for~$\alpha+1$.
We  only prove $A(\Pi_{\alpha+1} L) \subseteq \overline{K}$
and $f\circ \Pi_{\alpha+1}\varphi = \overline \psi \circ A|\Pi_{\alpha+1} L$;
the other part, $A(\Sigma_{\alpha+1} L) \subseteq \overline{K}$
and $f\circ \Sigma_{\alpha+1}\varphi 
= \overline\psi \circ A|\Sigma_{\alpha+1}L$
follows similarly.

Let $a\in \Pi_{\alpha+1} L$
be given. We need to prove that
\begin{equation}
\label{eq:L:ext1-2}
A(a)\in \overline{K}\qquad\text{and}\qquad
\overline\psi (A(a)) \,=\, f(\Pi_{\alpha+1}\varphi(a)).
\end{equation}
Recall that $\Pi_{\alpha+1}L = \Pi(\Sigma_\alpha L)$,
so write $a=\bw_n a_n$
for some 
$\Sigma_\alpha \varphi$-convergent
$a_1 \geq a_2 \geq \dotsb$
and note that $\Pi_{\alpha+1}\varphi (a) = \bw_n \Sigma_\alpha\varphi(a_n)$.
We have
\begin{alignat*}{3} 
f(\Pi_{\alpha+1}\varphi(a))
 \,&=\, f(\bw_n \Sigma_\alpha \varphi (a_n))
   && \\
 \,&=\, \bw_n f(\Sigma_\alpha\varphi(a_n))\qquad
   && \text{since $f$ is $\sigma$-preserving} \\
 \,&=\, \bw_n \overline{\psi} (A( a_n )) 
   && \text{since \eqref{eq:L:ext1-1} holds for~$\alpha$.} 
\end{alignat*}
So we see that $A(a_1)\geq A(a_2)\geq \dotsb$
is $\overline{\psi}$-convergent.
Since $\vs{W}{\overline{K}}{\overline{\psi}}{F}$ is complete,
this implies $\bw_n A(a_n) \in \overline{K}$
and $\bw_n\overline{\psi}(A(a_n))=\overline{\psi}(\bw_n A(a_n))$.
Because $A$ is $\sigma$-preserving,
we have $\bw_n A(a_n) = A(a)$.
Hence $A(a)\in \overline{K}$
and 
\begin{alignat*}{3}
f(\Pi_{\alpha+1}\varphi(a))
 \,&=\,  \bw_n \overline{\psi}(A(a_n)) \\
 \,&=\,  \overline{\psi} (\bw_n A(a_n)) \\
 \,&=\,  \overline{\psi} (A(a)).
\end{alignat*}
So we have proven Statement~\eqref{eq:L:ext1-2}.

\item
Let $\lambda$ be a limit ordinal,
and assume that \eqref{eq:L:ext1-1}
holds for all~$\alpha<\lambda$;
we prove that \eqref{eq:L:ext1-1} holds for~$\lambda$.
Since  $\Pi_\lambda \varphi = \Sigma_\lambda \varphi$,
we must prove that
\begin{equation}
A(\Pi_\lambda L)\subseteq \overline K
\qquad\text{and}\qquad
f\circ \Pi_\lambda \varphi \,=\, \overline\psi \circ A | \Pi_\lambda L.
\end{equation}
Let $a\in \Pi_\lambda L$ be given
in order to prove $A(a)\in \overline{K}$
and $\overline\psi(A(a))= f(\Pi_\lambda\varphi(a))$.
Recall that $\Pi_\lambda L = \bigcup_{\alpha < \lambda} \Pi_\alpha L$,
and $\Pi_\lambda \varphi \,|\,\Pi_\alpha L = \Pi_\alpha \varphi$
for all~$\alpha < \lambda$.
So choose $\alpha < \lambda$ such that
$a\in \Pi_\alpha L$.
Since \eqref{eq:L:ext1-1} holds for~$\alpha$,
we know that 
\begin{equation*}
A(\Pi_\alpha L) \subseteq \overline K
\qquad\text{and}\qquad
f\circ \Pi_\alpha \varphi = \overline\psi\circ A | \Pi_\alpha L.
\end{equation*}
Hence $A(a) \in A(\Pi_\alpha L) \subseteq \overline K$ and 
$f(\Pi_\lambda \varphi(a)) = f(\Pi_\alpha \varphi(a)) =
\overline\psi(A(a))$.\qedhere
\end{enumerate}
\end{proof}
%
%                  EXAMPLE ON RINGS A with X in A
%
\begin{ex}
\label{E:ring-ext-1}
Let $\mathcal{A}$
be a ring of subsets of~$X$.
Let $\mu\colon \mathcal A \ra \R$
be a positive and additive map.
Recall that~$\vs{\wp X}{\mathcal{A}}{\mu}{\R}$
is a valuation system (see Example~\ref{E:ring-system}).
Assume that~$\vs{\wp X}{\mathcal{A}}{\mu}{\R}$
is extendible.

We would like to prove that~$\overline{\mathcal{A}}$
is also a ring of subsets of~$X$ (as is~$\mathcal{A}$).
For the moment,
we will prove this under the assumption that~$X\in \mathcal{A}$,
see Example~\ref{E:ring-ext-2}.

To prove that~$\overline{\mathcal{A}}$
is a ring,
we need to show that $Z\backslash Y \in \overline{\mathcal{A}}$
for all~$Z,Y\in\overline{\mathcal{A}}$.
Note that $Z\backslash Y = (X\backslash Y)\cap Z$.
So it suffices to show that
$X\backslash Y \in\overline{\mathcal{A}}$
for all $Y\in\overline{\mathcal{A}}$.

Consider the order \emph{reversing} maps $A\colon \wp X \ra \wp X$
and $f\colon \R \ra \R$ given by
\begin{alignat*}{3}
A( Y ) &= X\backslash Y
\qquad&&(Y\subseteq X) \\
f(x) &= \mu(X) - x
\qquad&&(x\in \R).
\end{alignat*}
In order to apply Theorem~\ref{T:ext1}
to these maps,
let us rebaptise them as order preserving
maps $A\colon \wp X \ra (\wp X)^\mathrm{op}$
and $f\colon \R \ra \R^\mathrm{op}$
(see Example~\ref{E-val-opposite}).

We have the following situation.
\begin{equation*}
\xymatrix{
\wp X\ar[d]_{A} & 
  \mathcal A \ar[r]^{\mu} \ar[d]_{A|L} \ar @{_{(}->} [l] & 
  \R \ar[d]^{f} \\
(\wp X)^\mathrm{op} &
  {\mathcal{A}}^\mathrm{op} \ar[r]^{\mu} \ar @{_{(}->} [l] & 
  \R^\mathrm{op}
}\end{equation*}
We leave it to the reader to verify that
$\vs{(\wp X)^\mathrm{op}}{{\mathcal A}^\mathrm{op}}\mu{\R^\mathrm{op}}$
is again valuation system which is extendibe.
We have
 $A(\mathcal A)\subseteq \mathcal A$,
because
$X\backslash Y \in \mathcal{A}$
for all~$Y\in\mathcal{A}$
since $\mathcal{A}$ is a ring containing~$X$.
Further,
since $\mu$ is additive, we have
\begin{equation*}
\mu(A(Y))\,=\, \mu(X\backslash Y)
\,=\, \mu(X) - \mu(Y)
\,=\, f(\mu(Y)).
\end{equation*}
So $\mu\circ A$ and $f\circ \mu$ 
are identical on~$\mathcal A$.
Note that
\begin{equation*}
\textstyle{X\backslash\bigcup_n A_n \,=\, \bigcap_n X\backslash A_n}
\qquad\text{and}\qquad
\mu(X)-\bv_n x_n \,=\, \bw_n \,\mu(X) - x_n
\end{equation*}
where $Y_1 \subseteq Y_2 \subseteq \dotsb$
are from~$\wp X$
and $x_1 \leq x_2 \leq \dotsb$ 
is a bounded sequence in~$\R$.
So $A$ and $f$ are $\sigma$-preserving
(see Definition~\ref{D:sigma-preserving}).

Hence by Theorem~\ref{T:ext1},
we get $A(\overline{\mathcal{A}})\subseteq \overline{\mathcal{A}}$
and secondly
  $f\circ\overline\mu=\overline \mu \circ A$ on $\overline{\mathcal{A}}$.

From the first fact
we get that $X\backslash Y \in \overline{Y}$
for all $Y\in \mathcal{A}$,
and hence~$\overline{\mathcal{A}}$ is a ring.

From the second fact,
we get $\overline\mu (X\backslash Y)
= \overline\mu(X) - \overline\mu(Y)$
for all $Y\in \overline{\mathcal{A}}$.
From this,
one might say,
we see that~$\overline\mu$ is additive.
However, we already knew this
as $\overline\mu$ is modular
(and $\mu(\varnothing)=0$, see Definition~\ref{P:hier}).
\end{ex}
%
%                  SECOND EXTENSION THEOREM
%
\begin{thm}
\label{T:ext2}
Let $\vs{V}{L}{\varphi}{E}$ and
 $\vs{W}{K}{\psi}{F}$ be extendible valuation systems.
Let
$A\colon V\ra W$
be a $\sigma$-preserving map
such that $A(L)\subset K$.
Let $f\colon E\ra F$ be a $\sigma$-preserving group-homomorphism
such that
\begin{equation}
\label{eq:T:ext2-as}
\ld{\overline{\psi}}(A(c),A(d))\ \leq\  f(\ld{\overline{\varphi}}(c,d))
\end{equation}
for all $c,d\in V$ with $A(c),A(d)\in \overline K$.
\begin{equation*}
\xymatrix{
V\ar[d]_{A} & 
  L\ar[r]^{\varphi} \ar[d]_{A|L} \ar @{_{(}->}[l] & 
  E \ar[d]^f\\
W &
  K\ar[r]^{\psi} \ar @{_{(}->}[l] & 
  F
}\end{equation*}
Then $A(\overline{L})\subseteq \overline{K}$.
\end{thm}
\begin{proof}
With induction,
we prove that for every ordinal number~$\alpha$,
we have 
\begin{equation*}
A(\Pi_\alpha L)\subseteq \overline K
\qquad\text{and}\qquad
A(\Sigma_\alpha L)\subseteq \overline K.
\end{equation*}
This is sufficient, because $\overline L = \Pi_{\aleph_1} L$.

As one can  see,
such a proof might be quite similar to the proof of Theorem~\ref{T:ext1}.
Therefore,
we leave the details to the reader
and only prove the following statement.
\begin{equation}
\label{eq:T:ext2-1}
A(\Sigma_\alpha L) \subseteq \overline K
\quad\implies\quad
A(\Pi_{\alpha+1}L)\subseteq\overline K.
\end{equation}

Assume $A(\Sigma_\alpha L)\subseteq \overline{K}$
and let $a\in \Pi_{\alpha+1} L$ be given;
we must prove $A(a)\in \overline K$.
Write $a=\bw_n a_n$ for some $\Sigma_{\alpha}\varphi$-convergent
sequence
$a_1 \geq a_2 \geq \dotsb$
in~$\Sigma_\alpha L$.
Because we have assumed $A(\Sigma_\alpha L)\subseteq \overline{K}$,
we know that $A(a_i) \in \overline K$.
To prove $A(a)\in\overline K$,
it suffices to show that $A(a_1) \geq A(a_2) \geq \dotsb$
is $\overline{\psi}$-convergent.
Indeed, then
\begin{equation*}
A(a) \,\equiv\, A(\bw_n a_n) \,=\, \bw_n A(a_n)\,\in\, \overline{K},
\end{equation*}
because $A$ is $\sigma$-preserving and 
$\vs{W}{\overline{K}}{\overline{\psi}}{F}$
is complete.

To prove that the sequence $A(a_1)\geq A(a_2) \geq \dotsb$
is $\overline{\psi}$-convergent,
we must show that $\bw_n \overline \psi(A(a_n))$ exists.
Note that by Inequality~\eqref{eq:T:ext2-as},
we have
\begin{alignat*}{3}
\overline{\psi}(A(a_{n+1}))
\,-\, \overline{\psi}(A(a_n))
\ &=\ 
\ld{\overline\psi}(A(a_{n+1}),A(a_n))\\
\ &\leq\ 
f(\ld{\overline\varphi}(a_{n+1},a_n))
\ =\ 
f(\overline\varphi(a_{n+1},a_n)) 
\,-\, f(\overline\varphi(a_n)).
\end{alignat*}
So since~$F$ is $R$-complete (see Definition~\ref{D:R-complete}),
in order to show that $\bw_n\overline{\psi}(A(a_n))$
exists,
it suffices to prove that $\bw_n f(\overline{\varphi}(a_n))$
exists. 
For this
we need to prove that $\bw_n\overline\varphi(a_n)$ exists
(as  $f$ is $\sigma$-preserving).
That is,
we must show that $a_1 \geq a_2 \geq \dotsb$ is 
$\overline{\varphi}$-convergent.
Of course,
this follows quickly from the fact that $a_1 \geq a_2 \geq \dotsb$
is $\Sigma_{\alpha}\varphi$-convergent.
We have proven Statement~\eqref{eq:T:ext2-1}.
\end{proof}
%
%                  RESTRICTION PROPOSITION         
%
\begin{prop}
\label{P:restriction}
Let $\vs{V}{L}\varphi{E}$ be an extendible valuation system.
Note that its completion is denoted by
$\vs{V}{\overline L}{\overline \varphi}{E}$.
Given $\ell \leq u$ from~$L$,
consider
\begin{equation*}
\vs{[\ell,u]}{L\cap{[\ell, u]}}{\varphi|[\ell,u]}{E};
\end{equation*}
it is an extendible valuation system.
Note that its completion is denoted by
\begin{equation*}
\vs{[\ell,u]}{\overline{L\cap[\ell,u]}}{\overline{\varphi|[\ell,u]}}{E}.
\end{equation*}
We have $\overline{L\cap [\ell, u]} = \overline{L}\cap [\ell,u]$.
Moreover,
 $\overline{\varphi}$ and $\overline{\varphi|[\ell,u]}$
are identical on $\overline{L\cap [\ell,u]}$.
\end{prop}
\begin{proof}
One can easily see that $\overline\varphi | [\ell,u]$
extends $\varphi|[\ell,u]$
and that the valuation system
\begin{equation*}
\vs{[\ell,u]}{\overline{L}\cap [\ell,u]}{\overline\varphi | [\ell, u]}{E}
\end{equation*}
is complete.
Hence $\varphi|[\ell,u]$ is extendible,
and $\overline\varphi | [\ell,u]$ extends
$\overline{\varphi|[\ell,u]}$ (see Proposition~\ref{P:comp-monotonous}).
In particular,
$\overline{L\cap[\ell,u]}\subseteq \overline{L}\cap[\ell,u]$
and $\overline{\varphi}$ and $\overline{\varphi|[\ell,u]}$
are identical on~$\overline{L\cap[\ell,u]}$.
It remains to be shown that 
\begin{equation}
\label{eq:P:restriction}
\overline{L}\cap[\ell,u]\,\subseteq\,\overline{L\cap[\ell,u]}.
\end{equation}
To this end,
consider the map~$\varrho\colon V\ra [\ell, u]$
given by $\varrho(x)=\ell\vee (x\wedge u)$.
Note that $\varrho(x)=x$ for all~$x\in [\ell, u]$,
and thus $\varrho(\overline{L})=\overline{L}\cap[\ell,u]$.
So in order to prove~\eqref{eq:P:restriction},
we must show that
$\varrho(\overline{L})\subseteq\overline{L\cap[\ell,u]}$.
To do this, we apply Theorem~\ref{T:ext2}.
\begin{equation*}
\xymatrix{
V\ar[d]_{\varrho} &&
  L\ar[rr]^{\varphi} \ar[d]_{\varrho|L} \ar @{_{(}->}[ll] & &
  E\ar[d]^{1_E} \\
[\ell,u] &&
  L\cap[\ell,u]\ar[rr]^{\varphi|[\ell,u]} \ar @{_{(}->}[ll] &&
  E
}\end{equation*}
We must verify that $\varrho$ is
$\sigma$-preserving and that
\begin{equation}
\label{eq:P:restriction-1}
\ld{\overline{\varphi|[\ell,u]}}(\varrho(c),\varrho(d))
\ \leq\ 
\ld{\overline{\varphi}}(c,d)
\end{equation}
for all $c,d\in V$ with $\varrho(c),\varrho(d)\in \overline {L\cap [\ell,u]}$.
One can easily see that~$\varrho$ is $\sigma$-preserving,
because $V$ is $\sigma$-distributive
(see Definition~\ref{D:sigma-distributive}).
Concerning Inequality~\eqref{eq:P:restriction-1},
note that for $c,d\in V$
with $\varrho(c),\varrho(d)\in \overline {L\cap [\ell,u]}$
we have
\begin{alignat*}{3}
\ld{\overline\varphi|[\ell,u]}(\varrho(c),\varrho(d)) 
\ &=\ \ld{\overline\varphi}(\varrho(c),\varrho(d)) 
   && \text{since $\overline\varphi|\,\overline{L\cap[\ell,u]}
                           =\overline{\varphi| [\ell,u]}$}\\
\ &=\  \ld{\overline\varphi}(\,\ell\vee(c\wedge u),\,\ell\vee(d\wedge u)\,) 
   \quad&& \text{by definition of~$\varrho$} \\
\ &\leq\  \ld{\overline\varphi}(\,c\wedge u,\,d\wedge u\,) 
   && \text{by Lemma~\ref{L:wv-unif}} \\
\ &\leq\  \ld{\overline\varphi}(c,d)
   && \text{by Lemma~\ref{L:wv-unif}}.
\end{alignat*}
Hence Theorem~\ref{T:ext2} is applicable,
and we obtain Inequality~\eqref{eq:P:restriction}.
\end{proof}
%
%                  EXAMPLE ON RINGS
%
\begin{ex}
\label{E:ring-ext-2}
Let $\mathcal{A}$
be a ring of subsets of~$X$.
Let $\mu\colon \mathcal A \ra \R$
be a positive and additive map.
Recall that~$\vs{\wp X}{\mathcal{A}}{\mu}{\R}$
is a valuation system (see Example~\ref{E:ring-system}).
Assume that~$\vs{\wp X}{\mathcal{A}}{\mu}{\R}$
is extendible
(see Definition~\ref{D:extendible}).

We prove that~$\overline{\mathcal{A}}$ is a ring.
(In Example~\ref{E:ring-ext-1},
we saw that this
is the case if~$X\in \mathcal{A}$.)

Let $Y,Z\in\overline{\mathcal{A}}$ be given.
To prove that~$\overline{\mathcal{A}}$
is a ring,
we must show that
\begin{equation*}
Y\backslash Z\in\overline{\mathcal{A}}.
\end{equation*}

We restrict our attention to the interval $I\eqdf [\,\varnothing,\, Y\cup Z\,]$.
Note that $\mathcal{A}\cap I$ is a ring of subset of $Y\cup Z$
with $Y\cup Z\in\mathcal{A}\cap I$.
So by Example~\ref{E:ring-ext-1},
we know that~$\overline{\mathcal{A}\cap I}$ is a ring.
Note that 
$Y,Z\in\overline{\mathcal{A}\cap I}$
because $\overline{\mathcal A} \cap I 
= \overline{ \mathcal A \cap I}$
by Proposition~\ref{P:restriction}.
So we get $Y\backslash Z\in \overline{\mathcal{A}\cap I}$,
and thus $Y\backslash Z\in \overline{\mathcal{A}}\cap I$
by Proposition~\ref{P:restriction}.

Hence~$\overline{\mathcal{A}}$ is a ring of subsets of~$X$.
\end{ex}

%
%                  THIRD EXTENSION THEOREM
%
\begin{thm}
\label{T:ext3}
Let $\vs{V}{L}{\varphi}{E}$
and
 $\vs{W}{K}{\psi}{F}$ be extendible valuation systems.
Let $R$ be a sublattice of~$V$ with 
$L\subseteq R$.
Let
$f\colon E\ra F$ be a $\sigma$-preserving map,
and let $A\colon R\ra W$ be $\sigma$-preserving
with respect to~$V$.
Assume that $A(L)\subseteq K$
and  $f \circ \varphi = \psi \circ A|L$.
\begin{equation*}
\xymatrix{
V & 
  R\ar[d]_{A} \ar @{_{(}->}[l]& 
  L\ar[r]^{\varphi} \ar[d]_{A|L} \ar @{_{(}->}[l] & 
  E\ar[d]^{f} \\
& W &
  K\ar[r]^{\psi} \ar @{_{(}->}[l] & 
  F
}\end{equation*}
Assume that
$R$ is convex in~$V$,
and that
for every $c\in R$,
there are $\varphi$-convergent sequences $a_1 \leq a_2 \leq \dotsb$ 
and $b_1 \geq b_2 \geq \dotsb$
such that  $\bw_n a_n \leq c \leq \bv_n b_n$.

Then $A(\overline L\cap R)\subseteq \overline{K}$
and $f \circ \overline\varphi = \overline\psi\circ A$ on $\overline L \cap R$.
\end{thm}
\begin{proof}
Let us first prove the following special case.
\begin{equation}
\label{eq:T:ext3-claim-1}
\left[\quad
\begin{minipage}{.7\columnwidth}
Let $c\in \overline{L}\cap R$
with $\ell\leq c\leq u$ for some~$\ell,u\in L$.
Then 
\begin{equation*}
A(c)\in\overline{K}
\qquad\text{and}\qquad
f(\overline\varphi(c))= \overline\psi (A(c)).
\end{equation*}
\end{minipage}
\right.
\end{equation}
Let $c \in \overline{L}\cap R$
with $\ell \leq c \leq u$ for some $\ell,u\in L$ be given.
Then clearly $c\in [\ell, u]$.
Further, $[\ell,u]\subseteq R$ since~$R$ is convex
and $\ell,u\in R$ (as $L\subseteq R$).
So we have:
\begin{equation*}
\xymatrix{
[\ell,u]\ar[d]_{A|[\ell,u]} & 
  L\cap[\ell,u]\ar[r]^{\varphi|[\ell,u]} 
               \ar[d]_{A\,|\,L\cap[\ell,u]} \ar @{_{(}->}[l] & 
  E\ar[d]^{f} \\
W &
  K\ar[r]^{\psi} \ar @{_{(}->}[l] & 
  F
}\end{equation*}
Moreover, by Proposition~\ref{P:restriction},
we know that $c\in \overline{L\cap[\ell,u]}$
and $\overline{\varphi}(c) = \overline{\varphi|[\ell,u]}(c)$.
Hence Theorem~\ref{T:ext1}
yields $A(c)\in\overline{K}$
and $\overline{\psi}(A(c))
= f(\overline{\varphi|[\ell,u]}(c))$.
But then
$\overline{\psi}(A(c))
= f(\overline{\varphi}(c))$.
This proves Statement~\eqref{eq:T:ext3-claim-1}.

\vspace{.5em}
We proceed by proving another special case.
\begin{equation}
\label{eq:T:ext3-claim-2}
\left[\quad
\begin{minipage}{.7\columnwidth}
Let $c\in \overline{L}\cap R$
and suppose $c\geq \ell$ for some~$\ell\in L$.
Then 
\begin{equation*}
A(c)\in\overline{K}
\qquad\text{and}\qquad
f(\overline\varphi(c))= \overline\psi (A(c)).
\end{equation*}
\end{minipage}
\right.
\end{equation}
Let $c\in\overline{L}\cap R$ with $c\geq \ell$
for some~$\ell$ be given.
Pick $\varphi$-convergent $u_1 \leq u_2 \leq \dotsb$
such that $c \leq \bv_n u_n$.
Since $u_1 \geq u_2 \geq \dotsb$ is $\varphi$-convergent
and $c\in\overline{L}$,
we know that $c\wedge u_1 \leq c\wedge u_2 \leq \dotsb$
is $\overline{\varphi}$-convergent (see Proposition~\ref{P:R-main}).
Since $\vs{V}{\overline{L}}{\overline\varphi}{E}$ is complete,
this implies $\overline\varphi(c)
=\overline\varphi(\bv_n c\wedge u_n)
=\bv_n\overline\varphi(c\wedge u_n)$.
We get:
\begin{alignat*}{3}
f(\overline\varphi(c))
\,&=\, f(\bv_n\overline\varphi(c\wedge u_n)) &&  \\
\,&=\, \bv_n f(\overline\varphi(c\wedge u_n))
   \qquad && \text{since $f$ is $\sigma$-preserving}
\shortintertext{Note that $\ell\leq c\wedge u_n \leq u_n$.
So by \eqref{eq:T:ext3-claim-1}, we get $A(c\wedge u_n)\in\overline K$ and:}
f(\overline\varphi(c))
\,&=\, \bv_n \overline{\psi} (A (c\wedge u_n))  &&
\shortintertext{From this we 
see $A(c\wedge u_1) \leq A(c\wedge u_2)\leq \dotsb$
is $\overline\psi$-convergent.
Since $\vs{W}{\overline K}{\overline \psi}{F}$
is complete,
we get $f(\overline\varphi(c)=\bv_n A(c\wedge u_n) \in \overline K$
and}
f(\overline\varphi(c))
 \,&=\,  \overline\psi (\bv_n A(c\wedge u_n)) \\
   &=\,  \overline\psi (A(\bv_n c\wedge u_n) )
      && \text{since $A$ is $\sigma$-preserving} \\
   &=\,  \overline\psi (A(c)).
\end{alignat*}
This completes the proof of Statement~\eqref{eq:T:ext3-claim-2}.

\vspace{.5em}
We are now ready to give the proof of the general case.
Let $c\in R\cap \overline L$ be given.
We need to prove that $A(c)\in\overline K$
and $f(\overline{\varphi}(c))=\overline{\psi}(A(c))$.
Pick $\varphi$-convergent $\ell_1 \geq \ell_2 \geq \dotsb$
such that $\bw_n \ell_n \leq c$.
Since $\ell_1 \geq \ell_2 \geq \dotsb$
is $\varphi$-convergent and $c\in \overline L$,
we know that $\ell_1 \vee c \geq \ell_2 \vee c\geq \dotsb$ 
is $\overline\varphi$-convergent.
Since $\vs{V}{\overline L}{\overline\varphi}{E}$
is complete,
this implies that $\overline\varphi(c)
=\overline\varphi(\bw_n \ell_n \vee c)
=\bw_n\overline\varphi(\ell_n \vee c)$.
We get:
\begin{alignat*}{3}
f(\overline\varphi(c))
\,&=\, f(\bw_n\overline\varphi(\ell_n \vee c)) &&  \\
\,&=\, \bw_n f(\overline\varphi(\ell_n\vee c))
   \qquad && \text{since $f$ is $\sigma$-preserving}
\shortintertext{Note that $\ell_n\leq \ell_n \vee c$.
Further,
since $R$ is a sublattice of~$V$,
and $c\in R$, $\ell_n\in L\subseteq R$,
we get $\ell_n \vee c\in R$.
So by \eqref{eq:T:ext3-claim-2}, we have $A(\ell_n\vee c)\in\overline K$ and:}
f(\overline\varphi(c))
\,&=\, \bw_n \overline{\psi} (A (\ell_n \vee c))  &&
\shortintertext{From this we 
see $A(\ell_1 \vee c) \geq A(\ell_2 \vee c)\geq \dotsb$
is $\overline\psi$-convergent.
Since $\vs{W}{\overline K}{\overline \psi}{F}$
is complete,
we get $f(\overline\varphi(c)=\bw_n A(\ell_n \vee c) \in \overline K$
and}
f(\overline\varphi(c))
 \,&=\,  \overline\psi (\bw_n A(\ell_n \vee c)) \\
   &=\,  \overline\psi (A(\bw_n\ell_n \vee c) )
      && \text{since $A$ is $\sigma$-preserving} \\
   &=\,  \overline\psi (A(c)).
\end{alignat*}
We are done.
\end{proof}
%
%                  CLOSURE UNDER ADDITION
%
\begin{prop}
\label{P:subgroup}
Let $\vs{V}{L}{\varphi}{E}$ be an extendible valuation system.
Let $R$ be a sublattice of~$V$
endowed with a group structure.
Assume $L$ is a subgroup of~$R$ and that $\varphi$ is a group homomorphism
(recall that~$E$ is an ordered Abelian \emph{group}).

Further, assume 
that $R$ is convex and that for every $c\in R$,
there are $\varphi$-convergent sequences
$a_1 \geq a_2 \geq \dotsb$ and
$b_1 \leq b_2 \leq \dotsb$ 
such that $\bw_n a_n \leq c\leq \bv_n b_n$.

Then $\overline L \cap R$ is a subgroup of~$R$,
and $\overline \varphi | R$ is a group homomorphism.
\end{prop}
\begin{proof}
In order to show that $\overline L\cap R$ is a subgroup of~$R$,
we must prove the following.
\begin{enumerate}
\item \label{tp:P:subgroup-1}
If $a,b\in \overline{L}\cap R$, then $a+b\in \overline L$.

\item \label{tp:P:subgroup-2}
If $a \in \overline{L}\cap R$, then $-a\in \overline{L}$.
\end{enumerate}
We only give a proof for~\ref{tp:P:subgroup-1}.
It will then be clear how to prove~\ref{tp:P:subgroup-2}.

We aim to apply Theorem~\ref{T:ext3}.
To this end,
the reader can easily verify that
$\vs{V\times V}{L\times L}{\varphi\times\varphi}{E\times E}$
is an extendible valuation system;
that its completion is
$\vs{V\times V}{\overline L\times\overline L}
{\overline\varphi \times\overline\varphi}{E\times E}$;
that $R\times R$ is a convex sublattice of~$V\times V$;
that the assignment $(c,d)\mapsto c+d$
yields a $\sigma$-preserving map $A\colon R\times R\ra V$
 with respect to~$V\times V$;
that the map $f\colon E\times E\ra E$ given by
$f(x,y)=x+y$ is $\sigma$-preseving.

Further, note that $A(L\times L)\subseteq L$
because~$L$ is a subgroup of~$R$.
Note that for all $c_1,c_2\in R\times R$
there are $\varphi$-convergent  $\ell^i_1 \geq \ell^i_2 \geq \dotsb$
and $u^i_1 \leq u^i_2 \leq \dotsb$
such that $\bw_n \ell^i_n \leq c_i \leq \bv_n u^i_n$,
and thus $\bw_n (\ell^1_n,\ell^2_n) \leq (c_1,c_2) \leq \bv_n (u^1_n, u^2_n)$,
where $(\ell^1_1,\ell^2_1) \geq (\ell^1_2,\ell^2_2)\geq \dotsb$
and $(u^1_1,u^2_1)\leq (u^1_2,u^2_2)\leq \dotsb$
are $\varphi\times\varphi$-convergent.
Finally,
note that $f\circ (\varphi\times \varphi) = \varphi \circ A|(L\times L)$,
because $\varphi$ is a group homomorphism.
\begin{equation*}
\xymatrix{
V\times V & 
  R\times R\ar[d]_{+} \ar @{_{(}->}[l]& 
  L\times L\ar[r]^{\varphi\times\varphi} 
     \ar[d]_{+} \ar @{_{(}->}[l] & 
  E\times E\ar[d]^{+} \\
& V &
  L\ar[r]^{\varphi} \ar @{_{(}->}[l] & 
  E
}\end{equation*}

So we are in a position to apply Theorem~\ref{T:ext3}.
It gives us that $A(\overline L\times \overline L \cap R\times R)\subseteq L$ 
and 
$f\circ (\overline \varphi \times \overline \varphi)
= \overline \varphi \circ A$ on
 $\overline {L}\times \overline{L} \cap R\times R$.
In other words,
if $c,d\in \overline{L}\cap R$, then $c+d \in \overline{L}$,
and $\overline\varphi(c+d)=\overline\varphi(c)+\overline\varphi(d)$.
Hence we have proven statement~\ref{tp:P:subgroup-1},
and at the same time we have shown that~$\ol\varphi$ is a group homomorphism.
\end{proof}
%
%                  EXAMPLE ON RIESZ SPACE OF FUNCTIONS
%
\begin{ex}
\label{E:riesz-space-ext}
Let $X$ be a set.
Let $F$ be a Riesz space of functions on~$X$.
Let $\varphi\colon F\ra R$ be a positive linear map.
Recall that $\vs{[-\infty,\infty]^X}F{\varphi}\R$
is a valuation system.
Assume that $\varphi$ is extendible.

We would like to prove that $\overline F$ is a Riesz space of functions
and $\overline\varphi$ is linear.
However, since addition is only defined on $R\eqdf\R^X$,
we will instead show that
$\overline F\cap R$ is a Riesz space
of functions and that $\overline\varphi|R$ is linear.
Moreover,
we assume
that for every~$f\in\overline F \cap R$
there are $\varphi$-convergent sequences
 $\ell_1 \geq \ell_2 \geq \dotsb$
and $u_1 \leq u_2 \leq \dotsb$
such that $\bw_n \ell_n \leq f \leq \bv_n u_n$.

To prove that $\overline F \cap R$ is a Riesz space,
we must show that
\begin{enumerate}
\item
 $f+g\in \overline F$ 
for all $f,g\in \overline F \cap R$, and

\item
$\lambda \cdot f\in \overline F$
for all $\lambda\in \R$ and $f\in \overline F\cap R$.
\end{enumerate}
We only prove the first statement;
we leave it to the reader to prove the second.

Of course,
it suffices to establish that $\overline F\cap R$
is a subgroup of~$R$.
To do this,
we apply Proposition~\ref{P:subgroup}.
Indeed, one can easily
see that all the prerequisites are met.
To name a few: one sees that
$R$ is a sublattice of~$V$,
that $F$ is a subgroup of~$R$ (since $F$ is a Riesz space of functions),
that $\varphi$ is a group homomorphism (since $\varphi$ is linear),
and that $R$ is convex (since $\R$ is convex in $[-\infty,\infty]$).

Proposition~\ref{P:subgroup} not only gives us that $\overline F \cap R$
is a subgroup of~$R$, but
also that~$\varphi|R$ is a group homomorphism.
We leave it to the reader to prove that
$\overline\varphi|R$ is homogeneous,
i.e., $\overline\varphi(\lambda\cdot f) = \lambda \cdot \overline\varphi(f)$
for all~$f\in \overline F \cap R$ and $\lambda \in \R$.

Hence $\overline F\cap R$
is a Riesz space of functions, and $\overline \varphi |R$ is linear.
\end{ex}

\end{document}
