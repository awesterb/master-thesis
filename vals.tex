\documentclass[main.tex]{subfiles}
\begin{document}
\section{Valuations}
\noindent
%
%          DEFINITION OF MODULAR MAPS
%
\begin{dfn}
Let $L$ be a lattice, $E$ an ordered Abelian group,
and $\varphi\colon L \ra E$.
\begin{enumerate}
\item
$\varphi$ is \keyword{modular} provided that
\begin{equation*}
\varphi(a\wedge b) + \varphi(a \vee b)
\ =\ 
\varphi(a) + \varphi(b)
\qquad(a,b\in L).
\end{equation*}

\item
$\varphi$ is a \keyword{valuation}
provided that $\varphi$ is modular and order preserving.
\end{enumerate}
\end{dfn}

\begin{ex}
Let $\mathcal{F}$ be the set of finite subsets of~$\mathcal{N}$,
and for each $A\in \mathcal{F}$,
let $\#(A)$ be the number of elements of~$A$.
Then
\begin{equation*}
\#(A\cap B) + \#(A\cup B) \,=\, \#(A) + \#(B)
\qquad(A,B\in\mathcal{F})
\end{equation*}
So the map $\#\colon \mathcal{F}\rightarrow \N$
given by $A\mapsto \#(A)$ is modular.

Since $\#$ is also order preserving,
$\#$ is a valuation.
\end{ex}

\begin{ex}
Let $C$ be a chain;
i.e. a totally ordered set.
Then $C$ is a lattice with
\begin{equation*}
a\wedge b = \min\{a,b\},
 \qquad 
a\vee b = \max\{a,b\}.
\end{equation*}
\emph{Then any map $f\colon C\rightarrow E$
to an ordered Abelian group is modular}
Indeed,
let $a,b\in C$ be given.
Then either $a\leq b$ or $b\leq a$.
We need to prove that 
\begin{equation}
\label{eq:ex:chain}
f(a\wedge b) + f(a \vee b) \,=\, f(a) + f(b).
\end{equation}
Suppose $a\leq b$. Then $a\wedge b = a$ and $a \vee b= b$.
Hence Equation~\eqref{eq:ex:chain} holds.  
If $b\leq a$, then $a\wedge b = b$ and $a \vee b = b$,
so again Equation~\eqref{eq:ex:chain} holds.
\end{ex}

\begin{ex}
Let~$R$ be a \emph{lattice ordered} Abelian group.
Then we have the equality
\begin{equation*}
a\wedge b  + a\vee b \,=\, a+b \qquad(a,b\in R).
\end{equation*}
So \emph{any group homomorpism $f\colon R\rightarrow E$
to some ordered Abelian group
is modular.}
Indeed,
for any $a,b\in R$, we have
\begin{equation*}
f(a\wedge b) + f(a\vee b)
\,=\,f(a\wedge b + a\vee b) 
\,=\,f(a+ b)
\,=\, f(a) + f(b).
\end{equation*}
Similarly,
\emph{if $f\colon L\ra R$ is a lattice homomorphism on 
some lattice~$L$,
then $f$ is modular.}
Indeed, let $a,b\in L$, then 
\begin{equation*}
f(a\wedge b) + f(a \vee b)
\,=\,f(a)\wedge f(b) + f(a) \vee f(b)
\,=\, f(a) + f(b).
\end{equation*}
\end{ex}

\begin{ex}
Let~$X$ be a set and let~$\mathcal{A}$ be a ring of subsets of~$X$.
That is,
\begin{equation*}
A\cap B,\qquad\qquad A\cup B,\qquad\qquad A\backslash B
\end{equation*}
are in~$\mathcal{A}$ for all $A,B\in\mathcal{A}$.
Then clearly~$\mathcal{A}$ is a lattice.

Let $E$ be an ordered Abelian group
and let~$\mu\colon \mathcal{A}\rightarrow \R$ be a map.
Recall that~$\mu$ is \keyword{additive} if $\mu(A) + \mu(B) = \mu(A\cup B)$
for all $A,B\in\mathcal{A}$ with $A\cap B=\varnothing$.

\emph{If~$\mu$ is additive,
then $\mu$ is modular.}
Indeed,
let $A,B\in \mathcal{A}$ be given. We need to prove that
$\mu(A) + \mu(B) =\mu(A\cap B) + \mu(A\cup B)$
assuming~$\mu$ is additive.
We have
\begin{alignat*}{6}
\mu(A) + \mu(B) \,
  & =\, \mu(A\cap B \ \cup\ A\backslash B) + \mu(B) \\ 
  & =\, \mu(A\cap B) + \mu(A\backslash B)  + \mu(B)\qquad
    && \text{since } A\cap B \ \cap\ A\backslash B = \varnothing \\ 
  & =\, \mu(A\cap B) + \mu(A\backslash B \ \cup\ B ) 
    && \text{since } A\backslash B\ \cap\ B = \varnothing \\
  & =\, \mu(A\cap B) + \mu(A\cup B).
\end{alignat*}

Conversely,
\emph{if $\mu$ is modular
and $\mu(\varnothing) = 0$,
then $\mu$ is additive.}
Indeed,
assume~$\mu$ is modular and $\mu(\varnothing)=0$
and let $A,B\in\mathcal{A}$
with $A\cap B = \varnothing$ be given.
We need to prove $\mu(A\cup B) = \mu(A) + \mu(B)$.
We have $\mu(A\cap B) = \mu(\varnothing) = 0$,
so 
\begin{equation*}
\mu(A\cup B) \ =\  \mu(A\cap B) + \mu(A\cup B) \ =\  \mu(A) + \mu(B),
\end{equation*}
since $\mu$ is modular.
\end{ex}
%
%                  LEMMA ON MODULARITY FOR MODULAR MAPS
%
\begin{lem}
\label{L:modular-map-modular}
Suppose $\varphi$ is modular.
For $\ell,u\in L$ with $\ell\leq u$, we have
\begin{equation}
\label{eq:modular-map}
\varphi(\,\ell \vee (a \wedge u)\,) 
\ =\ 
\varphi(\,(\ell\vee a)\wedge u\,)
\qquad (a\in L).
\end{equation}
\begin{proof}
The trick is to consider the expression 
$\varphi(\ell) + \varphi(a) + \varphi(u)$.
On the one hand,
\begin{align*}
\varphi(\ell) + \varphi(a) + \varphi(u)
\ &=\ \varphi(\ell\wedge a) + \varphi(\ell \vee a) + \varphi(u) \\
\ &=\ \varphi(\ell \wedge a)
      + \varphi(\,(\ell\vee a)\wedge u\,)
      + \varphi(a\vee u),
\end{align*}
where we have applied Equation~\eqref{eq:modular-map} twice.
On the other hand,
\begin{align*}
\varphi(\ell) + \varphi(a) + \varphi(u)
\ &=\ \varphi(\ell) + \varphi(a\wedge u) + \varphi(a \vee u) \\
\ &=\ \varphi(\ell \wedge a)
      + \varphi(\,\ell\vee (a\wedge u)\,)
      + \varphi(a\vee u).
\end{align*}
The difference,
$\varphi(\,(\ell\vee a)\wedge u\,)
- \varphi(\,\ell\vee (a\wedge u)\,)$,
must be zero.
\end{proof}

\end{lem}

%
%                  SYSTEMS
%
\begin{dfn}
We say $\vs{V}{L}{\varphi}{E}$
 is a \keyword{valuation system}
provided that
\begin{enumerate}
\item
$V$ is a lattice such that
$\bigvee a_n$ exists for all $a_1 \leq a_2 \leq \dotsb$
and $\bigwedge b_n$ exists for all $b_1 \geq b_2 \geq \dotsb$;

\item
$L$ is a sublattice of~$V$;

\item
$E$ is an ordered Abelian group;

\item
$\varphi\colon L \ra E$ is a valuation.
\end{enumerate}
\end{dfn}

\begin{ex}
Let $\mathcal{S}$ be the set of all finite intervals of~$\R$,
\begin{equation*}
(a,b)\qquad [a,b)\qquad (a,b]\qquad [a,b].
\end{equation*}
Let~$\mathcal{R}_\mathrm{L}$ be rthe ring generated by~$\mathcal{S}$.
Every element~$A$ of~$\mathcal{R}_\mathrm{L}$ is of the form
\begin{equation*}
I_1 \cup \dotsb \cup I_N
\end{equation*}
where $I_1,\dotsc,I_N\in \mathcal{R}_\mathrm{L}$
are disjoint.
Let $\lambda(A)$ be given by
\begin{equation*}
\lambda(A) \ \eqdf\  |I_1| + \dotsb + |I_N|.
\end{equation*}
One can verify that the number $\varphi(A)$
only depends on~$A$ and not on the choice of~$I_1,\dotsc,I_N$.
Hence we obtain a map~$\lambda\colon \mathcal{R}_\mathrm{L} \ra \R$.
Almost by definition $\lambda$ is additive and positive.
So $\lambda$ is modular and order preserving.

Hence $\vs{\wp \R}{\mathcal{R}_\mathrm{L}}{\lambda}{\R}$
is a valuation system.
\end{ex}

\begin{nt}
Let $\vs{V}{L}\varphi{E}$ be a valuation system.
Let $a_1 \leq a_2 \leq \dotsb$ be from~$L$.
Then $a_1 \leq a_2 \leq \dotsb$ has a supremum
in~$V$ and might have a supremum in~$L$.
We ignore the latter:
\emph{With $\bv a_n$
we always mean the supremum of~$a_1\leq a_2\leq \dotsb $ in~$V$}.
\end{nt}

%
%                  THE R-property
%
\begin{dfn}
\label{D:R-property}
Let $E$ be an ordered Abelian group.
Consider the following.
\begin{equation*}
\left[\quad 
\begin{minipage}{.7\columnwidth}
Let $x_1 \leq x_2 \leq \dotsb$
and $y_1 \leq y_2 \leq \dotsb$ be from~$E$
such that
\begin{equation*}
x_{n+1} - x_n \ \leq\ y_{n+1} - y_n\qquad \text{for all }n.
\end{equation*}
Then $\bv y_n $ exists implies that $\bv x_n$ exists.
\end{minipage}
\right.
\end{equation*}
If the above statement holds,
we say~$E$ has the \keyword{$R$-property}.
\end{dfn}

\begin{exs}
\begin{enumerate}
\item
Any $\sigma$-Dedekind complete Riesz space has the $R$-property.

\item
The lexicographic plane~$\Lex$ (see Definition~\todo{add reference})
has the $R$-property,
but~$\Lex$ is not $\sigma$-Dedekind complete.
\end{enumerate}
\end{exs}


\end{document}
