\documentclass[main.tex]{subfiles}
\begin{document}
\section{Valuations}
\label{S:vals}
We derive some elementary properties of valuations in this section.
We start with some examples of valuations in Subsection~\ref{SS:vals_intro}.
We study the distance 
induced by a valuation~$\varphi$
in Subsection~\ref{SS:vals_d}, being
\begin{equation*}
\ld\varphi(x,y) \ =\ \varphi(x\vee y) - \varphi(x\wedge y).
\end{equation*}
In Subsection~\ref{SS:vals_eq}
we study the equivalence induced by this distance,
\begin{equation*}
x\approx y\quad\iff\quad \ld\varphi(x,y)=0.
\end{equation*}
The notion of distance 
is especially important to the remainder of this text.
%%%%%%%%%%%%%%%%%%%%%%%%%%%%%%%%%%%%%%%%%%%%%%%%%%%%%%%%%%%%%%%%%%%%%%%%%%%%%%%
%%%%%%%%%%%%%%%%%%%%%%%%%%%%%%%%%%%%%%%%%%%%%%%%%%%%%%%%%%%%%%%%%%%%%%%%%%%%%%%
%%%%%%%%%%%%%%%%%%%%%%%%%%%%%%%%%%%%%%%%%%%%%%%%%%%%%%%%%%%%%%%%%%%%%%%%%%%%%%%
%%%%%%%%%%%%%%%%%%%%%%%%%%%%%%%%%%%%%%%%%%%%%%%%%%%%%%%%%%%%%%%%%%%%%%%%%%%%%%%
\subsection{Introduction}
\label{SS:vals_intro}
\noindent
%
%          DEFINITION OF MODULAR MAPS
%
\begin{dfn}

\label{D:val}
Let $L$ be a lattice. 
Let $E$ an ordered Abelian group
(see Section~\ref{S:ag}).
Let  $\varphi\colon L \ra E$ be a map.
We say that
\begin{enumerate}
\item
\label{D:val-mod}
$\varphi$ is \keyword{modular} provided that
\begin{equation*}
\varphi(a\wedge b) + \varphi(a \vee b)
\ =\ 
\varphi(a) + \varphi(b)
\qquad(a,b\in L);
\end{equation*}

\item
\label{D:val-val}
$\varphi$ is a \keyword{valuation}
provided that $\varphi$ is modular and order preserving.
\end{enumerate}
\end{dfn}

\begin{ex}
Let $\mathcal{F}$ be the set of finite subsets of~$\N$,
and for each $A\in \mathcal{F}$,
let $\#(A)$ be the number of elements of~$A$.
Then have 
\begin{equation*}
\#(A\cap B) + \#(A\cup B) \,=\, \#(A) + \#(B)
\qquad(A,B\in\mathcal{F}),
\end{equation*}
so obviously the map $\mathcal{F} \ra \N$
given by by $A\mapsto \#(A)$ is a valuation.
\end{ex}

%
%                  LEBESGUE MEASURE IS A VALUATION
%
\begin{ex}
\label{E:lmeas-val}
Let $\LA$ be the set of Lebesgue measurable
subsets of~$\R$ with finite Lebesgue measure.
Then~$\LA$ is a lattice of subsets of~$\R$.
Given~$A\in\LA$,
let $\Lmu$ denote the Lebesgue measure of~$A$.
Then $A\subseteq B \implies \Lmu(A)\leq \Lmu (B)$
and 
\begin{equation*}
\Lmu(A\cap B) + \Lmu(A\cup B) \,=\, \Lmu(A) + \Lmu(B)
\qquad(A,B\in\mathcal{A}).
\end{equation*}
So~$\Lmu$ is a valuation.
\end{ex}

%
%                  LEBESGUE INTEGRAL IS A VALUATION
%
\begin{ex}
\label{E:int-val}
Let~$\LF$ be the set of Lebesgue integrable functions on~$\R$.
It is a lattice ordered Abelian group.
The assignment $f\mapsto \int f$ 
yields an order preserving group homomorphism
$\Lphi\colon \LF\ra \R$.
To see that $\Lphi$ is modular (and a valuation),
note
\begin{equation*}
\min\{x,y\}+\max\{x,y\} \,=\, x+y\qquad(x,y\in \R).
\end{equation*}
So given $f,g\in \LF$,
we have $f\wedge g + f \vee g = f+ g$,
and hence
\begin{equation*}
\Lphi(f\wedge g) + \Lphi(f\vee g) 
\,=\,\Lphi(f\wedge g + f\vee g)
\,=\, \Lphi(f+g)
\,=\, \Lphi (f)+\Lphi(g).
\end{equation*}
\end{ex}

%
%                  ANY MAP ON A CHAIN IS MODULAR
%
\begin{ex}
Let $C$ be a chain,
i.e. a totally ordered set.
Then $C$ is a lattice with
\begin{equation*}
a\wedge b = \min\{a,b\},
 \qquad 
a\vee b = \max\{a,b\}.
\end{equation*}
One quickly sees that
\emph{any map $f\colon C\rightarrow E$
to an ordered Abelian group is modular.}
\end{ex}

%
%                  RING OF SUBSETS
%
\begin{ex}
\label{E:ring-val}
Let~$X$ be a set and let~$\mathcal{A}$ be a ring of subsets of~$X$.
That is,
\begin{equation*}
A\cap B,\qquad\qquad A\cup B,\qquad\qquad A\backslash B
\end{equation*}
are in~$\mathcal{A}$ for all $A,B\in\mathcal{A}$.
Then clearly~$\mathcal{A}$ is a lattice.

Let $E$ be an ordered Abelian group
and let~$\mu\colon \mathcal{A}\rightarrow E$ be a map.
Recall that~$\mu$ is \keyword{additive} if $\mu(A) + \mu(B) = \mu(A\cup B)$
for all $A,B\in\mathcal{A}$ with $A\cap B=\varnothing$.

\emph{If~$\mu$ is additive,
then $\mu$ is modular.}
Indeed,
let $A,B\in \mathcal{A}$ be given. We need to prove that
$\mu(A) + \mu(B) =\mu(A\cap B) + \mu(A\cup B)$
assuming~$\mu$ is additive.
We have
\begin{alignat*}{6}
\mu(A) + \mu(B) \,
  & =\, \mu(A\cap B \ \cup\ A\backslash B) + \mu(B) \\ 
  & =\, \mu(A\cap B) + \mu(A\backslash B)  + \mu(B)\qquad
    && \text{since } A\cap B \ \cap\ A\backslash B = \varnothing \\ 
  & =\, \mu(A\cap B) + \mu(A\backslash B \ \cup\ B ) 
    && \text{since } A\backslash B\ \cap\ B = \varnothing \\
  & =\, \mu(A\cap B) + \mu(A\cup B).
\end{alignat*}

Recall that~$\mu$ is \keyword{positive} whenever
$\mu(A)\in E^+$ for all~$A\in\mathcal{A}$.

\emph{If~$\mu$ is additive and positive,
then~$\mu$ is a valuation.}
Since~$\mu$ is additive,
$\mu$ is modular.
It remains to be shown that~$\mu$ is order preserving
(see Definition~\ref{D:val}).
Let $A\subseteq B$ from~$\mathcal{A}$ be given
in order to prove $\mu(A)\leq \mu(B)$.
We have
\begin{equation*}
(B\backslash A)\,\cup\, A\,=\,B,\qquad\qquad 
(B\backslash A)\,\cap\, A\,=\,\varnothing.
\end{equation*}
So by additivity, 
$\mu(B)=\mu(B\backslash A)+\mu(A)$.
Then $\mu(B)\geq \mu(A)$, since $\mu(B\backslash A)\geq 0$.
\end{ex}

%
%                  RING OF SIMPLE LEBESGUE SUBSETS
%
\begin{ex}
\label{E:smeas-val}
We describe a ring of subsets of~$\R$
and a positive and additive 
map~$\Smu\colon \SA\ra \R$ that 
will eventually
lead to the Lebesgue measure.

Let $\mathcal{S}$ be the set of all finite intervals of~$\R$,
\begin{equation*}
(a,b)\qquad [a,b)\qquad (a,b]\qquad [a,b]
\qquad\qquad\text{where }a<b.
\end{equation*}
Let~$\SA$ be the ring generated by~$\mathcal{S}$.
Every element~$A$ of~$\SA$ is of the form
\begin{equation*}
I_1 \,\cup \,\dotsb \,\cup\, I_N
\end{equation*}
where $I_1,\dotsc,I_N\in \SA$
are disjoint.
Let $\Smu(A)$ be given by
\begin{equation*}
\mu_{\mathrm L}(A) \ \eqdf\  |I_1| + \dotsb + |I_N|.
\end{equation*}
One can verify that the number $\Smu(A)$
only depends on~$A$ and not on the choice of~$I_1,\dotsc,I_N$.
Hence we obtain a map~$\Smu\colon \SA \ra \R$.
Almost by definition $\Smu$ is additive and positive.
Hence $\Smu\colon\SA\ra \R$ is a valuation 
(see Example~\ref{E:ring-val}).
\end{ex}


In Example~\ref{E:int-val},
we saw a group homomorphism that is modular.
In fact any group homomorphism
on a lattice ordered Abelian group is modular
(see Corollary~\ref{C:hom-val}\ref{C:hom-val-group}).
\begin{ex}
\label{E:1-valuation}
Let~$R$ be a lattice ordered Abelian group.
Then the identity map $\mathrm{id}_R$ is a valuation.
Indeed, $\mathrm{id}_R$ is modular by Lemma~\ref{L:1-valuation},
and clearly order preserving.
\end{ex}

\begin{lem}
\label{L:mod-comp}
Suppose we have the following situation
\begin{equation*}
\xymatrix{
L' \ar[r]^f&
L \ar[r]^\varphi&
E \ar[r]^g&
E',
}
\end{equation*}
where $L$, $L'$ are lattices,
$E$, $E'$ are ordered Abelian groups,
$f$ is a lattice homomorphism,
$\varphi$ a map,
and $g$ is a group homomorphism.
Then
\begin{enumerate}
\item
\label{L:mod-comp-mod}
$g\circ \varphi \circ f$ is modular
provided that $\varphi$ is modular;
\item
\label{L:mod-comp-val}
$g\circ \varphi \circ f$ is a valuation
provided that $\varphi$ is a valuation
and~$g$ is positive.
\end{enumerate}
\end{lem}
\begin{proof}
\noindent
\ref{L:mod-comp-mod}
\  Suppose~$\varphi$ is modular.
Let $a,b\in L'$ be given.
Writing $\varphi'= g\circ\varphi \circ f$,
we need to prove that
$\varphi'(a\wedge b)+\varphi'(a\vee b)=\varphi'(a)+\varphi'(b)$.
We have
\begin{alignat*}{3}
\varphi'(a) + \varphi'(b)
\ &=\ g(\varphi(f(a))) \,+\, g(\varphi(f(b))) \\
  &=\ g(\ \varphi( f(a)) + \varphi( f(b))\ ) \\
  &=\ g(\ \varphi(f(a)\wedge f(b)) + \varphi(f(a)\vee f(b))\ ) \\
  &=\ g(\  \varphi(f(a\wedge b)) \,+\, \varphi(f(a\vee b))\ ) \\
  &=\ g(\varphi(f(a\wedge b))) \,+\, g(\varphi(f(a\vee b))) \\ 
  &=\ \varphi'(a\wedge b) + \varphi'(a \vee b)
\end{alignat*}
\ref{L:mod-comp-val}
\  Suppose~$\varphi$ is a valuation
and~$g$ is positive.
We need to prove that~$\varphi'\eqdf g\circ \varphi\circ f$
is a valuation.
By part~\ref{L:mod-comp-mod}
we know that~$\varphi'$ is modular.
It remains to be shown that~$\varphi'$ is order preserving.
This is easy: $g$, $\varphi$, and~$f$ are all order preserving.
So $\varphi'=g\circ\varphi\circ f$ must be order preserving too.
\end{proof}

\begin{cor}
\label{C:hom-val}
Let~$R$ be a lattice ordered Abelian group.
\begin{enumerate}
\item
\label{C:hom-val-lat}
Let~$L$ be a lattice.
Any lattice homomorphism $f\colon L\ra R$ 
is a valuation;

\item
\label{C:hom-val-group}
Let $E$ be an ordered Abelian group
and $g\colon R\ra E$ a group homomorphism.
Then $g$ is modular.
Moreover,
if~$g$ is positive,
then~$g$ is a valuation.
\end{enumerate}
\end{cor}
\begin{proof}
Apply Lemma~\ref{L:mod-comp} to the following situations.
\begin{equation*}
\xymatrix{
L\ar[r]^f &
R\ar[r]^{\mathrm{id}_R} &
R\ar[r]^{\mathrm{id}_R} &
R &&
R\ar[r]^{\mathrm{id}_R} &
R\ar[r]^{\mathrm{id}_R} &
R\ar[r]^{g} &
E}
\end{equation*}
(Recall that $\mathrm{id}_R$ is a valuation,
see Example~\ref{E:1-valuation}.)
\end{proof}

%
%                  RIESZ SPACE OF FUNCTIONS
%
\begin{ex}
\label{E:val-riesz-space-of-functions}
Let $X$ be a set.
We say that $F\subseteq \R^X$
is  \emph{Riesz space of functions} if
\begin{equation*}
f\vee g,\quad\qquad 
f\wedge g,\quad\qquad
f+g,\quad\qquad 
\lambda \cdot f
\end{equation*}
are all in~$F$
where $f,g\in F$ and $\lambda \in \R$.
Then~$F$ is a lattice ordered Abelian group.

Let~$E$ be an ordered Abelian group
and let $\varphi\colon F\ra E$ be a positive linear map.
We see that~$\varphi$ is a valuation
 by Corollary~\ref{C:hom-val}\ref{C:hom-val-group}.
\end{ex}

%
%                  RIESZ SPACE OF STEP FUNCTIONS
%
\begin{ex}
\label{E:sint-val}
We  describe a Riesz space
of functions~$F_{\mathrm{L}}$ on~$\R$
and a positive linear map~$\varphi_{\mathrm{L}}\colon F_{\mathrm{L}}\ra \R$
that will eventually lead to the Lebesgue integral.

A \emph{step function} is a function~$f\colon \R\ra\R$
for which there are $s_1 < s_2 <\dotsb <s_N$ in~$\R$
such that $f$ is constant on each~$(s_n,s_{n+1})$
and $f$ is zero outside $[s_1,s_N]$.

Let $F_\mathrm{L}$ be the set of step functions.
One can easily see that~$F_\mathrm{L}$ is a Riesz space of functions.
Let $f\in F_\mathrm{L}$.
Let $s_1 < s_2 <\dotsb <s_N$
be such that $f$ is constant, say $c_n\in \R$,
 on~$(s_n,s_{n+1})$
and $f$ is zero outside $[s_1,s_N]$.
One can prove that 
\begin{equation}
\label{exp:step}
\sum_{n=1}^{N-1} \, c_n\cdot(s_{n+1} - s_n)
\end{equation}
does not depend on the choice
of~$s_1 < s_2 <\dotsb <s_N$.
So Expression~\eqref{exp:step}
given a map $\varphi_\mathrm{L}\colon F_\mathrm{L} \ra \R$.
This map is easily seen to be linear.

Consequently, $\varphi_\mathrm{L}\colon F_\mathrm{L}\ra \R$
is a valuation (see Example~\ref{E:val-riesz-space-of-functions}).
\end{ex}

%
%                  TOTIENT FUNCTION
%
\begin{ex}
\todo{Write this example on the Euler Totient function.}
\end{ex}


%
%                  EXAMPLE ON VECTOR SPACES
%
\begin{ex}
Up to this point
we have only seen valuations on distributive lattices.
We will now give an example
of a valuation on a non-distributive lattice.

Let $W$ be a vector space.
Let $L$ be the set of finite-dimensional linear subspaces
of~$W$ ordered by inclusion.
  Then~$L$ is a lattice, and 
for all~$A,B\in L$,
\begin{equation*}
A\wedge B \ =\ A\cap B,
\qquad A\vee B \ =\ \left<A\cup B\right>
\end{equation*}
where $\left< S \right>$ denotes the smallest
linear subspace containing $S$.
We have
\begin{equation*}
\dim (A\wedge B) \,+\, \dim(A\vee B)
\ =\ 
\dim A \,+\, \dim B 
\qquad\qquad(A,B\in L).
\end{equation*}
To see this,
apply the dimension theorem
to the map~$f\colon A\times B\ra A\vee B$ given by  $(a,b)\mapsto a+b$.
Hence the assignment $A \mapsto \dim A$
gives a valuation $\dim\colon L\ra \N$.

The lattice~$L$ might be distributive.
For instance, if~$W=\{ 0 \}$.
This occurs only seldomly:
if $W$ contains two linearly independent vectors,
then~$L$ is non-distributive.

Indeed,
let $v_1,v_2\in W$ be linearly independent vectors
and consider  $w\eqdf v_1 + v_2$.
One can verify that $v_i, w$ are linearly independent too.
So $\left< v_i \right> \cap \left< w \right> = \{0\}$.
Hence
\begin{equation*}
\left< w \right> \wedge(\left< v_1 \right>\vee\left< v_2 \right>)
\ =\ 
\left< v_1, v_2 \right>
\ \neq\ 
\{0\}
\ =\ 
(\left<w\right> \wedge \left<v_1\right>)
\,\vee\, (\left<w\right> \wedge \left<v_2\right>).
\end{equation*}
\end{ex}

We end this subsection
with some tame examples of valuations we 
need lateron.
\begin{ex}
\label{E:val-product}
Let $I=\{ 1,2\}$.
For each $i\in I$, 
let $L_i$ be a lattice,
$E_i$ an ordered Abelian group,
and $\varphi_i \colon L_i \ra E_i$
a valuation.
Then 
the map 
\begin{equation*}
\varphi_1 \times \varphi_2 \colon \,
L_1 \times L_2 \,\longrightarrow\, E_1 \times E_2
\end{equation*}
given by $(\varphi_1 \times\varphi_2)(a_1,a_2) = (\varphi_1(a),\varphi_2(b))$
for all~$a_i\in L_i$, is a valuation.

We call the valuation $\varphi_1 \times \varphi_2$
the \emph{product} of $\varphi_1$ and $\varphi_2$.
Of course,
one can similarly define a product of an $I$-indexed family
of valuations for any set~$I$.
\end{ex}

\begin{ex}
\label{E-val-opposite}
Let $L$ be a lattice.
If we reverse the order on~$L$,
i.e., consider the partial order on~$L$ 
given by $\smash{a \leq_{L^\mathrm{op}} b}
\iff a\geq_L b$,
then if a subset~$S\subseteq L$
has a supremum, $\bigvee S$,
then $\bigvee S$ is the 
\emph{infimum} of~$S$
with respect to~$\leq^\mathrm{op}$.
So we see that $\leq^\mathrm{op}$
gives us a lattice, $L^\mathrm{op}$.
(The \emph{opposite} lattice.)

Let~$E$ be an ordered Abelian group.
If we reverse the order on~$E$,
we obtain an ordered Abelian group~$E^\mathrm{op}$
with the same group structure,
but whose positive elements, $\smash{(E^\mathrm{op})^+}$,
are precisely the negative elements of~$E$.

Let~$\varphi\colon L\ra E$ be a modular map
(see Definition~\ref{D:val}).
Then one quickly sees that $\varphi$ is also modular considered as a map
$L^\mathrm{op} \ra E$.
However,
$\varphi\colon L^\mathrm{op}\ra E$
is a valuation (that is, also order preserving)
if and only if $\varphi\colon L\ra E$
is order \emph{reversing},
i.e., $a\leq b\implies \varphi(a)\geq \varphi(b)$ for
all $a,b\in L$.

Of course,
if $\varphi\colon L\ra E$ is a valuation,
then $\varphi$ is a valuation $L^\mathrm{op}\ra E^\mathrm{op}$.
\end{ex}

It is interesting to note that
there are some `connections'
between
modular maps (see Definition~\ref{D:val})
and \emph{modular lattices}.
Recall that a lattice~$L$ is modular if
\begin{equation*}
\ell \vee (a \wedge u) \ =\ (\ell \vee a) \wedge u
\end{equation*}
for all~$\ell,u,a \in L$ with $\ell \leq u$.
One such connection is given by the following lemma.
%
%                  LEMMA ON MODULARITY FOR MODULAR MAPS
%
\begin{lem}
\label{L:modular-map-modular}
For $\ell,u\in L$ with $\ell\leq u$, we have
\begin{equation}
\label{eq:modular-map}
\varphi(\,\ell \vee (a \wedge u)\,) 
\ =\ 
\varphi(\,(\ell\vee a)\wedge u\,)
\qquad (a\in L).
\end{equation}
\end{lem}
\begin{proof}
The trick is to consider the expression 
$\varphi(\ell) + \varphi(a) + \varphi(u)$.
On the one hand,
\begin{align*}
\varphi(\ell) + \varphi(a) + \varphi(u)
\ &=\ \varphi(\ell\wedge a) + \varphi(\ell \vee a) + \varphi(u) \\
\ &=\ \varphi(\ell \wedge a)
      + \varphi(\,(\ell\vee a)\wedge u\,)
      + \varphi(a\vee u),
\end{align*}
where we have used modularity twice.
On the other hand,
\begin{align*}
\varphi(\ell) + \varphi(a) + \varphi(u)
\ &=\ \varphi(\ell) + \varphi(a\wedge u) + \varphi(a \vee u) \\
\ &=\ \varphi(\ell \wedge a)
      + \varphi(\,\ell\vee (a\wedge u)\,)
      + \varphi(a\vee u).
\end{align*}
The difference,
$\varphi(\,(\ell\vee a)\wedge u\,)
- \varphi(\,\ell\vee (a\wedge u)\,)$,
must be zero.
\end{proof}

%%%%%%%%%%%%%%%%%%%%%%%%%%%%%%%%%%%%%%%%%%%%%%%%%%%%%%%%%%%%%%%%%%%%%%%%%%%%%%%
%%%%%%%%%%%%%%%%%%%%%%%%%%%%%%%%%%%%%%%%%%%%%%%%%%%%%%%%%%%%%%%%%%%%%%%%%%%%%%%
%%%%%%%%%%%%%%%%%%%%%%%%%%%%%%%%%%%%%%%%%%%%%%%%%%%%%%%%%%%%%%%%%%%%%%%%%%%%%%%
%%%%%%%%%%%%%%%%%%%%%%%%%%%%%%%%%%%%%%%%%%%%%%%%%%%%%%%%%%%%%%%%%%%%%%%%%%%%%%%
%%%%%%%%%%%%%%%%%%%%%%%%%%%%%%%%%%%%%%%%%%%%%%%%%%%%%%%%%%%%%%%%j%
%
%
%                  DISTANCE INDUCED BY 
%                     a valuation
%
%
\subsection{Distance induced by a valuation}
\label{SS:vals_d}
In this subsection,
we derive some facts 
concerning the following notion
of distance induced by a valuation.
\begin{dfn}
\label{D:d}
Let $E$ be an ordered Abelian group.
Let $L$ be a lattice.\\
Let $\varphi\colon L \ra E$ be a valuation.
Define $\ld\varphi\colon L\times L \ra E$ by
\begin{equation*}
\ld\varphi(a,b)\ =\  \varphi(a\vee b) - \varphi(a \wedge b)
\qquad\quad(a,b\in L).
\end{equation*}
\end{dfn}

To give the name ``distance'' for~$\ld\varphi$ some credibility,
we will prove that $\ld\varphi$ 
is a pseudometric (see Lemma~\ref{L:d-metric}).
After that,
we turn our attention to the following fact,
we will use often.
Given~$a\in L$, the map $x\mapsto a\wedge x$ is a \emph{contraction}, i.e.,
\begin{equation*}
\ld\varphi(\,a\wedge x,\, a\wedge y\,)
\ \ \leq\ \ \ld\varphi(x,y)
\qquad\quad(x,y\in L).
\end{equation*}
In fact,
we will prove the following, stronger, statement
(see Lemma~\ref{L:curry-wc-unif}).
\begin{equation*}
\ld\varphi(\,a\wedge x,\, a\wedge y\,) \ +\ 
\ld\varphi(\,a\vee x,\, a\vee y\,)
\ \ \leq\ \ \ld\varphi(x,y)
\qquad\quad(x,y\in L).
\end{equation*}

Before we do all this,
let us consider some examples.
%
%                  EXAMPLES OF DISTANCE
%
\begin{ex}
\label{E:d-riesz}
Let~$E$ be an ordered Abelian group.
Let $F$ be a Riesz space of functions,
and let $\varphi\colon F\ra E$ be
a positive and linear map
(see Example~\ref{E:val-riesz-space-of-functions}).

Let $f,g\in F$ be given.
The distance between $f$ and $g$ is the usual one,
\begin{equation*}
\ld\varphi(f,g) \,=\, \|f-g\|_1 \ \eqdf\ \varphi(\,|f-g|\,).
\end{equation*}
To see this,
note that since~$\varphi$ is linear,
we have 
\begin{equation*}
\ld\varphi(f,g) \,=\, 
\varphi(f\vee g) - \varphi(f\wedge g)
\,=\, \varphi(f\vee g - f\wedge g).
\end{equation*}
Further,
since we have the identity $\max\{x,y\} - \min\{x,y\} = |x-y|$
for reals $x,y$,
we have the identity $f\vee g - f\wedge g = |f-g|$
for functions.
\end{ex}
\begin{ex}
Let $E$ be an ordered Abelian group.
Let $\mathcal{A}$ be a ring of sets,
and let $\mu\colon \mathcal{A}\ra E$
be a positive additive map
(see Example~\ref{E:ring-val}).
Let $A,B\in\mathcal{A}$.
We have
\begin{equation*}
\ld\mu(A,B) \ =\ \mu(A \ominus B),
\end{equation*}
where $A\ominus B \eqdf A\backslash B \,\cup\, B\backslash A$
is the \emph{symmetric difference} of~$A$ and~$B$.
To see this, note that $A\cup B$ is the disjoint union
of $A\ominus B$ and $A\cap B$. So
since $\mu$ is additive, 
\begin{equation*}
\mu(A\cup B) \ =\ \mu(A\ominus B) + \mu(A\cap B).
\end{equation*}
\end{ex}
%
%                  LEMMA ON THE DISTANCE
%
\begin{lem}
\label{L:d-metric}
Let $E$ be an ordered Abelian group.\\
Let $L$ be a lattice,
and let $\varphi\colon L \ra E$ be a valuation.
We have:
\begin{enumerate}
\item \label{d-metric_pos}
$\ld\varphi(a,b)\geq 0$ for all~$a,b\in L$.
\item\label{d-metric_self} 
$\ld\varphi(a,a)=0$ for all $a\in L$.
\item\label{d-metric_sym}
$\ld\varphi(a,b)=\ld\varphi(b,a)$ for all $a,b\in L$.
\item\label{d-metric_triangle}
$\ld\varphi(a,b)\leq\ld\varphi(a,z)+\ld\varphi(z,b)$
for all $a,b,z\in L$.
\end{enumerate}
\end{lem}
\begin{proof}
Only point~\ref{d-metric_triangle} requires some work.
Let $a,b,z\in L$ be given.
We want to show that $\ld\varphi(a,b) \leq \ld\varphi(a,z)+\ld\varphi(z,b)$.
In other words:
\begin{equation}
\label{eq:d-metric}
\varphi(a\vee b) + \varphi(a\wedge z) + \varphi (z\wedge b)
\ \leq\ 
\varphi (a\vee z) + \varphi(z\vee b) + \varphi (a\wedge b)\text{.}
\end{equation}
By modularity,
the left-hand side equals
\begin{equation*}
\varphi(a\vee b) 
 + \varphi(\,(a\wedge z)\vee(b\wedge z)\,)
 + \varphi(a\wedge b\wedge z).
\end{equation*}
On the other hand,
using modularity
the right-hand side of Inequality~\eqref{eq:d-metric} becomes
\begin{equation*}
\varphi(a\vee b\vee z)
 + \varphi(\,(a\vee z)\wedge(b\vee z)\,)
 + \varphi(a\wedge b).
\end{equation*}
Note that $a\vee b \leq a\vee b\vee z$,
and $(a\wedge z)\vee (b\wedge z) \leq z \leq (a\vee z)\wedge (b\vee z)$,
and $a\wedge b\wedge z \leq a\wedge b$,
so that the monotonicity of~$\varphi$ yields Inequality~\eqref{eq:d-metric}.
\end{proof}
%
%                  DEFINITION HAUSDORFF
%
We might have $\ld\varphi(a,b) = 0$ while $a\neq b$
(see Example~\ref{E:eq-int}).
So in general, $\ld\varphi$ is not a metric
(but merely a  \emph{pseudo}metric).
Those $\varphi$ for which~$\ld\varphi$ is a metric
turn out to be useful. So let us give them a name.
\begin{dfn}
\label{D:val_Hausdorff}
Let $L$ be a lattice.
Let $E$ be an ordered Abelian group.\\
Let $\varphi\colon L \ra E$ be a valuation.
We say $\varphi$ is \keyword{Hausdorff}
provided that
\begin{equation*}
\ld\varphi(a,b) = 0 
\quad \implies \quad a=b \qquad\qquad(a,b\in L).
\end{equation*}
\end{dfn}
We return to Hausdorff valuations in Subsection~\ref{SS:vals_eq}.
%
%                  CURRY-WC-UNIF
%
\begin{lem}
\label{L:curry-wc-unif}
Let $E$ be an ordered Abelian group.\\
Let $L$ be a lattice,
and $\varphi\colon L \ra E$ a valuation.
We have
\begin{equation*}
\ld\varphi(a\wedge z,\,b\wedge z)\,+\,
 \ld\varphi(a\vee z,b\vee z)\ \leq\ \ld\varphi(a,b),
\end{equation*}
where $a,b,z\in L$.
\end{lem}
\begin{proof}
By expanding Definition~\ref{D:d},
we see that 
we need to prove that
\begin{equation}
\label{eq:curry-wc-unif}
\begin{split}
\varphi(a\vee b\vee z) \,+\,
\varphi(\,(a\wedge z)\vee(b\wedge z)\,) & \,+\,
\varphi(a\wedge b) \\
\ \leq\ 
\varphi(a \vee b)  \,+\, &
\varphi(\,(a\vee z)\wedge(b\vee z)\,) \,+\,
\varphi(a\wedge b \wedge z)
\end{split}
\end{equation}
By modularity,
the left-hand side equals
\begin{equation*}
\varphi(a\vee b \vee z) \,+\,
\varphi(\, (a\wedge z) \vee (b\wedge z) \vee (a\wedge b)\,) \,+\,
\varphi(\,a\wedge b\wedge ( (a \wedge z) \vee (b\wedge z) )\,).
\end{equation*}
To simplify the above expression,
we prove that
$a\wedge b\wedge ( (a \wedge z) \vee (b\wedge z)=a\wedge b\wedge z$.
To this end, note that
$a\wedge z \,\leq\, (a\wedge z)\vee (b\wedge z) \leq z$
so that 
\begin{equation*}
a\wedge b\wedge z
\, =\, a\wedge b \wedge (a\wedge z)
\,\leq\, a\wedge b \wedge ((a\wedge z)\vee (b\wedge z))
\,\leq\, a\wedge b \wedge z.
\end{equation*}
Hence the left-hand side of Inequality~\eqref{eq:curry-wc-unif}
equals
\begin{equation*}
\varphi(a\vee b \vee z) \,+\,
\varphi(\, (a\wedge z) \vee (b\wedge z) \vee (a\wedge b)\,) \,+\,
\varphi(\,a\wedge b\wedge z\,).
\end{equation*}
In a similar fashion,
one can show that the right-hand side of Inequality~\eqref{eq:curry-wc-unif}
equals
\begin{equation*}
\varphi(a\vee b \vee z) \,+\,
\varphi(\, (a\vee z) \wedge (b\vee z) \wedge (a\vee b)\,) \,+\,
\varphi(\,a\wedge b\wedge z\,).
\end{equation*}
So in order to prove Inequality~\eqref{eq:curry-wc-unif},
we must show that
\begin{equation*}
\varphi(\, (a\wedge z) \vee (b\wedge z) \vee (a\wedge b)\,) 
\ \leq\ 
\varphi(\, (a\vee z) \wedge (b\vee z) \wedge (a\vee b)\,).
\end{equation*}
Since $\varphi$ is order preserving,
it suffices to show that 
\begin{equation*}
 (a\wedge z) \vee (b\wedge z) \vee (a\wedge b)
\ \leq\ 
  (a\vee z) \wedge (b\vee z) \wedge (a\vee b).
\end{equation*}
Writing $c_1 = a$, $c_2 = b$, and $c_3 = z$,
we must prove that 
\begin{equation*}
\bv_{i\neq j}\, c_i \wedge c_j \ \leq\ \bw_{k\neq \ell}\, c_k \vee c_\ell.
\end{equation*}
That is,
we must show that 
$c_i \wedge c_j \leq c_k \vee c_\ell$
for given $i\neq j$ and $k\neq \ell$.
Now, note
\begin{equation*}
\# (\{ i,j \} \cap \{ k,\ell\}) + \#\{i,j,k,\ell\}
\ = \ \# \{i,j\} + \# \{k,\ell \}
\ = \ 4.
\end{equation*}
Since  $\# \{ i,j,k,\ell \}\leq 3$,
we see that $\# \{ i,j \} \cap \{ k,\ell\} \geq 1$.
So pick $m\in  \{ i,j \} \cap \{ k,\ell\}$.
Then $c_i \wedge c_j \leq c_m \leq c_k \vee c_\ell$.
\end{proof}
%
%                  WC-UNIF
%
\begin{lem}
\label{L:wv-unif}
Let $E$ be an ordered Abelian group.\\
Let $L$ be a lattice,
and $\varphi\colon L \ra E$ a valuation.
Then we have
\begin{equation*}
\ld\varphi(a\wedge w,b\wedge z) \,+\, \ld\varphi(a \vee w,b\vee z)
\ \leq\ 
\ld\varphi(a,b) \,+\, \ld\varphi(w,z),
\end{equation*}
where $a,b,w,z\in L$.
\end{lem}
\begin{proof}
By the triangle inequality (point~\ref{d-metric_triangle}
of Lemma~\ref{L:d-metric}),
we have
\begin{equation}
\label{eq:L:d-metric-1}
\begin{alignedat}{3}
\ld\varphi(a\wedge w,\,b\wedge z)\ &\leq\ 
\ld\varphi(a\wedge w,\,b\wedge w)\,+\,
\ld\varphi(b\wedge w,\,b\wedge z), \\
\ld\varphi(a\vee w,\,b\vee z)\ &\leq\ 
\ld\varphi(a\vee w,\,b\vee w)\,+\,
\ld\varphi(b\vee w,\,b\vee z).
\end{alignedat}
\end{equation}
On the other hand, Lemma~\ref{L:curry-wc-unif} gives us
\begin{equation}
\label{eq:L:d-metric-2}
\begin{alignedat}{3}
\ld\varphi(a\wedge w,b\wedge w) + \ld\varphi(a\vee w,b\vee w)
   \ &\leq\ \ld\varphi(a,b), \\
\ld\varphi(b\wedge w,b\wedge z) + \ld\varphi(b\vee w, b\vee z)
   \ &\leq\ \ld\varphi(w,z).
\end{alignedat}
\end{equation}
The sum of the right-hand sides of Equation~\eqref{eq:L:d-metric-1}
equals the sum of the left-hand sides of Equation~\eqref{eq:L:d-metric-2}.
Hence
$\ld\varphi(a\wedge w,b\wedge z) + \ld\varphi(a \vee w,b\vee z)
\leq 
\ld\varphi(a,b) + \ld\varphi(w,z)$.
\end{proof}
%%%%%%%%%%%%%%%%%%%%%%%%%%%%%%%%%%%%%%%%%%%%%%%%%%%%%%%%%%%%%%%%%%%%%%%%%%%%%%%
%%%%%%%%%%%%%%%%%%%%%%%%%%%%%%%%%%%%%%%%%%%%%%%%%%%%%%%%%%%%%%%%%%%%%%%%%%%%%%%
%%%%%%%%%%%%%%%%%%%%%%%%%%%%%%%%%%%%%%%%%%%%%%%%%%%%%%%%%%%%%%%%%%%%%%%%%%%%%%%
%%%%%%%%%%%%%%%%%%%%%%%%%%%%%%%%%%%%%%%%%%%%%%%%%%%%%%%%%%%%%%%%%%%%%%%%%%%%%%%
%%%%%%%%%%%%%%%%%%%%%%%%%%%%%%%%%%%%%%%%%%%%%%%%%%%%%%%%%%%%%%%%%%%%%%%%%%%%%%%
%
%                  EQUIVALENCE INDUCED BY A VALUATION
%
\subsection{Equivalence induced by a valuation}
\label{SS:vals_eq}
In measure theory
two functions are considered equivalent
if they are equal almost everywhere.

In this subsection, we extend this notion of equivalence 
to valuations. 
%
%                  DEFINITION OF EQUIVALENCE
%
\begin{dfn}
\label{D:eq}
Let $E$ be an ordered Abelian group.
Let $L$ be a lattice.\\
Let $\varphi\colon L\ra E$ be a valuation.
We define $\approx$ to be 
the binary relation on~$L$ given by
\begin{equation*}
a \approx b
\quad\iff\quad
\ld\varphi(a,b)=0\qquad\qquad(a,b\in L).
\end{equation*}
\end{dfn}
\begin{rem}
$\varphi$ is Hausdorff (see Definition~\ref{D:val_Hausdorff})
iff $a \approx b\iff a = b$.
\end{rem}
%
%                  EXAMPLES OF EQUIVALENCE
%
\begin{ex}
\label{E:eq-int}
We consider the Lebesgue integral $\Lphi\colon \LF\ra \R$
(see Example~\ref{E:int-val}).

Let $f,g\in \LF$ be given.
By Example~\ref{E:d-riesz} 
we know that
\begin{equation*}
f\approx g\quad\iff\quad \Lphi(\,|f-g|\,)=0.
\end{equation*}
For any $h\in \LF$
with $h\geq 0$,
we have that $\Lphi(h)=0$
iff $h(x)=0$ for almost all~$x$.
(That is, $\smash{ \Lmu(\{ x\in\R\colon h(x)\neq 0 \})} $
is zero.)
So we see that 
\begin{equation*}
f\approx g\quad\iff\quad f(x) = g(x)\quad\text{for almost all~$x$}.
\end{equation*}
\end{ex}
%
%                  PROP ON EQUIVALENCE
%
\begin{prop}
\label{P:eq}
Let $E$ be an ordered Abelian group.
Let $L$ be a lattice.\\
Let $\varphi\colon L\ra E$ be a valuation.
Let $\approx$ be as in Definition~\ref{D:eq}.
\begin{enumerate}
\item
\label{P:eq-1}
The relation $\approx$ is an equivalence.
\item 
\label{P:eq-2}
Let~$a_1,a_2\in L$ with 
with $a_1\approx a_2$ be given.
Then $\varphi(a_1)=\varphi(a_2)$.
\item
\label{P:eq-3}
Let $a_1,a_2 \in L$ with $a_1\approx a_2$,
and let $b_1,b_2 \in L$ with $b_1 \approx b_2$ be given.
Then
\begin{equation*}
a_1 \wedge b_1 \,\approx\, a_2 \wedge b_2
\qquad\text{and}\qquad 
a_1 \vee b_1 \,\approx\, a_2 \vee b_2.
\end{equation*}
\item
\label{P:eq-4}
Let $a_1,a_2 \in L$ with $a_1\approx a_2$,
and let $b_1,b_2 \in L$ with $b_1 \approx b_2$ be given.
Then
\begin{equation*}
\ld\varphi(a_1,b_1)\ =\ \ld\varphi(a_2,b_2).
\end{equation*}
\end{enumerate}
\end{prop}
\begin{proof}
\ref{P:eq-1}\ 
The relation~$\approx$ is clearly reflexive
and symmetric. So to prove~$\approx$ is an equivalence relation,
we will only show that~$\approx$ is transitive.
Let $a,b,c\in L$ with $a\approx b\approx c$ be given.
We must show that $a\approx c$.
Or in other words, $\ld\varphi(a,c) = 0$.

By Lemma~\ref{L:d-metric}, points~\ref{d-metric_pos}
and~\ref{d-metric_triangle},
we get
\begin{equation}
\label{eq:P:eq-1}
0 \ \leq\ \ld\varphi(a,c) \ \leq\ 
\ld\varphi(a,b) + \ld\varphi(b,c).
\end{equation}
But 
$\ld\varphi(a,b)=0$ and $\ld\varphi(b,c)=0$,
since $a\approx b$ and $b\approx c$, respectively.

So we see that Statement~\eqref{eq:P:eq-1} implies $\ld\varphi(a,c)= 0$.
Hence $a\approx c$.

\vspace{.3em}
\ref{P:eq-2}\ 
Let $a_1,a_2\in L$ with $a_1\approx a_2$ be given.
We must prove $\varphi(a_1) = \varphi(a_2)$.

Let $i\in\{1,2\}$ be given. 
Note that $a_1 \wedge a_2 \leq a_i \leq a_1 \vee a_2$.
So we have 
\begin{equation}
\label{eq:P:eq-2}
\varphi(a_1 \wedge a_2) \ \leq\ \varphi(a_i) \ \leq\  \varphi(a_1 \vee a_2).
\end{equation}
Since $\ld\varphi(a_1,a_2)=0$,
we know that $\varphi(a_1 \vee a_2) = \varphi(a_1 \wedge a_2)$.
So Statement~\eqref{eq:P:eq-2}
implies that
$\varphi(a_1 \vee a_2) = \varphi(a_i) = \varphi(a_1 \wedge a_2)$.
Hence $\varphi(a_1) = \varphi(a_2)$.

\vspace{.3em}
\ref{P:eq-3}
Let $a_1,a_2 \in L$ with $a_1 \approx a_2$ be given.
Let $b_1,b_2 \in L$ with $b_1 \approx b_2$ be given.
We will only show that $a_1 \wedge b_1 \approx a_2 \wedge b_2$;
the proof of $a_1 \vee b_1 \approx a_2 \vee b_2$ is similar.

Note that we have the following inequalities by 
Lemma~\ref{L:d-metric}\ref{d-metric_pos}
and Lemma~\ref{L:wv-unif}.
\begin{equation}
\label{eq:P:eq-3}
0\ \leq\ 
\ld\varphi(a_1 \wedge b_1, a_2\wedge b_2)
\ \leq\ 
\ld\varphi(a_1,b_1) + \ld\varphi(a_2,b_2)
\end{equation}
Since $a_1 \approx a_2$ and $b_1 \approx b_2$,
we have $\ld\varphi(a_1,b_1)=0$
and $\ld\varphi(a_2,b_2)=0$,
respectively.
Hence Statement~\eqref{eq:P:eq-3} implies 
$\ld\varphi(a_1\wedge b_1, a_2 \wedge b_2)=0$.
So $a_1 \wedge b_1 \approx a_2 \wedge b_2$.

\vspace{.3em}
\ref{P:eq-4}
Let $a_1,a_2 \in L$ with $a_1 \approx a_2$ be given.
Let $b_1,b_2 \in L$ with $b_1 \approx b_2$ be given.
We must prove that $\ld\varphi(a_1,b_1)=\ld\varphi(a_2,b_2)$.
Note that
by point~\ref{P:eq-3}
we have
\begin{alignat*}{5}
a_1 \wedge b_1 &\approx a_2 \wedge b_2&
\qquad&\text{and}\qquad&
a_1 \vee b_1 &\approx a_2 \vee b_2.&
\shortintertext{%
So by point~\ref{P:eq-2} of this lemma, we get}
\varphi(a_1 \wedge b_1) &= \varphi( a_2 \wedge b_2)&
\qquad&\text{and}\qquad&
\varphi(a_1 \vee b_1)&=\varphi(a_2 \vee b_2).&
\end{alignat*}
So if we unfold Definition~\ref{D:d},
we see that
\begin{alignat*}{5}
\ld\varphi(a_1,b_1)
\ &=\ 
\varphi(a_1 \vee b_1) - \varphi(a_1 \vee b_1) \\
\ &=\ 
\varphi(a_2 \vee b_2) - \varphi(a_2 \vee b_2)
\ =\ 
\ld\varphi(a_2,b_2).\qedhere
\end{alignat*}
\end{proof}
%
%                  THE QUOTIENT LATTICE
%
When studying the Lebesgue integrable functions, $\LF$,
it is sometimes convenient
to consider the space~$\mathrm{L}^1 = \qvL{\LF}$
of integrable functions modulo equal-almost-everywhere
(see Example~\ref{E:eq-int}).
Of course,
one can consider the space $\qvL{L}$ 
for any valuation~$\varphi\colon L\ra E$.
We list some of the properties of~$\qvL{L}$
in Proposition~\ref{P:quotient-lattice}.
\begin{prop}
\label{P:quotient-lattice}
Let $E$ be an ordered Abelian group.\\
Let $L$ be a lattice.
Let $\varphi\colon L \ra E$ be a valuation.
Let $\approx$ be as in Definition~\ref{D:eq}.\\
Let $\qvL{L}$ denote the quotient set,
and let $q\colon L\ra \qvL{L}$ be the
quotient map. Then:
\begin{enumerate}
\item
The set
$\qvL{L}$ is lattice if
the operations are given by
\begin{equation*}
qa \wedge qb \,=\, q(a\wedge b),
\qquad
 qa \vee qb \,=\, q(a\vee b)\qquad\quad(a,b\in L).
\end{equation*} 
Then, in particular, $q\colon L\ra \qvL{L}$ is a lattice homomorphism.

\item
There is a unique map $\psi\colon\qvL{L}\ra E$
such that
\begin{equation*}
\psi(q(a)) \ =\ \varphi(a)\qquad\quad(a\in L).
\end{equation*}
This map ---
let us denote it by
 $\qvphi\varphi\colon \qvL{L}\ra E$ --- 
is a valuation.

\item
We have
\ $\ld{\qvphi{\varphi}}(\,qa,\,qb\,) \,=\,\ld\varphi(a,b)$\ \ 
for all \ $a,b\in L$.

\item
We have
\ $\ld{\qvphi{\varphi}}(\mathfrak{a},\mathfrak{b})=0
\ \implies \ 
\mathfrak{a} = \mathfrak{b}$\ \ 
for all \ $\mathfrak{a},\mathfrak{b}\in \qvL{L}$.
\end{enumerate}
\end{prop}
\begin{proof}
Follows from Proposition~\ref{P:eq}.
We leave the verification to the reader.
\end{proof}
\begin{rem}
Note that $\qvphi\varphi$ is Hausdorff 
(see Definition~\ref{D:val_Hausdorff}).
\end{rem}
%


\end{document}
