\documentclass[main.tex]{subfiles}
\begin{document}
\section{Valuations}
\noindent
%
%          DEFINITION OF MODULAR MAPS
%
\begin{dfn}
Let $L$ be a lattice, $E$ an ordered Abelian group,
and $\varphi\colon L \ra E$.
\begin{enumerate}
\item
$\varphi$ is \keyword{modular} provided that
\begin{equation*}
\varphi(a\wedge b) + \varphi(a \vee b)
\ =\ 
\varphi(a) + \varphi(b)
\qquad(a,b\in L).
\end{equation*}

\item
$\varphi$ is a \keyword{valuation}
provided that $\varphi$ is modular and order preserving.
\end{enumerate}
\end{dfn}
%
%                  LEMMA ON MODULARITY FOR MODULAR MAPS
%
\begin{lem}
\label{L:modular-map-modular}
Suppose $\varphi$ is modular.
For $\ell,u\in L$ with $\ell\leq u$, we have
\begin{equation}
\label{eq:modular-map}
\varphi(\,\ell \vee (a \wedge u)\,) 
\ =\ 
\varphi(\,(\ell\vee a)\wedge u\,)
\qquad (a\in L).
\end{equation}
\begin{proof}
The trick is to consider the expression 
$\varphi(\ell) + \varphi(a) + \varphi(u)$.
On the one hand,
\begin{align*}
\varphi(\ell) + \varphi(a) + \varphi(u)
\ &=\ \varphi(\ell\wedge a) + \varphi(\ell \vee a) + \varphi(u) \\
\ &=\ \varphi(\ell \wedge a)
      + \varphi(\,(\ell\vee a)\wedge u\,)
      + \varphi(a\vee u),
\end{align*}
where we have applied Equation~\eqref{eq:modular-map} twice.
On the other hand,
\begin{align*}
\varphi(\ell) + \varphi(a) + \varphi(u)
\ &=\ \varphi(\ell) + \varphi(a\wedge u) + \varphi(a \vee u) \\
\ &=\ \varphi(\ell \wedge a)
      + \varphi(\,\ell\vee (a\wedge u)\,)
      + \varphi(a\vee u).
\end{align*}
The difference,
$\varphi(\,(\ell\vee a)\wedge u\,)
- \varphi(\,\ell\vee (a\wedge u)\,)$,
must be zero.
\end{proof}

\end{lem}
\end{document}
