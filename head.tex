\documentclass[main.tex]{subfiles}
\begin{document} 

\title[A Generalisation of Measure and Integral]{Lattice Valuations,\\
A Generalisation of Measure and Integral}
\email{bram@westerbaan.name}

\author[A.A.~Westerbaan]{Bram Westerbaan}
%\date{\today \quad{\tiny \version}}


\vspace{1cm}
\input{circle.tex}

\vfill
\noindent
{\tiny
Thesis for the Master's Examination
Mathematics at the Radboud University Nijmegen,\\
supervised by prof.~dr.~A.C.M.~van Rooij
with second reader dr.~O.W.~van Gaans,\\
written by Abraham A.~Westerbaan, 
student number 0613622,
on November 16th 2012.

% The whitespace above is needed.
% If it is removed, the text above will be tiny,
% but the spacing between lines will be normal.
}
\thispagestyle{empty}
\clearpage
$\,$
\newpage
\begin{abstract}
Measure and integral are two closely related,
but distinct objects of study.
Nonetheless,
they are both real-valued \emph{lattice valuations}:
order preserving real-valued functions~$\varphi$
on a lattice~$L$
which are \emph{modular}, i.e.,
\begin{equation*}
\varphi(x) + \varphi(y) 
\,=\, 
\varphi(x\wedge y) + \varphi(x\vee y)\qquad(x,y\in L).
\end{equation*}
We unify measure and integral
by developing a theory for lattice valuations.
We allow these lattice valuations
to take their values from the reals,
or any suitable ordered Abelian group.
\end{abstract}

\clearpage
%
%                  FOREWORD
%
\section*{Foreword}
\noindent
In the summer of 2009  Bas Westerbaan
and I worked out an 
overly general approach
to the introduction of the Lebesgue measure and the Lebesgue integral
with the help of dr.~A.C.M.~van~Rooij.
The theory that is presented in this thesis
is based on the work done in that summer.

Since I was fortunate enough to be offered
a Ph.D.-position,
this thesis was written under time constraints.
Hence the text is not nearly as polished as 
I would like it to be,
and the proofs of some statements
have been left to the reader.
I hope the reader will be able to ignore the rough edges
and enjoy this fresh view on the old subject
of measure and integration.

I would like to thank  all my teachers
for showing me the beauty of mathematics.
In particular, I  thank
dr.~Mai Gehrke for showing me its elegance,
dr.~Wim Veldman for showing me its content, and
dr.~Henk Barendregt for showing me how it is written.
Furthermore, I am most grateful
to dr.~A.C.M.~van~Rooij
for his never relenting willingness 
to answer my questions
and note my errors.
\clearpage
\tableofcontents
\clearpage
\end{document}
