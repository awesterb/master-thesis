\documentclass[main.tex]{subfiles}
\begin{document}
\section{Epilogue}
\noindent
Starting from the similarity
between
 the Lebesgue measure
and the Lebesgue integral
as shown  on page~\pageref{S:intro}
I have tried to 
rebuild
a small part of the theory of
measure and integration 
in a more general setting.
When I look back at the result
I am most pleased that it was possible
to introduce the Lebesgue measure and the Lebesgue integral
with such natural and old primitives.
Indeed,
completeness and convexity
together
is nothing more than
the method of exhaustion
used by the ancient greeks 
to determine the area of the disk (see title page).

The price for simple primitives
seems to be that much more effort is
required to prove even the simplest statements,
as attested by the size of this text.
Of course,
the number of pages could be
greatly reduced if we worked with~$\R$
instead of any~$E$,
but even then
I doubt that the approach taken
in this thesis would
be suitable
for a first course on the Lebesgue measure
and the Lebesgue integral.

Whether the theory in this thesis
will bear any fruit
I cannot tell,
but nevertheless I am content,
because I have enjoyed writing it,
and I hope that you have enjoyed reading it as well.
\label{S:conclusion}
\end{document}
