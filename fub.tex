\documentclass[main.tex]{subfiles}
\begin{document}
\section{Fubini's Theorem}
\label{S:fub}
\noindent
In this section we study Fubini's Theorem.
We have not found a satisfactory generalisation of this theorem
to the setting of valuations.
However,
we will see that it is possible to split
the proof of Fubini's Theorem
into two parts,
so that the first part (Subsection~\ref{SS:fub_part1})
is algebraic in nature
and specific to the setting of step functions,
and the second part (Subsection~\ref{SS:fub_part2}) is more analytic in nature
and a consequence of a
general extension theorem for valuations (see Theorem~\ref{T:fubext}).

\subsection{Algebraic Part}
\label{SS:fub_part1}
Let us first formulate Fubini's Theorem.
This takes time.

Let $X$ be a set,
let $\mathcal{A}_X$ be a ring of subsets of~$X$,
and let 
\begin{equation*}
\mu_X\colon \mathcal{A}_X \ra \R
\end{equation*}
be a positive and additve map
(see Example~\ref{E:ring-val}).

Similarly, let $Y$ be a set,
let $\mathcal{A}_Y$ be a ring of subsets of~$Y$,
and let 
\begin{equation*}
\mu_Y\colon \mathcal{A}_Y \ra \R
\end{equation*}
be a positive and additve map.

Now, let $\mathcal{A}_{X\times Y}$
be the ring of subsets of~$X\times Y$
generated by the subsets
\begin{equation*}
\{\ \  A \times B\colon\ \  \text{$A\in \mathcal{A}_X$, \ \ 
 $B\in\mathcal{A}_Y$} \ \ \}
\end{equation*}
Let $\mu_{X\times Y} \colon \mathcal{A}_{X\times Y} \ra \R$
be the unique positive and additive map such that
\begin{equation*}
\mu_{X\times Y} (\,A\times B\,)
\ =\ \mu_X(A)\,\cdot\,\mu_Y(B)
\end{equation*}
for all~$A\in\mathcal{A}_X$ and $B\in\mathcal{A}_Y$.
Such $\mu_{X\times Y}$ exists,
as the reader can verify.

Let $F_X$ be the set of all $\mathcal{A}_X$-stepfunctions,
i.e., functions of the form
\begin{equation*}
\textstyle{\sum_{n=1}^N \ \lambda_n \cdot \mathbf{1}_{A_n}  },
\end{equation*}
where $A_1,\dotsc,A_N \in\mathcal{A}_X$
and $\lambda_i \in \R$.
As the reader may verify,
the expression
\begin{equation*}
\varphi_X(\,\textstyle{\sum_{n=1}^N \ \lambda_n \cdot \mathbf{1}_{A_n}}\,)
\ =\ \textstyle{\sum_{n=1}^N \ \lambda_n \cdot \mu_X (A_n)}
\end{equation*}
determines a unique positive and linear
map $\varphi_X\colon F_X \ra \R$. 

Similarly, we get
a map $\varphi_Y \colon F_Y \ra \R$,
and a map $\varphi_{X\times Y} \colon F_{X\times Y} \ra \R$.

One can verify that any $f\in F_{X\times Y}$
is of the form
\begin{equation*}
\textstyle{\sum_{n=1}^N} \ \lambda_n \cdot\mathbf{1}_{A_n \times B_n},
\end{equation*}
where $A_1,\dotsc,A_N\in \mathcal{A}_X$,
and $B_1,\dotsc,B_N\in\mathcal{A}_Y$,
and $\lambda_n \in \R$.

So it is not hard to verify that the  equality
\begin{equation*}
\mathcal{F}_X \bigl(\ 
\textstyle{\sum_{n=1}^N} \ \lambda_n \cdot\mathbf{1}_{A_n \times B_n}\ \bigr)
\ \ =\ \ 
\textstyle{\sum_{n=1}^N} \ \lambda_n \cdot \mu_X(A_n) \cdot \mathbf{1}_{B_n}
\end{equation*}
gives us positive and linear map 
$\mathcal{F}_X \colon F_{X\times Y} \ra F_Y$.

Let $f\in F_{X\times Y}$ be given.
For each~$y\in Y$,
define $f^y\in F_X$ by, for all~$x\in X$,
\begin{equation*}
f^y(x) \ =\  f(x,y)
\end{equation*}
One can easily verify that we have, for all~$y\in Y$,
\begin{equation*}
\mathcal{F}_X (f) (y)\ =\ \varphi_X (f^y).
\end{equation*}
So, informally, $\mathcal{F}_{X}(f) = \int f(x,y)\,dx$.

Since $\varphi_X$ is linear
one quickly sees that $\varphi_Y \circ \mathcal{F}_X = \varphi_{X\times Y}$.
Informally,
\begin{equation*}
\int\int f(x,y)\, \,dx \,dy
\ =\ 
\int f
\qquad(f\in F_{X\times Y}).
\end{equation*}
This is Fubini's Theorem for stepfunctions, $F_{X\times Y}$.

Of course, we want to prove
Fubini's Theorem
for the extension~$\ol{F}_{X\times Y}$.

So let us assume
$\varphi_X$, $\varphi_Y$, and $\varphi_{X\times Y}$ are extendible
(see Definition~\ref{D:extendible}).

Alternatively,
we can assume that 
$\varphi_Y$ and $\varphi_Y$
are continuous (see Definition~\ref{D:continuous});
we leave it to the reader to verify that then
 $\varphi_{X\times Y}$ is continuous
(which is not too easy),
and so~$\varphi_{X\times Y}$ is extendible,
since~$\R$ is benign (see Definition~\ref{D:benign}).

Note that it is not possible to find 
an $\ol{\mathcal{F}}_{Y}\colon \ol{F}_{X\times Y} \ra \ol{F}_Y$
such that, for all~$y\in Y$,
\begin{equation*}
\ol{\mathcal{F}}_Y(f)(y) \ = \  \ol{\varphi}_X(f^y).
\end{equation*}
So to formulate Fubini's Theorem for~$\ol{F}_{X\times Y}$
we need a slightly different approach than
the one we used for the stepfunctions.

Consider the space $E_Y \eqdf \qvL{\ol{F_Y}}$
(see Proposition~\ref{P:quotient-lattice}).
We leave it to the reader to
verify that~$E_Y$ can be endowed with 
the structure of an ordered Abelian group,
and a fitting uniformity
(see Definition~\ref{D:uniformity})
such that the map $F_{X\times Y} \ra E_Y$
given by $f\mapsto \mathcal{F}_X(f) /\approx$
is a group homomorphism.

We can now formulate Fubini's Theorem as follows.
\begin{equation}
\label{eq:fub}
\left[\quad
\begin{minipage}{.7\columnwidth}
The valuation 
\begin{equation*}
\mathcal{F}_X\colon F_{X\times Y} \ra \qvL{\ol{F_Y}}
\end{equation*}
is extendible, 
and $\mathrm{dom}(\ol{\varphi}_{X\times Y})
\,\subseteq\, 
 \mathrm{dom}(\ol{\mathcal{F}}_X)$,
and 
\begin{equation*}
\ol\varphi_{X\times Y}(f) \ =\ 
(\,\qvphi{\ol\varphi} \ \circ\  \ol{\mathcal{F}_X} \,)(f)
\end{equation*}
for all $f\in\ol{F}_{X\times Y}$.
\end{minipage}
\right.
\end{equation}
Of course,
to be true to the usual formulation of Fubini's Theorem
we would need to prove that
that $\ol{\mathcal{F}}_X(f)(y) = \ol{\varphi}_X(f^y)$
for \emph{almost} all~$y\in Y$.
We will not do this.

\subsection{Extension of Operations}
\label{SS:fub_part2}
Let $\vs{V}{L}\varphi{E}$ be a valuation system.
In Section~\ref{S:closedness}
we saw that the completion~$\ol{L}$ of~$\varphi$ is \emph{closed} under
various operations.
It is also possible to \emph{extend} operations
 to~$\ol{L}$, which are (initially only) defined on~$L$.
The aim of this subsection is to prove Theorem~\ref{T:fubext}
which is an example of this principle
in case that~$E$ has a fitting uniformity~$\Phi$
(see Definition~\ref{D:uniformity}).

It should be noted that from the methods
found in the proof of Theorem~\ref{T:fubext}
one can easily obtain a stronger version of this theorem.
More interestingly,
the patterns in the proof strongly suggest that
we should make a study of the uniform structure on~$L$
given by the relations $\ol{\ve}$ (where $\ve\in\Phi$) defined by
\begin{equation*}
a \ \ol{\ve}\ b \quad\iff\quad 0\ \ve\ \ld\varphi(a,b)\qquad\quad(a,b\in L).
\end{equation*}

However, we have refrained from proving a stronger version of the theorem
and introducing yet another notion of uniform structure.
Indeed,
we have not found a clear favorite 
 among the several approaches to the strengthening
of the theorem and the axiomatisation of the uniform structure on~$L$.
Accordingly,
we introduce few new notions,
and the proofs in this subsection are sometimes ad hoc.

One new notion we do present is that of
\emph{weak $\varphi$-convergence}
(see Definition~\ref{D:weak-phi-conv}).
As the name suggests,
$\varphi$-convergence (see Definition~\ref{D:seq-phi-conv})
implies weak $\varphi$-convergence
(see Lemma~\ref{L:weak-phi-conv-implies-phi-conv}),
but the reverse implication does not hold
(see Example~\ref{E:weak-phi-conv}).
Nevertheless,
any weakly $\varphi$-convergent sequence
has a $\varphi$-convergent subsequence
(see Proposition~\ref{P:weak-phi-conv-subseq}).

Due to this all 
the notions of $\varphi$-convergent
and weakly $\varphi$-convergent
can be used somewhat interchangeably.
The main merit of ``weak $\varphi$-convergent''
is that some useful statements concerning it
(see Lemma~\ref{L:phi-conv-diag} and  Lemma~\ref{L:phi-conv-inf}) 
can be easily proven,
while it is not clear if the same statement (or a variant) holds for
``$\varphi$-convergent''.

The  main application of Theorem~\ref{T:fubext} 
is the proof of Fubini's Theorem~\ref{T:fub}.
Let us start the work towards a proof.
%
%                  WEAK PHI-CONVERGENCE
%
\begin{dfn}
\label{D:weak-phi-conv}
\label{D:phi-conv-2}
Let $E$ be an ordered Abelian group.\\
Let $\Phi$ be a fitting uniformity on~$E$.\\
Let $L$ be a lattice, and let $\varphi\colon L \ra E$ be a valuation.\\
Let $a \in L$ and let $a_1,a_2,\dotsc$ be a sequence in~$L$.\\
We say $a_1,a_2,\dotsc$ 
\keyword{weakly $\varphi$-converges} to~$a$
if
\begin{equation*}
\forall \ve \in \Phi \quad \exists N \quad \forall n \geq N \quad
[\ \ 0\quad\ve\quad \ld\varphi(a_n,a)  \ \ ].
\end{equation*}
\end{dfn}
%
%                  WEAK PHI-CONVERGENCE IMPLIES PHI-CONVERGENCE
%
\begin{lem}
\label{L:weak-phi-conv-implies-phi-conv}
Let $E$ be an ordered Abelian group.\\
Let $\Phi$ be a fitting uniformity on~$E$.\\
Let $L$ be a lattice, and let $\varphi\colon L \ra E$ be a complete valuation.\\
Let $a \in L$ and let $a_1,a_2,\dotsc$ be a sequence in~$L$. We have:
\begin{equation*}
\text{$a_1,a_2,\dotsc$ $\varphi$-converges to~$a$}
\quad\implies\quad
\text{$a_1,a_2,\dotsc$ weakly $\varphi$-converges to~$a$}.
\end{equation*}
\end{lem}
\begin{proof}
Let $\ve \in \Phi$ be given.
To prove that $a_1,a_2,\dotsc$ weakly $\varphi$-converges to~$a$
we must find an~$N\in \N$ such that
$0 \ \ve\ \ld\varphi(a_n,a)$ for all~$n\geq N$.

To find such~$N$ takes some preparation,
so bear with us.

Since $a_1,a_2,\dotsc$  $\varphi$-converges to~$a$,
i.e., $a_1,a,a_2,a,\dotsc$ is $\varphi$-convergent,
we know that $a_1,a,a_2,a,\dotsc$ is upper $\varphi$-convergent.
That is, we know the following exists.
\begin{equation}
\label{eq:L:weak-phi-conv-implies-phi-conv-1}
u\ \eqdf \ 
\bw_N \bv_{n\geq N}\  \varphi(a\vee a_N\vee \dotsb \vee a_n)
\end{equation}
In particular,
we see that for each~$N\in \N$,
the sequence 
\begin{equation*}
a\vee a_N \,\leq\, a\vee a_N\vee a_{N+1}\,\leq\,\dotsb
\end{equation*}
is $\varphi$-convergent
(in the sense of Definition~\ref{D:phi-conv}).
Since $\varphi$ is complete,
we see that $\ol{a}_N \eqdf \bv_{n\geq N} \, a \vee a_n$ exists in~$L$
and that $\varphi(\ol{a}_N) 
= \bv_{n \geq N} \,\varphi(a \vee a_N \vee\dotsb \vee a_n)$.
Now, note that we
 can phrase Statement~\eqref{eq:L:weak-phi-conv-implies-phi-conv-1}
as $u = \bw_N \varphi(\ol{a}_N)$.

Since $a_1,a_2,\dotsc$ $\varphi$-converges to~$a$,
we know that the sequence $a_1,a,a_2,a,\dotsc$ is
lower $\varphi$-convergent. That is, the following exists.
\begin{equation}
\label{eq:L:weak-phi-conv-implies-phi-conv-2}
\ell\ \eqdf \ 
\bv_N \bw_{n\geq N}\  \varphi(a\wedge a_N\wedge \dotsb \wedge a_n)
\end{equation}
In particular we see that for each~$N\in\N$ the sequence
\begin{equation*}
a\wedge a_N \,\geq\, a\wedge a_N\wedge a_{N+1}\,\geq\,\dotsb
\end{equation*}
is $\varphi$-convergent.
As before, $\ul{a}_N \eqdf \bw_{n\geq N}\, a\wedge a_n$ exists,
and $\ell = \bv_N \varphi(\ul{a}_N)$.

Now, note that for each~$N\in \N$ and $n\geq N$ we have
\begin{equation*}
\ul{a}_N \ \leq\ a\wedge a_n  \ \leq\ a\vee a_n  \ \leq\ \ol{a}_N.
\end{equation*}
In particular, we have the following inequalities.
\begin{equation}
\label{eq:L:weak-phi-conv-implies-phi-conv-3}
\varphi(\ul{a}_N) \ \leq\ \varphi(a\wedge a_n)  
\ \leq\ \varphi(a\vee a_n)  \ \leq\ \varphi(\ol{a}_N).
\end{equation}
Recall that we want to prove (for some~$N$) that
$0 \ \ve\ \ld\varphi(a,a_n)$.
That is, we must show that $\varphi(a\wedge a_n) \ \ve\ \varphi(a\vee a_n)$
(see Definition~\ref{D:uniformity}\ref{E-add}).
To prove this it suffices to show 
that $\varphi(\ul{a}_N) \ \ve\ \varphi(\ol{a}_N)$
as we can see from Statement~\eqref{eq:L:weak-phi-conv-implies-phi-conv-3}
(see  Definition~\ref{D:uniformity}\ref{E-ord}).

So to complete the proof of this lemma,
we need to find an~$N\in\N$ with
\begin{equation*}
\varphi(\ul{a}_N) \ \ \ve\ \ \varphi(\ol{a}_N).
\end{equation*}

Since $a_1,a,a_2,a,\dotsc$
is $\varphi$-convergent,
we know that $u = \ell$.
Now, recall that we have $u = \bw_N \varphi(\ol{a}_N)$
and $\ell = \bw_N \varphi(\ol{a}_N)$.
Determine
an $N$ with
\begin{equation*}
\varphi(\ul{a}_N) \ \ \dt\ve\ \ \ell
\qquad\text{and}\qquad
u \ \ \dt\ve \ \ \varphi(\ol{a}_N)
\end{equation*}
using Definition~\ref{D:uniformity}\ref{E-inf-conv}
and Lemma~\ref{L:E-prop}\ref{E-sup-conv}.
Hence we see that $\varphi(\underline{a}_N) \ \ve \ \varphi(\ol{a}_N)$.
\end{proof}
%
%                  EXAMPLE ON WEAKLY PHI-CONVERGENCE
%
\begin{ex}
\label{E:weak-phi-conv}
Let $\varphi$ be a complete valuation.
We know that $\varphi$-convergence
implies weak $\varphi$-convergence (see Lemma~\ref{D:weak-phi-conv}).
The reverse implication does not hold.
Indeed,
consider the Lebesgue integral $\Lphi\colon \LF \ra \R$
and the sequence
$f_1,f_2, \dotsc$ of functions on~$\R$
given by $f_n = \frac{1}{n}\cdot\mathbf{1}_{[n,n+1]}$.
Note that 
\begin{equation*}
\textstyle{\ld\Lphi(f_n, \mathbf{0}) \ =\  \Lphi(|f_n|) 
\ =\ \frac{1}{n}.}
\end{equation*}
So we see that $f_1,f_2,\dotsc$ weakly $\Lphi$-converges to~$\mathbf{0}$.

However, we prove that the sequence $f_1,f_2,\dotsc$ 
does not $\Lphi$-converge to~$\mathbf{0}$.
Indeed, assume
(towards a contradiction) that
 $f_1,f_2,\dotsc$ does $\Lphi$-converge to~$\mathbf{0}$.
Then $f_1,f_2,\dotsc$ is $\Lphi$-convergent.
So in particular $f_1,f_2,\dotsc$ is upper $\Lphi$-convergent
(see Definition~\ref{D:seq-phi-conv}).
So we know that the following exists.
\begin{equation*}
\pulim{\Lphi}{n} f_n 
\ = \ \bw_N \bv_{n\geq N} \,\Lphi(f_N\vee \dotsb \vee f_n)
\end{equation*}
Now, note that any $N\in\N$ and $n\geq N$ we have
$f_N\vee \dotsb \vee f_n = f_N + \dotsb + f_n$,
so 
\begin{equation*}
\textstyle{\Lphi(f_N\vee \dotsb\vee f_n) 
\ =\ \Lphi(f_N) + \dotsb + \Lphi(f_n) \ = \ 
 \frac{1}{N} + \dotsb + \frac{1}{n}}.
\end{equation*}
%
%
%
So we see that  $\sum_{n} \frac{1}{n} 
= \bigvee_{n} \Lphi(f_1\vee\dotsb \vee f_n)$,
 which is absurd.
Hence $f_1,f_2,\dotsc$ 
does not $\Lphi$-converge to~$\mathbf{0}$.
\end{ex}
%
%                  LEMMA: SUMMABILITY + WEAK PHI-CONV ==> PHI-CONV
%
Let $\varphi$ be a complete valuation.
If $a_1,a_2,\dotsc$ weakly
$\varphi$-converges to~$a$,
then $a_1,a_2,\dotsc$ might not $\varphi$-converge to~$a$
(as we saw in Example~\ref{E:weak-phi-conv}).
However, 
there is always a subsequence of~$a_1,a_2,\dotsc$
which does $\varphi$-converge to~$a$
(see Proposition~\ref{P:weak-phi-conv-subseq}).
To prove this, we need a lemma.
\begin{lem}
\label{L:weak-phi-conv-sum}
Let $E$ be an ordered Abelian group.\\
Let $\Phi$ be a fitting uniformity on~$E$.\\
Let $L$ be a lattice, and let $\varphi\colon L \ra E$ be a valuation.\\
Let $a \in L$ and let $a_1,a_2,\dotsc$ be a sequence in~$L$ 
that weakly $\varphi$-converges to~$a$.\\
Assume that 
$\textstyle{\sum_{n}} \,\ld\varphi(a,a_n) 
\ \eqdf \ \bv_N  \sum_{n=1}^N  \,\ld\varphi(a,a_n)$ exists.\\
Then $a_1,a_2,\dotsc$ $\varphi$-converges to~$a$.
\end{lem}
\begin{proof}
To prove that $a_1,a_2,\dotsc$ $\varphi$-converges to~$a$,
we must show that $a_1,a,a_2,a,\dotsc$ is $\varphi$-convergent
(see Definition~\ref{D:seq-phi-conv-to}).
For this, 
we must show that the following exist,
\begin{alignat*}{3}
u \ &\eqdf\  \bw_N \bv_{n\geq N} \ \varphi(a \vee a_N \vee \dotsb \vee a_n)\\
\ell \ &\eqdf\  
\bv_N \bw_{n\geq N} \ \varphi(a \wedge a_N \wedge \dotsb \wedge a_n),
\end{alignat*}
and we must prove that $\ell = u$.

Let $N\in \N$ be given.
We prove that $\bv_{n\geq N} \ \varphi(a\vee a_N \vee \dotsb \vee a_n)$ exists.
Let us write $a_n ' \eqdf a\vee a_N \vee \dotsb \vee a_n$
for brevity. 
 To prove that $\bv_{n\geq N} \ \varphi(a_n')$ exists,
we want to use the fact that~$E$ is $R$-complete
(see Proposition~\ref{L:E-R-complete}).
So the task at hand is to study given $n\geq N$ the 
value $\varphi(a_{n+1}') - \varphi(a_n')$. Note that
\begin{equation}
\label{eq:L:weak-phi-conv-sum-1}
\varphi(a_{n+1}') - \varphi(a_n')
\ \ =\  \ \ld\varphi(\,a_{n+1}',\,a_n'\,)
\ \ =\ \ \ld\varphi(\ a_{n}'\vee a_{n+1},\  a_n'\vee a\ ),
\end{equation}
since $a_{n+1}' = a_{n}'\vee a_{n+1}$
and $a_n' = a_n'\vee a $ (as $a\leq a_n'$).
By Lemma~\ref{L:curry-wc-unif} we have
\begin{equation}
\label{eq:L:weak-phi-conv-sum-2}
\ld\varphi(\ a_n' \vee a_{n+1},\ a_n' \vee a\ )
\ \ \leq\ \ 
\ld\varphi(\,a_{n+1},\,a\,).
\end{equation}
So if we combine Statement~\eqref{eq:L:weak-phi-conv-sum-1}
and Statement~\eqref{eq:L:weak-phi-conv-sum-2}
we get 
\begin{equation}
\label{eq:L:weak-phi-conv-sum-3}
\varphi(a_{n+1}') - \varphi(a_n')
\ \ \leq\ \ 
\ld\varphi(a_{n+1},a).
\end{equation}
Recall that we have assumed that $\sum_n \ld\varphi(a_n,a)$ exists.
From this,
Statement~\eqref{eq:L:weak-phi-conv-sum-3},
and the fact that~$E$ is $R$-complete
it follows that $\bv_{n\geq N} \,\varphi(a_n')$ exists.

We prove that $u\eqdf 
\bw_N \bv_{n\geq N} \ \varphi(a\vee a_N\vee \dotsb \vee a_n)$ exists.
Again we use the fact that~$E$ is $R$-complete:
it is sufficient to prove that $\xi_N - \xi_{N+1} \leq \ld\varphi(a,a_N)$
where 
\begin{equation*}
\xi_N \ \eqdf\  \bv_{n\geq N} \ \varphi(a\vee a_N \vee \dotsb \vee a_n).
\end{equation*}
Let $n\in \N$ be given.
It is useful to begin by  considering the value
$\varphi(a_N'') - \varphi(a_{N+1}'')$
where 
$a_N''\eqdf a\vee a_N \vee \dotsb \vee a_n$
for all $N\leq n$. We obtain
\begin{equation*}
\varphi(a_N'') - \varphi(a_{N+1}'')
\ \ \leq\ \ \ld\varphi(a,a_N)
\end{equation*}
using a similar reasoning as before. 
Written differently, we have
\begin{equation*}
\varphi(a \vee a_N \vee\dotsb\vee a_n) 
\ \ \leq\ \ \ld\varphi(a,a_N) \ +\ 
\varphi(a \vee a_{N+1} \vee \dotsb \vee a_n)
\end{equation*}
for all~$N\in \N$ and $n\geq N$.
This implies
\begin{alignat*}{3}
\bv_{n\geq N}\ \varphi(a \vee a_N \vee\dotsb\vee a_n) 
\ \ &\leq\ \ 
\ld\varphi(a,a_N) \ +\ 
\varphi(a \vee a_{N+1} \vee \dotsb \vee a_n) \\
\ \ &\leq\ \ 
\ld\varphi(a,a_N) \ +\ 
\bv_{n\geq N+1}\ \varphi(a \vee a_{N+1} \vee \dotsb \vee a_n).
\end{alignat*}
Or in other words, $\xi_{N} \leq \ld\varphi(a,a_N) + \xi_{N+1}$.
Hence we have proven:
 \begin{equation*}
u\ \eqdf\ 
\bw_N \bv_{n\geq N} \ \varphi(a\vee a_N\vee \dotsb \vee a_n)
\quad\text{exists.}
\end{equation*}
Of course,
the above argument can be adapted to yield:
 \begin{equation*}
\ell\ \eqdf\ 
\bv_N \bw_{n\geq N} \ \varphi(a\wedge a_N\wedge \dotsb \wedge a_n)
\quad\text{exists.}
\end{equation*}

\vspace{.3em}
\noindent It remains to be shown that $\ell = u$.
Let $\ve \in \Phi$ be given.
Since reader can easily verify that $\ell \leq u$,
to prove that $\ell = u$,
it suffices to show that $\ell \se u$
(see Definition~\ref{D:uniformity}\ref{E-haus}).
Let~$N\in\N$ be given. Note that we have the following inequalities.
\begin{equation*}
\bw_{n\geq N}\ \varphi(a\wedge a_N\wedge \dotsb \wedge a_n)
\ \ \leq\ \ 
\ell \ \leq\ u
\ \ \leq\ \ 
\bv_{n\geq N}\ \varphi(a\vee a_N\vee \dotsb \vee a_n).
\end{equation*}
So to prove $\ell \se u$,
it suffices to show that for some~$N$
(see Definition~\ref{D:uniformity}\ref{E-ord}),
\begin{equation}
\label{eq:L:weak-phi-conv-sum-4}
\bw_{n\geq N}\ \varphi(a\wedge a_N\wedge \dotsb \wedge a_n)
\quad\ve\quad
\bv_{n\geq N}\ \varphi(a\vee a_N\vee \dotsb \vee a_n).
\end{equation}
Since $\sum_n \ld\varphi(a,a_n)$ exists,
we can find an~$N\in\N$ such that
(see Lemma~\ref{L:E-prop}\ref{E-sup-conv})
\begin{equation}
\label{eq:L:weak-phi-conv-sum-4b}
0\quad \dtn\ve4 \quad \ld\varphi(a,a_N) \,+\, \dotsb \,+\, \ld\varphi(a,a_n)
\qquad\quad(n\geq N).
\end{equation}
We will prove that Statement~\eqref{eq:L:weak-phi-conv-sum-4}
holds for this~$N$.
Since $\bw_{n\geq N}\ \varphi(a\wedge a_N\wedge \dotsb \wedge a_n)$
and 
$\bv_{n\geq N}\ \varphi(a\vee a_N\vee \dotsb \vee a_n)$
exist,
we can find $n\geq N$ such that
\begin{alignat*}{3}
\bw_{n\geq N}\ \varphi(a\wedge a_N\wedge \dotsb \wedge a_n)
\quad&\dtn\ve4 \quad
\varphi(a\wedge a_N\wedge \dotsb \wedge a_n)\\
\varphi(a\vee a_N\vee \dotsb \vee a_n)
\quad&\dtn\ve4 \quad
\bv_{n\geq N}\ \varphi(a\vee a_N\vee \dotsb \vee a_n).
\end{alignat*}
So to prove that Statement~\eqref{eq:L:weak-phi-conv-sum-4}
it suffices to show that
\begin{equation}
\label{eq:L:weak-phi-conv-sum-5}
\varphi(a\wedge a_N\wedge \dotsb \wedge a_n)
\quad\dt\ve\quad
\varphi(a\vee a_N\vee \dotsb \vee a_n).
\end{equation}
Note that we have the following inequalities.
\begin{equation*}
\varphi(a\wedge a_N\wedge \dotsb \wedge a_n)
\ \ \leq\ \ \varphi(a)\ \ \leq\ \ 
\varphi(a\vee a_N\vee \dotsb \vee a_n).
\end{equation*}
So to prove that Statement~\eqref{eq:L:weak-phi-conv-sum-5}
holds, we will show that
\begin{equation}
\label{eq:L:weak-phi-conv-sum-6}
\varphi(a\wedge a_N\wedge \dotsb \wedge a_n)
\quad\dtn\ve4\quad\varphi(a)\quad\dtn\ve4\quad
\varphi(a\vee a_N\vee \dotsb \vee a_n).
\end{equation}
Now, note that vteration of Lemma~\ref{L:wv-unif} yields
\begin{alignat*}{3}
\varphi(a\wedge a_N\wedge \dotsb \wedge a_n)
\,-\, \varphi(a)
\ \ &=\ \ 
\ld\varphi(\ a\wedge a_N\wedge\dotsb \wedge a_n,\ a\ )  \\
\ \ &=\ \ 
\ld\varphi(\ a\wedge a_N\wedge\dotsb \wedge a_n,
                      \ a\wedge a\wedge \dotsb \wedge a \  ) \\
\ \ &\leq\ \ 
\ld\varphi(a_N,a) \,+\,\dotsb \,+\,\ld\varphi(a_n,a).
\end{alignat*}
So by Statement~\eqref{eq:L:weak-phi-conv-sum-4b}
and Definition~\ref{D:uniformity}\ref{E-add} we get
\begin{equation*}
\varphi(a\wedge a_N\wedge \dotsb \wedge a_n)
\quad\dtn\ve4\quad\varphi(a).
\end{equation*}
Using a similar argument,
we obtain 
\begin{equation*}
\varphi(a)\quad\dtn\ve4\quad
\varphi(a\vee a_N\vee \dotsb \vee a_n).
\end{equation*}
So we have proven Statement~\eqref{eq:L:weak-phi-conv-sum-6} and 
thereby completed the proof.
\end{proof}
%
%                  PROPOSITION ON PHI-CONV SUBSEQUENCE OF A WEAK PHI-CONV SEQ
%
\begin{prop}
\label{P:weak-phi-conv-subseq}
Let $E$ be an ordered Abelian group.\\
Let $\Phi$ be a fitting uniformity on~$E$.\\
Let $L$ be a lattice, and $\varphi\colon L \ra E$ a valuation.\\
Let $a_1,a_2,\dotsc$ be a sequence in~$L$
that weakly $\varphi$-converges to some~$a\in L$.\\
Then: there are $j_1<j_2<\dotsb$ in~$\N$
such that $a_{j_1},a_{j_2},\dotsc$
$\varphi$-converges to~$a$.
\end{prop}
\begin{proof}
It suffices to find $j_1<j_2<\dotsb$ in~$\N$ such that
$\sum_k \ \ld\varphi(a,a_{j_k})$ exists.
Indeed,
then we have $a_{j_1},a_{j_2},\dotsc$
weakly $\varphi$-converging to~$a$
since $a_1,a_2,\dotsc$ $\varphi$-converges to~$a$.
So by Lemma~\ref{L:weak-phi-conv-sum} we 
$a_{j_1},a_{j_2},\dotsc$ $\varphi$-converges to~$a$, as we must prove.

Let $\ve_1',\ve_2',\dotsc$ be an enumeration of~$\Phi$.
Pick $\ve_1,\ve_2,\dotsc$ in~$\Phi$ such that for all~$n$,
\begin{equation*}
\ve_{n} \,\leq\, \ve_n ' \qquad\text{and}\qquad  \ve_{n+1}\,\leq\,\dt{\ve_{n}}.
\end{equation*}
Note that for all $N\in \N$ and $n\geq N+1$
we have
(see Notation~\ref{N:unif}),
\begin{equation}
\label{eq:P:weak-phi-conv-subseq-1}
\ve_{N+1} + \dotsb + \ve_{n} \ \leq\  \ve_N.
\end{equation}

Pick $j_1 < j_2 < \dotsb$ in~$\N$ such that for all~$k\in\N$, 
\begin{equation*}
0\quad\ve_{k+1}\quad \ld\varphi(a, a_{j_k}).
\end{equation*}
Then by Statement~\eqref{eq:P:weak-phi-conv-subseq-1} 
and Definition~\ref{D:uniformity}\ref{E-add} for all~$N\in\N$ and $n\geq N$,
\begin{equation}
\label{eq:P:weak-phi-conv-subseq-2}
0\quad\ve_{N}\quad 
\ld\varphi(a, a_{j_N}) \,+\, \dotsb \,+\, \ld\varphi(a,a_{j_n}).
\end{equation}

Recall that we need to prove that $\sum_k\ \ld\varphi(a,a_{j_k})$ exists.
Let $\ve \in \Phi$ be given.
By Lemma~\ref{L:E-prop}\ref{E-bound-sup}
it suffices to find an $N\in\N$ such that for all~$n\geq N$,
\begin{equation}
\label{eq:P:weak-phi-conv-subseq-3}
0\quad\ve\quad \ld\varphi(a,a_{j_N}) \,+\, \dotsb \,+\, \ld\varphi(a,a_{j_n}).
\end{equation}
Since $\ve_1',\ve_2',\dotsc$ enumerates~$\Phi$
we can find an $N\in \N$ such that $\ve_N'=\ve$.
Recall that $\ve \leq \ve_N' \leq \ve_N$.
So Statement~\eqref{eq:P:weak-phi-conv-subseq-3}
follows directly from Statement~\eqref{eq:P:weak-phi-conv-subseq-2}.
\end{proof}
%
%                  LEMMA ON DIAGONALISATION OF PHI-CONV SEQUENCES
%
\begin{lem}
\label{L:phi-conv-diag}
Let $E$ be an ordered Abelian group.\\
Let $\Phi$ be a fitting uniformity on~$E$.\\
Let $L$ be a lattice, and $\varphi\colon L \ra E$ a valuation.\\
Let $a_1,a_2,\dotsc$ be a sequence in~$L$
which weakly $\varphi$-converges to some~$a\in L$\\
For each~$N\in\N$, let $b^N_1, b^N_2, \dotsc$
be a sequence in~$L$ which weakly $\varphi$-converges to~$a_N$.\\
Then there are $j_1 < j_2 < \dotsb$ in~$\N$ such that
 $b^1_{j_1},\,b^2_{j_2},\,\dotsc$
weakly $\varphi$-converges to~$a$.
\end{lem}
\begin{proof}
To find a suitable sequence $j_1 < j_2 <\dotsb$ we need some preparation.

We know that~$\Phi$ is countable
(see Definition~\ref{D:fitting-uniformity}).
Let $\ve_1',\,\ve_2',\,\dotsc$ be an enumeration of~$\Phi$.
Define a sequence $\ve_1 \geq \ve_2 \geq \dotsb$ in~$\Phi$
(see Notation~\ref{N:unif-plus})
by 
\begin{equation*}
\ve_1 \ = \ \ve_1'
\qquad\text{and}\qquad 
\ve_{n+1} \ =\ \ve_n \wedge \ve_{n+1}'.
\end{equation*}
Note that we have $\ve_n \leq \ve_n'$ for all~$n$.

Let~$N\in\N$ be given.
Since $b^N_1,b^N_2,\dotsc$ weakly $\varphi$-converges to~$a_N$,
we know by Definition~\ref{D:weak-phi-conv} that there is an~$M\in\N$
such that $\ld\varphi(b^N_n, a_N)\ \ve_N\ 0$
for all~$n\geq M$.

Now,
choose $j_1<j_2<\dotsb$ such that 
$\ld\varphi(b^N_n, a_N)\ \ve_N\ 0$
for all~$n\geq j_N$.

We will prove that $b^1_{j_1},\,b^2_{j_2},\dotsc$ weakly $\varphi$-converges
to~$a$.
Let $\ve\in\Phi$ be given.
We must find an~$\mathfrak{n}\in\N$
such that $\ld\varphi(b^N_{j_N},a_N) \ \ve\ 0$
for all~$N\geq \mathfrak{n}$
(see Definition~\ref{D:phi-conv-2}).

Find an~$k\in\N$ such that $\dt\ve = \varepsilon_k'$.
(Recall that $\ve_1',\ve_2',\dotsc$ enumerates~$\Phi$.)

Pick~$\mathfrak{n}\geq k$
such that $\ld\varphi(a_N,a) \ \dt\ve\ 0$.
We prove that $\ld\varphi(b_{j_N}^N,a) \ \ve\ 0$
for all~$N\geq \mathfrak{n}$.
Let $N\geq\mathfrak{n}$ be given.
We have $\ld\varphi(b^N_{j_N},a_N) \ \ve_N \ 0$
by choice of~$j_N$.
So in particular $\ld\varphi(b^N_{j_N},a_N) \ \dt\ve \ 0$
since $\dt\ve = \ve_k'\geq\ve_k \geq \ve_{\mathfrak{n}} \geq \ve_N$.

Further,
we have $\ld\varphi(a_N,a) \ \dt\ve\ 0$
since $N\geq \mathfrak{n}$.

So by property~\ref{E-add} 
of a fitting uniformity (see Definition~\ref{D:fitting-uniformity})
we have
\begin{equation*}
\ld\varphi(b_{j_N}^N,a_N) \,+\,  \ld\varphi(a_N,a) 
\quad\dt\ve\quad 
\ld\varphi(a_N,a)
\quad\dt\ve\quad
0.
\end{equation*}
So by property~\ref{E-half} of a fitting uniformity
we have
\begin{equation*}
\ld\varphi(b_{j_N}^N,a_N) \,+\,  \ld\varphi(a_N,a) 
\quad\ve\quad 
0.
\end{equation*}
Now,
by points~\ref{d-metric_pos} and~\ref{d-metric_triangle}
of Lemma~\ref{L:d-metric}
we get
\begin{equation*}
0 \ \leq\ 
\ld\varphi(b_{j_N}^N, a) \ \leq\ 
\ld\varphi(b_{j_N}^N,a_N) \,+\,  \ld\varphi(a_N,a).
\end{equation*}
So by property~\ref{E-ord} we get $\ld\varphi(b^N_{j_N},a) \ \ve\ 0$.
\end{proof}
%
%                  LEMMA ON WEAK PHI-CONVERGENCE AND THE BINARY INFIMUM
%
\begin{lem}
\label{L:phi-conv-inf}
Let $E$ be an ordered Abelian group.\\
Let~$\Phi$ be a fitting uniformity on~$E$.\\
Let~$L$ be a lattice, and $\varphi\colon L\ra E$ a valuation.
Let $a,b\in L$ be given.\\
Let $a_1,a_2,\dotsc$ 
be a sequence in~$L$ which weakly $\varphi$-converges to~$a$.\\
Let $b_1,b_2,\dotsc$
be a sequence in~$L$ which weakly $\varphi$-converges to~$b$.\\
Then $a_1\wedge b_1,\ a_2\wedge b_2,\ \dotsc$
weakly $\varphi$-converges to~$a\wedge b$,\\
and $a_1\vee b_1,\ a_2\vee b_2,\ \dotsc$
weakly $\varphi$-converges to~$a\vee b$.
\end{lem}
\begin{proof}
We will only prove that $a_1\wedge b_1,\ a_2\wedge b_2,\ \dotsc$
weakly $\varphi$-converges to~$a\wedge b$.

Let $\ve \in \Phi$ be given.
To prove  $a_1\wedge b_1,\ a_2\wedge b_2,\ \dotsc$
weakly $\varphi$-converges to~$a\wedge b$,
we must find an~$N\in\N$
 such that 
\begin{equation}
\label{L:phi-conv-inf-1}
0\quad \ve\quad \ld\varphi(\,a_n\wedge b_n,\ a\wedge b\,)
\qquad\qquad(n\geq N).
\end{equation}
Since $a_1,a_2, \dotsc$ weakly $\varphi$-converges to~$a$
and $b_1,b_2,\dotsc$ weakly $\varphi$-converges to~$b$
we know
that there is an~$N\in\N$ such that 
\begin{equation}
\label{L:phi-conv-inf-2}
0\ \ \dt\ve\ \ \ld\varphi(a_n, a)
\qquad\text{and}\qquad
0\ \ \dt\ve\ \ \ld\varphi(b_n,b)
\qquad\qquad(n\geq N).
\end{equation}
We will prove that Statement~\eqref{L:phi-conv-inf-1}
holds for this~$N$.

To this end, note by Lemma~\ref{L:d-metric}\ref{d-metric_pos}
and Lemma~\ref{L:wv-unif} we have, for all~$n\in \N$,
\begin{equation*}
0 \ \ \leq\ \ 
\ld\varphi(\,a_n\wedge b_n,\ a\wedge b\,)
\ \ \leq\ \ 
\ld\varphi(a_n, a) \,+\,
\ld\varphi(b_n, b)
\end{equation*}
So by property~\ref{E-ord} of~$\Phi$
(see Def.~\ref{D:uniformity})
to prove~\eqref{L:phi-conv-inf-1}
it suffices to show  that
\begin{equation}
\label{L:phi-conv-inf-3}
0\quad\ve\quad
\ld\varphi(a_n, a) \,+\,
\ld\varphi(b_n, b)
\end{equation}
for any $n\geq N$.
By Statement~\eqref{L:phi-conv-inf-2} 
and property~\ref{E-add} of~$\Phi$, we get
\begin{equation*}
0 
\ \ \dt\ve  \ \ 
\ld\varphi(a_n, a)
\quad \dt\ve \quad
\ld\varphi(a_n, a) \,+\,
\ld\varphi(b_n, b)
\end{equation*}
for all~$n\geq N$. 
So we see that Statement~\eqref{L:phi-conv-inf-3} 
holds by property~\ref{E-add} of $\Phi$.
\end{proof}
%
%                  COUNTEREXAMPLE ON THE PREVIOUS LEMMA FOR PHI-CONV
%
\begin{ex}
\label{E:phi-conv-inf}
Given Lemma~\ref{P:R-main},
one might surmise that Lemma~\ref{L:phi-conv-inf}
holds if one replaces ``weakly $\varphi$-converges''
by
``$\varphi$-converges''.
This is not the case, as we will show.

Recall that  Lebesgue integral $\Lphi\colon \LF \ra \R$
is a valuation.
For all~$n\in \N$, define
\begin{equation*}
f_n \ = \ (-1)^{n} \cdot \textstyle{\frac{1}{n}}\cdot \mathbf{1}_{[n,n+1]}.
\end{equation*}
Then $f_n\in \LF$ for all~$n$, 
and the sequence $f_1,f_2,\dotsc $ $\Lphi$-converges to~$\mathbf{0}$.
As one expects,
the sequence $-f_1,-f_2,\dotsc$ 
also
 $\Lphi$-converges to~$\mathbf{0}$.
However, the sequence 
\begin{equation*} 
f_1\vee(-f_1),\  f_2\vee(-f_2),\ \dotsc
\end{equation*}
does not $\Lphi$-converge to~$\mathbf{0}$,
because $f_n \vee (-f_n) = \frac{1}{n}\cdot \mathbf{1}_{[n,n+1]}$
(see Example~\ref{E:weak-phi-conv}).
\end{ex}
%
%                  LEMMA ON WEAK DENSENESS 
%
\begin{lem}
\label{L:fitting-phi-conv-dense}
Let $E$ be an ordered Abelian group.\\
Let $\Phi$ be a fitting uniformity on~$E$.\\
Let $\vs{V}{L}\varphi{E}$ be an extendible valuation system.
Let $a\in \ol L$ be given.\\
Then there is a sequence
$a_1,a_2,\dotsc$ in~$L$
that weakly $\ol\varphi$-converges to~$a$.
\end{lem}
\begin{proof}
By Corollary~\ref{C:aleph1}
we know that~$\overline L = \Pi_{\aleph_1} L$.
So it suffices to prove that
the following statement holds for every ordinal number~$\alpha$.
\begin{equation*}
\left[\quad
\begin{minipage}{.7\columnwidth}
Let $a\in \Pi_\alpha L\cup \Sigma_\alpha L$ be given.\\
There is a sequence $a_1,a_2,\dotsc$ in~$L$
that $\ol\varphi$-converges to~$a$.
\end{minipage}
\right.
\end{equation*}
Let us name the above statement~$P(\alpha)$.
We prove $\forall\alpha\ P(\alpha)$ with induction.

Clearly, $P(0)$ holds,
since $\Pi_0 L = L = \Sigma_0 L$.

\vspace{.3em}
Let $\alpha$ be an ordinal  such that $P(\alpha)$ holds.
We prove that $P(\alpha+1)$ holds.
Let $a\in \Pi_{\alpha+1}L \cup \Sigma_{\alpha+1} L$ be given.
We must find a sequence in~$L$ that
$\ol\varphi$-converges to~$a$.

Assume that $a\in \Pi_{\alpha+1} L$.
There is a $\Sigma_\alpha\varphi$-convergent
sequence $b_1 \geq b_2 \geq \dotsb$ in $\Sigma_\alpha L$
such that $\bw_n b_n = a$.
In particular, $a_1,a_2,\dotsc$ $\ol\varphi$-converges to~$a$.
Since $P(\alpha)$ holds,
we can find for each~$N\in \N$ a
sequence $b^N_1, b^N_2,\dotsc$ in~$L$
that $\ol\varphi$-converges to~$a_N$.
Then by Lemma~\ref{L:phi-conv-diag}
there are $j_1 < j_2 < \dotsb$ in $\N$
such that $b^1_{j_1},\, b^2_{j_2},\,\dotsc$
$\ol\varphi$-converges to~$a$.
So we see that there is a sequence in~$L$
that $\ol\varphi$-converges to~$a$.

By a similar reasoning
we see that if $a\in \Sigma_{\alpha+1}L$
then there is a sequence in~$L$ 
that $\ol\varphi$-converges to~$a$.
Hence $P(\alpha+1)$.

\vspace{.3em}
Let $\lambda$ be a limit ordinal
such that $P(\alpha)$ holds
for all~$\alpha<\lambda$.
We must prove that $P(\lambda)$ holds.
Let $a\in \Pi_\lambda L\cup \Sigma_\lambda L$ be given.
We must find a sequence in~$L$ that $\ol\varphi$-converges to~$a$.
By definition of~$\Pi_\lambda L$ and $\Sigma_\lambda L$,
there is an~$\alpha<\lambda$
such that $a \in \Pi_\alpha L \cup \Sigma_\alpha L$
(see Definition~\ref{P:hier}).
Since we know that~$P(\alpha)$ holds,
there must be a sequence in~$L$ that $\ol\varphi$-converges to~$a$.
Hence~$P(\lambda)$.
\end{proof}
%
%                  TECHNICAL COROLLARY TO THE DENSENESS LEMMA
%
\begin{cor}
\label{C:fitting-phi-conv-dense}
Let $E$ be an ordered Abelian group.\\
Let $\Phi$ be a fitting uniformity on~$E$.\\
Let $\vs{V}{L}\varphi{E}$ be an extendible valuation system.
Let $a,b\in \ol L$ with $a\leq b$ be given.\\
Then there is a sequence
$a_1,a_2,\dotsc$ in~$L$ that weakly $\ol\varphi$-converges to~$a$,\\
and there is a sequence
$b_1,b_2,\dotsc$ in~$L$ that weakly $\ol\varphi$-converges to~$b$,\\
such that $a_n \leq b_n$ for all~$n\in\N$.
\end{cor}
\begin{proof}
Let $a,b\in\ol L$ with $a\leq b$ be given.
Using Lemma~\ref{L:fitting-phi-conv-dense}
find a sequence $a_1',a_2',\dotsc$ in~$L$
that weakly $\ol\varphi$-converges to~$a$
and find a sequence $b_1',b_2',\dotsc$ in~$L$
that weakly $\ol\varphi$-converges to~$b$.
Consider the sequences
$a_1,a_2,\dotsc$ and $b_1,b_2,\dotsc$  in~$L$ 
given by  
\begin{equation*}
a_n \eqdf a_n'\wedge b_n'
\qquad\text{and}\qquad
b_n \eqdf a_n' \vee b_n'
\qquad\qquad(n\in\N).
\end{equation*}
Clearly $a_n \leq b_n$ for all~$n\in \N$.
Moreover,
by Lemma~\ref{L:phi-conv-inf} 
we know that $a_1,a_2,\dotsc$ 
weakly $\ol\varphi$-converges to~$a = a\wedge b$
and weakly $b_1,b_2,\dotsc$ $\ol\varphi$-converges to~$b= a\vee b$.
\end{proof}
%
%                  DENSENESS PROPOSITION 
%
\begin{prop}
\label{P:fitting-phi-conv-dense}
Let $E$ be an ordered Abelian group.\\
Let $\Phi$ be a fitting uniformity on~$E$.\\
Let $\vs{V}{L}\varphi{E}$ be an extendible valuation system.
\begin{enumerate}
\item
Given $a\in \ol L$,
 there is a sequence
$a_1,a_2,\dotsc$ in~$L$
which $\ol\varphi$-converges to~$a$.

\item
Let $a,b\in \ol L$ with $a\leq b$ be given.\\
There is a sequence 
$a_1,a_2,\dotsc$ in~$L$
which $\varphi$-converges to~$a$, and \\
there is a sequence
$b_1,b_2,\dotsc$ in~$L$
which $\varphi$-converges to~$b$, such that
\begin{equation*}
a_n \ \leq\ b_n \qquad\quad(n\in\N).
\end{equation*}
\end{enumerate}
\end{prop}
\begin{proof}
Combine Lemma~\ref{L:fitting-phi-conv-dense},
Corollary~\ref{C:fitting-phi-conv-dense},
and Proposition~\ref{P:weak-phi-conv-subseq}.
\end{proof}

%
%                  EXTENSION THEOREM USED BY FUBINI
%
\begin{thm}
\label{T:fubext}
Let $E$ be a lattice ordered Abelian group.\\
Let $\Phi$ be a fitting uniformity on~$E$.\\
Let $\vs{V}{L}\varphi{E}$ be an extendible valuation system.\\
Let $\psi\colon C\ra E$ be 
a complete Hausdorff valuation.\\
Let $f\colon L\ra C$ be an order preserving map such that 
$\psi\circ f = \varphi$.\\
Then there is a unique
order preserving extension~$g\colon \ol L \ra C$ of~$f$
such that $\psi\circ g = \ol\varphi$.
\begin{equation*}
\xymatrix{
\ol L \ar [rd] |{\ol\varphi} 
\ar @{-->} [rrd] ^{g} \\
&E&C\ar[l]|{\psi}\\
L \ar [rru] _{f}
\ar [ru] |{\varphi}
\ar @{^(->} [uu] 
}
\end{equation*}
\end{thm}
\begin{proof}
(Uniqueness)
Let $g_1,g_2\colon \ol L \ra C$
be order preserving extensions of~$f$ such that
$\psi\circ g_i = \ol\varphi$.
We prove that $g_1 = g_2$.
Let~$a\in \ol L$ be given.
By Proposition~\ref{P:fitting-phi-conv-dense}
there is a $\varphi$-convergent sequence $a_1,a_2,\dotsc$
in~$L$ which $\ol\varphi$-converges to~$a$.

We must show that~$g_1(a) = g_2(a)$.
To this end we prove that $f(a_1),\,f(a_2),\,\dotsc$
weakly $\psi$-converges to~$g_i(a)$.
Then  $g_1(a) = g_2(a)$
because~$\psi$ is Hausdorff.

Let~$i\in\{1,2\}$ be given.
Note that $g_i(a_n) = f(a_n)$ because $g_i$ extends~$f$.
So we must prove that $g_i(a_1),\,g_i(a_2),\,\dotsc$
weakly $\psi$-converges to~$g_i(a)$.
Let~$\ve\in\Phi$ be given.
We must find 
an~$N\in \N$
such that
\begin{equation}
\label{eq:T:fubext-1}
\ld\psi(\,g_i(a_n),\,g_i(a)\,) \quad\ve\quad 0
\qquad\quad(n\geq N).
\end{equation}
Since $a_1,a_2,\dotsc$ $\ol\varphi$-converges to~$a$
we know that $a_1,a_2,\dotsc$ weakly $\ol\varphi$-converges to~$a$
so we know there is an $N\in \N$ with
\begin{equation*}
\ld{\ol\varphi}(a_n,a) \quad\ve\quad 0\qquad\quad(n\geq N).
\end{equation*}
We will prove that Statement~\eqref{eq:T:fubext-1}
holds for this~$N$.

Let $n\geq N$ be given.
Since $g$ is order preserving, we have
\begin{equation*}
g_i(a_n \wedge a) \ \leq\ g_i(a_n)\wedge g_i(a)
\qquad\qquad
g_i(a_n) \vee g_i(a) \ \leq\  g_i (a_n \vee a).
\end{equation*}
In particular
(recall that $\psi\circ g_i=\ol\varphi$),
\begin{equation}
\label{eq:T-fubext-3}
\begin{alignedat}{3}
\ld\psi(\,g_i(a_n),\,g_i(a)\,) 
\ &=\ 
\psi(g_i(a_n) \vee g_i(a)) \,-\, \psi(g_i(a_n)\wedge g_i(a)) \\
\ &\leq\ 
\psi(g_i(a_n\vee a)) \,-\, \psi(g_i(a_n \wedge a)) \\
\ &=\ 
\ol\varphi(a_n \vee a) \,-\, \ol\varphi(a_n\wedge a) 
\ =\ \ld{\ol\varphi}(a_n,a).
\end{alignedat}
\end{equation}
So we
know that $0\ \ve\ \ld{\ol\varphi}(a_n,a)$
and we have the following inequalities.
\begin{equation*}
0 \ \leq\ \ld{\psi}(\,g_i(a_n),\,g_i(a)\,) \ \leq\ \ld{\ol\varphi}(a_n,a)
\end{equation*}
Hence $0 \ \ve\ \ld{\psi}(g_i(a_n),g_i(a))$
by property~\ref{E-ord} of a fitting uniformity.
So we have shown that Statement~\eqref{eq:T:fubext-1} holds.
Thus, $g_1=g_2$.

\vspace{.3em}(Existence)
We will prove the following statement.
\begin{equation}
\label{eq:T-fubext-2}
\left[\quad
\begin{minipage}{.7\columnwidth}
Let $a\in \ol L$ be given.
There is a unique $b\in C$ 
such that for every sequence $a_1,a_2,\dotsc$ in~$L$
that $\ol\varphi$-converges to~$a$,
we have $f(a_1),\,f(a_2),\,\dotsc$
$\psi$-converges to~$b$.
\end{minipage}
\right.
\end{equation}
Of course,
we will later define~$g\colon \ol L \ra C$ by  $g(a) = b$.

Let $a\in\ol L$ be given.
For each~$i\in\{1,2\}$,
let $b_i\in C$
and $a^i_1,a^i_2,\dotsc\in L$
be given,
such that 
$a^i_1,a^i_2,\dotsc$
 $\ol\varphi$-converges to~$a$,
and $f(a^i_1),\,f(a^i_2),\dotsc$
$\psi$-converges to~$b_i$.

We must prove that $b_1=b_2$.
Let $\ve\in\Phi$ be given.
Since $\psi$ is Hausdorff,
it suffices to show that $0\ \ve\ \ld\psi(b_1,b_2)$
(see~\ref{D:hausdorff}).

Note that
by
points \ref{d-metric_pos}
and \ref{d-metric_triangle}
of Lemma~\ref{L:d-metric}
we have
\begin{equation*}
0 \ \leq\ 
\ld\psi(b_1,b_2)
\ \ \leq\ \ 
\ld\psi(b_1,\,f(a^1_n)\,)
\ +\  
\ld\psi(\,f(a^1_n),\,f(a^2_n)\,)
\ +\ 
\ld\psi(\,f(a^2_n),\,b_2).
\end{equation*}
So to prove 
 $0\ \ve\ \ld\psi(b_1,b_2)$,
it sufficient to find~$N\in\N$
such that for all~$n\geq N$
the following statement holds
(see Definition~\ref{D:uniformity},
points~\ref{E-ord}, \ref{E-half} and \ref{E-add}).
\begin{alignat*}{3}
0\quad &\dtn\ve4 \quad \ld\psi(b_1,\,f(a^1_n)\,),
\quad\text{and}\\
0\quad &\dtn\ve4 \quad \ld\psi(\,f(a^1_n),\,f(a^2_n)\,),
\quad\text{and}\\
0\quad&\dtn\ve4 \quad \ld\psi(\,f(a^2_n),\,b_2).
\end{alignat*}


Recall that $f(a^i_1),\,f(a^i_2),\,\dotsc$
$\psi$-converges to~$b_i$ for all~$i$.
Hence  $f(a^i_1),\,f(a^i_2),\,\dotsc$
weakly $\psi$-converges to~$b_i$ for all~$i$.
So we know  there is an~$N\in\N$ such that
$0 \ \dtn\ve4\ \ld\psi(f(a^i_n),\,b_i)$
for all~$n\geq N$ and~$i$.

It remains to be shown that there is an~$N\in \N$
such that 
$0\  \dtn\ve4 \  \ld\psi(f(a^1_n),\,f(a^2_n))$
for all $n\geq N$.
To this end, note that~$f$ is order preserving
and that $\psi \circ f = \varphi$.
So with a similar reasoning as before (see Statement~\eqref{eq:T-fubext-3}),
we see that
\begin{equation*}
\ld\psi(\,f(a^1_n),\,f(a^2_n)\,)
\ \leq\ 
\ld\varphi(a^1_n, a^2_n).
\end{equation*}
So to complete the proof Statement~\eqref{eq:T-fubext-2}
it suffices to find an~$N\in\N$ such that 
\begin{equation}
\label{eq:T:fubext-5}
0\quad\dtn\ve4\quad\ld\varphi(a^1_n, a^2_n)
\qquad\quad(n\geq N).
\end{equation}
Note that
by
points \ref{d-metric_pos}
and \ref{d-metric_triangle}
of Lemma~\ref{L:d-metric}
we have
\begin{equation*}
0\ \leq\ 
\ld\varphi(a_n^1,a_n^2) 
\ \leq\ 
\ld\varphi(a_n^1,a)
\,+\,
\ld\varphi(a,a_n^2).
\end{equation*}
Since the sequence $a^i_1,a^i_2,\dotsc$
$\ol\varphi$-converges to~$a$
(and hence also weakly)
we can find an~$N\in\N$
such that $0\ \dtn\ve4 \ \ld\varphi(a_n^i,a)$
 for all~$n\geq N$ and~$i\in\{1,2\}$.

So by points~\ref{E-ord}, \ref{E-half} and \ref{E-add}
of Definition~\ref{D:uniformity}
we see that Statement~\eqref{eq:T:fubext-5} holds.

Hence we have proven Statement~\eqref{eq:T-fubext-2}.
So we now know there is a unique map~$g\colon \ol L \ra C$
such that 
for every $a\in\ol L$
and every sequence $a_1,a_2,\dotsc$ in~$L$ that $\ol\varphi$-converges to~$a$
we have $f(a_1),\,f(a_2),\,\dotsc$ 
$\psi$-converges to~$g(a)$.

To complete the proof of this theorem,
we show that $g$ extends~$f$,
we show that $g$ is order preserving,
and that $\psi \circ g = \ol \varphi$.

Let $a\in L$ be given. To prove that $g$ extends~$f$
we show that $g(a)=f(a)$.
Note that $a,a,\dotsc$ $\ol\varphi$-converges to~$a$.
So by definition of~$g$ we know that $f(a),\,f(a),\,\dotsc$
$\psi$-converges to~$g(a)$.
But $f(a),\,f(a),\,\dotsc$ $\psi$-converges to~$f(a)$
too, and $\psi$ is Hausdorff.
So we see that $f(a)=g(a)$.

Let $a,b\in L$ with $a\leq b$ be given.
To prove that $g$ is order preserving 
we must show that $g(a)\leq g(b)$.
By Proposition~\ref{P:fitting-phi-conv-dense}
we can find a sequence $a_1,a_2,\dotsc$ in~$L$
that $\ol\varphi$-converges to~$a$
and a sequence $b_1,b_2,\dotsc$ in~$L$
that $\ol\varphi$-converges to~$b$
such that we have $a_n \leq b_n$ for all~$n\in\N$.
Now, note that  by Lemma~\ref{L:phi-conv-inf}
we know that
\begin{equation*}
f(a_1) \wedge f(b_1),\quad f(a_2)\wedge f(b_2),\quad \dotsc
\qquad\text{\emph{weakly} $\psi$-converges to}\quad g(a)\wedge g(b).
\end{equation*}
Let $n\in \N$ be given.
Since $f$ is order preserving
and $a_n \leq b_n$
we have $f(a_n)\leq f(b_n)$
and so $f(a_n) \wedge f(b_n) = f(a_n)$.
Hence  $f(a_1),\,f(a_2),\,\dotsc$
weakly $\psi$-converges to both $g(a)$ and $g(a)\wedge g(b)$.
So we see that $g(a) = g(a)\wedge g(b)$ 
and thus $g(a)\leq g(b)$.

Let $a\in\ol L$ be given.
We show that $\psi(g(a)) = \ol\varphi(a)$.
Find a sequence $a_1,a_2,\dotsc$ in~$L$
that $\ol\varphi$-converges to~$a$
(see Lemma~\ref{L:fitting-phi-conv-dense}).

Recall that $E$ is a \emph{lattice} ordered Abelian group.
By Theorem~\ref{T:lebesgue}
we see that
\begin{equation}
\label{eq:T:fubext-7}
\ol\varphi(a) \ = \ \textstyle{\lim_n} \varphi(a_n).
\end{equation}
By definition of~$g$
we have $f(a_1),\,f(a_2),\,\dotsc$
$\psi$-converges to~$g(a)$.
So by Theorem~\ref{T:lebesgue}
we have  $\psi(g(a)) = \lim_n \psi(f(a_n)) $.
But $\psi(f(a_n)) = \varphi(a_n)$ for all~$n\in\N$,
so we have
\begin{equation}
\label{eq:T:fubext-8}
\psi(f(a))
\ =\  \textstyle{\lim_n} \varphi(a_n).
\end{equation}
If we combine Equalities~\eqref{eq:T:fubext-7}
and~\eqref{eq:T:fubext-8} we get
$\ol\varphi(a) = \psi(f(a))$.
\end{proof}
%%%%%%%%%%%%%%%%%%%%%%%%%%%%%%%%%%%%%%%%%%%%%%%%%%%%%%%%%%%%%%%%%%%%%%%%%%%%%%%
%
 %                 FUBINI'S THEOREM
\begin{thm}
\label{T:fub}
Statement~\eqref{eq:fub} holds.
\end{thm}
\begin{proof}
We only give hints and leave the details to the reader.
With the notation of Subsection~\ref{SS:fub_part1}
apply Theorem~\eqref{T:fubext}
to the following situation.
\begin{equation*}
\xymatrix{
\ol{F}_{X\times Y} \ar [rd] |{\ol\varphi_{X\times Y}} 
\ar @{-->} [rrd] ^{\mathcal{G}} \\
&\R& \qvL{\ol{F}_Y} \ar [l]\\
F_{X\times Y} \ar [rru] _{\mathcal{F}_X}
\ar [ru] |{\varphi}
\ar @{^(->} [uu] 
}
\end{equation*}
Now, note that $\mathcal{G}$
is a complete valuation
which extends $\mathcal{F}_X$.
\end{proof}

\end{document}
