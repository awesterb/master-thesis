\documentclass[
  a4paper,
  draft,
  twoside,
  reqno
]{amsart}
\usepackage[english]{babel}
\usepackage{amssymb}
\usepackage{amsmath}
\usepackage{amsthm}
\usepackage{subfiles}
\usepackage{mathtools}

\newcommand{\R}{\mathbb{R}}
\newcommand{\N}{\mathbb{N}}
\newcommand{\Lex}{\mathbb{L}}
\newcommand{\ra}{\rightarrow}
\newcommand{\bv}{\textstyle{\bigvee}}
\newcommand{\bw}{\textstyle{\bigwedge}}
\newcommand{\eqdf}{:=}

\newcommand{\vs}[4]{#1 \supseteq #2\,\smash{\stackrel{#3}{\ra}}\,#4}

\newcommand{\keyword}[1]{\textbf{#1}}
\newcommand{\todo}[1]{\textbf{[todo\footnote{\textbf{todo:} #1}]}}

\theoremstyle{definition}
\newtheorem{dfn}{Definition}
\newtheorem{ex}{Example}
\newtheorem{exs}{Examples}

\theoremstyle{notation}
\newtheorem{nt}{Notation}

\theoremstyle{theorem}
\newtheorem{lem}{Lemma}
\newtheorem{cor}{Corollary}
\newtheorem{thm}{Theorem}
\newtheorem{prop}{Proposition}

\newcommand{\version}{ {\bf compiled} 2012-09-13}

\usepackage[final]{hyperref}

\begin{document}
\title[A Generalisation of Measure and Integral]{Lattice Valuations,\\
A Generalisation of Measure and Integral}
\email{bram@westerbaan.name}

\author[A.A.~Westerbaan]{Bram Westerbaan}
\date{\today \quad{\tiny \version}}
\maketitle

\begin{abstract}
In analysis,
measure and integral are two fundamental
and distinct
objects of study.
Note that they
both are \emph{lattice valuations}:
an order preserving function~$f$
on a lattice~$L$ (e.g., of sets or functions)
which is \emph{modular},
\begin{equation*}
f(x) + f(y) \,=\, f(x\wedge y) + f(x\vee y)\qquad(x,y\in L).
\end{equation*}
In this thesis,
we unify measure and integral
by developing a theory of suitable lattice valuations
and deriving some elementary results from
integration and measure theory.
\end{abstract}

{ \subfile{intro.tex} }


\bibliography{main}{}
\bibliographystyle{amsalpha}

\end{document}
