\documentclass[main.tex]{subfiles}
\begin{document}
%%%%%%%%%%%%%%%%%%%%%%%%%%%%%%%%%%%%%%%%%%%%%%%%%%%%%%%%%%%%%%%%%%%%%%%%%%
%
%
%                  C O M P L E T E      V A L U A T I O N S 
%
%
%
\section{Complete Valuations}
\label{S:complete-val}
\noindent
We now turn to the study of \emph{complete} valuations
(see Definition~\ref{D:complete-val}).
Among all valuations
the complete valuations resemble
the Lebesgue measure
and the Lebesgue integral most closely.
To support this claim,
we will
prove generalisations of
some of the classical convergence theorems of integration
in Subsection~\ref{SS:complete-val_convergence}.

But first, we give some examples of complete valuations
in Subsection~\ref{SS:complete-val_introduction}.

After that,
we  study a notion of completeness
for an ordered Abelian group~$E$,
called \emph{$R$-completeness},
in Subsection~\ref{SS:complete-val_R-completeness},
which will be useful later on.

The notion of complete valuation is not at the end of the 
road.  We will study the
slightly more sophisticated 
\emph{valuation systems}
(see Definition~\ref{D:system})
and \emph{complete valuation systems}
(see Definition~\ref{D:system-complete})
in Section~\ref{S:valuation-systems}.

%
%                  PHI-CONVERGENCE
%
\subsection{Introduction}
\label{SS:complete-val_introduction}
\begin{dfn}
\label{D:phi-conv}
Let $E$ be an ordered Abelian group.\\
Let $L$ be a lattice, and let $\varphi\colon L \ra E$ be a valuation.

Consider a sequence
$a_1 \geq a_2 \geq \dotsb$ from~$L$.
We say
\begin{equation*}
a_1 \geq a_2 \geq \dotsb \text{ is \keyword{$\varphi$-convergent}}
\qquad\text{if}\qquad \bw_n \varphi(a_n)\text{ exists.}
\end{equation*}

Similarly,
if
$b_1 \leq b_2 \leq \dotsb$ is
a sequence in~$L$, then 
\begin{equation*}
b_1 \leq b_2 \leq \dotsb \text{ is \keyword{$\varphi$-convergent}}
\qquad\text{if}\qquad \bv_n \varphi(b_n)\text{ exists.}
\end{equation*}
\end{dfn}
%
%                  COMPLETE VALUATION
%
\begin{dfn}
\label{D:complete-val}
Let $E$ be an ordered Abelian group. Let $L$ be a lattice.\\
Let $\varphi\colon L \ra E$ be a valuation.
We say $\varphi$ is \keyword{$\Pi$-complete} if
\begin{alignat*}{5}
a_1 \geq a_2 \geq \dotsb \text{$\varphi$-convergent }
  \quad &\implies \quad 
  & \bw_n a_n &\text{ exists,}\quad 
  &&\text{and}\quad
  &\varphi(\,\bw_n a_n\,) &= \bw_n \varphi(a_n).
\shortintertext{%
We say $\varphi$ is \keyword{$\Sigma$-complete} if
}
b_1 \leq b_2 \leq \dotsb \text{$\varphi$-convergent }
  \quad &\implies \quad 
  & \bv_n b_n &\text{ exists,}\quad 
  &&\text{and}\quad
  &\varphi(\,\bv_n b_n\,) &= \bv_n \varphi(b_n).
\end{alignat*}
We say $\varphi$ is \keyword{complete}
if $\varphi$ is both $\Pi$-complete and $\Sigma$-complete.
\end{dfn}
%
%                  THE LEBESGUE MEASURE IS A COMPLETE VALUATION
%
\begin{ex}
\label{E:lmeas-complete-val}
The Lebesgue measure~$\Lmu$
(see Example~\ref{E:lmeas-val})
is a complete valuation.
We must show that~$\Lmu$ is
both $\Pi$-complete and $\Sigma$-complete
(see Definition~\ref{D:complete-val}).

Let us prove~$\Lmu$ is $\Sigma$-complete.
Let $B_1 \subseteq B_2 \subseteq \dotsb$ in~$\LA$
be $\Lmu$-convergent.
We must prove that $\bv_n B_n$ exists in~$\LA$
and that 
$\Lmu(\bv_n B_n) = \bv_n \Lmu(B_n)$.

Note that $\bigcup_n B_n$
is Lebesgue measurable,
and that  $\bigcup_n B_n$ has (finite) Lebesgue measure~$\bv_n\Lmu(B_n)$.
Hence we have $\bigcup_n B_n \in \LA$, and,
\begin{equation}
\label{eq:E:lmeas-complete-val-1}
\textstyle
\Lmu(\,\bigcup_n B_n\,) \ =\  \bv_n \Lmu(B_n).
\end{equation}
So we are done if we prove that $\bigcup_n B_n = \bv_n B_n$.
Since $\bigcup_n B_n$ is the smallest subset of~$\R$
containing all~$B_n$
(i.e. $\bigcup_n B_n$ is the supremum of the~$B_n$ in~$\wp \R$),
$\bigcup_n B_n$ is also the smallest
subset \emph{of finite Lebesgue measure} containing all~$B_n$
(i.e. $\bigcup_n B_n$ is the supremum of the~$B_n$ in~$\LA$).
So
 $\bigcup_n B_n = \bv_n B_n$.
Hence $\Lmu$ is $\Sigma$-complete.

Using an easier reasoning one can prove that $\Lmu$
is $\Pi$-complete.
\end{ex}
%
%                  THE LEBESGUE INTEGRAL IS A COMPLETE VALUATION
%
\begin{ex}
\label{E:int-complete-val}
The Lebesgue integral~$\Lphi$
(see Example~\ref{E:int-val})
is a complete valuation.
We must show that~$\Lphi$ is
both $\Pi$-complete and $\Sigma$-complete
(see Definition~\ref{D:complete-val}).

Let us prove~$\Lphi$ is $\Sigma$-complete.
Let $f_1\leq f_2 \leq\dotsb$ in~$\LF$
be $\Lphi$-convergent.
We must prove that $\bv_n f_n$ exists in~$\LF$,
and that 
\begin{equation*}
\Lphi(\bv_n f_n) \ =\  \bv_n \Lphi(f_n).
\end{equation*}
Of course,
this follows immediately from
Levi's Monotone Convergence Theorem;
the supremum $\bv_n f_n$ in~$\LF$ is simply
the pointwise supremum
(which is the supremum of~$f_1\leq f_2 \leq\dotsb$ in $\E^\R$).
So we see $\Lphi$ is $\Sigma$-complete.

With a similar argument one can see that $\Lphi$ is $\Pi$-complete.
\end{ex}
%
%                  REMARK ON \E INSTEAD OF \R
%
\begin{rem}
\label{R:non-finite-functions}
Note that the restriction of~$\Lphi$ to~$\LF\cap \R^\R$
is not complete.

Indeed,
consider for instance
the following sequence.
\begin{equation*}
1\cdot\mathbf{1}_{\{0\}} 
\ \leq\   2\cdot\mathbf{1}_{\{0\}} 
\ \leq\   3\cdot\mathbf{1}_{\{0\}}
\ \leq\  \dotsb
\end{equation*}
It is~$\Lphi$-convergent
in~$\LF\cap \R^\R$,
but it has no supremum in~$\R^\R$.\\
On the other hand,
it does have a supremum in~$\LF$, 
namely~$+\infty\cdot \mathbf{1}_{\{0\}}$.

Because of the above observation,
we work with
the $\E$-valued Lebesgue integrable functions
instead of the $\R$-valued Lebesgue integrable functions.
\end{rem}
%
%                  THE SIMPLE MEASURABLE SETS ARE NOT COMPLETE
%
\begin{ex}
The valuation $\Smu$ 
(see Example~\ref{E:smeas-val}) is \emph{not} complete.

To see this,
we consider the sets~$A_1,A_2,\dotsc$ given by, for~$n\in\N$,
\begin{equation*}
A_n \ =\  \{1,\dotsc,n\}.
\end{equation*}
Then $A_n\in \SA$
and $\Smu(A_n) = 0$
for all~$n\in \N$.
So we see that
\begin{equation*}
A_1 \,\subseteq\, A_2 \,\subseteq\, \dotsb
\end{equation*}
is a $\Smu$-convergent sequence.
To prove that~$\Smu$ is not complete,
we  show that
the $\Smu$-convergent sequence~$A_1 \subseteq A_2 \subseteq \dotsb$
has no supremum in~$\SA$
(see Definition~\ref{D:complete-val}).

Suppose (towards a contradiction) that $A_1 \subseteq A_2 \subseteq\dotsb$
has a supremum~$B$ in~$\SA$.
Then in particular $A_n\subseteq B$ for all~$n\in \N$.
So we have
\begin{equation*}
\N \,=\, \textstyle{\bigcup_n A_n} \ \subseteq\  B.
\end{equation*}
Note that $B$ is the disjoint union of
elements from~$\mathcal{S}$ (see Example~\ref{E:smeas-val}).
Since all~$I\in \mathcal{S}$ are bounded, the set~$B$ is bounded.
That is, $B\subseteq [a,b]$ for some~$a,b\in \R$.

We now see that $\N\subseteq B\subseteq [a,b]$,
which is nonsense.
So $A_1 \subseteq A_2 \subseteq \dotsb$ has no supremum
in~$\SA$.
Hence $\Smu$ is not complete.
\end{ex}
%
%                  THE STEP FUNCTIONS ARE NOT COMPLETE
%
\begin{ex}
The valuation $\Sphi$ 
(see Example~\ref{E:sint-val}) is also \emph{not} complete.\\
We leave it to the reader to prove this fact.
\end{ex}

%
%                  PHI-CONVERGENCE FOR SIGMA-DEDEKIND COMPLETE VALUATIONS
%
\noindent
If $E=\R$,
or more generally,
if~$E$ is $\sigma$-Dedekind complete
(see Definition~\ref{D:sdc}),
then there is a nice description of $\varphi$-convergence,
see Proposition~\ref{P:phi-conv-dom}.
\begin{lem}
\label{L:phi-conv-dom}
Let $E$ be an ordered Abelian group.\\
Assume that~$E$ is $\sigma$-Dedekind complete
(see Definition~\ref{D:sdc}).\\
Let $L$ be a lattice,
and let $\varphi\colon L\ra E$ be a valuation.\\
Let $a_1 \geq a_2 \geq \dotsb$ be a sequence in~$L$.\\
Then $a_1 \geq a_2 \geq \dotsb$ 
is $\varphi$-convergent 
provided that~$a_1\geq a_2\geq \dotsb$ 
has a lower bound.
\end{lem}
\begin{proof}
Let $\ell\in L$ be a lower bound of~$a_1 \geq a_2 \geq \dotsb$,
that is, $\ell \leq a_n$ for all~$n\in\N$.
We must prove that~$a_1 \geq a_2 \geq \dotsb$ is
$\varphi$-convergent,
i.e., $\bw_n \varphi(a_n)$ exists.

Note that $\varphi(\ell)\leq \varphi(a_n)$ for all~$n\in \N$.
So $\varphi(a_1),\,\varphi(a_2),\, \dotsc$
has a lower bound.
But then $\bw_n \varphi(a_n)$ exists,
because~$E$ is $\sigma$-Dedekind complete
(see Remark~\ref{R:sdc}).
Hence $a_1 \geq a_2 \geq \dotsb$ is $\varphi$-convergent.
\end{proof}
\begin{prop}
\label{P:phi-conv-dom}
Let $E$ be an ordered Abelian group.\\
Assume that~$E$ is $\sigma$-Dedekind complete
(see Definition~\ref{D:sdc}).\\
Let $L$ be a lattice,
and let $\varphi\colon L\ra E$ be a complete valuation.\\
For a sequence $a_1 \geq a_2 \geq \dotsb$
in~$E$ the following are equivalent.
\begin{enumerate}
\item
\label{P:phi-conv-dom_i}
$a_1 \geq a_2 \geq \dotsb$
is $\varphi$-convergent.

\item
\label{P:phi-conv-dom_ii}
$a_1 \geq a_2 \geq \dotsb$
has a lower bound in~$L$.

\item
\label{P:phi-conv-dom_iii}
$a_1 \geq a_2 \geq \dotsb$
has an infimum, $\bw_n a_n$.
\end{enumerate}
\end{prop}
\begin{proof}
The implication 
``\ref{P:phi-conv-dom_i}$\ \Longleftarrow\ $\ref{P:phi-conv-dom_ii}''\ 
holds by Lemma~\ref{L:phi-conv-dom}.

\noindent
``\ref{P:phi-conv-dom_ii}$\ \Longleftarrow\ $\ref{P:phi-conv-dom_iii}''\ 
holds, because the infimum~$\bw_n a_n$ is a 
lower bound of $a_1 \geq a_2 \geq \dotsb$.

\noindent
``\ref{P:phi-conv-dom_iii}$\ \Longleftarrow\ $\ref{P:phi-conv-dom_i}''\ 
holds since $\varphi$ is complete
(see Definition~\ref{D:complete-val}).
\end{proof}
%
%                  EXAMPLE ON INFIMUM DOES NOT IMPLY PHI-CONVERGENCE
%
\noindent
The notion of $\varphi$-convergence
is less trivial in general
as the following example shows.
\begin{ex}
We will show that
the assumption 
that~$E$ is $\sigma$-Dedekind complete
in Proposition~\ref{P:phi-conv-dom}
is necessary for 
the implication 
``\ref{P:phi-conv-dom_iii}$\ \Longrightarrow\ $\ref{P:phi-conv-dom_i}''.\\
To this end, we extend the Lebesgue integral $\Lphi$
(see Example~\ref{E:int-complete-val})
to the set 
\begin{equation*}
\LF'\ \eqdf\  \LF\,\cup\,\{\,-\infty\cdot \mathbf{1}\,\}.
\end{equation*}
Note that~$\LF'$ is a sublattice of~$\E^\R$.
Let $\Lphi'\colon \LF'\rightarrow \Lex$
be the map,
where $\Lex$ is the \emph{lexicograpic plane} 
(see Example~\ref{E:oag}\ref{E:oag_lex}),
given by, for $f\in \LF'$,
\begin{equation*}
\Lphi'( f) \ =\ 
\begin{cases}
\quad (\,\rsub{0}{-1},\,\Lphi(f)\,)\,  \qquad
&\text{if $f\in\LF$}, \\
\quad (\,-1,\,0\,)  &\text{if $f=-\infty\cdot \mathbf{1}$}.
\end{cases}
\end{equation*}
Then $\Lphi'$ is a valuation.
In fact, 
$\Lphi'$ is a complete valuation
as the reader can verify 
using the following observation.
If $a_1 \geq a_2 \geq \dotsb$ from~$\LF'$
is $\Lphi'$-convergent, then:
\begin{enumerate}
\item
If $a_n \in \LF$ for all~$n\in\N$,
then $a_1 \geq a_2 \geq \dotsb$ is $\Lphi$-convergent.
\item
If $a_N = -\infty\cdot \mathbf{1}$
for some $N\in\N$,
then $a_n = -\infty\cdot \mathbf{1}$ for all~$n\geq N$.
\end{enumerate}
Now, 
consider the following sequence in $\LF'$.
\begin{equation*}
-1\cdot\mathbf{1}_{[0,1]} \ \geq\ 
-2 \cdot\mathbf{1}_{[0,1]} \ \geq \ 
-3 \cdot\mathbf{1}_{[0,1]}
\ \geq\ \dotsb.
\end{equation*}
This sequence has an infimum in~$\LF'$, 
namely $-\infty\cdot \mathbf{1}$.
Nevertheless,
the sequence is not $\Lphi'$-convergent(,
because $(0,-1) \,\geq\, (0,-2)\,\geq\, \dotsb$
has no infimum in~$\Lex$).
\end{ex}





%%%%%%%%%%%%%%%%%%%%%%%%%%%%%%%%%%%%%%%%%%%%%%%%%%%%%%%%%%%%%%%%%%%%%%%%%%
%%%%%%%%%%%%%%%%%%%%%%%%%%%%%%%%%%%%%%%%%%%%%%%%%%%%%%%%%%%%%%%%%%%%%%%%%%%
%
\subsection{$R$-completeness}
\label{SS:complete-val_R-completeness}
We now study a notion of completeness for ordered Abelian groups
called \emph{$R$-completeness}
that will be useful later on.

Let $\varphi\colon L \ra E$ be a valuation.
Let $a_1 \leq a_2 \leq \dotsb$
and $b_1 \leq b_2 \leq \dotsb$ be $\varphi$-convergent sequences in~$L$.
(see Definition~\ref{D:phi-conv}).

For the development of the theory,
it would be convenient if also 
\begin{equation}
\label{eq:SS-complete-val_R-completeness-1}
a_1 \vee b_1 \ \leq\  a_2 \vee b_2 \ \leq\  \dotsb
\qquad\text{is $\varphi$-convergent.}
\end{equation}
Unfortunately, 
this is not always the case (see Example~\ref{E:P:R-main}).
However,
if the space~$E$ is $\sigma$-Dedekind complete 
(see Appendix~\ref{S:ag}, Definition~\ref{D:sdc}),
for instance if~$E=\R$,
then one can prove that Statement~\eqref{eq:SS-complete-val_R-completeness-1}
holds.

In fact,
if we only assume that
$E$ is \emph{$R$-complete} (see Definition~\ref{D:R-complete}) ---
which is a weaker assumption than that~$E$ is Dedekind-complete ---
then we can still prove that 
that Statement~\eqref{eq:SS-complete-val_R-completeness-1}
holds (see Proposition~\ref{P:R-main}).
%
%                  R-completeness
%
\begin{dfn}
\label{D:R-complete}
Let $E$ be an ordered Abelian group.
Consider the following.
\begin{equation*}
\left[\quad 
\begin{minipage}{.7\columnwidth}
Let $x_1 \leq x_2 \leq \dotsb$
and $y_1 \leq y_2 \leq \dotsb$ be from~$E$
such that
\begin{equation*}
x_{n+1} - x_n \ \leq\ y_{n+1} - y_n\qquad \text{for all }n.
\end{equation*}
Then $\bv x_n $ exists whenever $\bv y_n$ exists.
\end{minipage}
\right.
\end{equation*}
If the above statement holds,
we say~$E$ is \keyword{$R$-complete}.
\end{dfn}

\begin{exs}
\begin{enumerate}
\item
The ordered Abelian group $\R$ is $R$-complete.

\item
In fact, any $\sigma$-Dedekind complete 
ordered Abelian group $E$ is $R$-complete.\\
Indeed,
let $x_1 \leq x_2 \leq \dotsb$ 
and $y_1 \leq y_2 \leq \dotsb$
be from~$E$
such that
\begin{equation}
\label{eq:E:R:sdc}
x_{n+1} - x_n \ \leq\ y_{n+1}-y_n
\qquad\text{for all }n,
\end{equation}
and assume that $\bv_n y_n$ exists.
We must show that $\bv_n x_n$ exists.

Let $n\in\N$ be given.
By Statement~\eqref{eq:E:R:sdc}
we see that
\begin{equation*}
x_{n+1} - y_{n+1} \ \leq\ x_n - y_n.
\end{equation*}
So with induction on~$n$, we get $x_n - y_n \ \leq\ x_1 - y_1$.
Then
\begin{equation*}
x_n \ \leq\ 
(x_1 - y_1) \,+\, y_n  \ \leq\ 
(x_1 - y_1) \,+\, \bv_m y_m.
\end{equation*}
So we see that the sequence $x_1,x_2,\dotsc$ 
has an upper bound.\\
So $\bv_n x_n$ exists,
as~$E$ is $\sigma$-Dedekind complete.
Hence~$E$ is $R$-complete.


\item
The lexicographic plane~$\Lex$ (see Examples~\ref{E:oag}\ref{E:oag_lex})
is $R$-complete,
but~$\Lex$ is not $\sigma$-Dedekind complete
(see Examples~\ref{E:sdc}\ref{E:sdc_lex}).

\item
The ordered Abelian group~$\Q$ is \emph{not} $R$-complete.\\
To see this,
and pick $q_1 \leq q_2\leq \dotsb$ in~$\Q$
with
\begin{equation*}
q_{n+1} - q_{n} \ \leq\ 2^{-(n+1)}
\qquad\quad\text{and}\quad\qquad  \text{$\bv_n q_n = \sqrt{2}$ in~$\R$.}
\end{equation*}
Note that $q_1\leq q_2 \leq \dotsb$
has no supremum in~$\Q$.

Now, let $y_n \eqdf 1-2^{-n}$ for all~$n\in\N$.
Then $y_1 \leq y_2 \leq \dotsb$ has an supremum, namely~$1$,
and we have $y_{n+1} - y_{n} = 2^{-(n+1)}$. So we see that
\begin{equation*}
q_{n+1} - q_{n} \ \leq\ y_{n+1} - y_n
\qquad\quad(n\in\N).
\end{equation*}
If~$\Q$ were~$\R$-complete,
then the above implies 
 $q_1 \leq q_2 \leq \dotsb$
would have a supremum in~$\Q$,
which it does not.
Hence~$\Q$ is not $R$-complete.
\end{enumerate}
\end{exs}

\begin{rem}
\label{R:R-dual}
Let~$E$ be an ordered Abelian group.
Using the map $x\mapsto -x$,
one can easily verify
that $E$ is $R$-complete
if and only if the following statement holds.
\begin{equation*}
\left[\quad 
\begin{minipage}{.7\columnwidth}
Let $x_1 \geq x_2 \geq \dotsb$
and $y_1 \geq y_2 \geq \dotsb$ be from~$E$
such that
\begin{equation*}
x_{n} - x_{n+1}\ \leq\ y_{n} - y_{n+1}\qquad \text{for all }n.
\end{equation*}
Then $\bw x_n $ exists whenever  $\bw y_n$ exists.
\end{minipage}
\right.
\end{equation*}
\end{rem}

\begin{prop}
\label{P:R-main}
Let  $E$ be an ordered Abelian group which is $R$-complete.\\
Let~$L$ be a lattice, and let $\varphi\colon L \ra E$ be a valuation.
\begin{enumerate}
\item
\label{P:R-main-descending}
If  $a_1 \geq a_2 \geq \dotsb$,
$b_1 \geq b_2 \geq \dotsb$
are  $\varphi$-convergent
sequences from~$L$,
then
\begin{equation*}
a_1 \wedge b_1 \,\geq\, a_2 \wedge b_2 \,\geq\, \dotsb
\qquad\text{and}\qquad
a_1 \vee b_1 \,\geq\, a_2 \vee b_2 \,\geq\, \dotsb
\end{equation*}
are $\varphi$-convergent.

\item
\label{P:R-main-ascending}
If  $a_1 \leq a_2 \leq \dotsb$,
$b_1 \leq b_2 \leq \dotsb$
are  $\varphi$-convergent
sequences from~$L$,
then
\begin{equation*}
a_1 \wedge b_1 \,\leq\, a_2 \wedge b_2 \,\leq\, \dotsb
\qquad\text{and}\qquad
a_1 \vee b_1 \,\leq\, a_2 \vee b_2 \,\leq\, \dotsb
\end{equation*}
are $\varphi$-convergent.
\end{enumerate}
\end{prop}
\begin{proof}
\ref{P:R-main-descending}\ 
We prove that $a_1 \wedge b_1 \geq a_2 \wedge b_2 \geq\dotsb$
is $\varphi$-convergent.
For this we need to show that $\bw_n \varphi(a_n\wedge b_n)$ exists.
Note that since $\bw_n \varphi (a_n)$
and $\bw_n \varphi(b_n)$ exist,
we know that $\bw_n \ (\varphi(a_n) + \varphi(b_n))$
exists (by Lemma~\ref{L:addition-of-infima}).
So by $R$-completeness,
in order to show $\bw_n\varphi(a_n \wedge b_n)$ exists,
it suffices to prove that (see Remark~\ref{R:R-dual}),
\begin{equation*}
\varphi(a_{n}\wedge b_{n}) \,-\, \varphi(a_{n+1} \wedge b_{n+1}) 
\ \leq\ 
(\,\varphi(a_{n}) + \varphi(b_{n})\,) 
\,-\, (\,\varphi(a_{n+1}) + \varphi(b_{n+1})\,).
\end{equation*}
Phrased differently
using ``$d_\varphi$''
(see Definition~\ref{D:d}),
we need to prove that
\begin{equation*}
d_\varphi(a_{n}\wedge b_{n},\, a_{n+1} \wedge b_{n+1}) 
\ \leq\ 
d_\varphi(a_{n},a_{n+1}) + d_\varphi(b_{n},b_{n+1}).
\end{equation*}
This follows  from Lemma~\ref{L:wv-unif}.

The 
proof that~$a_1 \vee b_1 \geq a_2 \vee b_2 \geq \dotsb$
is $\varphi$-convergent is similar.

\ref{P:R-main-ascending}.  Again, similar.
\end{proof}
%
%                  COUNTEREXAMPLE ON PROP R-MAIn
%
\begin{ex}
\label{E:P:R-main}
We will prove that
the assumption in Proposition~\ref{P:R-main},
that~$E$ is $R$-complete, is necessary.

Let $\mathcal{A}$ be the ring of subsets (see Example~\ref{E:ring-val})
of~$\R$ generated by the non-empty closed intervals with \emph{rational}
 endpoints,
i.e., subsets of the form $[q,r]$ where $q,r\in \Q$ and $q\leq r$.
Then there is a unique positive and additive map $\mu\colon\mathcal{A} \ra\Q$
such that
\begin{equation*}
\mu(\,[q,r]\,) \ =\ r-q\qquad\quad\text{for all $q\leq r$ from $\Q$}.
\end{equation*}

Recall that~$\Q$ is not $R$-complete.
To prove that the conclusion of Proposition~\ref{P:R-main}
does not hold for~$E=\Q$, we will find $\mu$-convergent 
sequences $A_1 \subseteq A_2 \subseteq \dotsb$
and $B_1 \subseteq B_2 \subseteq\dotsb$ such that
$A_1 \cup B_1 \ \subseteq\ A_2 \cup B_2 \ \subseteq\ \dotsb$
is not $\mu$-convergent.

If we have done this,
we see that the assumption ``$E$ is $R$-complete''
is necessary.

Find rational numbers $ \dotsb \leq r_2 \leq r_1 < q_1 \leq q_2 \leq \dotsb$
such that, in~$\R$,
\begin{equation*}
\bv_n q_n \ = \ \sqrt2\qquad\text{and}\qquad
\bw_n r_n \ = \ \sqrt2-1.
\end{equation*}
Now, let us define $A_1 \subseteq A_2 \subseteq \dotsb$
and $B_1 \subseteq B_2 \subseteq \dotsb$ in~$\mathcal{A}$
by, for~$n\in\N$,
\begin{equation*}
A_n  \ = \ [0, r_1]
\qquad\text{and}\qquad
B_n \ = \ [r_n,q_n].
\end{equation*}
Then clearly $A_1 \subseteq A_2 \subseteq \dotsb$
is $\varphi$-convergent.
Note that $\mu(B_n) = q_n - r_n$. So, in~$\R$,
\begin{equation*}
\bv_n \mu(B_n) \ =\ \bv_n q_n - \bw_n r_n \ =\ 1.
\end{equation*}
Hence $\bv_n \mu(B_n)=1$ in~$\Q$.
So  $B_1 \subseteq B_2 \subseteq \dotsb$
is a $\mu$-convergent sequence.

However $A_n \cup B_n = [0,q_n]$,
and thus $\mu(A_n \cup B_n) = q_n$.
So we see that, in~$\R$,
\begin{equation*}
\bv_n \mu(\,A_n \cup B_n \,) \ =\ \bv_n q_n \ =\ \sqrt{2}.
\end{equation*}
So $\mu(A_1 \cup B_1) \ \leq\  \mu(A_2 \cup B_2) \ \leq\ \dotsb$
has no supremum in~$\Q$.

Hence $A_1 \cup B_1 \ \subseteq\ A_2 \cup B_2 \ \subseteq\ \dotsb$
is not $\mu$-convergent.
\end{ex}

%%%%%%%%%%%%%%%%%%%%%%%%%%%%%%%%%%%%%%%%%%%%%%%%%%%%%%%%%%%%%%%%%%%%%%%%%%%%%%%
%%%%%%%%%%%%%%%%%%%%%%%%%%%%%%%%%%%%%%%%%%%%%%%%%%%%%%%%%%%%%%%%%%%%%%%%%%%%%%%

\subsection{Convergence Theorems}
\label{SS:complete-val_convergence}
%
%                  CONVERGENCE OF SEQUENCES
%
The notion of a complete valuation
has been based on Levi's Monotone Convergence Theorem
(see Example~\ref{E:int-complete-val}).
In this subsection,
we prove variants of some of the other classical convergence theorems
of integration theory.
For example, 
Lebesgue's Dominated Convergence Theorem.
It states:
\begin{equation}
\label{eq:Lebesgue}
\left[\quad
\begin{minipage}{.7\columnwidth}
Let $f_1,\,f_2,\,\dotsc$ be a sequence in~$\LF$.

Assume $f_1(x),\,f_2(x),\,\dotsc$
converges for almost all~$x\in\R$.

Assume that $f_1,\,f_2,\,\dotsc$
 is dominated in the sense that
 $|f_n|\leq D$ for all~$n$
for some $D\in\LF$.

Then there is an $f\in\LF$
with $f_1(x),\,f_2(x),\,\dotsc$ converges to $f(x)$ for 
for almost all~$x\in\R$,
and $\Lphi(f) = \lim_n\Lphi(f_n)$.
\end{minipage}
\right.
\end{equation} 
The difficulty 
in the setting of  valuation systems
is not the proof of the theorem,
but its formulation.
For instance, it 
not clear how we should interpret 
\begin{equation*}
\text{``$f_1(x),\,f_2(x),\,\dotsb$
converges for almost all~$x$''}
\end{equation*}
when the objects $f_n$ are not necessarily functions,
but elements of a lattice~$V$.

%
%                  CONVERGENCE IN A LATTICE
%
Let us begin by generalising the notion of convergence in~$\R$
to any lattice~$L$.
Recall that a sequence $a_1,a_2,\dotsc$ in~$\R$
is convergent (in the usual sense) if and only if 
the \emph{limit inferior}, $\lim_{N} \inf_{n\geq N}\,a_n$,
and the \emph{limit superior}, $\lim_{N} \sup_{n\geq N}\,a_n$,
exist and are equal.
This leads us to the following definitions.
\begin{dfn}
\label{D:conv}
Let~$L$ be a lattice.
Let $a_1,\,a_2,\,\dotsc$ be a sequence in~$L$.
\begin{enumerate}
\item
We say $a_1,\,a_2,\,\dotsc$ 
is \keyword{upper convergent}
if the following exists.
\begin{equation*}
\ulim{n} a_n \ \eqdf\ \bw_N\bv_{n\geq N}\ a_N \vee \dotsb \vee a_n.
\end{equation*}
Similarly,
we say $a_1,\,a_2,\,\dotsc$ is \keyword{lower convergent}
if the following exists.
\begin{equation*}
\llim{n} a_n \ \eqdf\ \bv_N\bw_{n\geq N}\ a_N \wedge \dotsb \wedge a_n.
\end{equation*}

\item
We say $a_1,\,a_2,\,\dotsc$ is \keyword{convergent}
if it is both upper and lower convergent,
and in addition $\ulim{n}a_n = \llim{n}a_n$.
In that case,
we write $\lim_n a_n\eqdf\ulim{n} a_n$.

\item
Let $a\in L$ be given.
We say $a_1,\,a_2,\,\dotsc$ 
\keyword{converges to}~$a$
if  $a=\lim_n a_n$.
\end{enumerate}
\end{dfn}
%
%                  REMARK ON THE INEQUALITY OF LIMSUP AND LIMINF
%
\begin{rem}
\label{R:conv}
Let~$L$ be a lattice.
Let $a_1,\,a_2,\,\dotsc$ be a sequence in~$L$,
which is  upper convergent and lower convergent.
Then we have the following inequality.
\begin{equation*}
\llim{n} a_n \ \leq\ \ulim{n} a_n
\end{equation*}
Indeed,
this follows immediately from the observation
that, for every~$N\in\N$,
\begin{equation*}
\bv_{n\geq N}\ a_N \wedge \dotsb \wedge a_n
\ \leq\  a_N\ \leq\ 
\bw_{n\geq N}\ a_N \vee \dotsb \vee a_n.
\end{equation*}
\end{rem}
%
%                  EXAMPLES OF CONVERGENCE IN A LATTICE
%
\begin{exs}
\label{E:conv}
\begin{enumerate}
\item 
In~$\R$ we have:
A sequence $a_1,\,a_2,\,\dotsc$
is convergent in the sense of Definition~\ref{D:conv}
if and only if $a_1,\,a_2,\,\dotsc$
is convergent as usual.\\
Moreover, if $a_1,\,a_2.\,\dotsc$
is convergent, then $\lim_n a_n$ from Definition~\ref{D:conv}
is also the limit of $a_1,\,a_2,\,\dotsc$
in the usual sense.

\item
Similarly,
in~$\R^X$, where~$X$ is any set,
``convergent'' from Definition~\ref{D:conv}
coincides with the usual ``pointwise convergent''.
\end{enumerate}
\end{exs}

\begin{ex}
Let $X$ be a set.
Let $A_1,\,A_2,\,\dotsc$ be subsets of~$X$.
Then $A_1,\,A_2,\,\dotsc$
is upper and lower convergent in the lattice~$\wp X$, 
and we have, for~$x\in X$,
\begin{alignat*}{3}
x \in \ulim{n} A_n 
\quad&\iff\quad  \forall\,N\ \ \exists\, n\geq N \quad x\in A_n, \\
x \in \llim{n} A_n 
\quad&\iff\quad  \exists\,N\ \ \forall\, n\geq N \quad x\in A_n.
\end{alignat*}
So we see that~$A_1,\,A_2,\,\dotsc$ is \emph{not} convergent
iff there is an~$\tilde{x}\in X$ such that
\begin{equation*}
\forall\,N\ \ \exists\, n\geq N \quad \tilde{x}\in A_n
\qquad\text{and}\qquad
\forall\,N\ \ \exists\, n\geq N \quad \tilde{x}\notin A_n.
\end{equation*}
\end{ex}

\begin{ex}
\label{E:conv_leb}
\textbf{To complete}
It is not hard to see
(cf. Example~\ref{E:int-complete-val}),
that in the space~$\LF$ of Lebesgue integrable functions,
``convergent'' from Definition~\ref{D:conv}
coincides with the usual ``pointwise convergent''.
\end{ex}

%
%                  PHI-CONVERGENCE
%
Let us now think about ``almost everywhere convergent''.
Let $\varphi\colon L \ra E$ be a valuation.
We would like to define a notion of convergence
for sequences in~$L$ 
which coincides with ``convergence
almost everywhere'' in the case of the  Lebesgue integral,
that is, for $\varphi = \Lphi$
(see Example~\ref{E:int-val}).

Unfortunately,
we have not found such a notion of convergence.
Instead,
we study a specific notion of convergence,
which we call \emph{$\varphi$-convergence}
(see Definition~\ref{D:seq-phi-conv})
that \emph{almost} coincides with ``convergence almost everywhere''.

More precisely,
given Lebesgue integrable functions $f_1,\,f_2,\,\dotsc$
and $f$, we have that the
$f_1,\,f_2,\,\dotsc$ $\Lphi$-converges to~$f$
if and only if 
\begin{equation*}
\left[\quad
\begin{minipage}{.7\columnwidth}
 $f_1(x),\,f_2(x),\,\dotsc$ 
converges to~$f(x)$ for allmost all~$x\in \R$,
\emph{and}
there is a Lebesgue integrable function~$D$
such that 
\begin{equation*}
|f_n|\,\leq\, D\qquad\quad(n\in\N).
\end{equation*}
\end{minipage}
\right.
\end{equation*}
We postpone the proof of this 
statement to Example~\ref{E:lebesgue}
as it requires some theory.
\begin{dfn}
\label{D:seq-phi-conv}
Let $E$ be an ordered Abelian group.
Let~$L$ be a lattice.\\
Let $\varphi\colon L \ra E$ be a valuation.
Let $a_1,\, a_2\, \dotsc\in L$ be given.
\begin{enumerate}
\item
We say $a_1,\,a_2,\, \dotsc$
is \keyword{upper $\varphi$-convergent}
if the following exists.
\begin{equation*}
\pulim\varphi{n}a_n \ \eqdf\ 
\bw_N \bv_{n\geq N}\ \varphi(a_N\vee\dotsb\vee a_n)
\end{equation*}
Similarly,
we say $a_1,\,a_2,\,\dotsc$ is \keyword{lower $\varphi$-convergent} if
the following exists.
\begin{equation*}
\pllim\varphi{n}a_n \ \eqdf\ 
\bv_N \bw_{n\geq N}\ \varphi(a_N\wedge\dotsb\wedge a_n)
\end{equation*}

\item
We say $a_1,\,a_2,\,\dotsc$
is \keyword{$\varphi$-convergent}
if it is lower and upper $\varphi$-convergent,
and in addition $\pulim\varphi{n}a_n = \pllim\varphi{n}a_n$.
\end{enumerate}
\end{dfn}
%
%                  REMARK ON INEQUALITY BETWEEN UPPER AND LOWER PHI-LIM
%
\begin{rem}
\label{R:seq-phi-conv}
Let $E$ be an ordered Abelian group.
Let~$L$ be a lattice.\\
Let $\varphi\colon L \ra E$ be a valuation.
Let $a_1,\,a_2,\,\dotsb$ be a sequence in~$L$,
which is upper and lower $\varphi$-convergent.
We have the following inequality
(cf. Remark~\ref{R:conv}).
\begin{equation*}
\pllim\varphi{n}a_n \ \leq\ \pulim\varphi{n}a_n.
\end{equation*}
\end{rem}
%
%                  EXAMPLE OF PHI-CONV SEQUENCE
%
\begin{ex}
\label{E:seq-phi-conv}
Let $f_1,f_2,\dotsc$ be  Lebesgue integrable functions
(see Example~\ref{E:int-val}). \\
We will prove the following.
\begin{equation}
\label{eq:E:seq-phi-conv}
\left[\quad
\begin{minipage}{.7\columnwidth}
 The sequence $f_1,f_2,\dotsc$
is  upper and lower $\varphi$-convergent.
\begin{center}
$\Longleftrightarrow$
\end{center}
There is a Lebesgue integrable~$D$ with $|f_n| \leq D$ for all~$n$.
\end{minipage}
\right.
\end{equation}

\vspace{.3em}
\emph{($\Longrightarrow$)}
Assume that  $f_1,\,f_2,\,\dotsc$ is upper and lower $\Lphi$-convergent.
We prove there is a Lebesgue integrable function~$D$ such that
$|f_n|\leq D$ for all~$n\in \N$.

Since $f_1,\,f_2,\,\dotsc$ is  upper $\Lphi$-convergent
(see Definition~\ref{D:seq-phi-conv}), 
we know that 
\begin{equation*}
\bv_n \ \Lphi(f_1 \vee \dotsb \vee f_n)\qquad\text{exists.}
\end{equation*}
In other words, we know that
\begin{equation*}
f_1 \ \leq\ f_1 \vee f_2 \ \leq\ \dotsb \qquad\text{is $\Lphi$-convergent.}
\end{equation*}
Since $\Lphi$ is complete
(see Example~\ref{E:int-complete-val}),
this implies that $\ol{D}\eqdf \bv_n f_n$ exists in~$\LF$.
Note that $f_n \leq \ol{D}$ for all~$n\in \N$.

By a similar reasoning, but using the fact that
$f_1,\,f_2,\,\dotsc$ is lower $\varphi$-convergent,
we can find a Lebesgue integrable function~$\underline{D}$
such that $\ul{D}\leq f_n$ for all~$n\in \N$.

Now define $D\eqdf  \ol{D} \vee (-\ul{D})$. We see that $|f_n|\leq D$
for all~$n\in \N$.

\vspace{.3em}

\emph{($\Longleftarrow$)}
Assume
 that $|f_n|\leq D$
for some Lebesgue integrable~$D$.

We prove that $f_1,\,f_2,\,\dotsc$
is upper $\Lphi$-convergent.
For this,
we must show that the following exists
(see Definition~\ref{D:seq-phi-conv}).
\begin{equation}
\label{eq:E:seq-phi-conv-1}
\bw_N \bv_{n\geq N}\ \Lphi(f_N\vee\dotsb\vee f_n)
\end{equation}
Let $N\in\N$ and $n\geq N$ be given.
Note that we have 
\begin{equation*}
f_N \vee \dotsb\vee f_n \ \leq\ D.
\end{equation*}
Since $\Lphi$ is order preserving, this gives us
\begin{equation*}
\Lphi(f_N \vee \dotsb \vee f_n)
\ \leq\  \Lphi(D).
\end{equation*}
So we see that 
$\Lphi(f_N) \ \leq\ 
\Lphi(f_N \vee f_{N+1}) \ \leq\ \dotsb$
is an ascending sequence in~$\R$
bounded from above by~$\Lphi(D)$.
Hence $\bv_{n\geq N} \, \Lphi(f_N\vee \dotsb \vee f_n)$ exists.
I.e.,
\begin{equation*}
f_N \ \leq \ f_N \vee f_{N+1} \ \leq\ \dotsb
\qquad\text{is $\Lphi$-convergent.}
\end{equation*}
Since $\Lphi$ is complete
(see Example~\ref{E:int-complete-val}),
this implies $\ol{f}_N \eqdf \bv_{n\geq N}\,f_n$ exists and
\begin{equation*}
\Lphi(\ol{f}_N)\ 
=\ \bv_{n\geq N}\ \Lphi(f_N \vee \dotsb \vee f_n).
\end{equation*}
Recall that we must prove that Expression~\eqref{eq:E:seq-phi-conv-1}
exists.
In other words,
we must show that $\bw_N \Lphi(\ol{f}_N)$ exists.
Note that $-D \leq  f_n$ for all~$n\in\N$.
Hence 
\begin{equation*}
-D \ \leq \   \ol{f}_N  = \bv_{n \geq N}\, f_n .
\end{equation*}
We now have a descending sequence
$\Lphi(\ol{f}_1) \,\geq\, \Lphi(\ol{f}_2) \,\geq\,\dotsb$
in~$\R$ that is bounded from below by $\Lphi(-D)$.
Hence $\bw_N \Lphi(\ol{f}_N)$ exists.

We have proven that $f_1,\,f_2,\,\dotsc$ 
is upper $\varphi$-convergent.
With a similar reasoning one can prove that
$f_1,\,f_2,\,\dotsc$
is lower $\varphi$-convergent.

This completes the proof of Statement~\eqref{eq:E:seq-phi-conv}.

\vspace{.3em}

Again, let $f_1,\,f_2,\,\dotsc$ be a sequence of
Lebesgue integrable functions.
Assume that $f_1,\,f_2,\,\dotsc$
is upper and lower $\Lphi$-convergent.
Note that the sequence $f_1,\,f_2,\,\dotsc$
need not be $\Lphi$-convergent.
That is,
we do not always have
 $\pulim\Lphi{n} f_n = \pllim\Lphi{n} f_n$.
Indeed, consider $f_1,\,f_2,\,\dotsc$
given by
\begin{equation*}
f_n \ =\  (-1)^{n}\cdot \mathbf{1}_{[0,1]}.
\end{equation*}
Then we have the following inequality
\begin{equation*}
\pllim\Lphi{n} f_n \ = \ -1 \ <\  +1 \ = \ 
\pulim\Lphi{n} f_n.
\end{equation*}

Indeed,
as has been remarked before,
we will lateron prove 
(see Example~\ref{E:lebesgue}) that such a sequence $f_1,\,f_2,\,\dotsc$
is $\varphi$-convergent if and only if 
\begin{equation*}
\text{ $f_n(x)$ converges for almost all~$x\in\R$.}
\end{equation*}
\end{ex}

%
%                  LEMMA OF FATOU
%
\noindent The following lemma
is a generalisation of the Lemma of Fatou.
\begin{lem}
\label{L:fatou}
Let $E$ be an ordered Abelian group.\\
Let $L$ be a lattice,
and let $\varphi\colon L \ra E$ be a 
complete valuation.\\
Let $a_1,a_2,\dotsc$ be an upper $\varphi$-convergent
sequence in~$L$
(see Definition~\ref{D:seq-phi-conv}) \\
Then $a_1,a_2,\dotsc$ is upper convergent
(see Definition~\ref{D:conv}),
and we have
\begin{equation*}
\varphi(\ulim{n}a_n) \ =\ 
\pulim\varphi{n} a_n.
\end{equation*}
Moreover,
if $E$ is a lattice, and
if $\ulim{n}\varphi(a_n)$ exists
(see Definition~\ref{D:conv}), then
\begin{equation*}
\pulim\varphi{n}a_n
\ \geq\  
\ulim{n}\varphi(a_n) .
\end{equation*}
\end{lem}
\begin{proof}
Let $a_1,\,a_2,\,\dotsc$ be a upper $\varphi$-convergent sequence.
We prove that $a_1,a_2,\dotsc$ is upper convergent
(see Definition~\ref{D:conv}),
and that $\varphi(\ulim{n}a_n) = \pulim\varphi{n}a_n$.

Let $N\in\N$ be given.
Note that $\bv_{N\geq n} \,\varphi(a_N\vee\dotsb\vee a_n)$
exists because the sequence~$a_1,\,a_2,\,\dotsc$ is upper $\varphi$-convergent
(see Definition~\ref{D:seq-phi-conv}).
So the sequence
\begin{equation*}
a_N \,\leq\ a_N \vee a_{N+1} 
    \ \leq\quad a_N\vee a_{N+1} \vee a_{N+2} 
    \quad \leq\qquad \dotsb
\end{equation*}
is $\varphi$-convergent (in the sense of
Definition~\ref{D:phi-conv}).
For brevity,
let us write
\begin{equation*}
\overline{a}_{N}^n \ \eqdf \ a_N\vee\dotsb\vee a_{N+n},
\end{equation*}
Since~$\varphi$ is complete,
and $\overline{a}_N^0 \leq \overline{a}_N^1 \leq\dotsb$
is $\varphi$-convergent,
we get $\bv_n\,\overline{a}_N^n$ exists,
and 
\begin{equation*}
\varphi(\overline{a}_N) 
\ =\ 
 \bv_{n}\ \varphi( \overline{a}_N^n).
\end{equation*}
For brevity, let us write
\begin{equation*}
\overline{a}_N \ \eqdf \ \bv_n \,\overline{a}_N^n.
\end{equation*}

Note that $\overline{a}_1 \geq \overline{a}_2 \geq\dotsb$
is $\varphi$-convergent,
because
\begin{equation*}
\pulim\varphi{n} a_n
\ \eqdf\ 
\bw_N\bv_n\, \varphi(\overline{a}^n_N)
\end{equation*}
exists as $a_1,\,a_2,\,\dotsc$ is upper $\varphi$-convergent.
Since $\varphi$ is complete,
this implies that
\begin{equation*}
\bw_n \overline{a}_N\text{ exists}
\qquad\text{and}\qquad \varphi(\bw_n \overline{a}_N)
\,=\,
\bw_n\varphi(\overline{a}_N).
\end{equation*}
Now,
note that  we have the following equality.
\begin{equation*}
\bw_N \overline{a}_N \ =\ 
\bw_N \bv_{n\geq N} \, a_n
\end{equation*}
So we see that $a_1,a_2,\dotsc$
is $\varphi$-convergent and that
\begin{equation*}
\varphi(\ulim{n}a_n)
\ =\ 
\bw_N\varphi(\overline{a}_N)
\ =\ 
\bw_N \bv_n \varphi(\overline{a}^n_N)
\ =\ 
\pulim\varphi{n} a_n.
\end{equation*}
We have proven the first part of the lemma.

Assume $E$ is a lattice and $\ulim{n}\varphi(a_n)$ exists
(see Definition~\ref{D:conv}).
To prove the remainder of the theorem,
we need to show that 
$\pulim\varphi{n}a_n \geq \ulim{n}\varphi(a_n)$.
That is,
\begin{equation*}
\bw_N \bv_{n\geq N} \ \varphi(a_N \vee \dotsb \vee a_n)
\ \geq \ 
\bw_N \bv_{n\geq N} \ \varphi(a_N) \vee \dotsb \vee \varphi(a_n).
\end{equation*}
This is easy.  It follows immediately
from the fact that
\begin{equation*}
\varphi(a_N\vee \dotsb\vee a_n)
\ \geq\ \varphi(a_N)\vee \dotsb \vee \varphi(a_n)
\end{equation*}
for all~$N\in\N$ and $n\geq N$.
\end{proof}

%
%                  LEBESGUE'S DOMINATED CONVERGENCE THEOREM
%
\noindent
The following theorem contains the gist
of Lebesgue's Dominated Convergence Theorem,
although there is no word of domination.
\todo{Add remark on
$E$ which are conditionally complete.}
\begin{thm}
\label{T:lebesgue}
Let $E$ be an ordered Abelian group.\\
Let $L$ be a lattice,
and let $\varphi\colon L \ra E$ be a 
complete valuation.\\
Let $a_1,a_2,\dotsc$ be a $\varphi$-convergent
sequence in~$L$
(see Definition~\ref{D:seq-phi-conv}). \\
Then $a_1,a_2,\dotsc$ is convergent
(see Definition~\ref{D:conv}),
and we have
\begin{equation*}
\varphi(\llim{n}a_n) \,=\, \varphi(\ulim{n} a_n) 
\ =\ 
\plim\varphi{n} a_n.
\end{equation*}
Moreover,
if $E$ is a lattice,
and if $\llim{n}\varphi(a_n)$ and $\ulim{n}\varphi(a_n)$ exists, 
then
\begin{equation*}
\plim\varphi{n}a_n \ =\ {\lim}_n \varphi(a_n).
\end{equation*}
\end{thm}
\begin{proof}
Let $a_1,a_2,\dotsc$ be a $\varphi$-convergent sequence.
The first part of the theorem follows immediately from Lemma~\ref{L:fatou}
and its dual.

Assume that $\llim{n}\varphi(a_n)$ and $\ulim{n}\varphi(a_n)$ exist.
By Lemma~\ref{L:fatou}, we see that
\begin{equation*}
\pllim\varphi{n}a_n \,\leq\,
\llim{n}\varphi(a_n) \,\leq\,
\ulim{n}\varphi(a_n) \,\leq\,
\pulim\varphi{n}a_n.
\end{equation*}
But $\pllim\varphi{n}a_n = \pulim\varphi{n}a_n$,
since $a_1,a_2,\dotsc$ is $\varphi$-convergent.
So we get 
\begin{equation*}
\pllim\varphi{n}a_n \,=\,
\llim{n}\varphi(a_n) \,=\,
\ulim{n}\varphi(a_n) \,=\,
\pulim\varphi{n}a_n.
\end{equation*}
In particular,
$\varphi(a_1),\,\varphi(a_2),\,\dotsc$
is convergent and ${\lim}_n\varphi(a_n) = \plim\varphi{n}a_n$.
\end{proof}
%
%                  EXAMPLE ON LEBESGUE'S THEOREM
%
\begin{ex}
\label{E:lebesgue}
Let us now prove Lebesgue's Dominated Convergence Theorem,
 see Statement~\eqref{eq:Lebesgue}, 
using Theorem~\ref{T:lebesgue}.
To do this, we first establish the relationship between
``$\Lphi$-convergent'' and ``convergent almost everywhere''.

Let $f_1,\,f_2,\,\dotsc$ be a sequence of Lebesgue integrable functions.
We will prove:
\begin{equation}
\label{eq:E:lebesgue-0}
\left[\quad
\begin{minipage}{.7\columnwidth}
The sequence $f_1,\,f_2,\,\dotsc$
is $\Lphi$-convergent
\begin{center}
if and only if
\end{center}
 $f_1(x),\,f_2(x),\,\dotsc$ converges
for allmost all~$x\in \R$,
and
there is a Lebesgue integrable function~$D$
such that 
\begin{equation*}
|f_n|\,\leq\, D\qquad\quad(n\in\N).
\end{equation*}
\end{minipage}
\right.
\end{equation}

\vspace{.3em}
\emph{($\Longrightarrow$)}
Suppose that $f_1,\,f_2,\,\dotsc$
is $\Lphi$-convergent.
By Example~\ref{E:seq-phi-conv}
we know that there is a Lebesgue integrable function~$D$
such that $|f_n|\leq D$ for all~$n\in\N$.
So we only need to show that $f_n(x)$ 
converges for almost all~$x\in\R$.

By Theorem~\ref{T:lebesgue},
we know that $f_1,\,f_2,\,\dotsc$
is upper convergent and lower convergent
(see Definition~\ref{D:conv}),
and that 
\begin{equation}
\label{eq:E:lebesgue}
\Lphi(\llim{n} f_n) \ =\  \Lphi(\ulim{n} f_n).
\end{equation}
Since $\llim{n} f_n \leq \ulim{n} f_n$
(see Remark~\ref{R:conv})
one can easily see that Statement~\eqref{eq:E:lebesgue}
implies that 
 $\llim{n} f_n \approx \ulim{n} f_n$
(see Definition~\ref{D:eq}).
Hence (see Example~\ref{E:eq-int})
\begin{equation*}
 (\,\llim{n} f_n\,)(x) \ =\  (\,\ulim{n} f_n\,)(x)
\qquad\text{for almost all~$x\in \R$.}
\end{equation*}
Or in other words
(cf.~Example~\ref{E:conv}) we have
\begin{equation*}
 \llim{n} (\,f_n(x)\,) \ =\  \ulim{n} (\,f_n(x)\,)
\qquad\text{for almost all~$x\in \R$.}
\end{equation*}
So we see that $f_1(x),\,f_2(x),\,\dotsc$
is convergent
for almost all~$x\in \R$.

\vspace{.3em}
\emph{($\Longleftarrow$)}
Suppose that there is a Lebesgue integrable function~$D$
such that $|f_n|\leq D$ for all~$n\in\N$,
and that  $f_1(x),\,f_2(x),\,\dotsc$
is convergent for almost all~$x\in\R$.

We must prove that $f_1,\,f_2,\,\dotsc$
is $\Lphi$-convergent (see Definition~\ref{D:seq-phi-conv}).

By Example~\ref{E:seq-phi-conv}
we know that $f_1,\,f_2,\,\dotsc$ is upper and lower
$\Lphi$-convergent.
So the only thing that we must still show is
that 
\begin{equation}
\label{eq:E:lebesgue-2}
\pulim\Lphi{n} f_n \ =\  \pllim\Lphi{n} f_n.
\end{equation}

By Lemma~\ref{L:fatou}, our generalisation of the Lemma of Fatou,
we know that $f_1,\,f_2,\,\dotsc$
is upper and lower convergent in~$\LF$,
and that 
\begin{equation*}
\Lphi(\,\llim{n} f_n\,) \ =\ 
\pllim\Lphi{n} f_n
\qquad\text{and}\qquad
\Lphi(\,\ulim{n} f_n\,) \ =\ 
\pulim\Lphi{n} f_n.
\end{equation*}
So to prove Equality~\eqref{eq:E:lebesgue-2},
it suffices to show that 
\begin{equation}
\label{eq:E:lebesgue-3}
\Lphi(\,\llim{n} f_n\,) \ =\  \Lphi(\,\ulim{n} f_n\,).
\end{equation}
Recall that $\llim{n}f_n \leq \ulim{n}f_n$
(see Remark~\ref{R:conv}).
So Statement~\ref{eq:E:lebesgue-3}
is equivalent to 
\begin{equation*}
\llim{n} f_n \ \approx\  \ulim{n} f_n,
\end{equation*}
where $\approx$ is from Definition~\ref{D:eq}.
So we must prove that (see Example~\ref{E:eq-int})
\begin{equation*}
(\,\llim{n} f_n\,) (x)\ \approx\  (\,\ulim{n} f_n\,)(x)
\qquad\text{for almost all $x\in\R$}.
\end{equation*}
The reader can verify that this is equivalent to
\begin{equation*}
\llim{n} (\, f_n (x)\,)\ \approx\  \ulim{n} (\,f_n(x)\,)
\qquad\text{for almost all $x\in\R$}.
\end{equation*}
That is (see Example~\ref{E:conv}),
we must prove that $f_1(x),\,f_2(x),\,\dotsc$
converges for almost all~$x\in\R$.
But this is exactly what we have assumed.

This concludes the proof of Statement~\eqref{eq:E:lebesgue-0}.

\vspace{.3em}
\noindent
We now prove Lebesgue's Dominated Convergence Theorem
(see Stat.~\eqref{eq:Lebesgue}).

Let $f_1,\,f_2,\,\dotsc$ be a sequence of Lebesgue integrable functions.
Assume there is an Lebesgue integrable function~$D$ such that
$|f_n|\leq D$ for all~$n\in\N$,
and assume that $f_1(x),\,f_2(x),\,\dotsc$ converges
for almost all~$x\in\R$.

We must prove that there is a Lebesgue integrable~$f$
such that $f_1(x),\,f_2(x),\,\dotsc$
converges to~$f(x)$ for almost all~$x\in\R$,
and $\Lphi(f) = \lim_n \Lphi(f_n)$.

By Statement~\eqref{eq:E:lebesgue-0},
we know that $f_1,\,f_2,\,\dotsc$
$\Lphi$-converges.
So by Theorem~\ref{T:lebesgue} we get that
$f_1,\,f_2,\,\dotsc$
is upper convergent
and that $\Lphi(\,\ulim{n}f_n\,) = \lim_n \Lphi(f_n)$.

So let us define $f\eqdf \ulim{n}f_n$.
To complete the proof,
we must show that $f_1(x),\,f_2(x),\,\dotsc$
converges to~$f(x)=\ulim{n}f_n(x)$
for almost all~$x\in \R$.
This is indeed the case since~$f_1(x),\,f_2(x),\,\dotsc$
converges for almost all~$x\in\R$.
\end{ex}

There are many variants of the classical convergence theorems of
integration. 
For instance,
a variant on Levi's Monotone Convergence Theorem
is the following.
\begin{equation*}
\left[\quad
\begin{minipage}{.7\columnwidth}
Let $f_1,\,f_2,\,\dotsc$ be
 Lebesgue integrable functions.\\
Assume that $\bv_n \Lphi(f_n)$ exists.\\
Assume that 
for every $n\in\N$,
\begin{equation*}
f_n(x) \,\leq\, f_{n+1}(x) \qquad\text{ for almost all~$x\in\R$.}
\end{equation*}
Then $\bv_n f_n$ is Lebesgue integrable
and 
\begin{equation*}
\Lphi(\bv_n f_n) \,=\, \bv_n \Lphi(f_n).
\end{equation*}
\end{minipage}
\right.
\end{equation*}
Note that to prove the above statement
it suffices to show that the valuation
\begin{equation*}
\qvphi{\Lphi}\colon \qvL{\LF} \ra \R
\end{equation*}
from Proposition~\ref{P:quotient-lattice} 
is complete. We will prove this in Proposition~\ref{P:quot-complete}.

Of course,
if we apply
Lemma~\ref{L:fatou}
and Theorem~\ref{T:lebesgue} 
to
$\qvphi{\Lphi}$,
we obtain variants of the Lemma of Fatou
and the Dominated Convergence Theorem of Lebesgue,
respectively. We leave this to the reader.

%
%                 QUOTIENT OF COMPLETE IS COMPLETE
%
\begin{prop}
\label{P:quot-complete}
Let~$L$ be a lattice. Let $E$ be an ordered Abelian group.\\
Let $\varphi\colon L\ra E$ be a complete valuation.
Then the valuation
\begin{equation*}
\qvphi\varphi\colon \qvL{L} \ra E
\end{equation*}
from Proposition~\ref{P:quotient-lattice}
is a complete valuation.
\end{prop}
\begin{proof}
We leave this to the reader.
\end{proof}

There is a small gap that needs to filled before
we continue with another topic.
Let $\varphi\colon L \ra E$ be a valuation.
We have defined what it means
for a sequence $a_1,\,a_2,\,\dotsc$ in~$L$ to be
$\varphi$-convergent
(see Definition~\ref{D:seq-phi-conv}),
but we have not yet given the meaning of
``$a_1,\,a_2,\,\dotsc$ converges \emph{to} $a$''.
We will do this in Definition~\ref{D:seq-phi-conv-to}.
%
%                  DEFINITION OF PHI-CONV TO AN ELEMENT
%
\begin{dfn}
\label{D:seq-phi-conv-to}
Let $L$ be a lattice.
Let $E$ be an ordered Abelian group.\\
Let $\varphi\colon L\ra E$ be a valuation.
Let $a_1,\,a_2,\,\dotsc$ be a sequence in~$L$.\\
Let $a\in L$ be given.
We say $a_1,\,a_2,\,\dotsc$
\keyword{$\varphi$-converges} to~$a$
provided that 
\begin{equation*}
a_1,\,a,\,a_2,\,a,\,\dotsc\qquad\text{is $\varphi$-convergent.}
\end{equation*}
\end{dfn}

Let $\varphi\colon L \ra E$ be a valuation.
While Definition~\ref{D:seq-phi-conv-to} is certainly reasonable,
it is also quite silly, 
and so one wonders if there is a more direct description
of when a sequence $\varphi$-converges to an element~$a\in L$.
If we assume~$\varphi$ is complete,
then there is a slightly 
better description (see Proposition~\ref{P:seq-phi-conv-to}).

In Section~\ref{S:fub} we study a notion of convergence
(see Definition~\ref{D:weak-phi-conv})
which was intended to be a more aesthetically pleasing
definition of $\varphi$-convergence,
but which turns out to be strictly weaker 
than $\varphi$-convergence
(see Example~\ref{E:weak-phi-conv}).
%
%                  EXAMPLE PHI-CONVERGENCE
%
\begin{prop}
\label{P:seq-phi-conv-to}
Let~$E$ be an ordered Abelian group.\\
Let~$L$ be a lattice, and $\varphi\colon L \ra E$ be a complete valuation.\\
Let $a_1,\,a_2,\,\dotsc$ be a $\varphi$-convergent sequence in~$L$,
and let $a\in L$.\\
Then $a_1,\,a_2,\,\dotsc$
$\varphi$-converges to~$a$
if and only if (see Definition~\ref{D:eq})
\begin{equation*}
a\ \approx\ \ulim{n}a_n.
\end{equation*}
(Recall that
$a_1,\,a_2,\,\dotsc$
is upper convergent
(see Definition~\ref{D:conv})
by Lemma~\ref{L:fatou}.)
\end{prop}
\begin{proof}
We leave this to the reader.
\end{proof}
\end{document}
