\documentclass[main.tex]{subfiles}
\begin{document}
\section{Extendibility}
\label{S:benign}
Let 
$\vs{V}{L}{\varphi}{E}$
be a valuation space,
and suppose we want to prove
that $\varphi$ can be extended to a complete valuation.
We have seen
that it suffices to prove that~$\varphi$
is $\Pi_{\aleph_1}$-extendible
(see Corollary~\ref{C:aleph1}).
However,
to prove $\varphi$ is $\Pi_{\aleph_1}$-extendible
already
seems like a monumental task
when one has only barely started
to unfold
the definition of ``$\varphi$ is $\Pi_{\aleph_1}$-extendible''
(see Definition~\ref{P:hier}):
\begin{alignat*}{3}
\varphi &\text{ is $\Pi$-extendible,}&
\quad&\text{ and }\quad& \varphi &\text{ is $\Sigma$-extendible;}\\
\Pi\varphi &\text{ is $\Pi$-extendible,}&
\quad&\text{ and }\quad& \Sigma\varphi &\text{ is $\Sigma$-extendible;}\\
\Pi_2\varphi &\text{ is $\Pi$-extendible,}&
\quad&\text{ and }\quad& \Sigma_2\varphi &\text{ is $\Sigma$-extendible;}\\
&\ \,\vdots&&\quad\vdots&&\ \,\vdots\\
\Pi_\omega\varphi &\text{ is $\Pi$-extendible,}&
\quad&\text{ and }\quad& \Sigma_\omega\varphi &\text{ is $\Sigma$-extendible;}\\
\Pi_{\omega+1}\varphi &\text{ is $\Pi$-extendible,}&
\quad&\text{ and }\quad& \Sigma_{\omega+1}\varphi
    &\text{ is $\Sigma$-extendible;}\\
&\ \,\vdots&&\quad\vdots&&\ \,\vdots
\end{alignat*}

It turns out
that for some~$E$
the situation is more tractable.
For instance,
we will see that if~$E=\R$,
then to prove that $\varphi$ is extendible
it suffices to show that~$\varphi$ is $\Pi_2$-extendible
or $\Sigma_2$-extendible.
Actually,
we have a sharper result:
it suffices to show that~$\varphi$ is \emph{continuous}
(see Definition~\ref{D:continuity}).
Those~$E$ for which we have 
\begin{equation*}
\varphi \text{ is continuous }
\quad\implies\quad
\varphi \text{ is extendible.}
\end{equation*}
will be called \emph{benign} (see Definition~\ref{D:benign}).
%%%%%%%%%%%%%%%%%%%%%%%%%%%%%%%%%%%%%%%%%%%%%%%%%%%%%%%%%%%%%%%%%%%%
%
%                  CONTINUITY
%
\subsection{Continuous Valuations}
Below we define what it means for a valuation system
to be continuous.
We will see that we have the following implications
\begin{equation*}
\xymatrix @=1em {
\Sigma_2\text{-extendible}\ar @{=>} [rd]
&&
\Sigma\text{-extendible}  \\
& \text{continuous} \ar @{=>} [rd] \ar @{=>} [ru]& \\
\Pi_2\text{-extendible}\ar @{=>} [ru]
&&
\Pi\text{-extendible}.
}
\end{equation*}
In fact,
we prove that $\varphi$ is continuous
if and only if it can be extended to~$\Pi L \cup \Sigma L$ in some sense
(see Lemma~\ref{L:continuity}),
so that we might have dubbed
it ``$\Pi\cup\Sigma$-extendible''.
\begin{dfn}
\label{D:continuity}
Let $\vs{V}{L}\varphi{E}$ be a valuation system.\\
We say $\varphi$ (or more precisely  $\vs{V}{L}\varphi{E}$)
is \keyword{continuous} provided that
\begin{equation*}
\bw_n a_n \,\leq\, \bv_n b_n 
\quad\implies\quad
\bw_n \varphi(a_n) \,\leq\, \bv_n \varphi(b_n)
\end{equation*}
for all $\varphi$-convergent $a_1 \geq a_2 \geq \dotsb$
and $\varphi$-convergent $b_1 \leq b_2 \leq \dotsb$.
\end{dfn}
\todo{Add examples:
when an elementary integral is continuous,
when an additive map on a ring is continuous.}
\begin{lem}
\label{L:continuity}
Let $\vs{V}{L}\varphi{E}$ be a valuation system.
The following are equivalent.
\begin{enumerate}
\item
\label{L:continuity-1}
$\varphi$ is continuous.
\item
\label{L:continuity-2}
$\varphi$ is $\Pi$-extendible
and $\Sigma$-extendible,
and there is an order preserving map
\begin{equation*}
f\colon \Pi L \cup \Sigma L \ra E
\end{equation*}
such that~$f$ extends both~$\Pi\varphi$ and $\Sigma\varphi$.
\end{enumerate}
\end{lem}
\begin{proof}
\noindent
\emph{\ref{L:continuity-1}
$\Longrightarrow$
\ref{L:continuity-2}}\ 
Suppose that~$\varphi$ is continuous.
By Lemma~\ref{L:Pi-continuity},
we see that $\varphi$ is $\Pi$-extendible.
Similarly, $\varphi$ must be $\Sigma$-extendible.
We need to find an order preserving map~$f\colon \Pi L \cup \Sigma L \ra E$
that extends both $\Pi\varphi$ and $\Sigma\varphi$.
We have little choice,
\begin{equation}
\label{eq:L:continuity-1}
f(c) \ \eqdf \ 
\begin{cases}
\ \Pi\varphi(c) \quad& \text{if }c\in \Pi L;\\
\ \Sigma\varphi(c) \quad& \text{if }c\in \Sigma L.
\end{cases}
\end{equation}
To see that Equation~\eqref{eq:L:continuity-1}
is a valid definition of a 
map $f\colon \Pi L \cup \Sigma L \ra E$,
we need to verify that $\Pi\varphi$ and $\Sigma\varphi$
are identical on~$\Pi L \cap \Sigma L$.
Let $c\in \Pi L \cap \Sigma L$
be given,
in order to prove that $\Pi\varphi(c) = \Sigma\varphi(c)$.
Choose $\varphi$-convergent
$a_1 \geq a_2 \geq \dotsb$ and
$\varphi$-convergent
$b_1 \leq b_2 \leq \dotsb$
such that $\bw_n a_n =c= \bv_n b_n$.

Then $b_n \leq a_n$ for all~$n$, so
$\varphi(b_n)\leq \varphi(a_n)$ for all~$n$.
Hence
\begin{equation*}
\Sigma\varphi(c)
=\bv_n \varphi(b_n)
\ \leq\ \bw_n \varphi(a_n)
=\Pi\varphi(c).
\end{equation*}

Conversely,
we have $\bw_n a_n \leq \bv_n b_n$,
so since $\varphi$ is continuous we get
\begin{equation*}
\Pi\varphi(c)
=\bw_n \varphi(a_n)
\ \leq\ \bv_n \varphi(b_n)
=\Sigma\varphi(c).
\end{equation*}
Hence $\Pi\varphi(c)=\Sigma\varphi(c)$.
So Equation~\eqref{eq:L:continuity-1}
is a valid definition of~$f$.

Since by defintion,
$f$ extends both $\Pi\varphi$ and $\Sigma\varphi$,
it only remains to be shown that~$f$ is order preserving.
Let $c,d\in \Pi L \cup \Sigma L$
with $c\leq d$ be given.
We prove
\begin{equation*}
f(c)\ \leq\  f(d).
\end{equation*}
Of course,
if $c,d$ are both in $\Pi L$,
then we done,
because $\Pi \varphi$
is order preserving 
and $f$ extends $\Pi\varphi$.
Similarly, if $c,d\in\Sigma L$, 
we also immediately get $f(c)\leq f(d)$.

Suppose $c\in \Pi L$ and $d\in \Sigma L$.
Choose $\varphi$-convergent
 sequences $b_1 \leq b_2 \leq \dotsb$
and  $a_1 \geq a_2 \geq \dotsb$ 
such that 
$\bv_n b_n = c$
and 
$\bw_n a_n = d$.
Then $b_m \leq \bv_n b_n \leq \bw_n a_n \leq a_m$
for all~$m$,
so $\varphi(b_m)\leq \varphi(a_m)$ for all~$m$,
and hence
\begin{equation*}
f(c) = \Sigma\varphi(c)
= \bv_n \varphi(b_n)
\ \leq\ 
\bw_n \varphi(a_n)
= \Pi \varphi(d)
= f(d).
\end{equation*}

Suppose $c\in \Pi L$ and $d\in \Sigma L$.
Choose $\varphi$-convergent sequences $a_1 \geq a_2 \geq \dotsb$ 
and $b_1 \leq b_2 \leq \dotsb$
such that $\bw_n a_n = c$ and $\bv_n b_n = d$.
Then $\bw_n a_n \leq \bv_n b_n $.
So since $f$ is continuous, 
we get 
$f(c)= \Pi\varphi(c) =\bw_n\varphi(a_n) 
\leq\bv_n\varphi(b_n) = \Sigma\varphi(d) = f(d)$.

\vspace{.3em}

\noindent\emph{\ref{L:continuity-2}
$\Longrightarrow$
\ref{L:continuity-1}}\ 
Let $f\colon \Pi L \cup \Sigma L \ra E$
be an order preserving map that extends both $\Pi\varphi$
and $\Sigma \varphi$.
We prove that~$\varphi$ is continuous.
Let $\varphi$-convergent sequences
$a_1 \geq a_2 \geq \dotsb$
and $b_1 \leq b_2 \leq \dotsb$
with $\bw_n a_n \leq \bv_n b_n$
be given.
We need to prove that  
\begin{equation*}
\bw_n \varphi(a_n)\ \leq\  \bv_n \varphi(b_n).
\end{equation*}
This is easy;
since~$f$ is order preserving
and extends $\Pi\varphi$ and $\Sigma\varphi$,
we get 
\begin{equation*}
\bw_n \varphi(a_n)
\,=\,
\Pi\varphi(\bw_n a_n)
\,=\,
 f(\bw_n a_n)
\ \leq\  
f(\bv_n b)
\,=\,
\Sigma\varphi(\bv_n b_n)
\,=\,
\bv_n \varphi(b_n).\qedhere
\end{equation*}
\end{proof}
%
%                  IMPLICATIONS
%
\begin{cor}
\label{C:cont-imp}
Let $\vs{V}{L}\varphi{E}$ be a valuation system. We have
\begin{enumerate}
\item 
\label{C:cont-imp-1}
$\varphi$ is continuous
implies that $\varphi$ is $\Pi$-extendible
and $\Sigma$-extendible.

\item 
\label{C:cont-imp-2}
$\varphi$ is continuous
provided that either $\varphi$ is $\Pi_2$-extendible
or $\Sigma_2$-extendible.
\end{enumerate}
\end{cor}
\begin{proof}
Point \ref{C:cont-imp-1}
follows immediately from Lemma~\ref{L:continuity}.
Point~\ref{C:cont-imp-2} is also
a consequence of Lemma~\ref{L:continuity}.
Indeed, 
assume $\varphi$ is $\Pi_2$-extendible
in order to prove that $\varphi$ is continuous.
Note that $\Pi_2 \varphi$
is order preserving and
extends both $\Pi\varphi$ and $\Sigma\varphi$.
Hence $\varphi$ satisfies condition~\ref{L:continuity-1}
of Lemma~\ref{L:continuity}.
Thus $\varphi$ is continuous.
\end{proof}
%
%                  LEMMA ON EXT OF CONTINUITY
%
\begin{lem}
\label{L:cont-ext}
Let $\vs{V}{L}\varphi{E}$
be a valuation system.\\
Let~$K$ be a sublattice of~$L$
such that $\psi \eqdf \varphi | K$
is continuous.\\
Then $\varphi$ is continuous
under the following assumptions.
\begin{enumerate}
\item\label{L:cont-ext-1}
Given a $\varphi$-convergent sequence $a_1 \geq a_2 \geq \dotsb$ 
in~$L$, we have
\begin{equation*}
\bw_n \varphi(a_n) \ = \ 
\bv\,\bigl\{\ \Pi \psi(\ell) \colon\ 
 \ell \in S \ \bigr\},
\end{equation*}
for some $S\subseteq \Pi K$
with $\ell \leq \bw_n a_n$ for all $\ell \in S$.

\item\label{L:cont-ext-2}
Given a 
$\varphi$-convergent sequence 
$b_1 \leq b_2 \leq \dotsb$ in~$L$,
we have
\begin{equation*}
\bv_n \varphi(b_n) \ = 
\ \bv \ \bigl\{ \ \Sigma\psi(u) \colon \ u \in T\ \bigr\},
\end{equation*}
for some $T\subseteq \Sigma K$ with $\bv_n b_n \leq u$
for all~$u\in T$.
\end{enumerate}
\end{lem}
\begin{proof}
Let $\varphi$-convergent sequences $a_1 \geq a_2 \geq \dotsb$
and $b_1 \leq b_2 \leq \dotsb$ from~$L$ 
with $\bw_n a_n \leq \bv_n b_n$ be given.
To prove that~$\varphi$
is continuous
(see Definition~\ref{D:continuity}),
we must show that $\bw_n \varphi(a_n) \leq \bv_n \varphi(b_n)$.
Let $\ell\in S$ and $u\in T$
be given.
Note that 
$\ell \leq \bw_n a_n \leq \bv_n b_n \leq u$,
so
$\Pi\psi(\ell) \leq \Sigma\psi (u)$
since $\psi$ is continuous.

Hence $\bw_n \varphi(a_n) \leq \bv_n \varphi(b_n)$
by Assumptions~\ref{L:cont-ext-1}
and~\ref{L:cont-ext-2}.
\end{proof}
%%%%%%%%%%%%%%%%%%%%%%%%%%%%%%%%%%%%%%%%%%%%%%%%%%%%%%%%%%%%%%%%%%%%%%%%%%%%%
%
%                  BENIGN E
%
\subsection{Benign~$E$}
\begin{dfn}
\label{D:benign}
Let $E$ be an ordered Abelian group.
We say~$E$ is \keyword{benign} provided 
that for every valuation system
$\vs{V}{L}\varphi{E}$,
we have
\begin{equation*}
\varphi\text{ is continuous}
\quad\implies\quad
\varphi\text{ is extendible}.
\end{equation*}
\end{dfn}
\todo{Add examples of 
benign spaces:
products of benign spaces;
``$\sigma$-sublattices'' of benign spaces.}


\end{document}
