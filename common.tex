\newcommand{\R}{\mathbb{R}}
\newcommand{\E}{{[-\infty,+\infty]}}
\newcommand{\EX}{\smash{\E^X}}
\newcommand{\N}{\mathbb{N}}
\newcommand{\Lex}{\mathbb{L}}
\newcommand{\ra}{\rightarrow}
\newcommand{\bv}{\textstyle{\bigvee}}
\newcommand{\bw}{\textstyle{\bigwedge}}
\newcommand{\eqdf}{:=}
\newcommand{\ve}{\varepsilon}
\newcommand{\se}{\,\ve\,}

\newcommand{\ld}[1]{d_{\smash{#1}}}
\newcommand{\vs}[4]{#1 \supseteq #2\,\smash{\stackrel{#3}{\ra}}\,#4}

\newcommand{\keyword}[1]{\textbf{#1}}
\newcommand{\todo}[1]{\textbf{[todo\footnote{\textbf{todo:} #1}]}}

\setlist[1]{label=(\roman*)}

\theoremstyle{definition}
\newtheorem{numbering}{Use to get sequential numbering}
\newtheorem{dfn}[numbering]{Definition}
\newtheorem{ex}[numbering]{Example}
\newtheorem{exs}[numbering]{Examples}

\theoremstyle{remark}
\newtheorem{rem}[numbering]{Remark}

\theoremstyle{notation}
\newtheorem{nt}[numbering]{Notation}

\theoremstyle{theorem}
\newtheorem{lem}[numbering]{Lemma}
\newtheorem{cor}[numbering]{Corollary}
\newtheorem{thm}[numbering]{Theorem}
\newtheorem{prop}[numbering]{Proposition}
