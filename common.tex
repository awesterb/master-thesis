% common mathematical symbols
\newcommand{\R}{\mathbb{R}}
\newcommand{\Prm}{\mathbb{P}}
\newcommand{\Z}{\mathbb{Z}}
\newcommand{\Q}{\mathbb{Q}}
\newcommand{\E}{{[-\infty,+\infty]}}
\newcommand{\EX}{\smash{\E^X}}
\newcommand{\N}{\mathbb{N}}
\newcommand{\Lex}{\mathbb{L}}
\newcommand{\QP}{\mathbb{Q}^{\bullet}}
\newcommand{\ra}{\rightarrow}
\newcommand{\ol}{\overline}
\newcommand{\ul}{\underline}
\newcommand{\bv}{\textstyle{\bigvee}}
\newcommand{\bw}{\textstyle{\bigwedge}}
\newcommand{\eqdf}{:=}

%  quotient valuations
\newcommand{\qvL}[1]{#1/{\approx}}
\newcommand{\qvphi}[1]{#1/{\approx}}

% names of common valuation systems
\newcommand{\vs}[4]{#1 \supseteq #2\,\smash{\stackrel{#3}{\ra}}\,#4}
\newcommand{\Lebesguesign}{\mathcal{L}}
\newcommand{\LA}{\mathcal{A}_{\Lebesguesign}}
\newcommand{\Lmu}{\mu_{\Lebesguesign}}
\newcommand{\LF}{F_{\Lebesguesign}}
\newcommand{\Lphi}{\varphi_{\Lebesguesign}}
\newcommand{\vsLA}{\vs{\wp{\R}}{\LA}{\Lmu}{\R}}
\newcommand{\vsLF}{\vs{\E^\R}{\LF}{\Lphi}{\R}}
\newcommand{\Simplesign}{\mathrm{S}}
\newcommand{\SA}{\mathcal{A}_{\Simplesign}}
\newcommand{\Smu}{\mu_{\Simplesign}}
\newcommand{\SF}{F_{\Simplesign}}
\newcommand{\Sphi}{\varphi_{\Simplesign}}
\newcommand{\vsSA}{\vs{\wp{\R}}{\SA}{\Sphi}{\R}}
\newcommand{\vsSF}{\vs{\E^\R}{\SF}{\Smu}{\R}}

% notation for limits
\newcommand{\ulim}[1]{\smash{\overline{\lim}_{#1}}}
\newcommand{\llim}[1]{\smash{\underline{\lim}_{#1}}}
\newcommand{\pulim}[2]{{#1}\text{-}\!\ulim{#2}}
\newcommand{\pllim}[2]{{#1}\text{-}\!\llim{#2}}
\newcommand{\plim}[2]{{#1}\text{-}\!{\lim}_{#2}}


% notation for fitting uniform spaces
\newcommand{\ve}{\varepsilon}
\newcommand{\se}{\,\ve\,}
\newcommand{\dtn}[2]{\smash{\nicefrac{\textstyle{#1}}{#2}}}
\newcommand{\dt}[1]{\dtn{#1}{2}}
\newcommand{\wt}[1]{\smash{\widetilde{#1}}}

\newcommand{\ld}[1]{d_{\smash{#1}}}


% other notation
\newcommand{\dv}{\preccurlyeq}
\newcommand{\lcm}{\operatorname{lcm}}
\newcommand{\keyword}[1]{\textbf{#1}}
\newcommand{\Zmod}[1]{\Z_{#1}}
\newcommand{\todo}[1]{\textbf{[todo\footnote{\textbf{todo:} #1}]}}

% macros for setting
\newcommand{\lsub}[2]{\mathrlap{#1}\phantom{#2}}
\newcommand{\rsub}[2]{\phantom{#2}\mathllap{#1}}


%
%                  CONFIGURATION
%
\setlist[1]{label=(\roman*)}

\theoremstyle{definition}
\newtheorem{numbering}{Use to get sequential numbering}
\newtheorem{dfn}[numbering]{Definition}
\newtheorem{ex}[numbering]{Example}
\newtheorem{exs}[numbering]{Examples}

\theoremstyle{remark}
\newtheorem{rem}[numbering]{Remark}

\theoremstyle{notation}
\newtheorem{nt}[numbering]{Notation}

\theoremstyle{theorem}
\newtheorem{lem}[numbering]{Lemma}
\newtheorem{cor}[numbering]{Corollary}
\newtheorem{thm}[numbering]{Theorem}
\newtheorem{prop}[numbering]{Proposition}


%depth of ToC
\setcounter{tocdepth}{1}
