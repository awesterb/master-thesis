\documentclass[main.tex]{subfiles}
\begin{document}
\section{Valuation Systems}
\label{S:valuation-systems}
\noindent
Note that the Lebesgue measure $\Lmu$
is a complete valuation (see Example~\ref{E:lmeas-complete-val}),
that extends the relatively simple valuation~$\Smu$
(see Example~\ref{E:smeas-val}).

We would like to consider~$\Lmu$ to be \emph{a completion} of~$\Smu$.
What should this mean?
The following definition seems obvious
when one thinks about valuations.
\begin{equation*}
\left[ \quad
\begin{minipage}{.7\columnwidth}
Let $E$ be an ordered Abelian group.\\
Let $L$ and~$K$ be lattices.\\
Let $\psi\colon K \ra E$
and $\varphi\colon L\ra E$
be valuations.\\
We say $\psi$ is \keyword{a completion} of~$\varphi$
provided that\\
$L$ is a sublattice of~$K$,
~$\psi$ extends~$\varphi$,
and  $\psi$ is complete.
\end{minipage}
\right.
\end{equation*}
However,
in the more concrete setting 
of measure theory this broad definition
of completion is not that useful.
After all,
if we are given a completion~$\psi\colon K\ra \R$ of~$\Smu$,
then we only know that~$K$ is a sublattice of~$\SA$,
while we would prefer~$K$ to be a sublattice of sets,
or resemble it.

To mend this problem
we might try to prove 
that any completion of~$\Smu$ is 
essentially a completion on a lattice of subsets.
Of course, the meaning of the previous statement
is not clear.
We suspect that if one gives it an exact meaning,
the statement will be either false or trivial.
So we will not follow this direction.

Instead,
we consider a different notion of completion
that involves the 
the surrounding lattice, $\wp \R$.
More precisely,
we will see that $\Lmu$ is a completion of~$\Smu$
\emph{relative to}~$\wp \R$,
which means 
that
$\Lmu$ extends~$\Smu$
and that $\Lmu$ is complete \emph{relative to~$\wp\R$}
(see Example~\ref{E:complete-lmeas}).
This naturally leads to the study of the following objects.
\begin{equation*}
\vsSA \qquad\qquad \vsLA.
\end{equation*}
That is, we are interested in objects of the following shape.
\begin{equation*}
\vs{V}{L}{\varphi}{E},
\end{equation*}
where $\varphi\colon L \ra E$ is a valuation,
and where $V$ is a lattice such that~$L$ is a sublattice of~$V$.
We call such objects \emph{valuation systems}
(see Definition~\ref{D:system}).

The drawback of this approach is that it requires
quite a bit of bookkeeping,
and so this section
is filled with definitions and examples,
but there is little theory.
We hope the reader will bear with us;
we are confident the reader will be rewarded
for his/her patience
in the next sections.

Since this section
 is already  administrative in nature,
we take this chance
to put 
 some additional restraints on
the notion of valuation system
which turns out to be useful later on.
Given a valuation system,
$\vs{V}{L}{\varphi}{E}$,
we  require that~$E$ is $R$-complete
(see Definition~\ref{D:R-complete}),
and that~$V$ is $\sigma$-distributive
(see Definition~\ref{D:sigma-distributive}).

Before we give a formal definition
of ``valuation system'' 
in Subsection~\ref{SS:valuation-systems},
and define ``complete valuation system''
in Subsection~\ref{SS:complete-valuation-systems},
we consider $\sigma$-distributive lattices
in Subsection~\ref{SS:sigma-distributive}.

We end the section
with ``convex valuation systems''
 in Subsection~\ref{SS:convex}.
%%%%%%%%%%%%%%%%%%%%%%%%%%%%%%%%%%%%%%%%%%%%%%%%%%%%%%%%%%%%%%%%%%%%%%%%%%%%%%%
%
%                  SIGMA DISTRIBUTIVITY 
%
\subsection{$\sigma$-Distributivity}
\label{SS:sigma-distributive}
%
%                  ADDITIONAL RESTRICIONS ON THE SYSTE
% 
\begin{dfn}
\label{D:sigma-distributive}
Let~$V$ be a lattice.
We say~$V$ is
\keyword{$\sigma$-distributive}
provided that
\begin{enumerate}
\item
$V$ is \keyword{$\sigma$-complete}, i.e.,
for every sequence $c_1,\,c_2,\,\dotsc$ in~$V$
we have 
\begin{equation*}
\text{ $\bw_n c_n$ exists\qquad and\qquad $\bv_n c_n$ exists, }
\end{equation*}
\item
and for every  $a\in V$ and $c_1,\,c_2,\,\dotsc\in V$,
we have,
\begin{equation*}
a \vee \bw_n c_n \,=\, \bw_n\  a\vee c_n
\qquad\text{and}\qquad
a \wedge \bv_n c_n \,=\, \bv_n\  a\wedge c_n.
\end{equation*}
\end{enumerate}
\end{dfn}
\begin{exs}
\label{E:sigma-distributive}
\begin{enumerate}
\item
\label{E:sigma-distributive-sets}
Let $X$ be a set. Then $\wp(X)$ is $\sigma$-distributive.
Indeed,  
\begin{equation*}
\textstyle{
A \cup \bigcap_n C_n \,=\, \bigcap_n A \cup C_n
\qquad
A\cap \bigcup_n C_n \,=\, \bigcup_n A \cap C_n}
\end{equation*}
for all $A,\, C_1,C_2,\dotsc \subseteq X$.
\item
Let $C$ be totally ordered
and $\sigma$-complete. Then $C$ is $\sigma$-distributive.

Indeed,
let $a,\,c_1,c_2,\dotsc \in C$ be given.
We need to prove that $a \vee \bw_n c_n$
is the supremum of~$a\vee c_1,\,a\vee c_2,\,\dotsc$.
To this end note that 
\begin{equation*}
b \leq d_1 \vee d_2 \quad\iff\quad 
b\leq d_1\quad\text{or}\quad b\leq d_2
\qquad\quad(b,d_i\in C).
\end{equation*}
(To see this, note that $d_1 \vee d_2 = \max\{d_1,d_2\}$.)
Now, for $\ell \in C$, we have
\begin{alignat*}{3}
\forall n [\ \ell \leq a \vee c_n \ ]
\quad&\iff\quad
\ell \leq a
    \quad\text{or}\quad
    \forall n[\ \ell \leq c_n\ ] \\
\quad&\iff\quad
\ell \leq a
    \quad\text{or}\quad
    \ell \leq \bw_n c_n \\
\quad&\iff\quad
\ell \leq a\vee \bw_n c_n.
\end{alignat*}
So we see 
that $a\vee\bw_n c_n$ is the greatest 
lower bound of~$a\vee c_1,\,a\vee c_2\,\dotsc$.

With the same argument,
one can prove that $a \wedge \bv_n c_n = \bv a \wedge c_n$
for all $a,\,c_1,c_2,\dotsc \in C$
such that $\bv_n c_n$ exists.
Hence $C$ is $\sigma$-distributive.

\item
The lattice of the real numbers~$\R$ is a chain
and so~$\R$ is $\sigma$-distributive
\emph{if}~$\R$ would be $\sigma$-complete.
However,
$\R$ is not $\sigma$-complete.
Indeed,
a sequence $c_1,c_2,\dotsc$ in~$\R$ has a supremum
if and only if it is bounded from above,
i.e., there is an~$a\in \R$ such that $c_n \leq a$
for all~$n$.
Similarly,
a sequence
$c_1,c_2,\dotsc\in \R$ has an infimum
if and only if it is bounded from below.

\item
Let $\E$ be the lattice of the extended real numbers.
Then~$\E$ is a chain and clearly $\sigma$-complete.
Hence $\E$ is $\sigma$-distributive.

\item
\label{E:sigma-distributive-product}
Let $I$ be a set,
and for each~$i\in I$,
let $L_i$ be a $\sigma$-distributive lattice.\\
Then the product $L\eqdf \prod_{i\in I} L_i$
is $\sigma$-distributive.

\item
\label{E:sigma-distributive-functions}
Let $X$ be a set.
Then lattice $\EX$ of functions from~$X$ to~$\E$
is $\sigma$-distributive.
\end{enumerate}
\end{exs}

\subsection{Valuation Systems}
\label{SS:valuation-systems}
%
%                  SYSTEMS
%

\begin{dfn}
\label{D:system}
We say $\vs{V}{L}{\varphi}{E}$
 is a \keyword{valuation system}
provided that
\begin{enumerate}
\item \label{D:simple-system-1}
$V$ is a $\sigma$-distributive lattce 
(see Definition~\ref{D:sigma-distributive});
\item \label{D:simple-system-2}
$L$ is a sublattice of~$V$;
\item \label{D:simple-system-3}
$E$ is an ordered Abelian group,
which is $R$-complete (see Definition~\ref{D:R-complete});
\item \label{D:simple-system-4}
$\varphi\colon L\ra E$ is a valuation.
\end{enumerate}
\end{dfn}
%
%                  RING AS SIMPLE VALUATION SYSTEM
%
\begin{ex}
\label{E:ring-system}
Let $E$ be an $R$-complete ordered Abelian group.
Let~$X$ be a set, 
$\mathcal{A}$ a ring of subsets of~$X$,
and $\mu\colon \mathcal{A}\ra E$
a positive and additive map
(see Example~\ref{E:ring-val}).

Then we have the following  valuation system.
\begin{equation*}
\vs{\wp X}{\mathcal{A}}\mu{E}
\end{equation*}
Indeed, $\wp X$ is lattice with 
$\bw_n A_n = \bigcap_n A_n$
and $\bv_n A_n = \bigcup_n A_n$ for all $A_i \in \wp X$,
the set
$\mathcal{A}$ is a sublattice of~$\wp X$
by definition,
$\wp X$ is $\sigma$-distributive
(see Examples~\ref{E:sigma-distributive}\ref{E:sigma-distributive-sets})
and we have already seen 
that $\mu\colon \mathcal{A}\ra E$ is a valuation
(in Example~\ref{E:ring-val}).

In particular,
we have the following valuation systems.
\begin{equation*}
\vsLA \qquad\qquad \vsSA
\end{equation*}
See Example~\ref{E:lmeas-val} and  Example~\ref{E:smeas-val}.
\end{ex}

%
%                  RIESZ SPACE OF FUNCTIONS AS SIMPLE VALUATION SYSTEM
%
\begin{ex}
\label{E:riesz-function-space-simple-system}
Let $E$ be an $R$-complete ordered Abelian group.
Let $F$ be a Riesz space of functions on a set~$X$
(see Example~\ref{E:val-riesz-space-of-functions}),
and let $\varphi\colon F\ra E$ be a positive and linear map.
Then we have the following  valuation system.
\begin{equation*}
\vs{[-\infty,\infty]^X}{F}\varphi{E}
\end{equation*}
Indeed, 
$[-\infty,\infty]^X$ is a $\sigma$-distributive
lattice 
(see Examples~\ref{E:sigma-distributive}\ref{E:sigma-distributive-functions}).
Further, $F$ is a sublattice of~$\R^X$
which is in turn a sublattice of $[-\infty,\infty]^X$,
and we already know that
$\varphi$ is a valuation (see Example~\ref{E:val-riesz-space-of-functions}).

In particular, since~$\R$ is $R$-complete,
we have the following valuation systems
\begin{equation*}
\vs{\E^X}{(\LF\cap\R^\R)}{\Lphi}{\R},
\qquad\qquad
\vsSF,
\end{equation*}
see Example~\ref{E:int-val} and Example~\ref{E:sint-val}.
Recall that $\LF\cap \R^\R$ is a Riesz space of functions on~$X$,
while~$\LF$ is not.
Of course,
we also have the following valuation sytem.
\begin{equation*}
\vsLF
\end{equation*}
\end{ex}

\begin{ex}
Let $I=\{1,2\}$.
For each~$i\in I$,
let $\vs{V_i}{L_i}{\varphi_i}{E_i}$
be a  valuation system.
Then we have the following  valuation system
(see Example~\ref{E:val-product}).
\begin{equation*}
\vs{V_1\times V_2}{L_1 \times L_2}{\varphi_1 \times \varphi_2}{E_1 \times E_2}
\end{equation*}
Indeed,
the lattice
$V_1\times V_2$ is $\sigma$-distributive
(see Examples~\ref{E:sigma-distributive}\ref{E:sigma-distributive-product}),
and the ordered Abelian group $E_1\times E_2$ is $R$-complete
(see Examples~\ref{E:R-complete}\ref{E:R-complete-product}).
We call this system
the \emph{product} of $\vs{V_1}{L_1}{\varphi_1}{E_1}$
and $\vs{V_2}{L_2}{\varphi_2}{E_2}$.
Of course,
one can similarly define the product of 
an $I$-indexed family of valuation systems
where~$I$ is any set.
\end{ex}

%
%                  NOTATION CONCERNING SUPREMA AND INFIMA IN L
%
\begin{nt}
\label{N:V-inf-sup}
Let $\vs{V}{L}\varphi{E}$ be a  valuation system.
Let $a_1, a_2, \dotsc$ be from~$L$.
Then $a_1, a_2,\dotsc$ has a supremum
in~$V$ and might have a supremum in~$L$.
We ignore the latter:
\textbf{with $\bv_n a_n$
we always mean the supremum of~$a_1, a_2,\dotsc $ in~$V$}.\\
Similarly, \textbf{with $\bw_n a_n$
we always mean the infimum of~$a_1, a_2,\dotsc $ in~$V$}.
\end{nt}
%%%%%%%%%%%%%%%%%%%%%%%%%%%%%%%%%%%%%%%%%%%%%%%%%%%%%%%%%%%%%%%%%%%%%%%%%%%%%%%
%
%                  COMPLETENESS
%
%
\subsection{Complete Valuation Systems}
\label{SS:complete-valuation-systems}
%
%                  COMPLETE SYSTEMS
%
\begin{dfn}
\label{D:system-complete}
Let $\vs{V}{L}\varphi{E}$ be a valuation system.
\begin{enumerate}
\item 
We say $\vs{V}{L}\varphi{E}$
is \keyword{$\Pi$-complete},
or  $\varphi$ is \keyword{$\Pi$-complete relative to} $V$,\\
or even $\varphi$ is  \keyword{$\Pi$-complete} (if no confusion should 
arise with
Definition~\ref{D:complete-val}),\\
if for every $\varphi$-convergent
$a_1\geq a_2 \geq \dotsb$ 
(see Definition~\ref{D:phi-conv}), we have,
\begin{equation*}
   \bw_n a_n \in L\quad 
  \text{and}\quad
  \varphi(\,\bw_n a_n\,) \ =\  \bw_n \varphi(a_n).
\end{equation*}
Here, $\bw_n a_n$
is the infimum of $a_1 \geq a_2 \geq \dotsb$ in~$V$
(see Notation~\ref{N:V-inf-sup}).

\item
Similarly,
we say $\vs{V}{L}\varphi{E}$
is \keyword{$\Sigma$-complete}, etc.,\\
provided that for every $\varphi$-convergent sequence
$b_1\leq b_2 \leq \dotsb$ we have
\begin{equation*}
   \bv_n b_n \in L\quad 
  \text{and}\quad
  \varphi(\,\bv_n b_n\,) \ =\  \bv_n \varphi(b_n).
\end{equation*}

\item
We say $\vs{V}{L}\varphi{E}$
is \keyword{complete}, etc.,\\
provided that 
$\vs{V}{L}\varphi{E}$
is both $\Pi$-complete and $\Sigma$-complete.
\end{enumerate}
\end{dfn}
%
%                  REMARK ON COMPLETE VALUATION SYSTEMS
%
\begin{rem}
Let $\vs{V}{L}\varphi{E}$ be a complete valuation
system
(see Definition~\ref{D:system-complete}).
Then the valuation $\varphi$
is also complete in the sense
of Definition~\ref{D:complete-val}.\\
We leave it to the reader to verify this.
\end{rem}
%
%                  THE LEBESGUE INTEGRAL IS COMPLETE
%
\begin{ex}
\label{E:complete-lint}
The Lebesgue integral
gives us the   valuation system
\begin{equation*}
\vsLF
\end{equation*}
(see Example~\ref{E:int-val} and
Example~\ref{E:riesz-function-space-simple-system});
we will prove that this system is complete.

Let $f_1 \leq f_2 \leq \dotsb$
be a $\Lphi$-convergent sequence (in $\LF$).
We must prove that 
\begin{equation*}
\bv_n f_n \in\LF
\qquad\text{and}\qquad\Lphi(\,\bv_n f_n\,) \ =\  \bv_n \Lphi (f_n).
\end{equation*}
This follows immediately from Levi's Monotone Convergence Theorem.

Similarly,
if $g_1 \geq g_2 \geq \dotsb$
is a $\Lphi$-convergent sequence,
then
\begin{equation*}
\bw_n g_n \in\LF
\qquad\text{and}\qquad\Lphi(\,\bw_n g_n\,) \ =\  \bw_n \Lphi (g_n).
\end{equation*}
So the  valuation system $\vsLF$ is complete.

Note that we have given a similar argument earlier 
(see Example~\ref{E:int-complete-val}) to prove
that the valuation $\Lphi$
is complete in the sense of Definition~\ref{D:complete-val}.
\end{ex}
%
%                  THE LEBESGUE MEASURE IS COMPLETE
%
\begin{ex}
\label{E:complete-lmeas}
The Lebesgue measure
given us the   valuation system
\begin{equation*}
\vsLA
\end{equation*}
(see Example~\ref{E:lmeas-val} and
Example~\ref{E:ring-system});
one can prove that this system is complete.

We leave this to the reader (cf.~Example~\ref{E:lmeas-complete-val}).
\end{ex}
%
%
%
\noindent
\begin{cnv}
The complete valuation systems play a far more important
role in the remainder of this thesis
than the complete valuations of Definition~\ref{D:complete-val}.
So:\\
\textbf{Whenever we later on write ``$\varphi$ is complete'' 
we mean that  $\varphi$ is complete relative to~$V$},
where~$V$ should be clear from the context.
\end{cnv}














\subsection{Convex Valuation Systems}
\label{SS:convex}
We have remarked
that the Lebesgue measure is
 complete (see Example~\ref{E:complete-lmeas}).
It should be noted that ``the Lebesgue measure is complete''
has a different meaning in the literature,
namely
that any subset of a Lebesgue neglegible set is neglegible itself.
We call this \emph{convexity} 
and we will briefly discuss this notion
in this subsection.
\begin{dfn}
\label{D:convex}
Let $\vs{V}{L}\varphi{E}$ be a valuation system.\\
We say 
that $\vs{V}{L}\varphi{E}$ is \keyword{convex},
if the following statement holds.
\begin{equation*}
\left[\quad
\begin{minipage}{.7\columnwidth}
Let~$a\leq b$ from~$L$
with $\varphi(a) = \varphi(b)$
be given. Then
\begin{equation*}
a \,\leq\,z\,\leq\, b
\qquad\implies\qquad z\in L,
\end{equation*}
where $z\in V$.
\end{minipage}
\right.
\end{equation*}
\end{dfn}
%
%                  EXAMPLES OF CONVEX VALUATION SYSTEMS
%
\begin{exs}
\begin{enumerate}
\item
The Lebesgue measure $\vsLA$ is convex.

\item
The Lebesgue integral $\vsLF$ is convex.
\end{enumerate}
\end{exs}
%
%                  CONVEXIFICATION
%
\begin{prop}
\label{P:convex-completion}
Let $\vs{V}{L}\varphi{E}$ be a valuation system.
\begin{enumerate}
\item
Let $L_\bullet$ be the subset of~$V$ given by,
for all $z\in V$,
\begin{equation*}
z\in L_\bullet \quad\iff\quad
\exists\, a,b\in L\ [
\quad a\leq z\leq b \quad\wedge\quad \varphi(a) = \varphi(b)\quad].
\end{equation*}
Then $L$ is a sublattice of~$L_\bullet$, which is a sublattice of~$V$.

\item
There is a unique order preserving map $\varphi_\bullet\colon L_\bullet\ra E$
which extends~$\varphi$.\\
Moreover, $\varphi_\bullet$
is a valuation,
and  $\vs{V}{L_\bullet}{\varphi_\bullet}{E}$ is convex.
\end{enumerate}
\end{prop}
\begin{proof}
We leave this to the reader.
\end{proof}
%
%
%
\begin{prop}
\label{P:convexification_versus_completion}
Let $\vs{V}{L}\varphi{E}$ be a complete valuation system.\\
Then 
the valuation system 
$\vs{V}{L_\bullet}{\varphi_\bullet}{E}$ is complete as well.
\end{prop}
\begin{proof}
We leave this to the reader.
\end{proof}

\end{document}
