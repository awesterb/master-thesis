\documentclass[main.tex]{subfiles}
\begin{document}
\section{Valuation Systems}
\label{S:valuation-systems}
Note that the Lebesgue measure $\Lmu$
is a complete valuation (see Example~\ref{E:lmeas-complete-val}),
that extends the relatively  valuation~$\Smu$
(see Example~\ref{E:smeas-val}).

We would like to consider~$\Lmu$ to be a \emph{completion} of~$\Smu$;
what should this mean?

To give a \emph{useful} answer to this question,
it seems we must involve the 
the surrounding lattice $\wp \R$.
We will see that $\Lmu$ is a completion of~$\Smu$
\emph{relative to}~$\wp \R$,
that is,
$\Lmu$ extends~$\Smu$
and $\Lmu$ is complete \emph{relative to~$\wp\R$}
(see Definition~\ref{D:system-complete}).

This naturally leads to the study of the following objects.
\begin{equation*}
\vsSA \qquad\qquad \vsLA.
\end{equation*}
That is, we are interested in objects of the following shape.
\begin{equation*}
\vs{V}{L}{\varphi}{E},
\end{equation*}
where $\varphi\colon L \ra E$ is a valuation,
and where $V$ is a lattice such that~$L$ is a sublattice of~$L$.
We call such objects \emph{valuation systems}
(see Definition~\ref{D:system}),
and we study these valuations systems in this section.

For the development of the theory
it is useful to put some additional restraints on
such valuation systems. 
We require~$E$ to be $R$-complete
(see Definition~\ref{D:R-complete}).
We also require~$V$ to be complete in a certain sense
(see Definition~\ref{D:system}).
Finally, we require~$V$ to be \emph{$\sigma$-distributive}
(see Definition~\ref{D:sigma-distributive}).

Before we give a formal definition
of ``valuation system'' (see Definition~\ref{D:system})
in Subsection~\ref{SS:valuation-systems}
we consider $\sigma$-distributive lattices
in Subsection~\ref{SS:sigma-distributive}.
%%%%%%%%%%%%%%%%%%%%%%%%%%%%%%%%%%%%%%%%%%%%%%%%%%%%%%%%%%%%%%%%%%%%%%%%%%%%%%%
%
%                  SIGMA DISTRIBUTIVITY 
%
\subsection{$\sigma$-distributivity}
%
%                  ADDITIONAL RESTRICIONS ON THE SYSTE
% 
\begin{dfn}
\label{D:sigma-distributive}
Let~$V$ be a lattice.
We say~$V$ is
\keyword{$\sigma$-distributive}
provided that
\begin{enumerate}
\item
$V$ is \keyword{$\sigma$-complete}, i.e.,
for every sequence $c_1,\,c_2,\,\dotsc$ in~$V$
we have 
\begin{equation*}
\text{ $\bw_n c_n$ exists\qquad and\qquad $\bv_n c_n$ exists, }
\end{equation*}
\item
and for every  $a\in V$ and $c_1,\,c_2,\,\dotsc$
we have
\begin{equation*}
a \vee \bw_n c_n \,=\, \bw_n\  a\vee c_n
\qquad\text{and}\qquad
a \wedge \bv_n c_n \,=\, \bv_n\  a\wedge c_n.
\end{equation*}
\end{enumerate}
\end{dfn}
\begin{exs}
\begin{enumerate}
\item
Let $X$ be a set. Then $\wp(X)$ is $\sigma$-distributive.
Indeed,  
\begin{equation*}
\textstyle{
A \cup \bigcap_n C_n \,=\, \bigcap_n A \cup C_n
\qquad
A\cap \bigcup_n C_n \,=\, \bigcup_n A \cap C_n}
\end{equation*}
for all $A,\, C_1,C_2,\dotsc \subseteq X$.
\item
Let $C$ be totally ordered
and $\sigma$-complete. Then $C$ is $\sigma$-distributive.

Indeed,
let $a,\,c_1,c_2,\dotsc \in C$ be given.
We need to prove that $a \vee \bw_n c_n$
is the supremum of~$a\vee c_1,\,a\vee c_2,\,\dotsc$.
To this end note that 
\begin{equation*}
b \leq d_1 \vee d_2 \quad\iff\quad 
b\leq d_1\quad\text{or}\quad b\leq d_2
\qquad\quad(b,d_i\in C).
\end{equation*}
(To see this, recall that $d_1 \vee d_2 = \max\{d_1,d_2\}$.)
Now, for $\ell \in C$, we have
\begin{alignat*}{3}
\forall n [\ \ell \leq a \vee c_n \ ]
\quad&\iff\quad
\ell \leq a
    \quad\text{or}\quad
    \forall n[\ \ell \leq c_n\ ] \\
\quad&\iff\quad
\ell \leq a
    \quad\text{or}\quad
    \ell \leq \bw_n c_n \\
\quad&\iff\quad
\ell \leq a\vee \bw_n c_n.
\end{alignat*}
So we see 
that $a\vee\bw_n c_n$ is the greatest 
lower bound of~$a\vee c_1,\,a\vee c_2\,\dotsc$.

With the same argument,
one can prove that $a \wedge \bv_n c_n = \bv a \wedge c_n$
for all $a,\,c_1,c_2,\dotsc \in C$
such that $\bv_n c_n$ exists.
Hence $C$ is $\sigma$-distributive.

\item
The lattice of the real numbers~$\R$ is a chain
and so~$\R$ is $\sigma$-distributive
if~$\R$ would be $\sigma$-complete.
However,
$\R$ is not $\sigma$-complete.
Indeed,
a sequence $c_1,c_2,\dotsc$ in~$\R$ has a supremum
if and only if it is bounded from above,
i.e. there is an~$a\in \R$ such that $c_n \leq a$
for all~$n$.
Similarly,
a sequence
$c_1,c_2,\dotsc\in \R$ has an infimum
if and only if it is bounded from below.

\item
Let $\E$ be the lattice of the extended real numbers.
Then~$\E$ is a chain and clearly $\sigma$-complete.
Hence $\E$ is $\sigma$-distributive.

\item
Let $I$ be a set,
and for each~$i\in I$,
let $L_i$ be a $\sigma$-distributive lattice.
Then the product $L\eqdf \prod_{i\in I} L_i$
is $\sigma$-distributive.

\item
Let $X$ be a set.
Then lattice $\EX$ of functions from~$X$ to~$\E$
is $\sigma$-distributive.
\end{enumerate}
\end{exs}

\subsection{Valuation systems}
\label{SS:valuation-systems}
%
%                  SYSTEMS
%

\begin{dfn}
\label{D:system}
We say $\vs{V}{L}{\varphi}{E}$
 is a \keyword{valuation system}
provided that
\begin{enumerate}
\item \label{D:simple-system-1}
$V$ is a $\sigma$-distributive lattce 
(see Definition~\ref{D:sigma-distributive});
\item \label{D:simple-system-2}
$L$ is a sublattice of~$V$;
\item \label{D:simple-system-3}
$E$ is an ordered Abelian group,
which is $R$-complete (see Definition~\ref{D:R-complete});
\item \label{D:simple-system-4}
$\varphi\colon L\ra E$ is a valuation.
\end{enumerate}
\end{dfn}
%
%                  RING AS SIMPLE VALUATION SYSTEM
%
\begin{ex}
\label{E:ring-system}
Let $E$ be an $R$-complete ordered Abelian group.
Let~$X$ be a set, 
$\mathcal{A}$ a ring of sets,
and $\mu\colon \mathcal{A}\ra E$
a positive and additive map.
(See Example~\ref{E:ring-val}.)

Then we have the following  valuation system.
\begin{equation*}
\vs{\wp X}{\mathcal{A}}\mu{E}
\end{equation*}
Indeed, $\wp X$ is lattice with 
$\bw_n A_n = \bigcap_n A_n$
and $\bv_n A_n = \bigcup_n A_n$ for all $A_i \in \wp X$,
$\mathcal{A}$ is a sublattice of~$\wp X$
by definition,
$\wp X$ is $\sigma$-distributve,
and we have already seen 
that $\mu\colon \mathcal{A}\ra E$ is a valuation
(in Example~\ref{E:ring-val}).

In particular,
we have the following valuation systems.
\begin{equation*}
\vsLA \qquad\qquad \vsSA.
\end{equation*}
See Example~\ref{E:lmeas-val} and  Example~\ref{E:smeas-val}.
\end{ex}

%
%                  RIESZ SPACE OF FUNCTIONS AS SIMPLE VALUATION SYSTEM
%
\begin{ex}
\label{E:riesz-function-space-simple-system}
Let $E$ be an $R$-complete ordered Abelian group.
Let $F$ be a Riesz space of functions on a set~$X$
(see Example~\ref{E:val-riesz-space-of-functions}),
and $\varphi\colon F\ra E$ be a positive and linear map.
Then we have the following  valuation system.
\begin{equation*}
\vs{[-\infty,\infty]^X}{F}\varphi{E}
\end{equation*}
Indeed, $[-\infty,\infty]$ is a $\sigma$-distributive
lattice,
and hence so is $[-\infty,\infty]^X$.
Further, $F$ is a sublattice of~$\R^X$
which is in turn a sublattice of $[-\infty,\infty]^X$,
and we already know that
$\varphi$ is a valuation (see Example~\ref{E:val-riesz-space-of-functions}).

In particular, since~$\R$ is $R$-complete,
we have the following valuation systems
\begin{equation*}
\vsLF
\qquad\qquad
\vsSF
\end{equation*}
see Example~\ref{E:int-val} and Example~\ref{E:sint-val}.
\end{ex}

\begin{ex}
Let $I=\{1,2\}$.
For each~$i\in I$,
let $\vs{V_i}{L_i}{\varphi_i}{E_i}$
be a  valuation system.
Then we have the following  valuation system
(see Example~\ref{E:val-product}).
\begin{equation*}
\vs{V_1\times V_2}{L_1 \times L_2}{\varphi_1 \times \varphi_2}{E_1 \times E_2}.
\end{equation*}
We call this system
the \emph{product} of $\vs{V_1}{L_1}{\varphi_1}{E_1}$
and $\vs{V_2}{L_2}{\varphi_2}{E_2}$.

Of course,
one can define a product of  valuation systems
for any set~$I$.
\end{ex}

%
%                  NOTATION CONCERNING SUPREMA AND INFIMA IN L
%
\begin{nt}
Let $\vs{V}{L}\varphi{E}$ be a  valuation system.
Let $a_1, a_2, \dotsc$ be from~$L$.
Then $a_1, a_2,\dotsc$ has a supremum
in~$V$ and might have a supremum in~$L$.
We ignore the latter:
\emph{With $\bv_n a_n$
we always mean the supremum of~$a_1, a_2,\dotsc $ in~$V$}.
\end{nt}
%%%%%%%%%%%%%%%%%%%%%%%%%%%%%%%%%%%%%%%%%%%%%%%%%%%%%%%%%%%%%%%%%%%%%%%%%%%%%%%
%
%                  COMPLETENESS
%
%
\subsection{Complete Valuation Systems}
%
%                  COMPLETE SYSTEMS
%
\begin{dfn}
\label{D:system-complete}
Let $\vs{V}{L}\varphi{E}$ be a valuation system.
\begin{enumerate}
\item 
We say $\vs{V}{L}\varphi{E}$
is \keyword{$\Pi$-complete},
or  $\varphi$ is \keyword{$\Pi$-complete relative to} $V$,\\
or even $\varphi$ is  \keyword{$\Pi$-complete} (if no confusion should 
arise with
Definition~\ref{D:complete-val}),\\
provided that for every $\varphi$-convergent sequence
$a_1\geq a_2 \geq \dotsb$ we have
\begin{equation*}
   \bw_n a_n \in L\quad 
  \text{and}\quad
  \varphi(\,\bw_n a_n\,) = \bw_n \varphi(a_n)
\end{equation*}

\item
Similarly,
we say $\vs{V}{L}\varphi{E}$
is \keyword{$\Pi$-complete}, etc.,\\
provided that for every $\varphi$-convergent sequence
$b_1\leq b_2 \leq \dotsb$ we have
\begin{equation*}
   \bv_n b_n \in L\quad 
  \text{and}\quad
  \varphi(\,\bv_n b_n\,) = \bv_n \varphi(b_n).
\end{equation*}

\item
We say $\vs{V}{L}\varphi{E}$
is \keyword{complete}, etc.,\\
provided that 
$\vs{V}{L}\varphi{E}$
is both $\Pi$-complete and $\Sigma$-complete.
\end{enumerate}
\end{dfn}
%
%                  REMARK ON COMPLETE VALUATION SYSTEMS
%
\begin{rem}
Let $\vs{V}{L}\varphi{E}$ be a complete valuation
system
(see Definition~\ref{D:system-complete}).
Then the valuation $\varphi$
is also complete (in the sense
of Definition~\ref{D:complete-val}).\\
We leave it to the reader to verify this.
\end{rem}
%
%                  THE LEBESGUE INTEGRAL IS COMPLETE
%
\begin{ex}
\label{E:complete-lint}
The Lebesgue integral
given us the   valuation system
\begin{equation*}
\vsLF
\end{equation*}
(see Example~\ref{E:int-val} and
Example~\ref{E:riesz-function-space-simple-system});
we will prove that this system is complete.

Let $f_1 \leq f_2 \leq \dotsb$
be a $\Lphi$-convergent sequence (in $\LF$).
We must prove that 
\begin{equation*}
\bv_n f_n \in\LF
\qquad\text{and}\qquad\Lphi(\bv_n f_n) = \bv_n \Lphi (f_n).
\end{equation*}
This follows immediately from Levi's Monotone Convergence Theorem.

Similarly,
if $g_1 \geq g_2 \geq \dotsb$
is a $\Lphi$-convergent sequence,
then
\begin{equation*}
\bw_n g_n \in\LF
\qquad\text{and}\qquad\Lphi(\bw_n g_n) = \bw_n \Lphi (g_n).
\end{equation*}
So the  valuation system $\vsLF$ is complete.

Note that we have given a similar argument earlier 
(see Example~\ref{E:int-complete-val}) to prove
that the valuation $\Lphi$
is complete (in the sense of Definition~\ref{D:complete-val})
\end{ex}
%
%                  THE LEBESGUE MEASURE IS COMPLETE
%
\begin{ex}
\label{E:complete-lmeas}
The Lebesgue measure
given us the   valuation system
\begin{equation*}
\vsLA
\end{equation*}
(see Example~\ref{E:lmeas-val} and
Example~\ref{E:ring-system});
one can prove that this system is complete.

We leave this to the reader (see Example~\ref{E:lmeas-complete-val}).
\end{ex}
%
%
%
The complete valuation systems play a far more important
role in the remainder of this thesis
than the complete valuations of Definition~\ref{D:complete-val}.
So whenever we lateron write ``$\varphi$ is complete'' 
we mean (unless stated otherwise) that  $\varphi$ is complete relative to~$V$,
where~$V$ should be clear from the context.
\end{document}
